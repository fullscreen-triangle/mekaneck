\section{Subjective Time as Circuit Completion Duration}

\subsection{Overview: From Physical to Phenomenological}

Section 3 established that physical geometric tracing requires temporal duration $\tau_{\text{circuit}} > 0$. We now make the crucial connection: \textbf{this physical duration is what consciousness experiences as subjective time}.

Three foundational results establish this connection:
\begin{enumerate}
\item \textbf{Internal time definition}: Subjective duration = sum of circuit completion times (Section 4.2)
\item \textbf{Specious present derivation}: ~100–1000 ms duration of "now" = average circuit completion time (Section 4.3)
\item \textbf{Temporal elasticity}: Time perception variability reflects oscillatory hole dynamics (Section 4.4)
\end{enumerate}

\subsection{Internal Time: The Fundamental Definition}

\subsubsection{Geometric Completion as Temporal Experience}

Section 3.2 established that physical tracing of geometric structures takes duration $\tau_{\text{trace}}$. The hypothesis: \textit{this tracing duration IS subjective time}.

\begin{definition}[Internal Time]
\label{def:internal_time}
The \textbf{internal time} $T_{\text{internal}}$ experienced by a conscious system is the cumulative duration of circuit completion events for all active oscillatory holes during the conscious moment:
\begin{equation}
\boxed{T_{\text{internal}} = \sum_{i=1}^{N_{\text{holes}}} \tau_{\text{circuit}}^{(i)}}
\end{equation}

where:
\begin{itemize}
\item $N_{\text{holes}}$ = number of oscillatory holes in active thought geometries
\item $\tau_{\text{circuit}}^{(i)}$ = circuit completion time for hole $i$
\end{itemize}
\end{definition}

\textbf{Physical interpretation}: Each oscillatory hole in a thought geometry requires electron stabilization. The time required for stabilization ($\tau_{\text{circuit}}$) is directly experienced as duration. Multiple holes complete in parallel and serially, and their total completion time constitutes the felt passage of time.

\begin{theorem}[Internal Time as Geometric Tracing]
\label{thm:internal_time}
Subjective temporal flow is the phenomenological correlate of geometric tracing during circuit completion. The experience of "duration" directly reflects the physical time required to trace oscillatory hole patterns.
\end{theorem}

\begin{proof}[Justification]
From Section 3:
\begin{itemize}
\item Mathematical structures exist timelessly (Theorem \ref{thm:mathematical_timelessness})
\item Physical instantiation requires tracing (Theorem \ref{thm:physical_tracing})
\item Oscillations necessarily have temporal extension (Proposition \ref{prop:oscillations_require_duration})
\end{itemize}

From Section 2:
\begin{itemize}
\item Consciousness involves oscillatory hole patterns (Definition \ref{def:thought_geometry})
\item Circuit completion stabilizes holes via electron transport (Definition \ref{def:circuit_completion})
\item Completion times are $\tau_{\text{circuit}} \sim 1$–100 ms (Proposition \ref{prop:circuit_time})
\end{itemize}

\textbf{The connection}: For a thought geometry $\mathcal{G}_{\text{thought}} = \{(\mathbf{r}_i, \mathcal{H}_i)\}$ to be consciously experienced:
\begin{enumerate}
\item Each hole $\mathcal{H}_i$ must be physically traced (instantiated)
\item Tracing requires circuit completion time $\tau_{\text{circuit}}^{(i)}$
\item Total duration for complete thought geometry: $T_{\text{internal}} = \sum_i \tau_{\text{circuit}}^{(i)}$
\item This duration is \textit{directly experienced} as subjective time
\end{enumerate}

The experience cannot be instantaneous because physical tracing cannot be instantaneous. The felt duration reflects actual physical duration. \qed
\end{proof}

\subsubsection{Arc Length Formulation}

We can express internal time more formally as geometric arc length in oscillatory hole configuration space.

\begin{proposition}[Time as Geometric Arc Length]
\label{prop:time_arc_length}
Internal time equals the arc length of the trajectory traced through oscillatory hole configuration space:
\begin{equation}
T_{\text{internal}} = \int_{\text{conscious stream}} \|\nabla_{\text{geometric}} \mathcal{H}(\tau)\| \, d\tau
\end{equation}

where:
\begin{itemize}
\item $\mathcal{H}(\tau)$ = oscillatory hole configuration at trace parameter $\tau$
\item $\nabla_{\text{geometric}}$ = geometric gradient operator
\item $\|\cdot\|$ = magnitude in configuration space
\end{itemize}
\end{proposition}

\begin{proof}
Circuit completion traces path through configuration space of oscillatory holes. At trace parameter $\tau$, system occupies configuration $\mathcal{H}(\tau)$.

The infinitesimal arc length element:
\begin{equation}
ds = \left\|\frac{d\mathcal{H}}{d\tau}\right\| d\tau = \|\nabla_{\text{geometric}} \mathcal{H}\| \, d\tau
\end{equation}

Integrating over conscious stream (from initial configuration $\mathcal{H}_0$ to final $\mathcal{H}_f$):
\begin{equation}
T_{\text{internal}} = \int_{\mathcal{H}_0}^{\mathcal{H}_f} ds = \int_{\text{stream}} \|\nabla_{\text{geometric}} \mathcal{H}(\tau)\| \, d\tau
\end{equation}

This is literally the "length" of the path traced through configuration space. Longer paths (more complex geometric patterns) take more time. \qed
\end{proof}

\textbf{Physical meaning}: Subjective time is not arbitrary but geometrically determined by the "distance" traversed in oscillatory hole space. Complex thoughts (long paths) feel longer than simple thoughts (short paths).

\subsection{The Specious Present: Deriving the ~100–1000 ms Duration}

\subsubsection{William James and the Specious Present}

Psychologist William James noted that conscious experience does not occur at an instant but extends over ~0.1–3 seconds \cite{james1890principles}:

\begin{quotation}
\textit{"The practically cognized present is no knife-edge, but a saddle-back, with a certain breadth of its own on which we sit perched, and from which we look in two directions into time... We do not first feel one end and then feel the other after it, and from the perception of the succession infer an interval of time between, but we seem to feel the interval of time as a whole, with its two ends embedded in it."} — William James
\end{quotation}

This extended present—the \textbf{specious present}—is the duration of immediate experience. Modern neuroscience confirms $\sim$100–3000 ms timescale \cite{poeppel2010temporal,buzsaki2006rhythms}.

\textbf{Question}: Why does the specious present have this particular duration? Why not 10 ms or 10 seconds?

\subsubsection{Derivation from Circuit Completion}

Our framework provides quantitative answer.

\begin{theorem}[Specious Present from Circuit Completion]
\label{thm:specious_present}
The duration of the specious present equals the average circuit completion time for coherent neural oscillatory hole ensembles:
\begin{equation}
\tau_{\text{specious}} = \langle \tau_{\text{circuit}} \rangle_{\text{ensemble}}
\end{equation}

For typical neural parameters, this yields $\tau_{\text{specious}} \sim 100$–1000 ms, matching empirical observations.
\end{theorem}

\begin{proof}
The specious present is the "unit" of conscious experience—the minimal felt duration that constitutes a complete conscious moment.

From Definition \ref{def:internal_time}, this corresponds to completing one coherent set of oscillatory holes:
\begin{equation}
\tau_{\text{specious}} = \sum_{i \in \text{ensemble}} \tau_{\text{circuit}}^{(i)}
\end{equation}

For ensemble of $N_{\text{holes}}$ holes completing in parallel:
\begin{equation}
\tau_{\text{specious}} \approx \max_i \{\tau_{\text{circuit}}^{(i)}\} \sim \langle \tau_{\text{circuit}} \rangle
\end{equation}

From Proposition \ref{prop:circuit_time}:
\begin{itemize}
\item Gamma oscillations (30–100 Hz): $\tau \sim 10$–30 ms
\item Theta oscillations (4–8 Hz): $\tau \sim 125$–250 ms
\item Delta oscillations (1–4 Hz): $\tau \sim 250$–1000 ms
\end{itemize}

Conscious experience integrates across multiple frequency bands. Average:
\begin{equation}
\langle \tau_{\text{circuit}} \rangle \sim \frac{1}{3}(30 + 200 + 500) \text{ ms} \sim 240 \text{ ms}
\end{equation}

This matches the empirical specious present of ~200–300 ms for typical conscious processing \cite{poeppel2010temporal}. \qed
\end{proof}

\subsubsection{Experimental Validation: Poeppel's Two-Window Model}

Neuroscientist David Poeppel proposed a "two-window" model of temporal integration \cite{poeppel2010temporal}:
\begin{itemize}
\item \textbf{Short window} (~20–50 ms): Phonemic processing, gamma oscillations
\item \textbf{Long window} (~150–300 ms): Syllabic processing, theta oscillations
\end{itemize}

These correspond precisely to circuit completion timescales:
\begin{itemize}
\item Short: Gamma-band circuits ($\tau \sim 10$–30 ms)
\item Long: Theta-band circuits ($\tau \sim 125$–250 ms)
\end{itemize}

The "window" is not arbitrary neural integration time but reflects actual circuit completion duration.

\subsection{Temporal Elasticity: Time Perception Variability}

\subsubsection{The Phenomenon of Time Dilation/Compression}

Subjective time is not uniform. Common experiences:
\begin{itemize}
\item \textbf{"Time flies"}: When engaged/entertained, time seems compressed
\item \textbf{"Time drags"}: When bored/waiting, time seems dilated
\item \textbf{Fear dilation}: During dangerous situations, time seems to slow dramatically
\item \textbf{Flow states}: Deep absorption where hours feel like minutes
\end{itemize}

\textbf{Question}: Why does subjective time vary if clock time is constant?

\subsubsection{Oscillatory Hole Dynamics Explanation}

From Definition \ref{def:internal_time}:
\begin{equation}
T_{\text{internal}} = \sum_{i=1}^{N_{\text{holes}}} \tau_{\text{circuit}}^{(i)}
\end{equation}

Internal time depends on two factors:
\begin{enumerate}
\item $N_{\text{holes}}$: Number of oscillatory holes being traced
\item $\tau_{\text{circuit}}^{(i)}$: Completion time per hole
\end{enumerate}

\textbf{Time dilation} (time slows down):
\begin{itemize}
\item \textbf{Increased $N_{\text{holes}}$}: More holes → more circuits → longer total duration
\item \textbf{Longer $\tau_{\text{circuit}}$}: Slower electron transport → longer per-hole duration
\end{itemize}

\textbf{Time compression} (time speeds up):
\begin{itemize}
\item \textbf{Decreased $N_{\text{holes}}$}: Fewer holes → fewer circuits → shorter total duration
\item \textbf{Shorter $\tau_{\text{circuit}}$}: Faster electron transport → shorter per-hole duration
\end{itemize}

\begin{theorem}[Temporal Elasticity]
\label{thm:temporal_elasticity}
Subjective time perception varies with oscillatory hole generation rate and electron transport efficiency:
\begin{equation}
\frac{T_{\text{internal}}}{T_{\text{clock}}} = \frac{N_{\text{holes}}(T_{\text{clock}}) \cdot \langle \tau_{\text{circuit}} \rangle}{T_{\text{clock}}}
\end{equation}

When $N_{\text{holes}}$ or $\tau_{\text{circuit}}$ increase, subjective time dilates relative to clock time.
\end{theorem}

\subsubsection{Specific Examples}

\textbf{Example 1—Fear response}:

During danger:
\begin{itemize}
\item Amygdala activation → massive norepinephrine release
\item Heightened attention → increased neural oscillatory activity
\item More oscillatory holes generated: $N_{\text{holes}}^{\text{fear}} \gg N_{\text{holes}}^{\text{baseline}}$
\item Result: $T_{\text{internal}} \gg T_{\text{clock}}$ (time seems to slow)
\end{itemize}

Quantitative estimate: If $N_{\text{holes}}$ increases 5-fold during fear, subjective duration stretches $5\times$ (3 clock seconds feel like 15 subjective seconds).

\textbf{Example 2—Flow states}:

During flow (deep absorption):
\begin{itemize}
\item Focused attention → reduced meta-cognitive monitoring
\item Fewer "self-referential" oscillatory holes: $N_{\text{holes}}^{\text{flow}} \ll N_{\text{holes}}^{\text{baseline}}$
\item Smooth geometric tracing (short arc length in configuration space)
\item Result: $T_{\text{internal}} \ll T_{\text{clock}}$ (time seems to fly)
\end{itemize}

Quantitative estimate: If $N_{\text{holes}}$ decreases 3-fold during flow, subjective duration compresses $3\times$ (60 clock minutes feel like 20 subjective minutes).

\textbf{Example 3—Boredom}:

During boredom:
\begin{itemize}
\item Lack of engaging input → reduced circuit completion efficiency
\item Longer per-hole completion time: $\tau_{\text{circuit}}^{\text{bored}} > \tau_{\text{circuit}}^{\text{baseline}}$
\item OR increased temporal meta-monitoring (checking time repeatedly) increases $N_{\text{holes}}$
\item Result: $T_{\text{internal}} > T_{\text{clock}}$ (time drags)
\end{itemize}

\subsection{Connection to Consciousness Geometry}

\subsubsection{The NOW Moment Revisited}

Our previous work established consciousness as the confluence of perception flux and thought geometry \cite{sachikonye2024consciousness}. Perception flux $\Psi_p(t)$ and thought geometry $\Theta_t(t)$ decay exponentially, intersecting at unique time $t^*$:
\begin{equation}
t^* = \frac{\tau_p \tau_t}{\tau_t - \tau_p} \ln\left(\frac{\Theta_0}{\Psi_0}\right)
\end{equation}

This $t^*$ defines the "NOW" moment—the experienced present.

\textbf{New insight}: The duration of the NOW moment (its "thickness") is precisely the circuit completion time:
\begin{equation}
\Delta t_{\text{NOW}} = \tau_{\text{circuit}}
\end{equation}

The NOW is not an instant but has finite width because circuit completion takes time. The specious present IS the circuit completion duration.

\subsubsection{Stream of Consciousness as Geometric Tracing}

William James described consciousness as a "stream" \cite{james1890principles}—continuous flow rather than discrete snapshots.

\textbf{Geometric explanation}: Consciousness is trajectory along confluence curve in $(t, \Psi, \Theta)$ space. As perception and thought evolve, the intersection point $t^*(t)$ traces a path—the conscious stream.

The "speed" of flow:
\begin{equation}
v_{\text{stream}} = \left|\frac{d\mathcal{C}}{dt}\right| = \sqrt{\left(\frac{dt^*}{dt}\right)^2 + \left(\frac{d\Psi}{dt}\right)^2 + \left(\frac{d\Theta}{dt}\right)^2}
\end{equation}

\textbf{Subjective flow rate}: When $v_{\text{stream}}$ is high (rapid geometric tracing), time seems compressed. When low (slow tracing), time seems dilated.

\textbf{Quantitative prediction}:
\begin{equation}
\frac{T_{\text{internal}}}{T_{\text{clock}}} \propto \frac{1}{v_{\text{stream}}}
\end{equation}

Faster geometric tracing → shorter subjective duration per clock interval.

\subsection{Mathematical Unification: The Complete Formula}

Synthesizing Sections 4.2–4.4:

\begin{equation}
\boxed{
\begin{aligned}
T_{\text{internal}} &= \int_{\text{conscious stream}} \|\nabla_{\text{geometric}} \mathcal{H}(\tau)\| \, d\tau \\
&= \sum_{i=1}^{N_{\text{holes}}(t)} \tau_{\text{circuit}}^{(i)} \\
&= N_{\text{holes}}(t) \cdot \langle \tau_{\text{circuit}} \rangle \\
&\approx \frac{1}{v_{\text{stream}}} \cdot T_{\text{clock}}
\end{aligned}
}
\end{equation}

where:
\begin{itemize}
\item Line 1: Arc length formulation (Proposition \ref{prop:time_arc_length})
\item Line 2: Circuit completion sum (Definition \ref{def:internal_time})
\item Line 3: Average over ensemble (Theorem \ref{thm:specious_present})
\item Line 4: Stream velocity relation (Section 4.5.2)
\end{itemize}

This unified formula connects:
\begin{itemize}
\item Geometric tracing (differential geometry)
\item Circuit completion (neural biophysics)
\item Oscillatory holes (quantum field theory)
\item Consciousness geometry (phenomenology)
\end{itemize}

All describing the same phenomenon: \textbf{subjective time as geometric tracing duration}.

\subsection{Summary: Subjective Time Explained}

We have established four foundational results:

\begin{enumerate}
\item \textbf{Internal time definition}: $T_{\text{internal}} = \sum_i \tau_{\text{circuit}}^{(i)}$ (Definition \ref{def:internal_time}, Theorem \ref{thm:internal_time})

\item \textbf{Specious present derivation}: $\tau_{\text{specious}} \sim \langle \tau_{\text{circuit}} \rangle \sim 100$–1000 ms (Theorem \ref{thm:specious_present})

\item \textbf{Temporal elasticity}: Time perception varies with $N_{\text{holes}}$ and $\tau_{\text{circuit}}$ (Theorem \ref{thm:temporal_elasticity})

\item \textbf{Geometric formulation}: $T_{\text{internal}} = \int \|\nabla \mathcal{H}\| d\tau$ (Proposition \ref{prop:time_arc_length})
\end{enumerate}

The mystery of subjective time dissolves: we experience temporal flow because consciousness is implemented through geometric tracing, and tracing necessarily takes measurable duration. The felt passage of time directly reflects physical circuit completion times—not an illusion, not mere psychology, but the phenomenological face of a fundamental physical process.

Section 5 validates these predictions experimentally.

