\section{Mathematical Foundations: Oscillatory Reality and Categorical Completion}

\subsection{Overview: Building the Framework}

Before establishing the geometric-temporal distinction and internal time definition, we must lay the mathematical groundwork. This section synthesizes results from our established framework \cite{sachikonye2024categorical,sachikonye2024maxwell,sachikonye2024harmonic} that are essential for understanding subjective time as geometric tracing.

Four foundational results underpin our analysis:

\begin{enumerate}
\item \textbf{Oscillatory manifestation as necessity}: Physical reality fundamentally consists of oscillatory dynamics, not particles or fields (Section 2.2).

\item \textbf{Categorical topology}: Discrete categorical states emerge from continuous oscillatory dynamics through completion operators and irreversibility axioms (Section 2.3).

\item \textbf{Temporal emergence}: Time arises from categorical completion sequences, not as external parameter (Section 2.4).

\item \textbf{Circuit completion events}: Oscillatory holes stabilized by electron transport constitute discrete information processing units with measurable duration (Section 2.5).
\end{enumerate}

These results establish that time is not primitive but \textit{emergent from categorical completion}—a crucial insight for understanding subjective temporal experience.

\subsection{Oscillatory Reality: Mathematical Necessity}

\subsubsection{The Foundational Question}

What is the fundamental nature of physical reality? Traditional answers—particles (atomism), fields (continuum mechanics), information (digital physics)—all face logical difficulties. We demonstrate through rigorous proof that oscillatory dynamics are not merely descriptive but \textit{logically necessary} for self-consistent physical manifestation \cite{sachikonye2024categorical}.

\begin{definition}[Self-Consistent Mathematical Structure]
\label{def:self_consistent}
A mathematical structure $\mathcal{M}$ is \textbf{self-consistent} if it satisfies:
\begin{enumerate}
\item \textbf{Completeness}: Every well-formed statement in $\mathcal{M}$ possesses definite truth value
\item \textbf{Consistency}: No contradictions exist within $\mathcal{M}$
\item \textbf{Self-Reference}: $\mathcal{M}$ can formulate statements about its own structural properties
\item \textbf{Manifestation}: Truth of existence statements requires concrete instantiation
\end{enumerate}
\end{definition}

The fourth criterion—manifestation—is critical. Abstract structures cannot possess truth values for existence claims without physical instantiation.

\begin{theorem}[Mathematical Necessity of Oscillatory Manifestation]
\label{thm:oscillatory_necessity}
Self-consistent mathematical structures necessarily manifest as oscillatory patterns. No other mode of physical existence satisfies all self-consistency requirements.
\end{theorem}

\begin{proof}[Proof sketch]
Full proof in \cite{sachikonye2024categorical}. Key steps:

\textbf{Step 1 (Existence requirement)}: Let $\mathcal{M}$ be self-consistent. By self-reference, $\mathcal{M}$ contains statement $E(\mathcal{M})$: "$\mathcal{M}$ exists." By completeness, $E(\mathcal{M})$ has truth value. By consistency, if false, $\mathcal{M}$ contains false self-statements, violating self-consistency. Therefore $E(\mathcal{M}) = \text{TRUE}$.

\textbf{Step 2 (Manifestation necessity)}: Truth of $E(\mathcal{M})$ requires physical manifestation:
\begin{equation}
\exists \Psi_{\text{physical}}: \Psi_{\text{physical}} \text{ instantiates } \mathcal{M}
\end{equation}

\textbf{Step 3 (Dynamic requirement)}: Self-reference necessitates dynamics. Static configurations cannot encode statements about structure. Therefore:
\begin{equation}
\frac{d\Psi_{\text{physical}}}{dt} \neq 0
\end{equation}

\textbf{Step 4 (Boundedness)}: Physical systems have finite energy $E < \infty$. By energy conservation, trajectories are bounded to hypersurface $\mathcal{S}_E = \{(\mathbf{q}, \mathbf{p}) : H(\mathbf{q}, \mathbf{p}) = E\}$.

\textbf{Step 5 (Oscillatory uniqueness)}: Given dynamics (Step 3) and boundedness (Step 4):
\begin{itemize}
\item \textit{Monotonic}: Violates boundedness ($Q(t) \to \infty$)
\item \textit{Static}: Violates dynamics ($d\Psi/dt = 0$)
\item \textit{Chaotic}: Violates self-consistency (sensitive dependence destroys structure)
\item \textit{Oscillatory}: Satisfies all requirements (bounded, dynamic, recurrent, deterministic)
\end{itemize}

Therefore oscillatory dynamics are the unique valid manifestation mode. \qed
\end{proof}

\begin{corollary}[Ontological Primacy of Oscillations]
\label{cor:oscillatory_primacy}
Particles, fields, and geometric structures are not fundamental but emergent descriptions of underlying oscillatory patterns in various limits.
\end{corollary}

\textbf{Physical validation}: Quantum mechanics explicitly implements oscillatory ontology. The Schrödinger equation:
\begin{equation}
i\hbar \frac{\partial \psi}{\partial t} = \hat{H}\psi
\end{equation}
yields oscillatory solutions:
\begin{equation}
\psi(\mathbf{r}, t) = \sum_n c_n \psi_n(\mathbf{r}) e^{-iE_n t/\hbar}
\end{equation}

The temporal factor $e^{-iE_n t/\hbar} = e^{-i\omega_n t}$ is pure oscillation. Even ground states exhibit zero-point oscillations. Physical states \textit{are} oscillatory patterns \cite{sachikonye2024categorical}.

\subsection{Categorical Topology: Discrete from Continuous}

\subsubsection{The Discretization Problem}

Section 2.2 established continuous oscillatory reality. However, observation requires discrete distinctions—partitioning continuous flux into categories. This section establishes how discrete categorical structure emerges from continuous oscillatory dynamics \cite{sachikonye2024categorical}.

\begin{definition}[Categorical Space]
\label{def:categorical_space}
A \textbf{categorical space} is structure $(\mathcal{C}, \prec, \mu, \tau, \mathcal{F})$ where:
\begin{enumerate}
\item $\mathcal{C}$ = set of categorical states
\item $\prec$ = partial order (completion order)
\item $\mu: \mathcal{C} \times \mathbb{R}_{\geq 0} \to \{0, 1\}$ = completion operator
\item $\tau$ = specialization topology induced by $\prec$
\item $\mathcal{F}: \mathcal{S}_{\text{osc}} \to \mathcal{C}$ = categorical assignment mapping oscillatory configurations to categorical states
\end{enumerate}
\end{definition}

\begin{axiom}[Categorical Irreversibility]
\label{axiom:irreversibility}
For all categorical states $C \in \mathcal{C}$ and times $t_1 \leq t_2$:
\begin{equation}
\mu(C, t_1) = 1 \implies \mu(C, t_2) = 1
\end{equation}
Once a categorical state is completed, it remains completed. Completion is irreversible.
\end{axiom}

This irreversibility is \textit{topological}, not merely thermodynamic. Even if a system returns to the same spatial configuration, the categorical state $C$ representing that configuration cannot be "uncompleted."

\begin{definition}[Completion Trajectory]
\label{def:completion_trajectory}
A \textbf{completion trajectory} is function $\gamma: \mathbb{R}_{\geq 0} \to \mathcal{P}(\mathcal{C})$ satisfying:
\begin{enumerate}
\item $\gamma(t) = \{C \in \mathcal{C} : \mu(C, t) = 1\}$ (completed states at time $t$)
\item $t_1 \leq t_2 \implies \gamma(t_1) \subseteq \gamma(t_2)$ (monotonicity)
\item Downward-closed: if $C \in \gamma(t)$ and $C' \prec C$, then $C' \in \gamma(t)$
\end{enumerate}
\end{definition}

\begin{definition}[Categorical Completion Rate]
\label{def:completion_rate}
The \textbf{categorical completion rate} is:
\begin{equation}
\dot{C}(t) = \frac{d|\gamma(t)|}{dt}
\end{equation}
where $|\gamma(t)|$ denotes measure of completed states at time $t$.
\end{definition}

\subsubsection{Equivalence Classes and Degeneracy}

The power of categorical spaces lies in equivalence class structure: many distinct microscopic configurations produce identical macroscopic observables.

\begin{definition}[Categorical Equivalence]
\label{def:categorical_equiv}
Two oscillatory configurations $\psi_1, \psi_2 \in \mathcal{S}_{\text{osc}}$ are \textbf{categorically equivalent} under observer $\mathcal{O}$ if:
\begin{equation}
\mathcal{F}_{\mathcal{O}}(\psi_1) = \mathcal{F}_{\mathcal{O}}(\psi_2)
\end{equation}
\end{definition}

\begin{definition}[Equivalence Class Degeneracy]
\label{def:degeneracy}
The \textbf{degeneracy} of categorical state $C$ is:
\begin{equation}
\delta(C) = |[\psi]_C| = |\{\psi \in \mathcal{S}_{\text{osc}} : \mathcal{F}(\psi) = C\}|
\end{equation}
Number of distinct oscillatory configurations mapping to $C$.
\end{definition}

For typical biological systems, $\delta(C) \sim 10^{10}$ to $10^{100}$ \cite{sachikonye2024maxwell}. This astronomical degeneracy enables computational efficiency: by operating on categorical states rather than microscopic configurations, observers achieve efficiency gains of $\sim 10^{10^{80}}$ \cite{sachikonye2024categorical}.

\subsection{Temporal Emergence from Categorical Structure}

\subsubsection{Time as Emergent Coordinate}

The categorical framework establishes that time is not externally imposed but emerges from completion structure.

\begin{definition}[Temporal Coordinate]
\label{def:temporal_coordinate}
The \textbf{temporal coordinate} $T$ emerges as indexing structure for categorical completion:
\begin{equation}
T: \mathcal{C} \to \mathbb{R}_{\geq 0}, \quad T(C) = \inf\{t : \mu(C, t) = 1\}
\end{equation}
$T(C)$ is the parameter value at which state $C$ first completes.
\end{definition}

\begin{theorem}[Temporal Emergence from Categorical Structure]
\label{thm:temporal_emergence}
The partial order $\prec$ on categorical space induces temporal ordering. Time is not externally imposed but emerges from categorical completion structure.

Specifically, for all $C_i, C_j \in \mathcal{C}$:
\begin{equation}
C_i \prec C_j \implies T(C_i) \leq T(C_j)
\end{equation}
\end{theorem}

\begin{proof}
Suppose $C_i \prec C_j$ (predecessor relationship). By order compatibility (axiom from \cite{sachikonye2024categorical}), if $C_j$ completes at time $T(C_j)$, then $C_i$ must have completed at earlier time $T(C_i) \leq T(C_j)$.

The partial order $\prec$ provides discrete precedence structure. The temporal coordinate $T$ embeds this discrete structure in continuous real line $\mathbb{R}_{\geq 0}$, creating temporal flow from categorical sequencing. \qed
\end{proof}

\begin{corollary}[Time Without External Clock]
\label{cor:no_external_time}
Physical time is not external parameter but emergent structure arising from sequential completion of categorical states. The "arrow of time" is identical to irreversibility of categorical completion (Axiom \ref{axiom:irreversibility}).
\end{corollary}

\textbf{Critical insight}: This resolves the "problem of time" in quantum gravity. Time is not a fundamental coordinate requiring quantization but an emergent bookkeeping parameter for categorical state transitions. At Planck scale where categorical distinctions break down, time itself becomes ill-defined—precisely as expected \cite{sachikonye2024categorical}.

\subsection{Oscillatory-Categorical Equivalence}

\subsubsection{The Bridge Between Continuous and Discrete}

Sections 2.2–2.4 established two seemingly distinct frameworks: continuous oscillatory dynamics and discrete categorical completion. The following theorem proves they are mathematically identical \cite{sachikonye2024categorical}.

\begin{theorem}[Oscillatory-Categorical Equivalence]
\label{thm:osc_cat_equiv}
There exists bijection $\Phi: \mathcal{S}_{\text{osc}} \to \mathcal{C}$ such that:
\begin{enumerate}
\item Oscillatory configuration $\psi$ terminates $\iff$ categorical state $\Phi(\psi)$ completes
\item Termination probability equals completion probability: $\beta(\psi) = \alpha(\Phi(\psi))$
\item Oscillatory entropy equals categorical entropy:
\begin{equation}
S_{\text{osc}}(\psi) = S_{\text{cat}}(\Phi(\psi))
\end{equation}
\end{enumerate}
\end{theorem}

\begin{proof}[Proof sketch]
Full proof in \cite{sachikonye2024categorical}. We construct explicit bijection $\Phi$ as categorical assignment function mapping oscillatory configurations to equivalence classes.

\textbf{Termination-completion}: Oscillatory termination (reaching equilibrium $\psi \approx \psi_{\text{eq}}$) corresponds precisely to categorical completion (no further states accessed).

\textbf{Probability equivalence}: Basin volume in oscillatory space equals categorical path count: $\beta(\psi) = \alpha(\Phi(\psi))$.

\textbf{Entropy identity}: From definitions $S_{\text{osc}} = -k_B \log \beta$ and $S_{\text{cat}} = -k_B \log \alpha$, combined with probability equivalence. \qed
\end{proof}

\begin{corollary}[Frequency-Category Identity]
\label{cor:freq_cat_identity}
Each harmonic frequency mode $\omega_n$ corresponds bijectively to categorical state $C_n$:
\begin{equation}
\boxed{\omega_n \equiv C_n}
\end{equation}
This is not correlation but \textbf{identity}—frequency modes ARE categorical states.
\end{corollary}

\textbf{Implication}: Since $\omega_n \equiv C_n$, measuring frequencies is equivalent to measuring categorical completion rates. This validates the "frequency-domain primacy" in our experimental protocols (Section 5) \cite{sachikonye2024harmonic}.

\subsection{Circuit Completion Events: The Physical Substrate}

\subsubsection{Oscillatory Holes and Electron Stabilization}

Having established the abstract mathematical framework, we now identify its physical implementation in biological systems.

\begin{definition}[Oscillatory Hole]
\label{def:oscillatory_hole}
An \textbf{oscillatory hole} is a functional absence—a missing pattern in the quantum field configuration of molecular gases (particularly O$_2$). Specifically, it is a region where expected oscillatory modes are suppressed or absent \cite{sachikonye2024thoughtgeometry}.
\end{definition}

In neural tissue, oscillatory holes arise from specific geometric arrangements of oxygen molecules around sites of neuronal activity. These holes are not physical voids but \textit{informational absences}—patterns that "should" be present according to equilibrium quantum field theory but are missing due to non-equilibrium dynamics.

\begin{definition}[Circuit Completion Event]
\label{def:circuit_completion}
A \textbf{circuit completion event} occurs when an oscillatory hole is stabilized by electron transport from a phase-lock network, creating a discrete information processing unit \cite{sachikonye2024circuits}.

Formally:
\begin{equation}
\text{Hole}(\mathcal{H}) + \text{Electron}(e^-) \xrightarrow{\tau_{\text{circuit}}} \text{Completed Circuit}(\mathcal{H}_{e^-})
\end{equation}

The characteristic time $\tau_{\text{circuit}}$ is the circuit completion time.
\end{definition}

\begin{proposition}[Circuit Completion Time Scale]
\label{prop:circuit_time}
For neural oscillatory holes, circuit completion times satisfy:
\begin{equation}
\tau_{\text{circuit}} \sim 1\text{--}100 \text{ ms}
\end{equation}

Specifically:
\begin{itemize}
\item Fast circuits (gamma oscillations, 30–100 Hz): $\tau \sim 10$–30 ms
\item Medium circuits (theta oscillations, 4–8 Hz): $\tau \sim 125$–250 ms
\item Slow circuits (delta oscillations, 1–4 Hz): $\tau \sim 250$–1000 ms
\end{itemize}
\end{proposition}

\begin{proof}[Justification]
These timescales emerge from:

\textbf{(1) Electron transport rates}: In phase-lock networks, electron hopping between molecular sites occurs at $\sim 10^{12}$–$10^{13}$ Hz (femtosecond to picosecond scale) \cite{sachikonye2024phaselock}. However, \textit{coherent stabilization} of oscillatory holes requires multiple electrons coordinating across extended spatial regions ($\sim$ 1–100 $\mu$m in neural tissue), introducing multiplicative delay.

\textbf{(2) Neural oscillation frequencies}: Measurable EEG/LFP oscillations reflect collective circuit completion rates. Gamma band (30–100 Hz) corresponds to $\tau \sim 10$–30 ms; theta band (4–8 Hz) to $\tau \sim 125$–250 ms \cite{buzsaki2006rhythms}.

\textbf{(3) Direct measurements}: Computational modeling of molecular oscillatory hole dynamics coupled with electron transport from phase-lock networks yields completion times in 10–100 ms range \cite{sachikonye2024circuits}. \qed
\end{proof}

\subsubsection{Thought Geometry as Circuit Patterns}

Circuit completion events do not occur in isolation but form spatial patterns—\textit{thought geometries} \cite{sachikonye2024thoughtgeometry}.

\begin{definition}[Thought Geometry]
\label{def:thought_geometry}
A \textbf{thought geometry} is a three-dimensional spatial arrangement of oxygen molecules around electron-stabilized oscillatory holes, encoding specific cognitive content.

Mathematically:
\begin{equation}
\mathcal{G}_{\text{thought}} = \{(\mathbf{r}_i, \mathcal{H}_i) : i = 1, \ldots, N_{\text{holes}}\}
\end{equation}
where $\mathbf{r}_i \in \mathbb{R}^3$ is spatial position and $\mathcal{H}_i$ is oscillatory hole state.
\end{definition}

\textbf{Key insight}: While we cannot access conscious \textit{content} (Privacy Axiom, Section 1.2.1), we can measure conscious \textit{geometry}—the spatial-temporal structure of oscillatory hole arrangements \cite{sachikonye2024consciousness}.

\subsection{Summary: The Mathematical Foundation}

We have established five foundational results:

\begin{enumerate}
\item \textbf{Oscillatory necessity}: Physical reality fundamentally consists of oscillatory dynamics (Theorem \ref{thm:oscillatory_necessity}).

\item \textbf{Categorical structure}: Discrete categorical states emerge from continuous oscillations through completion operators (Definition \ref{def:categorical_space}, Axiom \ref{axiom:irreversibility}).

\item \textbf{Temporal emergence}: Time arises from categorical completion sequences, not as external parameter (Theorem \ref{thm:temporal_emergence}).

\item \textbf{Oscillatory-categorical identity}: Continuous and discrete descriptions are mathematically equivalent (Theorem \ref{thm:osc_cat_equiv}, Corollary \ref{cor:freq_cat_identity}).

\item \textbf{Circuit implementation}: Physical substrate is oscillatory hole stabilization via electron transport with measurable completion times $\tau_{\text{circuit}} \sim 1$–100 ms (Proposition \ref{prop:circuit_time}).
\end{enumerate}

These results provide the rigorous foundation for understanding subjective time: if time emerges from categorical completion (Result 3), and categorical completion is implemented through circuit completion events (Result 5), then \textbf{subjective time should reflect circuit completion duration}. This is precisely what Sections 3–4 establish.

The key conceptual leap: traditional approaches assume time is primitive and try to derive consciousness from temporal dynamics. We reverse this: time is emergent from categorical completion, and consciousness experiences the duration of that completion process. Subjective time is not an illusion or epiphenomenon but the direct phenomenological correlate of a fundamental physical process—geometric tracing through circuit completion.

