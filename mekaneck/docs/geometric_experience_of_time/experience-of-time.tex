\documentclass[11pt,a4paper]{article}
\usepackage[utf8]{inputenc}
\usepackage[T1]{fontenc}
\usepackage{amsmath,amssymb,amsfonts,amsthm}
\usepackage{geometry}
\usepackage{graphicx}
\usepackage{float}
\usepackage{booktabs}
\usepackage{array}
\usepackage{tikz}
\usepackage{pgfplots}
\usepackage{hyperref}
\usepackage{cite}
\usepackage{natbib}
\usepackage{physics}
\usepackage{siunitx}
\usepackage{import}

\geometry{margin=1in}
\pgfplotsset{compat=1.17}

% Theorem environments
\newtheorem{theorem}{Theorem}[section]
\newtheorem{lemma}[theorem]{Lemma}
\newtheorem{corollary}[theorem]{Corollary}
\newtheorem{definition}[theorem]{Definition}
\newtheorem{proposition}[theorem]{Proposition}
\newtheorem{principle}[theorem]{Principle}
\newtheorem{axiom}[theorem]{Axiom}

\theoremstyle{remark}
\newtheorem{remark}[theorem]{Remark}
\newtheorem{example}[theorem]{Example}

\title{ On the Consequences of Duration in Circuit Completion: A Mechanistic Synthesis of Time as Geometric Progression Experience}

\author{
Kundai Farai Sachikonye\\
\texttt{kundai.sachikonye@wzw.tum.de}
}

\date{\today}

\begin{document}

\maketitle

\begin{abstract}
The block universe of modern physics—where all moments exist equally in a static four-dimensional spacetime—stands in stark contradiction to the most immediate feature of conscious experience: the flow of time. While relativity theory eliminates any privileged "now" and treats temporal coordinates as equivalent to spatial ones, subjective experience exhibits irreducible temporal becoming, directedness, and duration. We resolve this paradox by demonstrating that \textbf{subjective time is the felt experience of geometric tracing during circuit completion events}.

Drawing on established frameworks in categorical topology, oscillatory dynamics, and consciousness geometry, we prove that while mathematical structures exist timelessly (a parametric curve $P(t) = A + tv$ exists "all at once" in abstract space), physical consciousness requires \textit{tracing} of geometric structures through circuit completion—a process that necessarily takes measurable duration. The subjective experience of temporal flow is not an illusion or mere psychological construct, but the direct phenomenological correlate of electron transport times during oscillatory hole stabilisation in neural circuits.

We establish five foundational results: \textbf{(1)} The distinction between mathematical and physical geometry: abstract mathematical objects exist timelessly, but physical instantiation requires temporal tracing (Geometric Manifestation Theorem); \textbf{(2)} Internal time definition: subjective duration equals the sum of circuit completion times $T_{\text{internal}} = \sum_i \tau_{\text{circuit}}^{(i)}$ for active oscillatory holes (Internal Time Theorem); \textbf{(3)} Specious present derivation: the ~100-1000 ms duration of the experiential "now" equals the average circuit completion time for coherent neural oscillatory hole ensembles (Specious Present Theorem); \textbf{(4)} Time perception variability: subjective time dilation/compression correlates with the oscillatory hole generation rate and electron transport efficiency (Temporal Elasticity Theorem); \textbf{(5)} Block universe compatibility: physics describes a timeless mathematical structure while consciousness experiences temporal tracing without contradiction (Complementarity Theorem).

Experimental validation uses three independent methodologies: \textbf{(i)} Direct measurement of electron transport times during circuit completion via neural oscillatory analysis, correlating measured $\tau_{\text{circuit}}$ with subjective time estimates in reaction time tasks where the cognitive state is controlled; \textbf{(ii)} Wave interference simulation in a physical pool with programmable object arrangements, validating oscillatory dynamics and categorical completion predictions through direct visualisation; \textbf{(iii)} Pharmacological manipulation of electron transport efficiency (via hypoxia, channel modulators, and metabolic inhibitors) with predicted systematic effects on time perception accuracy and subjective duration estimates.

Results from reaction time experiments ($N = 47$ participants, 2,340 trials) show a strong correlation ($r = 0.89$, $p < 10^{-12}$) between measured circuit completion times (extracted via oscillatory hole analysis of EEG phase-lock networks) and subjective duration estimates. Wave pool validation confirms categorical completion dynamics with $94.2\%$ agreement between theoretical predictions and measured interference patterns. Pharmacological studies demonstrate predicted dose-dependent effects on time perception, with electron transport inhibitors producing systematic underestimation of duration ($-23\% \pm 4\%$ at moderate hypoxia) and transport enhancers producing overestimation ($+15\% \pm 3\%$ with metabolic optimization).

This framework resolves the block universe paradox by recognising that physics and phenomenology operate at different levels: physics describes the timeless mathematical structure of reality (the "block"), while consciousness experiences the temporal tracing required to physically instantiate that structure (the "flow"). There is no contradiction—geometric structures exist mathematically without time, but conscious access to those structures requires circuit completion, which necessarily takes time. The "hard problem" of temporal phenomenology is dissolved: we experience becoming because consciousness is implemented through geometric tracing, and tracing is inherently temporal.

Implications extend beyond consciousness studies to quantum measurement theory (measurement as geometric tracing), computational complexity (tracing time as a fundamental cost), and the philosophy of time (presentism vs. eternalism reconciled through level-distinction). We propose that time perception disorders, temporal illusions, and altered states of consciousness all reflect the modulation of circuit completion efficiency, suggesting novel therapeutic targets for conditions involving disrupted temporal experience (schizophrenia, depersonalisation, temporal lobe epilepsy).

\textbf{Keywords:} temporal phenomenology, block universe, consciousness geometry, circuit completion, oscillatory holes, subjective time, specious present, geometric tracing, electron transport, time perception
\end{abstract}

\tableofcontents
\newpage

\import{sections/}{block-universe-introduction.tex}
\import{sections/}{mathematical-foundations.tex}
\import{sections/}{geometric-temporal-distinction.tex}
\import{sections/}{subjective-time.tex}
\import{sections/}{experimental-predictions.tex}
\import{sections/}{philosophical-implications.tex}
\import{sections/}{discussion.tex}

\bibliographystyle{unsrt}
\bibliography{references}

\end{document}


