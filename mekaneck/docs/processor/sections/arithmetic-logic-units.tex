% ============================================================================
% ARITHMETIC LOGIC UNITS
% ============================================================================

\section{Arithmetic Logic Unit Architecture}
\label{sec:alu}

\subsection{BMD Transistor}

The Biological Maxwell Demon (BMD) transistor is the fundamental switching element of the categorical processor \citep{sachikonye2025semiconductor}. Unlike conventional transistors that switch based on voltage thresholds, BMD transistors switch based on pattern recognition. The concept derives from Maxwell's demon thought experiment \citep{maxwell1867}, with the resolution provided by Szilard \citep{szilard1929} and Landauer \citep{landauer1961}.

\subsubsection{Structure and Operation}

A BMD transistor consists of three terminals: gate, source, and drain. The gate is characterised by a pattern signature $\mathcal{P}_{\text{gate}}$, which is an oscillatory field configuration:
\begin{equation}
\mathcal{P}_{\text{gate}} = \sum_{k=1}^{K} a_k \cos(\omega_k t + \phi_k)
\label{eq:gate_pattern}
\end{equation}
where $K$ is the number of frequency components, $a_k$ the amplitudes, $\omega_k$ the frequencies, and $\phi_k$ the phases.

The input signal $\mathcal{S}_{\text{in}}$ is compared to the gate pattern through the overlap integral:
\begin{equation}
\mathcal{M} = \frac{|\langle \mathcal{S}_{\text{in}} | \mathcal{P}_{\text{gate}} \rangle|^2}{\langle \mathcal{S}_{\text{in}} | \mathcal{S}_{\text{in}} \rangle \langle \mathcal{P}_{\text{gate}} | \mathcal{P}_{\text{gate}} \rangle}
\label{eq:match_score}
\end{equation}
where the inner product is defined as
\begin{equation}
\langle f | g \rangle = \frac{1}{T} \int_0^T f(t) g^*(t) \, dt
\label{eq:inner_product}
\end{equation}
with $T$ being the integration period.

The transistor switches ON when $\mathcal{M} > \theta$, where $\theta = 0.5$ is the switching threshold.

\begin{figure*}[htbp]
\centering
\includegraphics[width=0.95\textwidth]{figures/maxwell_demon_resolution.png}
\caption{\textbf{Resolution of Maxwell's Demon: Seven-Fold Dissolution Through Phase-Lock Network Topology and Categorical Completion.}
\textbf{(A)} Temperature independence of network topology. Dual-axis plot showing network edges (blue circles, left axis) and kinetic energy (purple squares, right axis) versus temperature. Network edges remain constant at $N_{\text{edges}} \approx 106$ across all temperatures (horizontal blue line), while kinetic energy increases linearly as $E_{\text{kin}} \propto T$ (purple line, equipartition theorem). Annotation: "Edge variance: 0.00e+00 ($\sim 0$ confirms independence)" validates $\partial G/\partial E_{\text{kin}} = 0$. This proves that categorical network topology is independent of thermal energy, resolving the first dissolution argument.
\textbf{(B)} Kinetic independence: $\partial G/\partial E_{\text{kin}} = 0$ with $r = 0.0460$. Scatter plot showing network edges (y-axis) versus kinetic energy (x-axis). Data points (orange) show no correlation, with fitted slope $= 1.35 \times 10^{-3}$ (red dashed line, nearly horizontal). Pearson correlation $r = 0.0460$ (annotation) confirms statistical independence. The flat trend validates that network topology does not depend on kinetic energy, enabling temperature-independent computation.
\textbf{(C)} Distance inequivalence: $r = 0.413$ between spatial and categorical distances. Scatter plot showing categorical distance (y-axis) versus spatial distance (x-axis), colored by kinetic distance (colorbar). Points form a diffuse cloud with weak correlation $r = 0.413$ (annotation box: "Three metrics measure different properties"). This demonstrates that spatial, kinetic, and categorical distances are inequivalent: molecules can be spatially close but categorically distant, or vice versa. The weak correlation validates that categorical completion operates independently of spatial proximity.
\textbf{(D)} Temperature emergence from cluster statistics. Histogram showing frequency (y-axis) versus cluster temperature (x-axis). Distribution is peaked at mean cluster $T = 1.85$ (orange bars, annotation: $\mu = 1.848$, $\sigma = 0.821$), while global temperature is $T = 2.00$ (blue horizontal line). The emergence of temperature from categorical clustering validates that thermodynamic properties arise from phase-lock network statistics, not fundamental microscopic dynamics.
\textbf{(E)} "Sorting" increases entropy: $\Delta S = +0.0400$. Time series showing network density (y-axis, proxy for entropy) versus sorting attempts (x-axis). Density increases from $\sim 1.08$ to $\sim 1.30$ (purple shaded area), representing entropy increase $\Delta S = +0.0400$ (red annotation). This validates dissolution argument 6: apparent sorting increases total entropy through network densification, resolving the thermodynamic paradox.
\textbf{(F)} Velocity-blind completion: categorical paths identical 100\%. Scatter plot showing velocity difference (y-axis) versus temperature (x-axis, varies). All points (orange) cluster at low velocity differences ($< 7$ arbitrary units), with annotation: "100\% perfect match across all trials". This validates dissolution argument 4: categorical completion is independent of molecular velocities, requiring no velocity measurement.
\textbf{(Inset)} Information complementarity summary. Text box explaining the two conjugate faces: Kinetic face (observable): velocities, energy, temperature, momentum space; Categorical face (observable): network topology, distances, clustering, configuration space. Hidden properties are complementary. Complementarity principle (analogous to ammeter/voltmeter): cannot observe both faces simultaneously; measurement incompatibility; conjugate observables. Resolution: Maxwell observed only the kinetic face; categorical dynamics were hidden; "Demon" = projection of hidden categorical face onto observable kinetic face; not an agent, but a shadow of complementary dynamics. Final statement: "No demon exists / Only categorical completion through phase-lock networks".}
\label{fig:demon_resolution}
\end{figure*}

\subsubsection{Switching Characteristics}

The on/off current ratio is
\begin{equation}
\frac{I_{\text{on}}}{I_{\text{off}}} = G \cdot \frac{\mathcal{M} - \theta}{1 - \mathcal{M} + \theta}
\label{eq:on_off_ratio}
\end{equation}
where $G$ is the gain factor.

For the BMD transistor implementation:
\begin{itemize}
\item Gain: $G = 1000$
\item Threshold: $\theta = 0.5$
\item On/Off ratio: $I_{\text{on}}/I_{\text{off}} = 42.1$
\item Switching time: $\tau_{\text{switch}} < \SI{1}{\micro\second}$
\end{itemize}

\subsubsection{Information Catalysis}

The BMD transistor amplifies information content through catalytic action. The output information $I_{\text{out}}$ is related to the input information $I_{\text{in}}$ by
\begin{equation}
I_{\text{out}} = I_{\text{in}} \cdot (1 + \log_2 \eta)
\label{eq:info_catalysis}
\end{equation}
where $\eta$ is the catalytic efficiency in bits per molecule.

With $\eta = 3000$ bits/molecule:
\begin{equation}
I_{\text{out}} = I_{\text{in}} \cdot (1 + \log_2 3000) = I_{\text{in}} \cdot 12.55
\label{eq:catalytic_amplification}
\end{equation}

This represents a 12.55-fold information amplification per transistor stage.

\subsection{Tri-Dimensional Logic Gates}

Logic gates in the categorical processor operate in the three-dimensional S-coordinate space $(S_k, S_t, S_e)$ rather than on binary values.

\subsubsection{Gate Definition}

A tri-dimensional logic gate $\mathcal{G}$ maps two S-coordinate inputs to one S-coordinate output:
\begin{equation}
\mathcal{G}: \mathbb{R}^3 \times \mathbb{R}^3 \to \mathbb{R}^3
\label{eq:gate_def}
\end{equation}

The gate operation is defined by three component functions:
\begin{align}
(S_k^{\text{out}}, S_t^{\text{out}}, S_e^{\text{out}}) = \mathcal{G}(&(S_k^A, S_t^A, S_e^A), \nonumber \\
&(S_k^B, S_t^B, S_e^B))
\label{eq:gate_components}
\end{align}

\subsubsection{AND Gate}

The tri-dimensional AND gate computes the component-wise minimum:
\begin{align}
S_k^{\text{out}} &= \min(S_k^A, S_k^B) \label{eq:and_k} \\
S_t^{\text{out}} &= \frac{S_t^A + S_t^B}{2} \label{eq:and_t} \\
S_e^{\text{out}} &= S_e^A \cdot S_e^B \label{eq:and_e}
\end{align}

For binary inputs ($S_k \in \{0, 1\}$), this reduces to the classical AND operation.

\subsubsection{OR Gate}

The three-dimensional OR gate computes the component-wise maximum:
\begin{align}
S_k^{\text{out}} &= \max(S_k^A, S_k^B) \label{eq:or_k} \\
S_t^{\text{out}} &= \frac{S_t^A + S_t^B}{2} \label{eq:or_t} \\
S_e^{\text{out}} &= 1 - (1 - S_e^A)(1 - S_e^B) \label{eq:or_e}
\end{align}

\subsubsection{XOR Gate}

The three-dimensional XOR gate computes a symmetric difference:
\begin{align}
S_k^{\text{out}} &= |S_k^A - S_k^B| \label{eq:xor_k} \\
S_t^{\text{out}} &= |S_t^A - S_t^B| \label{eq:xor_t} \\
S_e^{\text{out}} &= S_e^A (1 - S_e^B) + S_e^B (1 - S_e^A) \label{eq:xor_e}
\end{align}

\subsection{ALU Operations}

The ALU performs arithmetic and logical operations on oscillatory data encoded in S-coordinates.

\subsubsection{Addition}

The addition of two oscillatory quantities $A$ and $B$ is implemented through frequency superposition:
\begin{equation}
A + B \to \mathcal{S}_{A+B} = (\omega_A + \omega_B, \phi_A, A_A + A_B)
\label{eq:alu_add}
\end{equation}

The corresponding S-coordinate transformation is:
\begin{align}
S_k^{A+B} &= \frac{\ln(1 + \omega_A + \omega_B)}{\ln(10^{15})} \label{eq:add_sk} \\
S_t^{A+B} &= \phi_A \mod 2\pi / (2\pi) \label{eq:add_st} \\
S_e^{A+B} &= \tanh(A_A + A_B) \label{eq:add_se}
\end{align}

\begin{figure*}[htbp]
\centering
\includegraphics[width=0.90\textwidth]{figures/ic_bmd_transistor.png}
\caption{\textbf{Integrated Circuit Component 1: Biological Maxwell Demon (BMD) Transistor—42.1× On/Off Ratio with Sub-Microsecond Switching and $10^{12}$ Probability Enhancement.}
\textbf{(A)} BMD transistor structure. Schematic showing three-terminal device analogous to bipolar junction transistor: Collector (output, top), Base/Gate (control, left), and Emitter (input, bottom). The circular junction represents the biological p-n junction where oscillatory holes (p-type) and molecular carriers (n-type) interact. Base current modulates collector-emitter conductivity through phase-lock control. Annotation box shows specifications: On/Off ratio = 42.1, switching time $< 1$ μs, probability enhancement = $10^{12}\times$. Unlike electronic transistors operating through charge carrier injection, BMD transistors operate through categorical phase-lock modulation, enabling therapeutic switching without thermal dissipation.
\textbf{(B)} On/Off current ratio validation. Bar chart showing OFF-state current (blue bar, $I_{\text{OFF}} \approx 10^{-8}$ A) and ON-state current (red bar, $I_{\text{ON}} \approx 10^{-6}$ A) on log scale. The ratio $I_{\text{ON}}/I_{\text{OFF}} = 42.1$ (green arrow annotation) exceeds the design specification of 42.1 (horizontal dashed line). The $42.1\times$ enhancement validates therapeutic switching: OFF state blocks unwanted signal propagation; ON state enables categorical computation. The log-scale separation demonstrates clean switching behavior with minimal leakage current.
\textbf{(C)} Switching dynamics with sub-microsecond response. Normalized response (y-axis, 0 to 1) versus time (μs, x-axis) showing sigmoid switching curve. Rise time $t_{\text{rise}} = 0.66$ μs (vertical dashed line, annotation) represents the time to transition from 10\% to 90\% of final response. The smooth exponential approach (orange shaded region) follows $R(t) = 1 - \exp(-t/\tau)$ where $\tau \approx 0.3$ μs is the time constant. Horizontal dotted lines mark 10\% and 90\% thresholds. The sub-microsecond switching validates that BMD transistors operate at biological clock frequency $f_0 = 758$ Hz with sufficient bandwidth for categorical computation.
\textbf{(D)} Probability enhancement through catalytic action. Bar chart comparing transition probability without BMD (blue bar, $p_0 \approx 10^{-15}$) versus with BMD (green bar, $p_{\text{BMD}} \approx 10^{-3}$) on log scale. The enhancement factor $p_{\text{BMD}}/p_0 = 10^{12}$ (red annotation: "$10^{12}\times$ enhancement!") demonstrates catalytic action: the BMD transistor increases transition probability by 12 orders of magnitude without being consumed. This validates the information catalysis mechanism: categorical phase-lock networks amplify rare events through resonance cascades. The $10^{12}\times$ enhancement enables detection and amplification of single-molecule events, crucial for therapeutic sensing applications. Without BMD, transition probability is $\sim 10^{-15}$ (annotation: "1e-15"), making detection impossible; with BMD, probability reaches $\sim 10^{-3}$, enabling reliable operation.}
\label{fig:bmd_transistor}
\end{figure*}

\subsubsection{Multiplication}

Multiplication is implemented through frequency modulation:
\begin{equation}
A \times B \to \mathcal{S}_{A \times B} = (\omega_A \cdot \omega_B / \omega_{\text{ref}}, \phi_A + \phi_B, A_A \cdot A_B)
\label{eq:alu_mult}
\end{equation}
where $\omega_{\text{ref}}$ is a reference frequency for dimensional consistency.

\subsubsection{Phase Shift}

Phase shift by angle $\Delta\phi$ is a fundamental operation:
\begin{equation}
\text{PhaseShift}(\Delta\phi): \mathcal{S} \to (\omega, \phi + \Delta\phi \mod 2\pi, A)
\label{eq:phase_shift}
\end{equation}

\subsubsection{Frequency Modulation}

Frequency modulation by factor $\alpha$ scales the oscillation rate:
\begin{equation}
\text{FreqMod}(\alpha): \mathcal{S} \to (\alpha \omega, \phi, A)
\label{eq:freq_mod}
\end{equation}

\subsection{ALU Performance}

The complete ALU is constructed from 47 BMD transistors arranged in the following configuration:
\begin{itemize}
\item Input stage: 8 transistors (pattern recognition)
\item Logic stage: 24 transistors (tri-dimensional gates)
\item Arithmetic stage: 12 transistors (frequency operations)
\item Output stage: 3 transistors (result encoding)
\end{itemize}

Performance metrics:
\begin{itemize}
\item Operation time: $\tau_{\text{ALU}} < \SI{100}{\nano\second}$
\item Throughput: $> 10^7$ operations per second
\item Power consumption: $P_{\text{ALU}} = 47 \times E_{\text{ATP}} \times f_{\text{op}} = \SI{2.4}{\pico\watt}$ at $10^6$ ops/s
\end{itemize}

The operation time is determined by the slowest gate in the critical path:
\begin{equation}
\tau_{\text{ALU}} = n_{\text{stages}} \cdot \tau_{\text{switch}} = 4 \times \SI{25}{\nano\second} = \SI{100}{\nano\second}
\label{eq:alu_time}
\end{equation}

