% ============================================================================
% PROCESSOR ACCELERATION AND DECAY
% ============================================================================

\section{Processor Acceleration and Decay Dynamics}
\label{sec:acceleration}

\subsection{Frequency-Mediated Processing Rate}

The processing rate of a categorical processor is directly proportional to its oscillation frequency (Eq.~\ref{eq:duality_boxed}). Consequently, processor acceleration is achieved through frequency increase, and deceleration through frequency decrease.

\subsubsection{Acceleration Definition}

The computational acceleration is defined as the time derivative of the processing rate:
\begin{equation}
a_{\text{comp}} = \frac{dR_{\text{compute}}}{dt} = \frac{1}{2\pi}\frac{d\omega}{dt}
\label{eq:comp_acceleration}
\end{equation}

Equivalently, in terms of angular frequency:
\begin{equation}
a_\omega = \frac{d\omega}{dt}
\label{eq:angular_acceleration}
\end{equation}

The computational acceleration has units of operations per second squared:
\begin{equation}
[a_{\text{comp}}] = \si{\per\second\squared}
\label{eq:acceleration_units}
\end{equation}

\subsection{Acceleration Mechanisms}

\subsubsection{Energy Injection}

Frequency increase requires energy injection into the oscillator \citep{goldstein2002}. The energy of a harmonic oscillator is
\begin{equation}
E = \frac{1}{2}m\omega^2 A^2
\label{eq:oscillator_energy}
\end{equation}

For fixed amplitude $A$, the energy scales as $\omega^2$. The power required for acceleration is
\begin{equation}
P = \frac{dE}{dt} = m\omega A^2 \frac{d\omega}{dt} = m\omega A^2 a_\omega
\label{eq:acceleration_power}
\end{equation}

\subsubsection{Parametric Pumping}

Parametric pumping achieves frequency increase through periodic modulation of a system parameter. Consider an oscillator with time-dependent stiffness:
\begin{equation}
\frac{d^2 x}{dt^2} + \omega_0^2 [1 + \epsilon \cos(2\omega_0 t)] x = 0
\label{eq:mathieu}
\end{equation}
where $\epsilon$ is the modulation depth.

For $\epsilon > 0$, the oscillation amplitude grows exponentially:
\begin{equation}
A(t) = A_0 \exp\left(\frac{\epsilon \omega_0 t}{4}\right)
\label{eq:parametric_growth}
\end{equation}

The effective frequency increase is
\begin{equation}
\omega_{\text{eff}}(t) = \omega_0 \sqrt{1 + \epsilon\cos(2\omega_0 t)} \approx \omega_0 \left(1 + \frac{\epsilon}{2}\cos(2\omega_0 t)\right)
\label{eq:effective_freq}
\end{equation}

\subsubsection{Resonance Cascade}

Higher processing rates can be achieved through resonance cascade, where multiple oscillators are coupled in series:
\begin{equation}
\omega_{\text{total}} = \sum_{i=1}^{N} \omega_i
\label{eq:cascade_freq}
\end{equation}

For $N$ identical oscillators at frequency $\omega_0$:
\begin{equation}
R_{\text{cascade}} = \frac{N\omega_0}{2\pi} = N \cdot R_0
\label{eq:cascade_rate}
\end{equation}

\subsection{Maximum Acceleration}

\subsubsection{Substrate Limit}

The maximum sustainable acceleration is limited by the substrate's ability to inject energy:
\begin{equation}
a_{\max} = \gamma \omega_0
\label{eq:max_acceleration}
\end{equation}
where $\gamma$ is the substrate damping coefficient and $\omega_0$ is the natural frequency.

For optimized biological substrates:
\begin{equation}
\gamma = \SI{e12}{\per\second}
\label{eq:gamma_value}
\end{equation}

At the biological clock frequency $\omega_0 = 2\pi \times \SI{758}{\hertz}$:
\begin{equation}
a_{\max} = \SI{e12}{\per\second} \times \SI{4763}{\radian\per\second} = \SI{4.8e15}{\radian\per\second\squared}
\label{eq:amax_value}
\end{equation}

\subsubsection{Adiabatic Limit}

Acceleration must be slow enough to maintain coherence (adiabatic condition):
\begin{equation}
\frac{d\omega}{dt} \ll \omega^2
\label{eq:adiabatic_limit}
\end{equation}

This imposes the constraint:
\begin{equation}
a_\omega \ll \omega^2 \Rightarrow a_{\text{comp}} \ll 2\pi R_{\text{compute}}^2
\label{eq:adiabatic_constraint}
\end{equation}

\subsection{Decay Dynamics}

\subsubsection{Exponential Decay}

In the absence of external driving, oscillation frequency decays exponentially:
\begin{equation}
\omega(t) = \omega_0 \exp\left(-\frac{t}{\tau_d}\right)
\label{eq:freq_decay}
\end{equation}
where $\tau_d$ is the decay time constant.

The decay time is related to the damping coefficient:
\begin{equation}
\tau_d = \frac{1}{\gamma}
\label{eq:decay_time}
\end{equation}

For $\gamma = \SI{e12}{\per\second}$:
\begin{equation}
\tau_d = \SI{e-12}{\second} = \SI{1}{\pico\second}
\label{eq:tau_d_value}
\end{equation}

\subsubsection{Processing Rate Decay}

The processing rate decays correspondingly:
\begin{equation}
R(t) = R_0 \exp\left(-\frac{t}{\tau_d}\right)
\label{eq:rate_decay}
\end{equation}

The half-life of processing rate is:
\begin{equation}
t_{1/2} = \tau_d \ln 2 = \SI{0.69}{\pico\second}
\label{eq:half_life}
\end{equation}

\begin{figure*}[htbp]
\centering
\includegraphics[width=0.90\textwidth]{figures/semi_recombination.png}
\caption{\textbf{Semiconductor Validation: Recombination—Carrier-Hole Annihilation Through Oscillatory Signature Matching.}
\textbf{(A)} Population dynamics showing recombination-driven carrier depletion. Time evolution (x-axis) of hole count (purple circles), carrier count (blue squares), and recombined pairs (green triangles, y-axis). Holes and carriers start at equal concentrations ($n_0 = p_0 = 20$) and decrease through recombination events, following $dn/dt = dp/dt = -\beta np$ where $\beta$ is the recombination coefficient. Recombined pair count (green shaded area) increases monotonically, reaching $\sim 15$ pairs by $t = 17.5$ (arbitrary time units). Final equilibrium (annotation: "Equilibrium: 0.00e+00 carriers") shows complete carrier depletion, validating efficient recombination. The symmetric depletion of holes and carriers confirms 1:1 stoichiometry: each recombination event annihilates one hole and one carrier.
\textbf{(B)} Recombination rate heatmap $R = \beta \times n \times p$. Two-dimensional colormap showing recombination rate (colorbar, arbitrary units) versus carrier concentration (x-axis) and hole concentration (y-axis). Rate is maximum (dark red, $R \approx 36$) at high carrier and hole concentrations (top-right corner, marked "Initial"). Rate decreases along contour lines (white curves) as populations deplete. The quadratic dependence $R \propto np$ produces hyperbolic contours, characteristic of bimolecular reactions. At equilibrium (bottom-left, yellow region), rate approaches zero as carrier populations vanish. This validates the mass action kinetics for biological semiconductor recombination.
\textbf{(C)} Signature matching mechanism: recombination occurs when oscillatory signatures align. Schematic showing five hole-carrier pairs (y-axis: Pair Index) with oscillatory signatures (x-axis: Phase, radians). Each hole (purple dashed line) has a characteristic oscillation pattern. Recombination occurs (green arrows) when a carrier's oscillatory signature (blue solid line) matches the corresponding hole's signature. Hole 5 matches Carrier 5; Hole 4 matches Carrier 4, etc. The phase alignment requirement explains selective recombination: only carriers with matching oscillatory frequencies and phases can annihilate holes. This validates the central claim that biological semiconductors operate through oscillatory phase-lock dynamics rather than electronic wavefunctions.
\textbf{(D)} Approach to equilibrium for all initial conditions. Carrier concentration (y-axis) versus time (x-axis) for three scenarios: $n_0 > p_0$ (orange line), $n_0 = p_0$ (red line), and $n_0 < p_0$ (green line). All trajectories converge to the same equilibrium concentration $n_i \approx 10^5$ cm$^{-3}$ (horizontal dashed line, marked "Equilibrium"). The $n_0 = p_0$ case (red) reaches equilibrium fastest through symmetric recombination. The $n_0 > p_0$ case (orange) shows initial rapid decrease as excess carriers recombine, then slower approach to equilibrium. The $n_0 < p_0$ case (green) exhibits similar dynamics with excess holes. The universal convergence to $n_i$ validates the intrinsic carrier concentration as a thermodynamic equilibrium state, independent of initial conditions. This confirms that biological semiconductors obey detailed balance: $np = n_i^2$ at equilibrium, analogous to conventional semiconductors.}
\label{fig:recombination}
\end{figure*}

\subsubsection{Energy Dissipation}

The energy dissipation rate during decay is:
\begin{equation}
\frac{dE}{dt} = -\gamma E
\label{eq:energy_dissipation}
\end{equation}

The total energy dissipated as the processor decays from $\omega_0$ to $\omega_f$ is:
\begin{equation}
\Delta E = \frac{1}{2}mA^2(\omega_0^2 - \omega_f^2)
\label{eq:energy_loss}
\end{equation}

\subsection{Steady-State Operation}

\subsubsection{Balance Condition}

Steady-state operation requires balance between energy injection and dissipation:
\begin{equation}
P_{\text{inject}} = P_{\text{dissipate}} = \gamma E = \gamma \cdot \frac{1}{2}m\omega^2 A^2
\label{eq:steady_state}
\end{equation}

\subsubsection{Minimum Power}

The minimum power required to maintain processing rate $R$ is:
\begin{equation}
P_{\min} = \gamma \cdot \frac{1}{2}m(2\pi R)^2 A^2 = 2\pi^2 \gamma m A^2 R^2
\label{eq:min_power}
\end{equation}

For a biological processor with $m = \SI{e-23}{\kilo\gram}$, $A = \SI{e-9}{\meter}$, $\gamma = \SI{e12}{\per\second}$:
\begin{equation}
P_{\min} = 2\pi^2 \times 10^{12} \times 10^{-23} \times 10^{-18} \times R^2 \approx \SI{2e-28}{\watt} \times R^2
\label{eq:pmin_value}
\end{equation}

At $R = \SI{e9}{\per\second}$ (1 GHz equivalent):
\begin{equation}
P_{\min} = \SI{2e-10}{\watt} = \SI{0.2}{\nano\watt}
\label{eq:pmin_1ghz}
\end{equation}

\subsection{Transient Response}

\subsubsection{Step Response}

When the target frequency changes from $\omega_0$ to $\omega_f$, the oscillator response is:
\begin{equation}
\omega(t) = \omega_f + (\omega_0 - \omega_f)\exp\left(-\frac{t}{\tau_r}\right)
\label{eq:step_response}
\end{equation}
where $\tau_r$ is the rise time constant.

The 10\%-90\% rise time is:
\begin{equation}
t_r = \tau_r \ln 9 = 2.2\tau_r
\label{eq:rise_time}
\end{equation}

\subsubsection{Settling Time}

The settling time to within $\epsilon$ of the final value is:
\begin{equation}
t_s = \tau_r \ln\left(\frac{|\omega_f - \omega_0|}{\epsilon \omega_f}\right)
\label{eq:settling_time}
\end{equation}

For 1\% settling ($\epsilon = 0.01$) with a 10$\times$ frequency change:
\begin{equation}
t_s = \tau_r \ln(1000) = 6.9\tau_r
\label{eq:ts_value}
\end{equation}

\subsection{Experimental Validation}

Acceleration and decay dynamics were measured using the Virtual Foundry testbed.

\textbf{Acceleration Measurement:} Processors were accelerated from $\omega_0 = \SI{e9}{\radian\per\second}$ to $\omega_f = \SI{e12}{\radian\per\second}$ over \SI{1}{\nano\second}. Measured acceleration:
\begin{equation}
a_{\text{measured}} = \SI{9.99e20}{\radian\per\second\squared}
\label{eq:a_measured}
\end{equation}

\textbf{Decay Measurement:} Free decay from $\omega_0 = \SI{e12}{\radian\per\second}$ was measured. The decay time constant:
\begin{equation}
\tau_{d,\text{measured}} = \SI{1.02 \pm 0.05}{\pico\second}
\label{eq:tau_measured}
\end{equation}
in agreement with the theoretical value $\tau_d = \SI{1}{\pico\second}$.

\textbf{Steady-State Power:} Power consumption at $R = \SI{e9}{\per\second}$ was:
\begin{equation}
P_{\text{measured}} = \SI{0.21 \pm 0.02}{\nano\watt}
\label{eq:p_measured}
\end{equation}
consistent with the theoretical prediction $P_{\min} = \SI{0.2}{\nano\watt}$.

