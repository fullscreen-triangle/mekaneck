\section{Sensory Sufficiency Principle}
\label{sec:sufficiency_detailed}

\subsection{The Core Insight}

Traditional theories of perception assume that sensory systems aim for \emph{complete} information about stimuli---as much detail as possible. We establish the opposite: perception aims for \emph{sufficient} information for categorical completion.

\begin{axiom}[Sensory Sufficiency]
Perception provides sufficient information for behavioral completion, not complete information about stimuli.
\end{axiom}

\subsection{The Lion Example}

Consider three observers at a zoo when a lion escapes:

\textbf{Observer A}: Trained zoologist
\begin{itemize}
\item Perceives: ``Male African lion, Panthera leo, approximately 5 years old, 180 kg''
\item Categorical completion: $\Gstate_A = \text{``dangerous predator''}$
\item Action: Run
\end{itemize}

\textbf{Observer B}: Child who only knows ``cats''
\begin{itemize}
\item Perceives: ``Big cat, looks scary''
\item Categorical completion: $\Gstate_B = \text{``dangerous animal''}$
\item Action: Run
\end{itemize}

\textbf{Observer C}: Looking at phone, doesn't see lion
\begin{itemize}
\item Perceives: ``My companions are running''
\item Categorical completion: $\Gstate_C = \text{``danger (unspecified)''}$
\item Action: Run
\end{itemize}

\begin{theorem}[Completion Equivalence]
All three perceptual states resolve to equivalent categorical completions:
\begin{equation}
\Gstate_A \equiv \Gstate_B \equiv \Gstate_C = \text{``flee''}
\end{equation}
despite vastly different perceptual content.
\end{theorem}

\subsection{Mathematical Formalization}

\begin{definition}[Sufficiency Map]
The sufficiency map $\Sigma$ takes perceptual states to behavioral completions:
\begin{equation}
\Sigma: \mathcal{P}_{\text{all}} \to \mathcal{B}_{\text{complete}}
\end{equation}
where $\mathcal{P}_{\text{all}}$ is the space of all possible percepts and $\mathcal{B}_{\text{complete}}$ is the space of behavioral completions.
\end{definition}

\begin{theorem}[Many-to-One Mapping]
The sufficiency map is many-to-one:
\begin{equation}
|\Sigma^{-1}(\Gstate)| \gg 1 \quad \forall \Gstate \in \mathcal{B}
\end{equation}
Many perceptual states map to each behavioral completion.
\end{theorem}

\subsection{Modality Equivalence}

Different sensory modalities can achieve identical categorical completions.

\begin{example}[Hot Stove Detection]
Consider detecting a hot stove:

\textbf{Visual}: See glowing red element
\begin{equation}
\Pcat_{\text{visual}} \xrightarrow{\Sigma} \Gstate_{\text{heat-danger}}
\end{equation}

\textbf{Tactile}: Touch hot surface
\begin{equation}
\Pcat_{\text{tactile}} \xrightarrow{\Sigma} \Gstate_{\text{heat-danger}}
\end{equation}

\textbf{Chemical}: Smell burning
\begin{equation}
\Pcat_{\text{olfactory}} \xrightarrow{\Sigma} \Gstate_{\text{heat-danger}}
\end{equation}

All modalities achieve identical completion.
\end{example}

\begin{theorem}[Modality Independence]
For any categorical completion $\Gstate$, there exist percepts from multiple modalities mapping to it:
\begin{equation}
\exists \, \Pcat_{\text{vis}}, \Pcat_{\text{tac}}, \Pcat_{\text{aud}}, \Pcat_{\text{olf}}: \Sigma(\Pcat_i) = \Gstate \quad \forall i
\end{equation}
\end{theorem}

\subsection{Implications for Drug Effects}

This has profound implications for neuropharmacology:

\begin{corollary}[Drug-Modality Equivalence]
A drug that mimics the categorical completion of a sensory percept is perceptually equivalent to that percept:
\begin{equation}
\Sigma(\Pcat_{\text{drug}}) = \Sigma(\Pcat_{\text{natural}}) \implies \text{equivalent experience}
\end{equation}
\end{corollary}

\begin{example}[Capsaicin]
Capsaicin activates heat receptors without actual heat:
\begin{equation}
\Pcat_{\text{capsaicin}} \xrightarrow{\Sigma} \Gstate_{\text{heat}} = \Sigma(\Pcat_{\text{actual-heat}})
\end{equation}
The brain cannot distinguish because sufficiency, not completeness, determines perception.
\end{example}

\subsection{The Evolutionary Rationale}

Why would evolution select for sufficiency over completeness?

\begin{enumerate}
\item \textbf{Speed}: Sufficient information enables faster response
\item \textbf{Efficiency}: Less neural resource required
\item \textbf{Robustness}: Multiple paths to same completion
\item \textbf{Flexibility}: Novel stimuli can still trigger appropriate responses
\end{enumerate}

\begin{theorem}[Sufficiency Optimality]
Under time pressure and resource constraints, sufficiency-based perception is evolutionarily optimal:
\begin{equation}
\max_{\Sigma} \frac{\text{Survival benefit}}{\text{Processing cost} \times \text{Response time}}
\end{equation}
\end{theorem}

\subsection{Connection to Categorical Distance}

From the cellular observation equations framework, categorical distance is independent of spatial distance. Similarly:

\begin{theorem}[Perceptual Categorical Distance]
The categorical distance between a percept and its completion is independent of sensory modality:
\begin{equation}
d_{\text{cat}}(\Pcat_{\text{vis}}, \Gstate) = d_{\text{cat}}(\Pcat_{\text{tac}}, \Gstate) = d_{\text{cat}}(\Pcat_{\text{drug}}, \Gstate)
\end{equation}
All pathways to the same completion have equal categorical distance.
\end{theorem}
