\section{Trajectory-Based Mental State Identification}
\label{sec:trajectory_detailed}

\subsection{The Poincar\'{e} Computing Paradigm}

Traditional neuroscience asks: ``Given initial conditions, what state will the brain reach?'' This is forward simulation from initial conditions.

The Poincar\'{e} computing paradigm inverts this: ``Given a completion condition, what trajectory must have been taken?'' This is backward constraint satisfaction.

\begin{principle}[Backward Determination]
Specifying the completion state determines what the trajectory MUST have been:
\begin{equation}
\Gstate_f \implies \gamma^* \quad \text{(unique trajectory)}
\end{equation}
\end{principle}

\subsection{Mental States as Trajectory-Terminus Pairs}

\begin{theorem}[Mental State Structure]
A mental state is not merely a state---it is a trajectory-terminus pair:
\begin{equation}
\boxed{\Mstate = (\gamma, \Gstate_f)}
\end{equation}
where:
\begin{itemize}
\item $\gamma: [0, T] \to \Sspace_N$ is the trajectory through S-entropy space
\item $\Gstate_f = \gamma(T)$ is the final (terminus) state
\end{itemize}
\end{theorem}

\subsection{Why Trajectory Matters}

\begin{example}[Same Thought, Different States]
Consider the thought ``cup on chair'':
\begin{itemize}
\item While drunk at a party
\item During an exam about furniture
\item While searching for lost keys
\item While reminiscing about grandmother's house
\end{itemize}

The thought \emph{content} is identical: $\Tcat_{\text{cup-chair}}$

But the \emph{mental states} differ:
\begin{align}
\Mstate_{\text{drunk}} &= (\gamma_{\text{party}}, \Gstate_{\text{cup-chair}}) \\
\Mstate_{\text{exam}} &= (\gamma_{\text{academic}}, \Gstate_{\text{cup-chair}}) \\
\Mstate_{\text{search}} &= (\gamma_{\text{goal-directed}}, \Gstate_{\text{cup-chair}}) \\
\Mstate_{\text{memory}} &= (\gamma_{\text{nostalgic}}, \Gstate_{\text{cup-chair}})
\end{align}
\end{example}

\begin{theorem}[Mental State Non-Identity]
Identical terminus states with different trajectories are different mental states:
\begin{equation}
\gamma_1 \neq \gamma_2 \implies \Mstate_1 \neq \Mstate_2 \quad \text{even if } \Gstate_{f,1} = \Gstate_{f,2}
\end{equation}
\end{theorem}

\subsection{Trajectory as Context}

The trajectory encodes:
\begin{enumerate}
\item \textbf{Emotional context}: How $\Hfield$ evolved during approach
\item \textbf{Cognitive history}: What thoughts preceded this one
\item \textbf{Perceptual pathway}: What sensory modalities were active
\item \textbf{Goal structure}: What completion was being sought
\end{enumerate}

\begin{definition}[Trajectory Context]
The trajectory context is the integral of state evolution:
\begin{equation}
\text{Context}(\gamma) = \int_0^T \frac{d\gamma}{dt} \otimes \Hfield(t) \, dt
\end{equation}
\end{definition}

\subsection{The Ternary Trisection Structure}

From the trajectory computing framework, trajectories in S-entropy space have ternary structure:

\begin{theorem}[Trajectory Address]
Each trajectory corresponds to a ternary address:
\begin{equation}
\gamma \leftrightarrow (a_1, a_2, \ldots, a_k) \quad \text{where } a_i \in \{0, 1, 2\}
\end{equation}
The address IS the path (trajectory-position identity).
\end{theorem}

\begin{corollary}[O($\log n$) Navigation]
Navigation through mental state space has complexity:
\begin{equation}
\text{Complexity} = O(\log_3 n)
\end{equation}
where $n$ is the number of distinct states. This is 37\% more efficient than binary.
\end{corollary}

\subsection{Penultimate State Principle}

\begin{principle}[Penultimate Determination]
To identify a mental state, determine the penultimate state from which it was reached:
\begin{equation}
\Mstate = (\gamma, \Gstate_f) \iff \Gstate_{f-1} \xrightarrow{\delta\gamma} \Gstate_f
\end{equation}
\end{principle}

This is the key inversion: instead of asking ``where will this go?'' (forward), ask ``where did this come from?'' (backward).

\begin{theorem}[Backward Uniqueness]
Given $\Gstate_f$ and the constraint structure, the penultimate state $\Gstate_{f-1}$ is unique:
\begin{equation}
\Gstate_f + \text{constraints} \implies \Gstate_{f-1} \quad \text{(unique)}
\end{equation}
\end{theorem}

\subsection{Mental State Equivalence Classes}

\begin{definition}[Trajectory Equivalence]
Two trajectories are equivalent if they reach the same terminus with equivalent contexts:
\begin{equation}
\gamma_1 \sim \gamma_2 \iff \Gstate_{f,1} = \Gstate_{f,2} \land |\text{Context}(\gamma_1) - \text{Context}(\gamma_2)| < \epsilon
\end{equation}
\end{definition}

\begin{theorem}[Equivalence Classes]
Mental states partition into equivalence classes based on:
\begin{enumerate}
\item Terminus identity: Same $\Gstate_f$
\item Context similarity: Similar trajectory integrals
\item Emotional proximity: Similar $\Hfield$ profiles
\end{enumerate}
\end{theorem}

\subsection{Implications for Neuropharmacology}

\begin{corollary}[Drug Effects as Trajectory Modifiers]
Drugs modify mental states by altering trajectories, not just termini:
\begin{equation}
\Pop_{\text{drug}}: \gamma \to \gamma' \implies \Mstate \to \Mstate'
\end{equation}
Even if the final thought is ``unchanged,'' the mental state differs because the trajectory differs.
\end{corollary}

\begin{example}[Anxiolytic Effect]
An anxiolytic doesn't change what you think, but changes HOW you get there:
\begin{align}
\gamma_{\text{anxious}} &\to \gamma_{\text{calm}} \\
\Gstate_f &= \Gstate_f \quad \text{(same thought)} \\
\Mstate_{\text{anxious}} &\neq \Mstate_{\text{calm}} \quad \text{(different mental states)}
\end{align}
\end{example}

\subsection{Complete Mental State Specification}

A complete mental state specification requires:
\begin{equation}
\Mstate = \left\{
\begin{array}{l}
\gamma: \text{trajectory through } \Sspace_N \\
\Gstate_f: \text{terminus state} \\
\Hfield: \text{emotional field} \\
\Pdecay: \text{perceptual constraint} \\
\Tdecay: \text{thought dynamics}
\end{array}
\right\}
\end{equation}

This is the fundamental unit of the Virtual Brain.
