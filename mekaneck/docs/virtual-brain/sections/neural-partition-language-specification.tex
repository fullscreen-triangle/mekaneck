\section{Neural Partition Language: Complete Specification}
\label{sec:npl_specification}

Building on the partition-based equations of state for hybrid microfluidic circuits, we establish a complete programming language for the Virtual Brain. This language implements Poincar\'{e} computing where programs specify completion conditions and the runtime navigates backward through S-entropy space to satisfy constraints.

\subsection{Foundational Principles}

\begin{principle}[No Null State]
At every computation step, the system must occupy exactly one category. There is no ``undefined'' or ``null'' state.
\end{principle}

\begin{principle}[Triple Equivalence]
All operations can be expressed equivalently in oscillatory, categorical, or partition form:
\begin{equation}
\Sosc = \Scat = \Spart = \kB M \ln n
\end{equation}
\end{principle}

\begin{principle}[Backward Determination]
Programs specify completion conditions $\Gstate_f$; the runtime determines what trajectory $\gamma$ must have been taken.
\end{principle}

\subsection{Type System}

\subsubsection{Coordinate Types}

\begin{definition}[Partition Coordinate Type]
\begin{verbatim}
type PartitionCoord = {
    n : Nat+,           -- depth (n >= 1)
    l : Fin(n),         -- complexity (0 <= l < n)
    m : Int[-l..+l],    -- orientation (-l <= m <= +l)
    s : Spin            -- chirality {-1/2, +1/2}
}
-- Capacity: C(n) = 2n²
\end{verbatim}
\end{definition}

\begin{definition}[S-Entropy Coordinate Type]
\begin{verbatim}
type SCoord = {
    Sk : Real[0,1],     -- knowledge entropy
    St : Real[0,1],     -- temporal entropy
    Se : Real[0,1]      -- evolution entropy
}
-- S-space: [0,1]³
\end{verbatim}
\end{definition}

\begin{definition}[Ternary Address Type]
\begin{verbatim}
type TernaryAddr(k : Nat) = Array[k] of {0, 1, 2}
-- Maps to 3^k cells in S-space
-- Address IS the path (trajectory-position identity)
\end{verbatim}
\end{definition}

\begin{definition}[Gyrometric Coordinate Type]
\begin{verbatim}
type GyroCoord = {
    J : Array[N] of Nat,    -- rotational quantum numbers
    omega : Array[N] of Real,-- angular frequencies
    gamma : Array[N,N] of Real -- damping coefficients
}
\end{verbatim}
\end{definition}

\subsubsection{State Types}

\begin{definition}[Mental State Type]
\begin{verbatim}
type MentalState = {
    gamma : Trajectory,      -- path through S-space
    Gamma_f : PartitionState,-- terminus state
    M : Memory,              -- accumulated emotional change
    H : EmotionalField,      -- current H+ field state
    P_decay : DecayCurve,    -- perception decay
    T_decay : DecayCurve     -- thought decay
}
\end{verbatim}
\end{definition}

\begin{definition}[Circuit State Type]
\begin{verbatim}
type CircuitState = {
    rho : ChargeDistribution,-- charge density field
    D : Real[0,1],           -- hierarchical depth
    R : Real[0,1],           -- phase coherence
    sigma2 : Real+,          -- variance
    regime : CircuitRegime   -- operational regime
}

type CircuitRegime =
    | Coherent(R > 0.8)
    | Turbulent(R < 0.3)
    | Hierarchical(D >= 0.6)
    | ApertureDominated
    | PhaseLocked(K > sigma_omega)
\end{verbatim}
\end{definition}

\begin{definition}[Thought Type]
\begin{verbatim}
type Thought = {
    O2_config : Array[N] of Vec3,  -- O2 positions
    electron_pos : Vec3,            -- stabilizing electron
    hole_phase : Real,              -- oscillatory phase
    signature : Array[30] of Real   -- 30-dim oscillatory signature
}
\end{verbatim}
\end{definition}

\begin{definition}[Perception Type]
\begin{verbatim}
type Perception = {
    modality : SensoryModality,
    input : SensoryInput,
    completion : PartitionState,
    sufficient : Bool              -- sufficiency achieved
}

type SensoryModality =
    | Visual | Auditory | Tactile | Olfactory
    | Gustatory | Proprioceptive | Drug
\end{verbatim}
\end{definition}

\subsection{Core Operators}

\subsubsection{Partition Operations}

\begin{definition}[Partition Operators]
\begin{verbatim}
-- Partition state transition
PARTITION(sigma_1, sigma_2) : PartitionState -> PartitionState
    requires: d_cat(sigma_1, sigma_2) = 1  -- adjacent categories
    ensures: No Null State maintained

-- Categorical distance
D_CAT(sigma_1, sigma_2) : PartitionState * PartitionState -> Real
    = sqrt((n1-n2)² + (l1-l2)² + (m1-m2)² + (s1-s2)²)

-- Partition capacity at depth n
CAPACITY(n : Nat+) : Nat = 2 * n²

-- Partition coordinate extraction
COORDS(state : PartitionState) : PartitionCoord
\end{verbatim}
\end{definition}

\subsubsection{S-Entropy Operations}

\begin{definition}[S-Entropy Operators]
\begin{verbatim}
-- Navigate in S-space
NAVIGATE(S_current : SCoord, S_target : SCoord) : Trajectory
    -- Uses ternary trisection: O(log₃ n) complexity

-- S-entropy gradient
GRAD_S(field : SCoord -> Real) : SCoord -> Vec3

-- Knowledge entropy update
UPDATE_SK(observation : Perception) : SCoord -> SCoord
    Sk_new = Sk_old - I(observation)  -- information reduces uncertainty

-- Temporal entropy update
UPDATE_ST(circuit_completion : Duration) : SCoord -> SCoord
    St_new = St_old + tau_circuit / tau_max

-- Evolution entropy update
UPDATE_SE(trajectory_step : TrajectoryStep) : SCoord -> SCoord
    Se_new = Se_old + |dS/dt| * dt
\end{verbatim}
\end{definition}

\subsubsection{Ternary Encoding Operations}

\begin{definition}[Ternary Operators]
\begin{verbatim}
-- Encode S-coordinate to ternary address
ENCODE(S : SCoord, precision : Nat) : TernaryAddr(precision)
    -- Each trit specifies which third of remaining interval

-- Decode ternary address to S-coordinate
DECODE(addr : TernaryAddr(k)) : SCoord
    -- Address IS the path: trajectory-position identity

-- Trisect interval (37% fewer iterations than binary)
TRISECT(interval : Interval) : (Interval, Interval, Interval)

-- Hierarchical navigation
NAVIGATE_TERNARY(addr_from, addr_to : TernaryAddr(k)) : Trajectory
    complexity: O(k) = O(log₃(resolution))
\end{verbatim}
\end{definition}

\subsection{Poincar\'{e} Computing Primitives}

\begin{definition}[Trajectory Completion]
\begin{verbatim}
-- Core Poincaré computing operation
COMPLETE(constraints : Constraints,
         Gamma_f : PartitionState) : Trajectory
    -- Finds trajectory gamma satisfying:
    --   (1) ||gamma(T) - S_0|| < epsilon (recurrence)
    --   (2) constraints(gamma) = true
    returns: trajectory that MUST have been taken

-- Specify completion condition
TARGET(Gamma_f : PartitionState,
       epsilon : Real+) : CompletionCondition

-- Constraint satisfaction
SATISFY(C : Constraints, gamma : Trajectory) : Bool
    -- Checks if trajectory satisfies all constraints

-- Equilibrium as recurrence
EQUILIBRIUM(S_0 : SCoord, epsilon : Real+) : CompletionCondition
    = TARGET(Gamma_f where ||Gamma_f.S - S_0|| < epsilon)
\end{verbatim}
\end{definition}

\begin{definition}[Free Energy Operations]
\begin{verbatim}
-- Helmholtz free energy (trajectory completion criterion)
HELMHOLTZ(U : Energy, T : Temperature, S : Entropy) : Energy
    = U - T * S

-- Gibbs free energy
GIBBS(H : Enthalpy, T : Temperature, S : Entropy) : Energy
    = H - T * S

-- Minimize free energy (find completion)
MINIMIZE_F(F : FreeEnergy, constraints : Constraints) : Trajectory
\end{verbatim}
\end{definition}

\subsection{Circuit Dynamics Operators}

\subsubsection{Variance Minimization}

\begin{definition}[Variance Operators]
\begin{verbatim}
-- Current variance
VARIANCE(state : CircuitState) : Real+
    = <(rho - <rho>)²>

-- Minimum achievable variance
VAR_MIN(T : Temperature, K : CouplingStrength) : Real+
    = k_B * T / K

-- Variance reduction dynamics
REDUCE_VAR(state : CircuitState,
           target : Real+) : CircuitState
    -- Thermodynamic optimization toward target variance

-- Geometric aperture selection (variance filter)
APERTURE(omega_set : Set[Frequency],
         threshold : Real+) : Set[Frequency]
    = { omega : sigma²(phi | omega) < threshold }
    -- Catalytic reduction factor ~10³⁸
\end{verbatim}
\end{definition}

\subsubsection{Phase-Lock Operations}

\begin{definition}[Phase-Lock Operators]
\begin{verbatim}
-- Phase coherence measure
COHERENCE(phases : Array[N] of Phase) : Real[0,1]
    R = (1/N) * |sum_j exp(i * phi_j)|

-- Phase-lock coupling
PHASE_LOCK(omega_1, omega_2 : Frequency,
           K : CouplingStrength) : Bool
    requires: |omega_1 - omega_2| < Delta_omega_c

-- Kuramoto synchronization
KURAMOTO(oscillators : Array[N] of Oscillator,
         K : CouplingStrength) : PhaseState
    -- d(phi_i)/dt = omega_i + (K/N) * sum_j sin(phi_j - phi_i)

-- Phase propagation speed
V_PHASE(K : CouplingStrength,
        D_O2 : DiffusionCoeff) : Velocity
    = sqrt(K * D_O2)
\end{verbatim}
\end{definition}

\subsubsection{Hierarchical Cascade Operations}

\begin{definition}[Hierarchy Operators]
\begin{verbatim}
-- Hierarchical depth
DEPTH(fluxes : Array[N] of Flux,
      threshold : Flux) : Real[0,1]
    D = (1/N) * sum_i indicator(F_i > threshold)

-- Information compression
COMPRESS(flux_in, flux_out : Array[N] of Flux,
         alpha : Array[N] of Real) : Information
    I = sum_i alpha_i * log2(flux_in_i / flux_out_i)

-- Multi-scale coupling
CASCADE(levels : Array[N] of Level) : HierarchicalState
    -- Couples adjacent levels through variance minimization

-- Depth transition (critical at D ≈ 0.4)
TRANSITION(state : CircuitState) : CircuitRegime
    if D > 0.6 then Hierarchical
    else if R > 0.8 then Coherent
    else if R < 0.3 then Turbulent
    else ApertureDominated
\end{verbatim}
\end{definition}

\subsection{Thermodynamic Operations}

\begin{definition}[Temperature Operations]
\begin{verbatim}
-- Categorical thermometry (zero backaction)
TEMPERATURE_CAT(Se_ref, Se_current : Real,
                T_0 : Temperature) : Temperature
    T = T_0 * exp(Se_current - Se_ref)
    -- Resolution: ~17 picokelvin
    -- Backaction: Delta_p = 0

-- Temperature as scaling factor
SCALE(observable : Dimensionless,
      T : Temperature) : Energy
    = k_B * T * observable

-- Universal equation of state
EOS(P : Pressure, V : Volume, N : Nat,
    T : Temperature,
    structure : PartitionGeometry) : Constraint
    P * V = N * k_B * T * S(structure)
    -- S is temperature-independent structural factor
\end{verbatim}
\end{definition}

\subsection{Gyrometric Dynamics}

\begin{definition}[Gyrometric Operators]
\begin{verbatim}
-- Rotational quantum number dynamics
GYRO_EVOLVE(J : GyroCoord,
            lambda : AffineParameter,
            F : ExternalForce) : GyroCoord
    -- d²J_i/dlambda² = -omega²_Ji * (J_i - J_eq,i)
    --                  - sum_j gamma_ij * dJ_j/dlambda + F_i
    -- Damped, driven oscillation in rotational state space

-- Gyrometric equilibrium
GYRO_EQ(omega : Array[N] of Real) : GyroCoord
    J_eq where d²J/dlambda² = 0

-- Angular momentum coupling
COUPLE_J(J_1, J_2 : GyroCoord,
         coupling : CouplingMatrix) : GyroCoord
\end{verbatim}
\end{definition}

\subsection{Measurement Operations}

\begin{definition}[Quintupartite + Categorical Thermometry]
\begin{verbatim}
-- Six-modality measurement framework
type MeasurementModality =
    | Optical(ambiguity: ~10⁶⁰)
    | Spectral(ambiguity: ~10⁴⁵)
    | Vibrational(ambiguity: ~10³⁰)
    | MetabolicGPS(ambiguity: ~10¹⁵)
    | TemporalCausal(ambiguity: ~10⁰)
    | CategoricalThermo(exclusion: ~10⁻³)

-- Sequential exclusion
MEASURE(state : CircuitState,
        modalities : List[MeasurementModality]) : UniqueState
    -- Each modality reduces ambiguity by epsilon_i ~ 10⁻¹⁵
    -- N_0 ~ 10⁶⁰ -> N_6 = 1 unique determination
    -- Effective resolution: delta_x ~ 0.08 nm

-- Zero-backaction categorical observation
OBSERVE_CAT(state : PartitionState) : PartitionState
    -- [O_cat, O_phys] = 0 (commutes with physical observables)
    -- Measures WHERE electron IS NOT (exhaustive exclusion)
\end{verbatim}
\end{definition}

\subsection{Neural-Specific Operations}

\begin{definition}[Consciousness Operations]
\begin{verbatim}
-- Consciousness as intersection
CONSCIOUSNESS(P_decay : DecayCurve,
              T_decay : DecayCurve) : ConsciousnessState
    C = P_decay ∩ T_decay
    -- Exists where/when perception constrains thought

-- Decay curve evolution
DECAY_EVOLVE(curve : DecayCurve,
             tau : TimeConstant,
             input : Optional[InputRate]) : DecayCurve
    d(curve)/dt = -curve/tau + input

-- Consciousness frequency
OMEGA_C(omega_T, omega_P : Frequency) : Frequency
    = (omega_T * omega_P) / (omega_T + omega_P)
    -- ~2.5 Hz
\end{verbatim}
\end{definition}

\begin{definition}[Memory Operations]
\begin{verbatim}
-- Memory as accumulated change
MEMORY(H_field : EmotionalField,
       t_start, t_end : Time) : Memory
    M = integral(dH/dt, t_start, t_end)
    -- "How the field got here"

-- Temporal differentiation
DISTINGUISH(thought : Thought,
            H : EmotionalField,
            M_1, M_2 : Memory) : Bool
    -- (T, H, M_1) ≠ (T, H, M_2) even if T, H identical

-- Memory retrieval
RETRIEVE(M : Memory,
         H_now : EmotionalField,
         t_0 : Time) : EmotionalField
    H(t_0) = H_now - (M(now) - M(t_0))

-- Prediction from memory
PREDICT(M : Memory,
        H_now : EmotionalField,
        delta_t : Duration) : EmotionalField
    H(t + delta_t) ≈ H_now + (dM/dt) * delta_t
\end{verbatim}
\end{definition}

\begin{definition}[Dream Operations]
\begin{verbatim}
-- Dream mode (unbounded thought)
DREAM(thought : Thought) : Trajectory
    -- P_decay = 0, no perceptual constraint
    -- Trajectory navigates full S-space
    domain: S_N - S_perceptible

-- Wake mode (constrained thought)
WAKE(thought : Thought,
     perception : Perception) : Trajectory
    -- P_decay > 0, perceptual constraint active
    domain: S_N ∩ S_perceptible
\end{verbatim}
\end{definition}

\subsection{Charge-Circuit Integration}

\begin{definition}[Charge Operations]
\begin{verbatim}
-- Charge conservation (no ground)
CONSERVE(rho : ChargeDistribution) : Constraint
    integral(rho, V) = Q_total = const

-- Autocatalytic redistribution
REDISTRIBUTE(rho : ChargeDistribution,
             imbalance : LocalImbalance) : ChargeDistribution
    -- Variance minimization drives redistribution
    -- Creates new imbalance -> cycle continues

-- Soul-consciousness coupling
COUPLE_SC(rho : ChargeDistribution,  -- soul
          C : ConsciousnessState)     -- consciousness
    : UnifiedState
    -- Same non-grounded circuit at different levels
\end{verbatim}
\end{definition}

\subsection{Program Structure}

\begin{definition}[NPL Program]
\begin{verbatim}
program VirtualBrainProcess {
    -- Declare target state (completion condition)
    target: PartitionState

    -- Declare constraints
    constraints: List[Constraint]

    -- Initial state (optional - determined by backward completion)
    initial: Optional[MentalState]

    -- The program body specifies WHAT, not HOW
    body: CompletionSpecification

    -- Runtime navigates S-space to find trajectory
    -- satisfying target + constraints
}

-- Example: Conscious perception process
program ConsciousPerception {
    target: Gamma_f where sufficient(perception)

    constraints: [
        P_decay > 0,           -- perception active
        T_decay > 0,           -- thought active
        C = P_decay ∩ T_decay, -- consciousness exists
        M = integral(dH/dt)    -- memory accumulates
    ]

    body: COMPLETE(constraints, target)

    returns: (gamma, Gamma_f, M)  -- mental state triple
}
\end{verbatim}
\end{definition}

\subsection{Execution Model}

\begin{theorem}[NPL Execution]
NPL programs execute through:
\begin{enumerate}
\item \textbf{Target specification}: Program declares completion condition $\Gstate_f$
\item \textbf{Constraint collection}: All constraints accumulated
\item \textbf{Backward navigation}: Runtime navigates S-space backward from target
\item \textbf{Trajectory determination}: Unique trajectory $\gamma$ satisfying constraints is found
\item \textbf{State extraction}: Mental state $\Mstate = (\gamma, \Gstate_f, M)$ returned
\end{enumerate}
Complexity: $O(\log_3 n)$ via ternary trisection.
\end{theorem}

\begin{theorem}[Computational Universality]
NPL achieves computational universality through:
\begin{enumerate}
\item \textbf{Controllability}: Arbitrary state transformations via aperture modulation
\item \textbf{Memory persistence}: Phase-locked states stable against thermal fluctuations
\item \textbf{Conditional operations}: Phase threshold dynamics
\item \textbf{Hierarchical composability}: Multi-scale coupling
\end{enumerate}
\end{theorem}

\subsection{Efficiency Properties}

\begin{theorem}[NPL Efficiency]
\begin{enumerate}
\item \textbf{Ternary advantage}: 37\% fewer iterations than binary search
\item \textbf{Categorical exclusion}: $10^{60} \to 1$ unique determination
\item \textbf{Aperture catalysis}: $10^{38}$ reduction factor
\item \textbf{Hierarchical compression}: $10^{44} \to 10^6$ configurations
\item \textbf{Overall efficiency}: $10^{22}$ improvement over explicit enumeration
\end{enumerate}
\end{theorem}
