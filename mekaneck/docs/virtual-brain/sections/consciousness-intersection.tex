\section{Consciousness as Decay Curve Intersection}
\label{sec:consciousness_detailed}

\subsection{Fundamental Architecture}

The central insight of this framework is that consciousness is not a substance, location, or emergent property---it is a \emph{geometric intersection}. Specifically:

\begin{theorem}[Consciousness Intersection]
Consciousness $\Ccat$ occurs at the intersection of the perception decay curve $\Pdecay(t)$ and the thought decay curve $\Tdecay(t)$:
\begin{equation}
\Ccat(t) = \Pdecay(t) \cap \Tdecay(t)
\end{equation}
\end{theorem}

\subsection{Why Intersection?}

Consider what happens without intersection:

\textbf{Thought without Perception} ($\Pdecay = 0$, $\Tdecay > 0$):
\begin{itemize}
\item Thoughts evolve freely through S-entropy space
\item No external constraint on trajectory
\item This IS dreaming (Section~\ref{sec:dreams})
\item Not conscious in the waking sense
\end{itemize}

\textbf{Perception without Thought} ($\Pdecay > 0$, $\Tdecay = 0$):
\begin{itemize}
\item Sensory input arrives but no cognitive processing
\item Pure stimulus-response (reflexive)
\item Not conscious---automatic reaction
\end{itemize}

\textbf{Intersection} ($\Pdecay > 0$ AND $\Tdecay > 0$):
\begin{itemize}
\item Perception constrains thought trajectories
\item Thoughts interpret perceptual input
\item The constraint relationship IS consciousness
\end{itemize}

\subsection{Mathematical Structure}

\begin{definition}[Decay Curve Product Space]
Define the decay product space:
\begin{equation}
\mathcal{D} = \Pdecay \times \Tdecay \subset [0,1]^2
\end{equation}
Consciousness lives on the diagonal where both are non-zero.
\end{definition}

\begin{theorem}[Consciousness Measure]
The ``amount'' of consciousness at time $t$ is:
\begin{equation}
|\Ccat(t)| = \min(\Pdecay(t), \Tdecay(t))
\end{equation}
Consciousness is limited by the weaker of perception or thought.
\end{theorem}

\subsection{Dynamics of the Intersection}

The decay curves evolve according to:

\begin{align}
\frac{d\Pdecay}{dt} &= -\frac{\Pdecay}{\tau_P} + I_{\text{sensory}}(t) \\
\frac{d\Tdecay}{dt} &= -\frac{\Tdecay}{\tau_T} + \Tdecay^{\infty} + \Hfield(t)
\end{align}

where:
\begin{itemize}
\item $\tau_P \sim 50$ ms is the perceptual integration time
\item $\tau_T \sim 100$ ms is the thought decay time
\item $I_{\text{sensory}}$ is sensory input rate
\item $\Tdecay^{\infty}$ is background thought activity
\item $\Hfield$ is the emotional field driving thought
\end{itemize}

\subsection{The Consciousness Window}

\begin{definition}[Consciousness Window]
The consciousness window is the temporal region where both decay curves exceed threshold $\theta$:
\begin{equation}
W_C = \{t : \Pdecay(t) > \theta \land \Tdecay(t) > \theta\}
\end{equation}
\end{definition}

\begin{theorem}[Window Duration]
The consciousness window has characteristic duration:
\begin{equation}
\Delta t_C = \frac{\tau_P \tau_T}{\tau_P + \tau_T} \ln\left(\frac{\Pdecay(0) \Tdecay(0)}{\theta^2}\right)
\end{equation}
For typical values, $\Delta t_C \sim 100$--$500$ ms.
\end{theorem}

\subsection{Distinction from Thought Geometry}

The geometric properties of thought paper established that thoughts are measurable geometric objects---O$_2$ configurations with 30-dimensional oscillatory signatures. But that paper explicitly noted:

\begin{quote}
``The geometric framework characterises \emph{thoughts}---specific O$_2$ molecular arrangements around electron-stabilised holes---but does \textbf{not} explain \emph{consciousness}.''
\end{quote}

Here we complete the picture: consciousness is not the thoughts themselves but the \emph{intersection of thought evolution with perceptual constraint}. Thoughts are the objects; consciousness is the relationship between those objects and external reality.

\subsection{Self-Reference and the Meta-Level}

The consciousness intersection enables self-reference because:

\begin{enumerate}
\item Perception can include perception of one's own body
\item Thought can include thoughts about one's own thoughts
\item The intersection creates a feedback loop
\end{enumerate}

\begin{equation}
\Ccat_{\text{self}} = \Pdecay(\text{self}) \cap \Tdecay(\text{self-thought})
\end{equation}

This is how consciousness becomes aware of itself.
