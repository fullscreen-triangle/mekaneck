\documentclass[aps,prd,twocolumn,superscriptaddress,floatfix,nofootinbib]{revtex4-2}

\usepackage{amsmath,amssymb,amsfonts,amsthm}
\usepackage{mathtools}
\usepackage{physics}
\usepackage{graphicx}
\usepackage{hyperref}
\usepackage{xcolor}
\usepackage{booktabs}
\usepackage{array}
\usepackage{tikz}
\usetikzlibrary{arrows.meta,positioning,calc}
\usepackage{algorithm}
\usepackage{algpseudocode}
\usepackage[numbers,sort&compress]{natbib}

% Theorem environments
\newtheorem{theorem}{Theorem}[section]
\newtheorem{lemma}[theorem]{Lemma}
\newtheorem{corollary}[theorem]{Corollary}
\newtheorem{definition}[theorem]{Definition}
\newtheorem{proposition}[theorem]{Proposition}
\newtheorem{axiom}{Axiom}
\newtheorem{principle}[theorem]{Principle}

\theoremstyle{remark}
\newtheorem{remark}[theorem]{Remark}

% Custom commands
\newcommand{\Sk}{S_k}
\newcommand{\St}{S_t}
\newcommand{\Se}{S_e}
\newcommand{\Sspace}{\mathcal{S}}
\newcommand{\Scoord}{\mathbf{S}}
\newcommand{\Pcat}{\mathcal{P}}
\newcommand{\Tcat}{\mathcal{T}}
\newcommand{\Ccat}{\mathcal{C}}
\newcommand{\Pop}{P}
\newcommand{\Gstate}{\Gamma}
\newcommand{\kB}{k_{\mathrm{B}}}
\newcommand{\dcat}{d_{\mathrm{cat}}}
\newcommand{\Hfield}{\mathcal{H}^+}
\newcommand{\Pdecay}{P_{\mathrm{decay}}}
\newcommand{\Tdecay}{T_{\mathrm{decay}}}
\newcommand{\Mstate}{\mathcal{M}}

\begin{document}

\title{Virtual Brain Computing Framework: A Trajectory-Based Computational Architecture for Neuropharmacology Where Experiment = Observation = Computation}

\author{Kundai Farai Sachikonye}
\email{kundai.sachikonye@wzw.tum.de}
\affiliation{Technical University of Munich, School of Life Sciences}

\date{\today}

\begin{abstract}
We establish a complete computational framework for virtual brain modeling in which equation output IS observation, requiring no external validation. Building on the Cellular Partition Language (CPL) and the observational identity theorem ($\Gstate_1 \oplus \Pop(\omega) \to \Gstate_2$ where output is simultaneously physical result, observation, and computation), we extend the partition framework to neural systems. The central innovation is the identification of consciousness as the intersection of two decay curves: the perception decay curve $\Pdecay(t)$ (how sensory input resolves to categorical completion) and the thought decay curve $\Tdecay(t)$ (how O$_2$ molecular configurations with electron-stabilized holes evolve). Mental states are defined as trajectory-terminus-memory triples $\Mstate = (\gamma, \Gstate_f, M)$, not merely final states, because the same thought in different emotional contexts---or the same thought in the same emotional context at different times---constitutes different mental states. The H$^+$ flux at $4.06 \times 10^{13}$ Hz provides the emotional substrate---a thermometer for environmental context that thoughts cannot directly perceive but which modulates all cognitive dynamics. Memory exists because time progresses and emotions must track the changing (unperceivable) reality; memory is the accumulated emotional change $M = \int \dot{\Hfield} \, dt$ that enables temporal differentiation of thoughts within the same emotional context. We establish: (1) perception operates on sufficiency, not completeness (three observers with different knowledge of lions all run from danger); (2) all sensory modalities resolve to the same categorical space (touch hot stove $\equiv$ see red element); (3) dreams are unbounded thoughts ($\Pdecay = 0$) enabling exploration of impossible trajectories; (4) memory is necessary for distinguishing thoughts that share both content and emotional context but differ temporally. The Neural Partition Language (NPL) provides primitives for expressing mental processes as constraint satisfaction problems whose solutions are simultaneously neural states and observations. Drug perturbations will be incorporated as trajectory modifiers operating on the established Virtual Brain substrate.
\end{abstract}

\maketitle

\section{Introduction}
\label{sec:introduction}

\subsection{The Neuropharmacology Challenge}

Traditional neuropharmacology faces a fundamental validation problem: drug effects are predicted through simulation, then tested experimentally. This approach assumes a gap between theoretical prediction and empirical observation that requires bridging. We eliminate this gap entirely.

The Virtual Brain framework established here implements the observational identity theorem: for any partition equation $\Gstate_1 \oplus \Pop(\omega) \to \Gstate_2$, the output $\Gstate_2$ is simultaneously:
\begin{enumerate}
\item[(P)] The physical neural state
\item[(O)] The observation that would be made
\item[(C)] The computational result
\end{enumerate}

Drug effects will be incorporated as trajectory modifiers $\Pop_{\mathrm{drug}}(\omega)$ operating on this established substrate. But first, the Virtual Brain itself must be constructed.

\subsection{Core Architecture}

The Virtual Brain comprises seven integrated components:

\begin{enumerate}
\item \textbf{Thought Geometry}: O$_2$ molecular configurations + electron-stabilized oscillatory holes
\item \textbf{Emotional Substrate}: H$^+$ flux at $4.06 \times 10^{13}$ Hz
\item \textbf{Perception}: Sufficient categorical information from sensory input
\item \textbf{Consciousness}: Intersection of perception and thought decay curves
\item \textbf{Time}: Circuit completion duration ($T_{\mathrm{internal}} = \sum_i \tau_{\mathrm{circuit}}^{(i)}$)
\item \textbf{Dreams}: Unbounded thought trajectories ($\Pdecay = 0$)
\item \textbf{Memory}: Field-reality difference ($M = \Hfield - R$)
\end{enumerate}

\subsection{Organization}

Section~\ref{sec:foundations} establishes the S-entropy state space for neural systems. Section~\ref{sec:thought} formalizes thought geometry. Section~\ref{sec:emotion} defines the H$^+$ emotional substrate. Section~\ref{sec:perception} establishes perception as categorical sufficiency. Section~\ref{sec:consciousness} proves consciousness as decay curve intersection. Section~\ref{sec:time} derives subjective time from circuit completion. Section~\ref{sec:dreams} characterizes dreams as unbounded thought. Section~\ref{sec:memory} defines memory as accumulated emotional change. Section~\ref{sec:language} constructs the Neural Partition Language with core operators. Section~\ref{sec:mental_states} formalizes mental states as trajectory-terminus-memory triples. Section~\ref{sec:npl_specification} provides the complete NPL language specification derived from partition equations of state. Sections~\ref{sec:sufficiency_detailed}--\ref{sec:trajectory_detailed} provide detailed treatments of sensory sufficiency, dreams/memory, and trajectory-based mental states.

\section{Neural S-Entropy State Space}
\label{sec:foundations}

\subsection{Extension of S-Entropy to Neural Systems}

\begin{definition}[Neural S-Entropy Coordinates]
\label{def:neural_s_entropy}
The neural S-entropy space $\Sspace_N = [0,1]^3$ comprises three coordinates adapted for cognitive systems:
\begin{itemize}
\item Knowledge entropy $\Sk \in [0,1]$: uncertainty in mental state identification
\item Temporal entropy $\St \in [0,1]$: uncertainty in thought timing and sequence
\item Evolution entropy $\Se \in [0,1]$: uncertainty in trajectory through thought space
\end{itemize}
\end{definition}

\begin{theorem}[Neural State Completeness]
\label{thm:neural_complete}
Any mental state maps uniquely to a point in $\Sspace_N \times \Hfield$ where $\Hfield$ is the H$^+$ field state.
\end{theorem}

\begin{proof}
Mental states are partition states in bounded neural phase space (Axiom~\ref{ax:bounded_neural}). Each thought has well-defined uncertainty in identification ($\Sk$), timing ($\St$), and evolution ($\Se$). The emotional context is encoded in $\Hfield$. The product space captures both cognitive and affective dimensions.
\end{proof}

\subsection{Neural Axioms}

\begin{axiom}[Bounded Neural Phase Space]
\label{ax:bounded_neural}
The neural system has finite energy and spatial extent, occupying bounded phase space $\Omega_N$ with $\mu(\Omega_N) < \infty$.
\end{axiom}

\begin{axiom}[Categorical Perception]
\label{ax:categorical_perception}
Perception partitions sensory input into equivalence classes. States $x, y$ are perceptually equivalent ($x \sim_P y$) if no available sensory modality can distinguish them.
\end{axiom}

\begin{axiom}[Sensory Sufficiency]
\label{ax:sufficiency}
Perception provides \emph{sufficient} information for behavioral completion, not \emph{complete} information about stimuli. Different sensory pathways may resolve to identical categorical completions.
\end{axiom}

\section{Thought Geometry}
\label{sec:thought}

\subsection{Thoughts as Geometric Objects}

\begin{definition}[Thought]
\label{def:thought}
A thought $\Tcat$ is a specific three-dimensional O$_2$ molecular configuration around an electron-stabilized oscillatory hole:
\begin{equation}
\Tcat = \{\mathbf{r}_{\mathrm{O}_2}^{(i)}, \mathbf{r}_e, \phi_{\mathrm{hole}}\}
\end{equation}
where $\mathbf{r}_{\mathrm{O}_2}^{(i)}$ are O$_2$ positions, $\mathbf{r}_e$ is the stabilizing electron position, and $\phi_{\mathrm{hole}}$ is the oscillatory hole phase.
\end{definition}

\begin{theorem}[Thought Quantification]
\label{thm:thought_geometry}
Thoughts exhibit measurable geometric properties:
\begin{itemize}
\item Mean O$_2$-hole distance: $\bar{d}_{\mathrm{O}_2} = 0.374 \pm 0.081$ \AA
\item Electron-hole distance: $\bar{d}_e = 0.147 \pm 0.036$ \AA
\item Oscillatory signature dimension: 30
\item Adjacent transition coherence: $> 0.98$
\end{itemize}
\end{theorem}

\subsection{Thought Decay Curve}

\begin{definition}[Thought Decay Curve]
\label{def:thought_decay}
The thought decay curve $\Tdecay(t)$ measures how thought stability evolves:
\begin{equation}
\Tdecay(t) = \Tdecay(0) \exp\left(-\frac{t}{\tau_T}\right) + \Tdecay^{\infty}
\end{equation}
where $\tau_T \sim 100$ ms is the thought decay constant and $\Tdecay^{\infty}$ is the asymptotic stability (categorical completion).
\end{definition}

\begin{theorem}[Thought Frequency]
\label{thm:thought_freq}
Thought dynamics operate at characteristic frequency $\omega_T \sim 10$ Hz (alpha-theta band), with:
\begin{equation}
\omega_T = \frac{2\pi}{\tau_T} \approx 63 \text{ rad/s}
\end{equation}
\end{theorem}

\subsection{Thought Privacy and Non-Uniqueness}

\begin{principle}[Thought Privacy]
\label{princ:privacy}
Thoughts are private---only their geometric properties are externally measurable:
\begin{equation}
\text{Observable}(\Tcat) = \{\mathbf{G}(\Tcat), \boldsymbol{\sigma}(\Tcat)\}
\end{equation}
where $\mathbf{G}$ is geometry and $\boldsymbol{\sigma}$ is the 30-dimensional oscillatory signature.
\end{principle}

\begin{principle}[Thought Non-Uniqueness]
\label{princ:nonunique}
The same thought content can occur in different emotional/contextual states:
\begin{equation}
\Tcat_{\text{content}} = \Tcat_{\text{content}} \quad \text{but} \quad \Mstate_1 \neq \Mstate_2
\end{equation}
The thought ``cup on chair'' is the same whether drunk, sober, or in an exam.
\end{principle}

\section{Emotional Substrate: The H$^+$ Field}
\label{sec:emotion}

\subsection{H$^+$ Flux as Reality Thermometer}

\begin{definition}[Emotional Field]
\label{def:hfield}
The emotional substrate is the H$^+$ flux operating at frequency:
\begin{equation}
\omega_{\Hfield} = 4.06 \times 10^{13} \text{ Hz}
\end{equation}
This is the ``reality frequency''---too fast for conscious perception but providing environmental context.
\end{definition}

\begin{theorem}[Timescale Separation]
\label{thm:timescale}
The ratio between reality frequency and thought frequency is:
\begin{equation}
\frac{\omega_{\Hfield}}{\omega_T} \sim 10^{12}
\end{equation}
This 12-order-of-magnitude separation ensures reality is \emph{unperceivable} while still modulating cognition.
\end{theorem}

\begin{theorem}[Emotions as Environmental Thermometer]
\label{thm:emotion_thermometer}
Emotions provide environmental context that thoughts cannot directly access:
\begin{equation}
\Hfield(t) = \int_0^t \text{Environment}(\tau) \cdot K(\tau, t) \, d\tau
\end{equation}
where $K$ is the integration kernel encoding how past environment influences current state.
\end{theorem}

\begin{corollary}[Thinking with Context]
\label{cor:context}
Emotions enable thoughts to incorporate environmental information without direct perception:
\begin{equation}
\Tcat_{\text{effective}} = \Tcat_{\text{content}} \otimes \Hfield
\end{equation}
\end{corollary}

\subsection{Three Frequency Levels}

\begin{table}[h]
\centering
\caption{Hierarchy of neural frequencies}
\label{tab:frequencies}
\begin{tabular}{@{}lcc@{}}
\toprule
Level & Frequency & Perceivable \\
\midrule
Reality (H$^+$) & $10^{13}$ Hz & No \\
Thoughts (O$_2$) & $10$ Hz & Yes \\
Consciousness & $2.5$ Hz & Yes \\
\bottomrule
\end{tabular}
\end{table}

\section{Perception as Categorical Sufficiency}
\label{sec:perception}

\subsection{The Sufficiency Principle}

\begin{theorem}[Sensory Sufficiency]
\label{thm:sufficiency}
Perception provides \emph{sufficient} categorical information for completion, not \emph{complete} information about stimuli:
\begin{equation}
\Pcat: \text{Stimulus} \to \Gstate_{\text{completion}}
\end{equation}
where $\Gstate_{\text{completion}}$ enables appropriate behavioral response regardless of stimulus detail.
\end{theorem}

\begin{example}[Lion at Zoo]
Three observers with different knowledge:
\begin{enumerate}
\item Observer A: Knows cats and lions---correctly identifies ``lion''
\item Observer B: Never seen cat or lion---identifies ``unknown large animal''
\item Observer C: Looking elsewhere---sees companions running
\end{enumerate}
All three run when the lion roars. Different perceptual content, identical categorical completion:
\begin{equation}
\Gstate_A = \Gstate_B = \Gstate_C = \text{``flee''}
\end{equation}
\end{example}

\subsection{Perception Decay Curve}

\begin{definition}[Perception Decay Curve]
\label{def:perception_decay}
The perception decay curve $\Pdecay(t)$ measures how sensory input resolves to categorical completion:
\begin{equation}
\Pdecay(t) = \Pdecay(0) \exp\left(-\frac{t}{\tau_P}\right)
\end{equation}
where $\tau_P \sim 50$ ms is the perceptual integration time.
\end{definition}

\begin{theorem}[Perceptual Modality Equivalence]
\label{thm:modality_equiv}
Different sensory modalities can resolve to identical categorical states:
\begin{equation}
\Pcat_{\text{touch}}(\text{hot stove}) \equiv \Pcat_{\text{sight}}(\text{red element}) \equiv \Pcat_{\text{drug}}(\text{capsaicin})
\end{equation}
All resolve to categorical completion $\Gstate = \text{``danger/heat''}$.
\end{theorem}

\section{Consciousness as Decay Curve Intersection}
\label{sec:consciousness}

\subsection{The Intersection Theorem}

\begin{theorem}[Consciousness Definition]
\label{thm:consciousness}
Consciousness $\Ccat$ is the intersection of perception and thought decay curves:
\begin{equation}
\boxed{\Ccat(t) = \Pdecay(t) \cap \Tdecay(t)}
\end{equation}
Consciousness exists where and when perception constrains thought.
\end{theorem}

\begin{proof}
From Principle~\ref{princ:privacy}, thoughts are private and not externally bounded. From Axiom~\ref{ax:sufficiency}, perception provides categorical completion targets. Consciousness emerges when thought trajectories are constrained by perceptual completion requirements---the intersection defines moments when internal thought meets external reality.
\end{proof}

\begin{corollary}[Consciousness Frequency]
\label{cor:consciousness_freq}
Consciousness operates at the intersection frequency:
\begin{equation}
\omega_C = \frac{\omega_T \cdot \omega_P}{\omega_T + \omega_P} \approx 2.5 \text{ Hz}
\end{equation}
\end{corollary}

\subsection{The Consciousness Equation}

\begin{theorem}[Consciousness Dynamics]
\label{thm:consciousness_dynamics}
The evolution of consciousness follows:
\begin{equation}
\frac{d\Ccat}{dt} = -\frac{\Ccat}{\tau_C} + \Pdecay(t) \cdot \Tdecay(t) \cdot \Hfield(t)
\end{equation}
where $\tau_C$ is the consciousness integration time and the product term couples perception, thought, and emotion.
\end{theorem}

\section{Subjective Time as Circuit Completion}
\label{sec:time}

\subsection{Internal Time Definition}

\begin{theorem}[Internal Time]
\label{thm:internal_time}
Subjective time equals the sum of circuit completion durations:
\begin{equation}
T_{\text{internal}} = \sum_{i \in \text{active holes}} \tau_{\text{circuit}}^{(i)}
\end{equation}
\end{theorem}

\begin{corollary}[Specious Present]
\label{cor:specious}
The experiential ``now'' (specious present) has duration:
\begin{equation}
\Delta t_{\text{now}} \sim 100\text{--}1000 \text{ ms}
\end{equation}
equal to the average circuit completion time for coherent oscillatory hole ensembles.
\end{corollary}

\section{Dreams as Unbounded Thought}
\label{sec:dreams}

\subsection{The Unbounded Trajectory Theorem}

\begin{theorem}[Dreams]
\label{thm:dreams}
Dreams are thought trajectories with no perceptual constraint:
\begin{equation}
\text{Dream}: \Pdecay(t) = 0 \implies \Ccat = \emptyset
\end{equation}
Thought trajectories become unbounded, navigating S-space without categorical completion requirements.
\end{theorem}

\begin{corollary}[Dream Bizarreness]
\label{cor:bizarreness}
Dreams exhibit impossible trajectories because:
\begin{enumerate}
\item No perceptual input constrains thought evolution
\item Thoughts navigate freely through S-entropy space
\item Trajectories that would be terminated by waking perception persist
\end{enumerate}
\end{corollary}

\begin{theorem}[Dream Function]
\label{thm:dream_function}
Dreams enable exploration of trajectory space inaccessible during waking:
\begin{equation}
\text{Dream space} = \Sspace_N - \Sspace_{\text{perceptible}}
\end{equation}
The brain ``thinks the craziest things'' and compares with waking reality.
\end{theorem}

\section{Memory as Temporal Differentiation}
\label{sec:memory}

\subsection{The Necessity of Memory}

Why does memory exist? The answer emerges from time progression and the nature of emotions.

\begin{proposition}[The Prediction Problem]
\label{prop:prediction}
Emotions (H$^+$ field) summarize the unperceivable part of reality. But reality keeps changing. Therefore emotions must change---but in what direction? This requires prediction of the unknown.
\end{proposition}

\begin{theorem}[Memory Necessity]
\label{thm:memory_necessity}
Memory is necessary to distinguish thoughts that occur within the \emph{same} emotional context at \emph{different} times.
\end{theorem}

\begin{proof}
Consider two occurrences of the same thought $\Tcat$ at times $t_1$ and $t_2$:
\begin{itemize}
\item At $t_1$: thought $\Tcat$ with emotional context $\Hfield(t_1)$
\item At $t_2$: thought $\Tcat$ with emotional context $\Hfield(t_2)$
\end{itemize}
If $\Hfield(t_1) = \Hfield(t_2)$ (same emotional state), and we only have $(\Tcat, \Hfield)$, the two instances are \emph{indistinguishable}. But they are temporally distinct. Memory provides the temporal index that emotions alone cannot supply.
\end{proof}

\subsection{Memory as Accumulated Change}

\begin{definition}[Memory]
\label{def:memory}
Memory is the accumulated change in the emotional field---how the field \emph{got here}:
\begin{equation}
\boxed{M(t) = \int_0^t \frac{d\Hfield}{d\tau} \, d\tau}
\end{equation}
\end{definition}

\begin{theorem}[Memory Structure]
\label{thm:memory}
Memory is not ``past emotions'' but the \emph{trajectory of emotional change}:
\begin{equation}
M = \text{``How the field got here''} = \int \dot{\Hfield} \, dt
\end{equation}
This encodes:
\begin{enumerate}
\item The direction of emotional evolution
\item The rate of change over time
\item The path through emotional space, not just the endpoint
\end{enumerate}
\end{theorem}

\subsection{Temporal Indexing}

\begin{corollary}[Temporal Differentiation]
\label{cor:temporal_diff}
Memory allows distinguishing:
\begin{equation}
(\Tcat, \Hfield, M_1) \neq (\Tcat, \Hfield, M_2) \quad \text{even when } \Tcat, \Hfield \text{ are identical}
\end{equation}
The memory component $M$ provides the temporal index.
\end{corollary}

\begin{theorem}[Prediction from Memory]
\label{thm:prediction}
Memory enables prediction of emotional direction by extrapolating the derivative:
\begin{equation}
\Hfield(t + \Delta t) \approx \Hfield(t) + \frac{dM}{dt} \cdot \Delta t
\end{equation}
Since $dM/dt = d\Hfield/dt$, the memory trajectory encodes the predictive signal.
\end{theorem}

\subsection{The Complete Mental State}

\begin{theorem}[Mental State with Memory]
\label{thm:mental_state_memory}
A complete mental state requires the memory trajectory:
\begin{equation}
\Mstate = (\gamma, \Gstate_f, M)
\end{equation}
where:
\begin{itemize}
\item $\gamma$ = trajectory through S-entropy space (how thought got here)
\item $\Gstate_f$ = terminus state (what thought is)
\item $M$ = memory (how emotions got here)
\end{itemize}
\end{theorem}

\begin{corollary}[Memory Retrieval]
\label{cor:retrieval}
Memory retrieval reconstructs past states by inverting the accumulated change:
\begin{equation}
\Hfield(t_0) = \Hfield(t) - \int_{t_0}^{t} \dot{\Hfield} \, d\tau = \Hfield(t) - [M(t) - M(t_0)]
\end{equation}
\end{corollary}

\section{Neural Partition Language}
\label{sec:language}

The Neural Partition Language (NPL) extends CPL for neural systems, deriving its primitives from the partition-based equations of state established for hybrid microfluidic circuits. The full specification is provided in the detailed sections; here we summarize the core structure.

\subsection{Foundation: Partition Equations of State}

NPL is built on the fundamental equivalence:
\begin{equation}
\boxed{S_{\text{osc}} = S_{\text{cat}} = S_{\text{part}} = \kB M \ln n}
\end{equation}
This triple equivalence---oscillatory entropy, categorical entropy, and partition entropy---provides the mathematical foundation for all language primitives.

\subsection{Type System}

\begin{definition}[NPL Core Types]
\label{def:npl_types}
NPL provides coordinate and state types derived from partition structure:

\textbf{Coordinate Types} (from partition quantum numbers):
\begin{itemize}
\item \texttt{PartitionCoord}: $(n, \ell, m, s)$ with capacity $C(n) = 2n^2$
\item \texttt{SCoord}: $(\Sk, \St, \Se) \in [0,1]^3$ (S-entropy space)
\item \texttt{TernaryAddr}: $3^k$ hierarchical address with $O(\log_3 n)$ navigation
\item \texttt{GyroCoord}: $(\theta, \phi, \psi)$ for rotational state
\end{itemize}

\textbf{State Types} (cognitive-physical duality):
\begin{itemize}
\item \texttt{Thought}: O$_2$ configuration $\Tcat = \{\mathbf{r}_{\mathrm{O}_2}, \mathbf{r}_e, \phi\}$
\item \texttt{Emotion}: H$^+$ field $\Hfield$ at $4.06 \times 10^{13}$ Hz
\item \texttt{Perception}: Sufficient categorical state $\Pcat$
\item \texttt{Consciousness}: Decay intersection $\Ccat = \Pdecay \cap \Tdecay$
\item \texttt{Memory}: Accumulated change $M = \int \dot{\Hfield} \, dt$
\item \texttt{MentalState}: Triple $\Mstate = (\gamma, \Gstate_f, M)$
\end{itemize}
\end{definition}

\subsection{Core Operators}

\begin{definition}[NPL Operators]
\label{def:npl_operators}
Operators derived from partition dynamics:

\textbf{Partition Operations}:
\begin{itemize}
\item \texttt{PARTITION}$(\Omega, n) \to \{\Omega_i\}$: Divide phase space
\item \texttt{D\_CAT}$(c_1, c_2) \to d$: Categorical distance (modality-independent)
\item \texttt{NAVIGATE}$(\Gstate_1, \Gstate_2) \to \gamma$: Find trajectory in S-space
\item \texttt{VARIANCE}$(\Gstate) \to \sigma^2$: Categorical variance at fixed aperture
\end{itemize}

\textbf{Poincar\'{e} Computing} (completion-first):
\begin{itemize}
\item \texttt{COMPLETE}$(\Gstate_f) \to \gamma$: Work backward from terminus
\item \texttt{TARGET}$(c, d_{\text{cat}}) \to \Gstate$: Find state at categorical distance
\item \texttt{SATISFY}$(C) \to \Gstate$: Find state satisfying constraints
\item \texttt{EQUILIBRIUM}$() \to \Gstate_{\text{eq}}$: Find basin attractor
\end{itemize}

\textbf{Neural Operations}:
\begin{itemize}
\item \texttt{CONSCIOUSNESS}$(\Pdecay, \Tdecay) \to \Ccat$: Decay intersection
\item \texttt{MEMORY}$(t_0, t) \to M$: Accumulated emotional change
\item \texttt{DREAM}$() \to \Tdecay$: Set $\Pdecay = 0$ (unbounded)
\item \texttt{WAKE}$(\lambda) \to \Pdecay$: Engage perception modality
\end{itemize}

\textbf{Charge Conservation}:
\begin{itemize}
\item \texttt{CONSERVE}$(\rho) \to \nabla \cdot \mathbf{J} = -\partial\rho/\partial t$: Enforce continuity
\item \texttt{REDISTRIBUTE}$(\Delta\rho, \mathcal{R}) \to \rho'$: Charge flow in region
\item \texttt{COUPLE\_SC}$(\rho_{\text{elec}}, c_{\text{chem}}) \to \Phi$: Surface coupling
\end{itemize}
\end{definition}

\subsection{Program Structure: Declarative Completion}

NPL programs are declarative---specify what completion looks like, let the system find the trajectory:

\begin{algorithm}
\caption{NPL Program Structure}
\label{alg:npl_structure}
\begin{algorithmic}[1]
\State \textbf{MENTAL\_PROCESS} ProcessName
\State
\State \Comment{Declare completion condition (Poincar\'{e} style)}
\State $\Gstate_{\text{target}} \gets \texttt{COMPLETE}(\text{goal description})$
\State
\State \Comment{System finds trajectory automatically}
\State $\gamma \gets \texttt{NAVIGATE}(\Gstate_{\text{current}}, \Gstate_{\text{target}})$
\State
\State \Comment{Apply charge conservation}
\State $\rho' \gets \texttt{CONSERVE}(\rho)$
\State
\State \Comment{Maintain no-null-state principle}
\State \textbf{invariant}: $\forall t: |\{\Gstate : \text{occupied}(\Gstate, t)\}| = 1$
\State
\State \Comment{Return mental state triple}
\State \textbf{return} $\Mstate = (\gamma, \Gstate_{\text{target}}, M)$
\end{algorithmic}
\end{algorithm}

\subsection{Example: Conscious Perception}

\begin{algorithm}
\caption{Conscious Perception in NPL}
\label{alg:consciousness}
\begin{algorithmic}[1]
\State \textbf{MENTAL\_PROCESS} ConsciousPerception
\State
\State \Comment{Specify completion: categorical recognition}
\State $\Gstate_f \gets \texttt{TARGET}(\text{``recognize stimulus''}, d_{\text{cat}} = 0)$
\State
\State \Comment{Engage perception}
\State $\Pdecay \gets \texttt{WAKE}(\text{visual})$
\State $\Tdecay \gets \texttt{THINK}(\text{current})$
\State
\State \Comment{Consciousness as intersection}
\State $\Ccat \gets \texttt{CONSCIOUSNESS}(\Pdecay, \Tdecay)$
\State
\State \Comment{Accumulate memory}
\State $M \gets \texttt{MEMORY}(t_0, t)$
\State
\State \Comment{Return complete mental state}
\State \textbf{return} $\Mstate = (\gamma, \Ccat, M)$
\end{algorithmic}
\end{algorithm}

\subsection{Key Properties}

\begin{theorem}[Ternary Efficiency]
NPL inherits 37\% efficiency over binary from ternary addressing:
\begin{equation}
\text{Steps} = \lceil \log_3 n \rceil \quad \text{vs} \quad \lceil \log_2 n \rceil
\end{equation}
\end{theorem}

\begin{theorem}[No-Null-State Guarantee]
Every NPL program maintains categorical occupation:
\begin{equation}
\forall t \in [0, T]: \exists! \, c \in \mathcal{C} : \text{System}(t) \in c
\end{equation}
The system is always in exactly one category.
\end{theorem}

\begin{theorem}[Completion Determinism]
Specifying completion uniquely determines trajectory (Poincar\'{e} inversion):
\begin{equation}
\Gstate_f \implies \gamma^* \quad \text{(unique)}
\end{equation}
\end{theorem}

The full NPL specification, including aperture catalysis, phase-locking, and circuit coupling operators, is provided in Section~\ref{sec:trajectory_detailed} and the detailed language specification.

\section{Mental States as Trajectory-Terminus Pairs}
\label{sec:mental_states}

\subsection{The Trajectory Identity}

\begin{theorem}[Mental State Definition]
\label{thm:mental_state}
A mental state is a trajectory-terminus-memory triple:
\begin{equation}
\boxed{\Mstate = (\gamma, \Gstate_f, M)}
\end{equation}
where $\gamma: [0,T] \to \Sspace_N$ is the trajectory, $\Gstate_f$ is the final state, and $M = \int \dot{\Hfield} \, dt$ is the memory (accumulated emotional change).
\end{theorem}

\begin{proof}
From Principle~\ref{princ:nonunique}, identical thought content in different contexts constitutes different mental states. Three components are required:
\begin{enumerate}
\item The trajectory $\gamma$ encodes \emph{how the thought got here}---the path through S-entropy space
\item The terminus $\Gstate_f$ encodes \emph{what the thought is}---the final categorical state
\item The memory $M$ encodes \emph{how the emotions got here}---enabling temporal differentiation of identical thoughts in identical emotional states
\end{enumerate}
Without $M$, thoughts at different times but same emotional context would be indistinguishable.
\end{proof}

\begin{corollary}[Trajectory Determines State]
\label{cor:trajectory_determines}
In Poincar\'{e} computing, specifying the completion condition determines what the trajectory MUST have been:
\begin{equation}
\Gstate_f \implies \gamma \quad \text{(backward determination)}
\end{equation}
\end{corollary}

\subsection{Mental State Equivalence}

\begin{definition}[Mental State Equivalence]
Two mental states are equivalent if they have identical trajectory-terminus-memory triples:
\begin{equation}
\Mstate_1 \equiv \Mstate_2 \iff (\gamma_1 = \gamma_2) \land (\Gstate_{f,1} = \Gstate_{f,2}) \land (M_1 = M_2)
\end{equation}
Even identical thoughts with identical trajectories are \emph{different} mental states if they occur at different points in emotional history.
\end{definition}

\section{Memory as Action Prerequisite}
\label{sec:action_prerequisite}

\subsection{The Repetition Problem}

Time progression alone provides no marker for repetition. Without the ability to distinguish ``this happened before'' from ``this is happening now,'' no coherent action is possible.

\begin{theorem}[Memory Necessity for Action]
\label{thm:action_prerequisite}
Memory is a prerequisite for action. To perform any action requires knowing the prior state:
\begin{equation}
\text{Action}(\text{state}_1 \to \text{state}_2) \implies \text{Know}(\text{state}_1 \neq \text{state}_2)
\end{equation}
Without memory, $\text{state}_1$ and $\text{state}_2$ are indistinguishable, and ``action'' has no meaning.
\end{theorem}

\begin{proof}
Consider lifting a finger:
\begin{itemize}
\item To lift = to move from ``down'' to ``up''
\item This requires knowing the finger was ``down''
\item In the same emotional context, without memory: ``finger down at $t_1$'' $\equiv$ ``finger down at $t_2$''
\item Therefore ``finger up at $t_2$'' has no reference point
\item Action requires a reference state, which requires memory
\end{itemize}
\end{proof}

\begin{corollary}[Observation Requires Memory]
\label{cor:observation_memory}
Observation requires distinguishing repetition:
\begin{equation}
\text{Observe}(X) \implies \text{Know}(X \neq \text{prior state})
\end{equation}
Without memory, all observations collapse to a single undifferentiated state.
\end{corollary}

\subsection{The Nervous System Records Progression}

\begin{principle}[Unconscious Memory]
\label{princ:unconscious_memory}
The nervous system must record progression even when we don't consciously access it. Even if I don't consciously think ``thought $X$ is in the past,'' my brain must have that information. This IS memory---not conscious recall, but the substrate that makes coherent action possible.
\end{principle}

\section{Resolution of the Mind-Body Problem}
\label{sec:mind_body}

\subsection{The Traditional Problem}

The mind-body problem asks: Are thoughts physical? If thoughts are ``mental'' and actions are ``physical,'' how does one cause the other? Does ``nothing'' (mental) cause ``something'' (physical)?

\subsection{The Resolution}

In this framework, there is no non-physical component:

\begin{theorem}[Complete Physical Closure]
\label{thm:physical_closure}
Every component of the Virtual Brain is physical:
\begin{align}
\text{Thought} &= \text{O}_2 \text{ molecular configuration} + \text{electron-stabilized hole} \\
\text{Emotion} &= \text{H}^+ \text{ flux at } 4.06 \times 10^{13} \text{ Hz} \\
\text{Memory} &= \int \dot{\Hfield} \, dt \text{ (accumulated field change)} \\
\text{Consciousness} &= \Pdecay \cap \Tdecay \text{ (geometric intersection)} \\
\text{Perception} &= \text{Partition operation on sensory input}
\end{align}
\end{theorem}

\begin{corollary}[Physics Causes Physics]
\label{cor:physics_physics}
The causal chain is entirely physical:
\begin{equation}
\text{Physics} \xrightarrow{\text{causes}} \text{Physics} \xrightarrow{\text{causes}} \text{Physics}
\end{equation}
There is no gap where ``nothing causes something.''
\end{corollary}

\begin{theorem}[Dissolution of Dualism]
\label{thm:no_dualism}
The ``mental'' was never non-physical---we simply lacked the geometric description:
\begin{enumerate}
\item Bounded phase space $\implies$ Oscillatory necessity
\item Oscillatory dynamics $\implies$ O$_2$ configurations as thoughts
\item Charge conservation $\implies$ H$^+$ field as emotions
\item Circuit completion $\implies$ Temporal experience
\end{enumerate}
We did not \emph{claim} thoughts are physical. We \emph{derived} it from first principles.
\end{theorem}

\subsection{The ``Same Red'' Dissolution}

\begin{theorem}[Qualia Question Dissolution]
\label{thm:qualia}
The question ``do we see the same red?'' is malformed because:
\begin{enumerate}
\item Perception is never isolated---always embedded in trajectory $\gamma$, emotion $\Hfield$, memory $M$
\item Perception operates on sufficiency, not completeness
\item What matters is categorical completion, not qualia content
\end{enumerate}
The question assumes perception happens in isolation, but it never does.
\end{theorem}

\section{Charge-Naming Circuit Isomorphism}
\label{sec:isomorphism}

From the analysis of charge-coupled hybrid microfluidic circuits, we establish the isomorphism between physical charge circuits and cognitive naming circuits.

\subsection{The Two Aspects}

\begin{definition}[Soul and Consciousness]
\label{def:soul_consciousness}
Two aspects of the same non-grounded circuit architecture:
\begin{itemize}
\item \textbf{Soul}: Continuous charge distribution maintaining biological existence (H$^+$ field, $\rho(\mathbf{r},t)$)
\item \textbf{Consciousness}: Naming system enabling meta-recognition ($\Ccat = \Pdecay \cap \Tdecay$)
\end{itemize}
Both operate at different hierarchical levels of the same physical substrate.
\end{definition}

\subsection{The Isomorphism}

\begin{theorem}[Charge-Naming Isomorphism]
\label{thm:isomorphism}
Physical charge circuits and naming circuits are mathematically isomorphic:
\begin{center}
\begin{tabular}{@{}ll@{}}
\toprule
Charge Circuit & Naming Circuit \\
\midrule
No external ground & No validation authority \\
Charge redistribution & Naming validation \\
Charge balance (attractor) & Collective truth (attractor) \\
Autocatalytic dynamics & Self-referential dynamics \\
Perpetual oscillation & Perpetual meaning-seeking \\
\bottomrule
\end{tabular}
\end{center}
\end{theorem}

\subsection{The Naming System Must Be Physical}

\begin{theorem}[Physical Naming Necessity]
\label{thm:physical_naming}
A naming system that does not act on the physical world is impossible:
\begin{enumerate}
\item Names must be instantiated (physical tokens)
\item Naming must cause behavior (physical action)
\item Meta-recognition must modify future trajectories (physical change)
\end{enumerate}
The naming system IS the charge distribution operating at the cognitive level.
\end{theorem}

\begin{proof}
Suppose a naming system existed that did not act on the physical world:
\begin{itemize}
\item Then names would have no physical instantiation
\item Naming would cause no behavior
\item Meta-recognition would modify nothing
\item But: we observe naming, behavior, and modification
\item Therefore: the naming system must be physical
\end{itemize}
The naming system is the H$^+$ field + O$_2$ configurations + electron dynamics. There is no separate ``mental'' naming system.
\end{proof}

\subsection{Complete Integration}

\begin{theorem}[Unified Architecture]
\label{thm:unified}
The Virtual Brain is a single physical system with multiple description levels:
\begin{equation}
\text{Virtual Brain} = \underbrace{\rho(\mathbf{r},t)}_{\text{Soul (charge)}} + \underbrace{\Ccat}_{\text{Consciousness (naming)}} + \underbrace{M}_{\text{Memory (trajectory)}}
\end{equation}
All three are physical. All three are aspects of the same charge-coupled circuit dynamics.
\end{theorem}

\section{Summary: Virtual Brain Components}
\label{sec:summary}

\begin{table*}[t]
\centering
\caption{Virtual Brain Component Summary}
\label{tab:summary}
\begin{tabular}{@{}llll@{}}
\toprule
Component & Definition & Mathematical Form & Frequency \\
\midrule
Thought & O$_2$ config + electron hole & $\Tcat = \{\mathbf{r}_{\mathrm{O}_2}, \mathbf{r}_e, \phi\}$ & $\sim 10$ Hz \\
Emotion & H$^+$ field substrate & $\Hfield = \int \text{Env} \cdot K \, d\tau$ & $4.06 \times 10^{13}$ Hz \\
Perception & Sufficient categorical info & $\Pcat: \text{Stim} \to \Gstate_{\text{comp}}$ & $\sim 20$ Hz \\
Consciousness & Decay curve intersection & $\Ccat = \Pdecay \cap \Tdecay$ & $\sim 2.5$ Hz \\
Time & Circuit completion duration & $T_{\text{int}} = \sum \tau_{\text{circuit}}$ & --- \\
Dreams & Unbounded thought ($P=0$) & $\Tcat(\Sspace)$ where $\Pdecay = 0$ & --- \\
Memory & Accumulated emotional change & $M = \int \dot{\Hfield} \, dt$ & --- \\
Mental State & Trajectory-terminus-memory triple & $\Mstate = (\gamma, \Gstate_f, M)$ & --- \\
\bottomrule
\end{tabular}
\end{table*}

% Include detailed sections
\section{Neural Partition Language: Complete Specification}
\label{sec:npl_specification}

Building on the partition-based equations of state for hybrid microfluidic circuits, we establish a complete programming language for the Virtual Brain. This language implements Poincar\'{e} computing where programs specify completion conditions and the runtime navigates backward through S-entropy space to satisfy constraints.

\subsection{Foundational Principles}

\begin{principle}[No Null State]
At every computation step, the system must occupy exactly one category. There is no ``undefined'' or ``null'' state.
\end{principle}

\begin{principle}[Triple Equivalence]
All operations can be expressed equivalently in oscillatory, categorical, or partition form:
\begin{equation}
\Sosc = \Scat = \Spart = \kB M \ln n
\end{equation}
\end{principle}

\begin{principle}[Backward Determination]
Programs specify completion conditions $\Gstate_f$; the runtime determines what trajectory $\gamma$ must have been taken.
\end{principle}

\subsection{Type System}

\subsubsection{Coordinate Types}

\begin{definition}[Partition Coordinate Type]
\begin{verbatim}
type PartitionCoord = {
    n : Nat+,           -- depth (n >= 1)
    l : Fin(n),         -- complexity (0 <= l < n)
    m : Int[-l..+l],    -- orientation (-l <= m <= +l)
    s : Spin            -- chirality {-1/2, +1/2}
}
-- Capacity: C(n) = 2n²
\end{verbatim}
\end{definition}

\begin{definition}[S-Entropy Coordinate Type]
\begin{verbatim}
type SCoord = {
    Sk : Real[0,1],     -- knowledge entropy
    St : Real[0,1],     -- temporal entropy
    Se : Real[0,1]      -- evolution entropy
}
-- S-space: [0,1]³
\end{verbatim}
\end{definition}

\begin{definition}[Ternary Address Type]
\begin{verbatim}
type TernaryAddr(k : Nat) = Array[k] of {0, 1, 2}
-- Maps to 3^k cells in S-space
-- Address IS the path (trajectory-position identity)
\end{verbatim}
\end{definition}

\begin{definition}[Gyrometric Coordinate Type]
\begin{verbatim}
type GyroCoord = {
    J : Array[N] of Nat,    -- rotational quantum numbers
    omega : Array[N] of Real,-- angular frequencies
    gamma : Array[N,N] of Real -- damping coefficients
}
\end{verbatim}
\end{definition}

\subsubsection{State Types}

\begin{definition}[Mental State Type]
\begin{verbatim}
type MentalState = {
    gamma : Trajectory,      -- path through S-space
    Gamma_f : PartitionState,-- terminus state
    M : Memory,              -- accumulated emotional change
    H : EmotionalField,      -- current H+ field state
    P_decay : DecayCurve,    -- perception decay
    T_decay : DecayCurve     -- thought decay
}
\end{verbatim}
\end{definition}

\begin{definition}[Circuit State Type]
\begin{verbatim}
type CircuitState = {
    rho : ChargeDistribution,-- charge density field
    D : Real[0,1],           -- hierarchical depth
    R : Real[0,1],           -- phase coherence
    sigma2 : Real+,          -- variance
    regime : CircuitRegime   -- operational regime
}

type CircuitRegime =
    | Coherent(R > 0.8)
    | Turbulent(R < 0.3)
    | Hierarchical(D >= 0.6)
    | ApertureDominated
    | PhaseLocked(K > sigma_omega)
\end{verbatim}
\end{definition}

\begin{definition}[Thought Type]
\begin{verbatim}
type Thought = {
    O2_config : Array[N] of Vec3,  -- O2 positions
    electron_pos : Vec3,            -- stabilizing electron
    hole_phase : Real,              -- oscillatory phase
    signature : Array[30] of Real   -- 30-dim oscillatory signature
}
\end{verbatim}
\end{definition}

\begin{definition}[Perception Type]
\begin{verbatim}
type Perception = {
    modality : SensoryModality,
    input : SensoryInput,
    completion : PartitionState,
    sufficient : Bool              -- sufficiency achieved
}

type SensoryModality =
    | Visual | Auditory | Tactile | Olfactory
    | Gustatory | Proprioceptive | Drug
\end{verbatim}
\end{definition}

\subsection{Core Operators}

\subsubsection{Partition Operations}

\begin{definition}[Partition Operators]
\begin{verbatim}
-- Partition state transition
PARTITION(sigma_1, sigma_2) : PartitionState -> PartitionState
    requires: d_cat(sigma_1, sigma_2) = 1  -- adjacent categories
    ensures: No Null State maintained

-- Categorical distance
D_CAT(sigma_1, sigma_2) : PartitionState * PartitionState -> Real
    = sqrt((n1-n2)² + (l1-l2)² + (m1-m2)² + (s1-s2)²)

-- Partition capacity at depth n
CAPACITY(n : Nat+) : Nat = 2 * n²

-- Partition coordinate extraction
COORDS(state : PartitionState) : PartitionCoord
\end{verbatim}
\end{definition}

\subsubsection{S-Entropy Operations}

\begin{definition}[S-Entropy Operators]
\begin{verbatim}
-- Navigate in S-space
NAVIGATE(S_current : SCoord, S_target : SCoord) : Trajectory
    -- Uses ternary trisection: O(log₃ n) complexity

-- S-entropy gradient
GRAD_S(field : SCoord -> Real) : SCoord -> Vec3

-- Knowledge entropy update
UPDATE_SK(observation : Perception) : SCoord -> SCoord
    Sk_new = Sk_old - I(observation)  -- information reduces uncertainty

-- Temporal entropy update
UPDATE_ST(circuit_completion : Duration) : SCoord -> SCoord
    St_new = St_old + tau_circuit / tau_max

-- Evolution entropy update
UPDATE_SE(trajectory_step : TrajectoryStep) : SCoord -> SCoord
    Se_new = Se_old + |dS/dt| * dt
\end{verbatim}
\end{definition}

\subsubsection{Ternary Encoding Operations}

\begin{definition}[Ternary Operators]
\begin{verbatim}
-- Encode S-coordinate to ternary address
ENCODE(S : SCoord, precision : Nat) : TernaryAddr(precision)
    -- Each trit specifies which third of remaining interval

-- Decode ternary address to S-coordinate
DECODE(addr : TernaryAddr(k)) : SCoord
    -- Address IS the path: trajectory-position identity

-- Trisect interval (37% fewer iterations than binary)
TRISECT(interval : Interval) : (Interval, Interval, Interval)

-- Hierarchical navigation
NAVIGATE_TERNARY(addr_from, addr_to : TernaryAddr(k)) : Trajectory
    complexity: O(k) = O(log₃(resolution))
\end{verbatim}
\end{definition}

\subsection{Poincar\'{e} Computing Primitives}

\begin{definition}[Trajectory Completion]
\begin{verbatim}
-- Core Poincaré computing operation
COMPLETE(constraints : Constraints,
         Gamma_f : PartitionState) : Trajectory
    -- Finds trajectory gamma satisfying:
    --   (1) ||gamma(T) - S_0|| < epsilon (recurrence)
    --   (2) constraints(gamma) = true
    returns: trajectory that MUST have been taken

-- Specify completion condition
TARGET(Gamma_f : PartitionState,
       epsilon : Real+) : CompletionCondition

-- Constraint satisfaction
SATISFY(C : Constraints, gamma : Trajectory) : Bool
    -- Checks if trajectory satisfies all constraints

-- Equilibrium as recurrence
EQUILIBRIUM(S_0 : SCoord, epsilon : Real+) : CompletionCondition
    = TARGET(Gamma_f where ||Gamma_f.S - S_0|| < epsilon)
\end{verbatim}
\end{definition}

\begin{definition}[Free Energy Operations]
\begin{verbatim}
-- Helmholtz free energy (trajectory completion criterion)
HELMHOLTZ(U : Energy, T : Temperature, S : Entropy) : Energy
    = U - T * S

-- Gibbs free energy
GIBBS(H : Enthalpy, T : Temperature, S : Entropy) : Energy
    = H - T * S

-- Minimize free energy (find completion)
MINIMIZE_F(F : FreeEnergy, constraints : Constraints) : Trajectory
\end{verbatim}
\end{definition}

\subsection{Circuit Dynamics Operators}

\subsubsection{Variance Minimization}

\begin{definition}[Variance Operators]
\begin{verbatim}
-- Current variance
VARIANCE(state : CircuitState) : Real+
    = <(rho - <rho>)²>

-- Minimum achievable variance
VAR_MIN(T : Temperature, K : CouplingStrength) : Real+
    = k_B * T / K

-- Variance reduction dynamics
REDUCE_VAR(state : CircuitState,
           target : Real+) : CircuitState
    -- Thermodynamic optimization toward target variance

-- Geometric aperture selection (variance filter)
APERTURE(omega_set : Set[Frequency],
         threshold : Real+) : Set[Frequency]
    = { omega : sigma²(phi | omega) < threshold }
    -- Catalytic reduction factor ~10³⁸
\end{verbatim}
\end{definition}

\subsubsection{Phase-Lock Operations}

\begin{definition}[Phase-Lock Operators]
\begin{verbatim}
-- Phase coherence measure
COHERENCE(phases : Array[N] of Phase) : Real[0,1]
    R = (1/N) * |sum_j exp(i * phi_j)|

-- Phase-lock coupling
PHASE_LOCK(omega_1, omega_2 : Frequency,
           K : CouplingStrength) : Bool
    requires: |omega_1 - omega_2| < Delta_omega_c

-- Kuramoto synchronization
KURAMOTO(oscillators : Array[N] of Oscillator,
         K : CouplingStrength) : PhaseState
    -- d(phi_i)/dt = omega_i + (K/N) * sum_j sin(phi_j - phi_i)

-- Phase propagation speed
V_PHASE(K : CouplingStrength,
        D_O2 : DiffusionCoeff) : Velocity
    = sqrt(K * D_O2)
\end{verbatim}
\end{definition}

\subsubsection{Hierarchical Cascade Operations}

\begin{definition}[Hierarchy Operators]
\begin{verbatim}
-- Hierarchical depth
DEPTH(fluxes : Array[N] of Flux,
      threshold : Flux) : Real[0,1]
    D = (1/N) * sum_i indicator(F_i > threshold)

-- Information compression
COMPRESS(flux_in, flux_out : Array[N] of Flux,
         alpha : Array[N] of Real) : Information
    I = sum_i alpha_i * log2(flux_in_i / flux_out_i)

-- Multi-scale coupling
CASCADE(levels : Array[N] of Level) : HierarchicalState
    -- Couples adjacent levels through variance minimization

-- Depth transition (critical at D ≈ 0.4)
TRANSITION(state : CircuitState) : CircuitRegime
    if D > 0.6 then Hierarchical
    else if R > 0.8 then Coherent
    else if R < 0.3 then Turbulent
    else ApertureDominated
\end{verbatim}
\end{definition}

\subsection{Thermodynamic Operations}

\begin{definition}[Temperature Operations]
\begin{verbatim}
-- Categorical thermometry (zero backaction)
TEMPERATURE_CAT(Se_ref, Se_current : Real,
                T_0 : Temperature) : Temperature
    T = T_0 * exp(Se_current - Se_ref)
    -- Resolution: ~17 picokelvin
    -- Backaction: Delta_p = 0

-- Temperature as scaling factor
SCALE(observable : Dimensionless,
      T : Temperature) : Energy
    = k_B * T * observable

-- Universal equation of state
EOS(P : Pressure, V : Volume, N : Nat,
    T : Temperature,
    structure : PartitionGeometry) : Constraint
    P * V = N * k_B * T * S(structure)
    -- S is temperature-independent structural factor
\end{verbatim}
\end{definition}

\subsection{Gyrometric Dynamics}

\begin{definition}[Gyrometric Operators]
\begin{verbatim}
-- Rotational quantum number dynamics
GYRO_EVOLVE(J : GyroCoord,
            lambda : AffineParameter,
            F : ExternalForce) : GyroCoord
    -- d²J_i/dlambda² = -omega²_Ji * (J_i - J_eq,i)
    --                  - sum_j gamma_ij * dJ_j/dlambda + F_i
    -- Damped, driven oscillation in rotational state space

-- Gyrometric equilibrium
GYRO_EQ(omega : Array[N] of Real) : GyroCoord
    J_eq where d²J/dlambda² = 0

-- Angular momentum coupling
COUPLE_J(J_1, J_2 : GyroCoord,
         coupling : CouplingMatrix) : GyroCoord
\end{verbatim}
\end{definition}

\subsection{Measurement Operations}

\begin{definition}[Quintupartite + Categorical Thermometry]
\begin{verbatim}
-- Six-modality measurement framework
type MeasurementModality =
    | Optical(ambiguity: ~10⁶⁰)
    | Spectral(ambiguity: ~10⁴⁵)
    | Vibrational(ambiguity: ~10³⁰)
    | MetabolicGPS(ambiguity: ~10¹⁵)
    | TemporalCausal(ambiguity: ~10⁰)
    | CategoricalThermo(exclusion: ~10⁻³)

-- Sequential exclusion
MEASURE(state : CircuitState,
        modalities : List[MeasurementModality]) : UniqueState
    -- Each modality reduces ambiguity by epsilon_i ~ 10⁻¹⁵
    -- N_0 ~ 10⁶⁰ -> N_6 = 1 unique determination
    -- Effective resolution: delta_x ~ 0.08 nm

-- Zero-backaction categorical observation
OBSERVE_CAT(state : PartitionState) : PartitionState
    -- [O_cat, O_phys] = 0 (commutes with physical observables)
    -- Measures WHERE electron IS NOT (exhaustive exclusion)
\end{verbatim}
\end{definition}

\subsection{Neural-Specific Operations}

\begin{definition}[Consciousness Operations]
\begin{verbatim}
-- Consciousness as intersection
CONSCIOUSNESS(P_decay : DecayCurve,
              T_decay : DecayCurve) : ConsciousnessState
    C = P_decay ∩ T_decay
    -- Exists where/when perception constrains thought

-- Decay curve evolution
DECAY_EVOLVE(curve : DecayCurve,
             tau : TimeConstant,
             input : Optional[InputRate]) : DecayCurve
    d(curve)/dt = -curve/tau + input

-- Consciousness frequency
OMEGA_C(omega_T, omega_P : Frequency) : Frequency
    = (omega_T * omega_P) / (omega_T + omega_P)
    -- ~2.5 Hz
\end{verbatim}
\end{definition}

\begin{definition}[Memory Operations]
\begin{verbatim}
-- Memory as accumulated change
MEMORY(H_field : EmotionalField,
       t_start, t_end : Time) : Memory
    M = integral(dH/dt, t_start, t_end)
    -- "How the field got here"

-- Temporal differentiation
DISTINGUISH(thought : Thought,
            H : EmotionalField,
            M_1, M_2 : Memory) : Bool
    -- (T, H, M_1) ≠ (T, H, M_2) even if T, H identical

-- Memory retrieval
RETRIEVE(M : Memory,
         H_now : EmotionalField,
         t_0 : Time) : EmotionalField
    H(t_0) = H_now - (M(now) - M(t_0))

-- Prediction from memory
PREDICT(M : Memory,
        H_now : EmotionalField,
        delta_t : Duration) : EmotionalField
    H(t + delta_t) ≈ H_now + (dM/dt) * delta_t
\end{verbatim}
\end{definition}

\begin{definition}[Dream Operations]
\begin{verbatim}
-- Dream mode (unbounded thought)
DREAM(thought : Thought) : Trajectory
    -- P_decay = 0, no perceptual constraint
    -- Trajectory navigates full S-space
    domain: S_N - S_perceptible

-- Wake mode (constrained thought)
WAKE(thought : Thought,
     perception : Perception) : Trajectory
    -- P_decay > 0, perceptual constraint active
    domain: S_N ∩ S_perceptible
\end{verbatim}
\end{definition}

\subsection{Charge-Circuit Integration}

\begin{definition}[Charge Operations]
\begin{verbatim}
-- Charge conservation (no ground)
CONSERVE(rho : ChargeDistribution) : Constraint
    integral(rho, V) = Q_total = const

-- Autocatalytic redistribution
REDISTRIBUTE(rho : ChargeDistribution,
             imbalance : LocalImbalance) : ChargeDistribution
    -- Variance minimization drives redistribution
    -- Creates new imbalance -> cycle continues

-- Soul-consciousness coupling
COUPLE_SC(rho : ChargeDistribution,  -- soul
          C : ConsciousnessState)     -- consciousness
    : UnifiedState
    -- Same non-grounded circuit at different levels
\end{verbatim}
\end{definition}

\subsection{Program Structure}

\begin{definition}[NPL Program]
\begin{verbatim}
program VirtualBrainProcess {
    -- Declare target state (completion condition)
    target: PartitionState

    -- Declare constraints
    constraints: List[Constraint]

    -- Initial state (optional - determined by backward completion)
    initial: Optional[MentalState]

    -- The program body specifies WHAT, not HOW
    body: CompletionSpecification

    -- Runtime navigates S-space to find trajectory
    -- satisfying target + constraints
}

-- Example: Conscious perception process
program ConsciousPerception {
    target: Gamma_f where sufficient(perception)

    constraints: [
        P_decay > 0,           -- perception active
        T_decay > 0,           -- thought active
        C = P_decay ∩ T_decay, -- consciousness exists
        M = integral(dH/dt)    -- memory accumulates
    ]

    body: COMPLETE(constraints, target)

    returns: (gamma, Gamma_f, M)  -- mental state triple
}
\end{verbatim}
\end{definition}

\subsection{Execution Model}

\begin{theorem}[NPL Execution]
NPL programs execute through:
\begin{enumerate}
\item \textbf{Target specification}: Program declares completion condition $\Gstate_f$
\item \textbf{Constraint collection}: All constraints accumulated
\item \textbf{Backward navigation}: Runtime navigates S-space backward from target
\item \textbf{Trajectory determination}: Unique trajectory $\gamma$ satisfying constraints is found
\item \textbf{State extraction}: Mental state $\Mstate = (\gamma, \Gstate_f, M)$ returned
\end{enumerate}
Complexity: $O(\log_3 n)$ via ternary trisection.
\end{theorem}

\begin{theorem}[Computational Universality]
NPL achieves computational universality through:
\begin{enumerate}
\item \textbf{Controllability}: Arbitrary state transformations via aperture modulation
\item \textbf{Memory persistence}: Phase-locked states stable against thermal fluctuations
\item \textbf{Conditional operations}: Phase threshold dynamics
\item \textbf{Hierarchical composability}: Multi-scale coupling
\end{enumerate}
\end{theorem}

\subsection{Efficiency Properties}

\begin{theorem}[NPL Efficiency]
\begin{enumerate}
\item \textbf{Ternary advantage}: 37\% fewer iterations than binary search
\item \textbf{Categorical exclusion}: $10^{60} \to 1$ unique determination
\item \textbf{Aperture catalysis}: $10^{38}$ reduction factor
\item \textbf{Hierarchical compression}: $10^{44} \to 10^6$ configurations
\item \textbf{Overall efficiency}: $10^{22}$ improvement over explicit enumeration
\end{enumerate}
\end{theorem}

\section{Consciousness as Decay Curve Intersection}
\label{sec:consciousness_detailed}

\subsection{Fundamental Architecture}

The central insight of this framework is that consciousness is not a substance, location, or emergent property---it is a \emph{geometric intersection}. Specifically:

\begin{theorem}[Consciousness Intersection]
Consciousness $\Ccat$ occurs at the intersection of the perception decay curve $\Pdecay(t)$ and the thought decay curve $\Tdecay(t)$:
\begin{equation}
\Ccat(t) = \Pdecay(t) \cap \Tdecay(t)
\end{equation}
\end{theorem}

\subsection{Why Intersection?}

Consider what happens without intersection:

\textbf{Thought without Perception} ($\Pdecay = 0$, $\Tdecay > 0$):
\begin{itemize}
\item Thoughts evolve freely through S-entropy space
\item No external constraint on trajectory
\item This IS dreaming (Section~\ref{sec:dreams})
\item Not conscious in the waking sense
\end{itemize}

\textbf{Perception without Thought} ($\Pdecay > 0$, $\Tdecay = 0$):
\begin{itemize}
\item Sensory input arrives but no cognitive processing
\item Pure stimulus-response (reflexive)
\item Not conscious---automatic reaction
\end{itemize}

\textbf{Intersection} ($\Pdecay > 0$ AND $\Tdecay > 0$):
\begin{itemize}
\item Perception constrains thought trajectories
\item Thoughts interpret perceptual input
\item The constraint relationship IS consciousness
\end{itemize}

\subsection{Mathematical Structure}

\begin{definition}[Decay Curve Product Space]
Define the decay product space:
\begin{equation}
\mathcal{D} = \Pdecay \times \Tdecay \subset [0,1]^2
\end{equation}
Consciousness lives on the diagonal where both are non-zero.
\end{definition}

\begin{theorem}[Consciousness Measure]
The ``amount'' of consciousness at time $t$ is:
\begin{equation}
|\Ccat(t)| = \min(\Pdecay(t), \Tdecay(t))
\end{equation}
Consciousness is limited by the weaker of perception or thought.
\end{theorem}

\subsection{Dynamics of the Intersection}

The decay curves evolve according to:

\begin{align}
\frac{d\Pdecay}{dt} &= -\frac{\Pdecay}{\tau_P} + I_{\text{sensory}}(t) \\
\frac{d\Tdecay}{dt} &= -\frac{\Tdecay}{\tau_T} + \Tdecay^{\infty} + \Hfield(t)
\end{align}

where:
\begin{itemize}
\item $\tau_P \sim 50$ ms is the perceptual integration time
\item $\tau_T \sim 100$ ms is the thought decay time
\item $I_{\text{sensory}}$ is sensory input rate
\item $\Tdecay^{\infty}$ is background thought activity
\item $\Hfield$ is the emotional field driving thought
\end{itemize}

\subsection{The Consciousness Window}

\begin{definition}[Consciousness Window]
The consciousness window is the temporal region where both decay curves exceed threshold $\theta$:
\begin{equation}
W_C = \{t : \Pdecay(t) > \theta \land \Tdecay(t) > \theta\}
\end{equation}
\end{definition}

\begin{theorem}[Window Duration]
The consciousness window has characteristic duration:
\begin{equation}
\Delta t_C = \frac{\tau_P \tau_T}{\tau_P + \tau_T} \ln\left(\frac{\Pdecay(0) \Tdecay(0)}{\theta^2}\right)
\end{equation}
For typical values, $\Delta t_C \sim 100$--$500$ ms.
\end{theorem}

\subsection{Distinction from Thought Geometry}

The geometric properties of thought paper established that thoughts are measurable geometric objects---O$_2$ configurations with 30-dimensional oscillatory signatures. But that paper explicitly noted:

\begin{quote}
``The geometric framework characterises \emph{thoughts}---specific O$_2$ molecular arrangements around electron-stabilised holes---but does \textbf{not} explain \emph{consciousness}.''
\end{quote}

Here we complete the picture: consciousness is not the thoughts themselves but the \emph{intersection of thought evolution with perceptual constraint}. Thoughts are the objects; consciousness is the relationship between those objects and external reality.

\subsection{Self-Reference and the Meta-Level}

The consciousness intersection enables self-reference because:

\begin{enumerate}
\item Perception can include perception of one's own body
\item Thought can include thoughts about one's own thoughts
\item The intersection creates a feedback loop
\end{enumerate}

\begin{equation}
\Ccat_{\text{self}} = \Pdecay(\text{self}) \cap \Tdecay(\text{self-thought})
\end{equation}

This is how consciousness becomes aware of itself.

\section{Sensory Sufficiency Principle}
\label{sec:sufficiency_detailed}

\subsection{The Core Insight}

Traditional theories of perception assume that sensory systems aim for \emph{complete} information about stimuli---as much detail as possible. We establish the opposite: perception aims for \emph{sufficient} information for categorical completion.

\begin{axiom}[Sensory Sufficiency]
Perception provides sufficient information for behavioral completion, not complete information about stimuli.
\end{axiom}

\subsection{The Lion Example}

Consider three observers at a zoo when a lion escapes:

\textbf{Observer A}: Trained zoologist
\begin{itemize}
\item Perceives: ``Male African lion, Panthera leo, approximately 5 years old, 180 kg''
\item Categorical completion: $\Gstate_A = \text{``dangerous predator''}$
\item Action: Run
\end{itemize}

\textbf{Observer B}: Child who only knows ``cats''
\begin{itemize}
\item Perceives: ``Big cat, looks scary''
\item Categorical completion: $\Gstate_B = \text{``dangerous animal''}$
\item Action: Run
\end{itemize}

\textbf{Observer C}: Looking at phone, doesn't see lion
\begin{itemize}
\item Perceives: ``My companions are running''
\item Categorical completion: $\Gstate_C = \text{``danger (unspecified)''}$
\item Action: Run
\end{itemize}

\begin{theorem}[Completion Equivalence]
All three perceptual states resolve to equivalent categorical completions:
\begin{equation}
\Gstate_A \equiv \Gstate_B \equiv \Gstate_C = \text{``flee''}
\end{equation}
despite vastly different perceptual content.
\end{theorem}

\subsection{Mathematical Formalization}

\begin{definition}[Sufficiency Map]
The sufficiency map $\Sigma$ takes perceptual states to behavioral completions:
\begin{equation}
\Sigma: \mathcal{P}_{\text{all}} \to \mathcal{B}_{\text{complete}}
\end{equation}
where $\mathcal{P}_{\text{all}}$ is the space of all possible percepts and $\mathcal{B}_{\text{complete}}$ is the space of behavioral completions.
\end{definition}

\begin{theorem}[Many-to-One Mapping]
The sufficiency map is many-to-one:
\begin{equation}
|\Sigma^{-1}(\Gstate)| \gg 1 \quad \forall \Gstate \in \mathcal{B}
\end{equation}
Many perceptual states map to each behavioral completion.
\end{theorem}

\subsection{Modality Equivalence}

Different sensory modalities can achieve identical categorical completions.

\begin{example}[Hot Stove Detection]
Consider detecting a hot stove:

\textbf{Visual}: See glowing red element
\begin{equation}
\Pcat_{\text{visual}} \xrightarrow{\Sigma} \Gstate_{\text{heat-danger}}
\end{equation}

\textbf{Tactile}: Touch hot surface
\begin{equation}
\Pcat_{\text{tactile}} \xrightarrow{\Sigma} \Gstate_{\text{heat-danger}}
\end{equation}

\textbf{Chemical}: Smell burning
\begin{equation}
\Pcat_{\text{olfactory}} \xrightarrow{\Sigma} \Gstate_{\text{heat-danger}}
\end{equation}

All modalities achieve identical completion.
\end{example}

\begin{theorem}[Modality Independence]
For any categorical completion $\Gstate$, there exist percepts from multiple modalities mapping to it:
\begin{equation}
\exists \, \Pcat_{\text{vis}}, \Pcat_{\text{tac}}, \Pcat_{\text{aud}}, \Pcat_{\text{olf}}: \Sigma(\Pcat_i) = \Gstate \quad \forall i
\end{equation}
\end{theorem}

\subsection{Implications for Drug Effects}

This has profound implications for neuropharmacology:

\begin{corollary}[Drug-Modality Equivalence]
A drug that mimics the categorical completion of a sensory percept is perceptually equivalent to that percept:
\begin{equation}
\Sigma(\Pcat_{\text{drug}}) = \Sigma(\Pcat_{\text{natural}}) \implies \text{equivalent experience}
\end{equation}
\end{corollary}

\begin{example}[Capsaicin]
Capsaicin activates heat receptors without actual heat:
\begin{equation}
\Pcat_{\text{capsaicin}} \xrightarrow{\Sigma} \Gstate_{\text{heat}} = \Sigma(\Pcat_{\text{actual-heat}})
\end{equation}
The brain cannot distinguish because sufficiency, not completeness, determines perception.
\end{example}

\subsection{The Evolutionary Rationale}

Why would evolution select for sufficiency over completeness?

\begin{enumerate}
\item \textbf{Speed}: Sufficient information enables faster response
\item \textbf{Efficiency}: Less neural resource required
\item \textbf{Robustness}: Multiple paths to same completion
\item \textbf{Flexibility}: Novel stimuli can still trigger appropriate responses
\end{enumerate}

\begin{theorem}[Sufficiency Optimality]
Under time pressure and resource constraints, sufficiency-based perception is evolutionarily optimal:
\begin{equation}
\max_{\Sigma} \frac{\text{Survival benefit}}{\text{Processing cost} \times \text{Response time}}
\end{equation}
\end{theorem}

\subsection{Connection to Categorical Distance}

From the cellular observation equations framework, categorical distance is independent of spatial distance. Similarly:

\begin{theorem}[Perceptual Categorical Distance]
The categorical distance between a percept and its completion is independent of sensory modality:
\begin{equation}
d_{\text{cat}}(\Pcat_{\text{vis}}, \Gstate) = d_{\text{cat}}(\Pcat_{\text{tac}}, \Gstate) = d_{\text{cat}}(\Pcat_{\text{drug}}, \Gstate)
\end{equation}
All pathways to the same completion have equal categorical distance.
\end{theorem}

\section{Dreams as Unbounded Thought and Memory as Temporal Differentiation}
\label{sec:dreams_memory_detailed}

\subsection{The Unbounded Thought State}

\begin{definition}[Dream State]
A dream state occurs when the perception decay curve reaches zero while thought dynamics persist:
\begin{equation}
\text{Dream}: \Pdecay(t) = 0, \quad \Tdecay(t) > 0
\end{equation}
\end{definition}

\subsection{Why Dreams Are ``Crazy''}

\begin{theorem}[Dream Trajectory Freedom]
In dream states, thought trajectories are unconstrained:
\begin{equation}
\gamma_{\text{dream}}: [0, T] \to \Sspace_N \quad \text{(no constraints)}
\end{equation}
compared to waking:
\begin{equation}
\gamma_{\text{wake}}: [0, T] \to \Sspace_N \cap \Sspace_{\text{perceptible}}
\end{equation}
\end{theorem}

\begin{corollary}[Impossible Trajectories]
Dreams can navigate trajectories that would be impossible during waking:
\begin{enumerate}
\item Physical impossibilities (flying, teleportation)
\item Logical inconsistencies (person is both X and not-X)
\item Temporal violations (past and present intermixed)
\item Identity fluidity (self becomes other)
\end{enumerate}
\end{corollary}

\subsection{The Function of Dreams}

\begin{theorem}[Dream Exploration]
Dreams serve to explore the full trajectory space $\Sspace_N$:
\begin{equation}
\text{Dream space} = \Sspace_N - \Sspace_{\text{perceptible}}
\end{equation}
This is the complement of waking-accessible states.
\end{theorem}

\begin{proposition}[Dream-Reality Comparison]
The brain uses dreams to:
\begin{enumerate}
\item Generate ``impossible'' thought configurations
\item Compare these with waking reality
\item Calibrate the boundary between possible and impossible
\end{enumerate}
\begin{equation}
\text{Calibration} = \|\gamma_{\text{dream}}\| - \|\gamma_{\text{wake}}\|
\end{equation}
\end{proposition}

\section{The Necessity of Memory}

\subsection{Why Memory Must Exist}

\begin{theorem}[Memory Necessity from Time Progression]
Memory is a necessary consequence of:
\begin{enumerate}
\item Time progresses---reality changes
\item Emotions (H$^+$ field) summarize the unperceivable part of reality
\item Emotions must change to track changing reality
\item But: emotions are already trying to ``predict'' what they cannot perceive
\item Therefore: a record of emotional change is required
\end{enumerate}
\end{theorem}

\begin{proposition}[The Prediction Problem]
Emotions summarize reality at $10^{13}$ Hz---too fast to consciously perceive. Reality keeps changing. Emotions must update. But in what direction? This requires predicting the unknown. The only available information is: how emotions have changed so far.
\end{proposition}

\subsection{Memory as Accumulated Emotional Change}

\begin{definition}[Memory]
Memory is the accumulated change in the emotional field:
\begin{equation}
\boxed{M(t) = \int_0^t \frac{d\Hfield}{d\tau} \, d\tau = \int_0^t \dot{\Hfield} \, d\tau}
\end{equation}
\end{definition}

\begin{theorem}[Memory Structure]
Memory is NOT ``past emotions'' but the \emph{trajectory of emotional change}:
\begin{equation}
M = \text{``How the field GOT HERE''}
\end{equation}
This encodes:
\begin{enumerate}
\item The direction of emotional evolution
\item The rate of change over time
\item The path through emotional space (not just the endpoint)
\end{enumerate}
\end{theorem}

\subsection{Temporal Differentiation Within Same Context}

\begin{theorem}[The Core Function of Memory]
Memory exists to distinguish thoughts that occur within the \emph{same} emotional context at \emph{different} times.
\end{theorem}

\begin{proof}
Consider thought $\Tcat$ occurring twice:
\begin{itemize}
\item At $t_1$: $(\Tcat, \Hfield(t_1))$
\item At $t_2$: $(\Tcat, \Hfield(t_2))$
\end{itemize}

Case 1: $\Hfield(t_1) \neq \Hfield(t_2)$ --- Distinguished by emotional context alone.

Case 2: $\Hfield(t_1) = \Hfield(t_2)$ --- Same thought, same emotional state. Without memory, these are \emph{indistinguishable}.

But with memory:
\begin{itemize}
\item At $t_1$: $(\Tcat, \Hfield, M(t_1))$
\item At $t_2$: $(\Tcat, \Hfield, M(t_2))$
\end{itemize}

Since $M(t_2) = M(t_1) + \int_{t_1}^{t_2} \dot{\Hfield} \, d\tau \neq M(t_1)$ (time passed, emotional field evolved), the two instances are distinguishable.
\end{proof}

\subsection{Prediction from Memory}

\begin{theorem}[Predictive Function]
Memory enables prediction of emotional direction by extrapolating the derivative:
\begin{equation}
\Hfield(t + \Delta t) \approx \Hfield(t) + \frac{dM}{dt} \cdot \Delta t
\end{equation}
Since $\frac{dM}{dt} = \frac{d\Hfield}{dt}$, the memory trajectory encodes the predictive signal for where emotions are \emph{going}.
\end{theorem}

\begin{corollary}[Emotional Prediction]
The brain can predict the direction of emotional change by examining the recent history encoded in $M$:
\begin{equation}
\text{Predicted direction} = \text{sign}\left(\frac{dM}{dt}\right) = \text{sign}\left(\dot{\Hfield}\right)
\end{equation}
\end{corollary}

\subsection{Memory Retrieval}

\begin{theorem}[Retrieval by Inversion]
Memory retrieval reconstructs past emotional states by inverting the accumulated change:
\begin{equation}
\Hfield(t_0) = \Hfield(t) - \int_{t_0}^{t} \dot{\Hfield} \, d\tau = \Hfield(t) - [M(t) - M(t_0)]
\end{equation}
\end{theorem}

\begin{corollary}[Context-Dependent Recall]
Retrieval depends on current emotional state $\Hfield(t)$:
\begin{equation}
\text{Recall}(t_0; \Hfield_{\text{now}}) = \Hfield_{\text{now}} - \Delta M
\end{equation}
The same memory recalled in different emotional contexts produces different reconstructions.
\end{corollary}

\subsection{Dreams, Memory, and the Field}

\begin{theorem}[Dream-Memory Relationship]
During dreams, with perception $\Pdecay = 0$ (no reality input):
\begin{itemize}
\item Thoughts are unbounded
\item The emotional field $\Hfield$ continues to evolve
\item Memory $M$ continues to accumulate: $\dot{M} = \dot{\Hfield} \neq 0$
\end{itemize}
Dreams contribute to memory even though perception is absent.
\end{theorem}

\begin{corollary}[Dream Content]
Dream content reflects the emotional trajectory:
\begin{equation}
\text{Dream content} \propto \frac{dM}{dt} = \dot{\Hfield}
\end{equation}
Emotionally significant periods (high $|\dot{\Hfield}|$) generate more memorable dreams.
\end{corollary}

\subsection{Memory Consolidation}

\begin{theorem}[Sleep and Memory]
Sleep enables memory consolidation by:
\begin{enumerate}
\item Removing perceptual constraint ($\Pdecay = 0$)
\item Allowing emotional field to ``relax'' toward equilibrium
\item Integrating the day's emotional changes into stable memory
\end{enumerate}
\begin{equation}
M_{\text{consolidated}} = M_{\text{pre-sleep}} + \int_{\text{sleep}} \dot{\Hfield}_{\text{relaxation}} \, dt
\end{equation}
\end{theorem}

\subsection{Forgetting}

\begin{definition}[Forgetting]
Forgetting occurs when the accumulated change $M$ loses resolution---fine temporal distinctions blur:
\begin{equation}
\text{Forgetting}: \frac{\partial^2 M}{\partial t^2} \to 0 \quad \text{(derivative information lost)}
\end{equation}
\end{definition}

\begin{theorem}[Forgetting Dynamics]
Forgetting rate depends on emotional significance:
\begin{equation}
\frac{d(\text{resolution})}{dt} = -\frac{1}{\tau_{\text{forget}}} + |\dot{\Hfield}|
\end{equation}
High emotional change ($|\dot{\Hfield}|$ large) maintains memory resolution; low emotional change leads to forgetting.
\end{theorem}

\subsection{The Complete Picture}

\begin{align}
\text{Waking} &: \Pdecay > 0, \, \Ccat = \Pdecay \cap \Tdecay, \, \dot{M} = \dot{\Hfield} \\
\text{Dreaming} &: \Pdecay = 0, \, \Ccat = \emptyset, \, \dot{M} = \dot{\Hfield}_{\text{relaxation}} \\
\text{Memory} &: M = \int \dot{\Hfield} \, dt \quad \text{(how field got here)} \\
\text{Mental State} &: \Mstate = (\gamma, \Gstate_f, M) \\
\text{Prediction} &: \Hfield(t+\Delta t) \approx \Hfield(t) + \dot{M} \cdot \Delta t
\end{align}

Memory is not storage of the past---it is the \emph{trajectory through emotional space} that allows the mind to know where it is in time.

\section{Trajectory-Based Mental State Identification}
\label{sec:trajectory_detailed}

\subsection{The Poincar\'{e} Computing Paradigm}

Traditional neuroscience asks: ``Given initial conditions, what state will the brain reach?'' This is forward simulation from initial conditions.

The Poincar\'{e} computing paradigm inverts this: ``Given a completion condition, what trajectory must have been taken?'' This is backward constraint satisfaction.

\begin{principle}[Backward Determination]
Specifying the completion state determines what the trajectory MUST have been:
\begin{equation}
\Gstate_f \implies \gamma^* \quad \text{(unique trajectory)}
\end{equation}
\end{principle}

\subsection{Mental States as Trajectory-Terminus Pairs}

\begin{theorem}[Mental State Structure]
A mental state is not merely a state---it is a trajectory-terminus pair:
\begin{equation}
\boxed{\Mstate = (\gamma, \Gstate_f)}
\end{equation}
where:
\begin{itemize}
\item $\gamma: [0, T] \to \Sspace_N$ is the trajectory through S-entropy space
\item $\Gstate_f = \gamma(T)$ is the final (terminus) state
\end{itemize}
\end{theorem}

\subsection{Why Trajectory Matters}

\begin{example}[Same Thought, Different States]
Consider the thought ``cup on chair'':
\begin{itemize}
\item While drunk at a party
\item During an exam about furniture
\item While searching for lost keys
\item While reminiscing about grandmother's house
\end{itemize}

The thought \emph{content} is identical: $\Tcat_{\text{cup-chair}}$

But the \emph{mental states} differ:
\begin{align}
\Mstate_{\text{drunk}} &= (\gamma_{\text{party}}, \Gstate_{\text{cup-chair}}) \\
\Mstate_{\text{exam}} &= (\gamma_{\text{academic}}, \Gstate_{\text{cup-chair}}) \\
\Mstate_{\text{search}} &= (\gamma_{\text{goal-directed}}, \Gstate_{\text{cup-chair}}) \\
\Mstate_{\text{memory}} &= (\gamma_{\text{nostalgic}}, \Gstate_{\text{cup-chair}})
\end{align}
\end{example}

\begin{theorem}[Mental State Non-Identity]
Identical terminus states with different trajectories are different mental states:
\begin{equation}
\gamma_1 \neq \gamma_2 \implies \Mstate_1 \neq \Mstate_2 \quad \text{even if } \Gstate_{f,1} = \Gstate_{f,2}
\end{equation}
\end{theorem}

\subsection{Trajectory as Context}

The trajectory encodes:
\begin{enumerate}
\item \textbf{Emotional context}: How $\Hfield$ evolved during approach
\item \textbf{Cognitive history}: What thoughts preceded this one
\item \textbf{Perceptual pathway}: What sensory modalities were active
\item \textbf{Goal structure}: What completion was being sought
\end{enumerate}

\begin{definition}[Trajectory Context]
The trajectory context is the integral of state evolution:
\begin{equation}
\text{Context}(\gamma) = \int_0^T \frac{d\gamma}{dt} \otimes \Hfield(t) \, dt
\end{equation}
\end{definition}

\subsection{The Ternary Trisection Structure}

From the trajectory computing framework, trajectories in S-entropy space have ternary structure:

\begin{theorem}[Trajectory Address]
Each trajectory corresponds to a ternary address:
\begin{equation}
\gamma \leftrightarrow (a_1, a_2, \ldots, a_k) \quad \text{where } a_i \in \{0, 1, 2\}
\end{equation}
The address IS the path (trajectory-position identity).
\end{theorem}

\begin{corollary}[O($\log n$) Navigation]
Navigation through mental state space has complexity:
\begin{equation}
\text{Complexity} = O(\log_3 n)
\end{equation}
where $n$ is the number of distinct states. This is 37\% more efficient than binary.
\end{corollary}

\subsection{Penultimate State Principle}

\begin{principle}[Penultimate Determination]
To identify a mental state, determine the penultimate state from which it was reached:
\begin{equation}
\Mstate = (\gamma, \Gstate_f) \iff \Gstate_{f-1} \xrightarrow{\delta\gamma} \Gstate_f
\end{equation}
\end{principle}

This is the key inversion: instead of asking ``where will this go?'' (forward), ask ``where did this come from?'' (backward).

\begin{theorem}[Backward Uniqueness]
Given $\Gstate_f$ and the constraint structure, the penultimate state $\Gstate_{f-1}$ is unique:
\begin{equation}
\Gstate_f + \text{constraints} \implies \Gstate_{f-1} \quad \text{(unique)}
\end{equation}
\end{theorem}

\subsection{Mental State Equivalence Classes}

\begin{definition}[Trajectory Equivalence]
Two trajectories are equivalent if they reach the same terminus with equivalent contexts:
\begin{equation}
\gamma_1 \sim \gamma_2 \iff \Gstate_{f,1} = \Gstate_{f,2} \land |\text{Context}(\gamma_1) - \text{Context}(\gamma_2)| < \epsilon
\end{equation}
\end{definition}

\begin{theorem}[Equivalence Classes]
Mental states partition into equivalence classes based on:
\begin{enumerate}
\item Terminus identity: Same $\Gstate_f$
\item Context similarity: Similar trajectory integrals
\item Emotional proximity: Similar $\Hfield$ profiles
\end{enumerate}
\end{theorem}

\subsection{Implications for Neuropharmacology}

\begin{corollary}[Drug Effects as Trajectory Modifiers]
Drugs modify mental states by altering trajectories, not just termini:
\begin{equation}
\Pop_{\text{drug}}: \gamma \to \gamma' \implies \Mstate \to \Mstate'
\end{equation}
Even if the final thought is ``unchanged,'' the mental state differs because the trajectory differs.
\end{corollary}

\begin{example}[Anxiolytic Effect]
An anxiolytic doesn't change what you think, but changes HOW you get there:
\begin{align}
\gamma_{\text{anxious}} &\to \gamma_{\text{calm}} \\
\Gstate_f &= \Gstate_f \quad \text{(same thought)} \\
\Mstate_{\text{anxious}} &\neq \Mstate_{\text{calm}} \quad \text{(different mental states)}
\end{align}
\end{example}

\subsection{Complete Mental State Specification}

A complete mental state specification requires:
\begin{equation}
\Mstate = \left\{
\begin{array}{l}
\gamma: \text{trajectory through } \Sspace_N \\
\Gstate_f: \text{terminus state} \\
\Hfield: \text{emotional field} \\
\Pdecay: \text{perceptual constraint} \\
\Tdecay: \text{thought dynamics}
\end{array}
\right\}
\end{equation}

This is the fundamental unit of the Virtual Brain.


\section{Conclusion and Future Directions}
\label{sec:conclusion}

We have established the Virtual Brain computing framework in which:

\begin{enumerate}
\item \textbf{Equation output IS observation}---no external validation required
\item \textbf{Consciousness} = intersection of perception and thought decay curves
\item \textbf{Mental states} = trajectory-terminus-memory triples $(\gamma, \Gstate_f, M)$
\item \textbf{Perception} operates on sufficiency, not completeness
\item \textbf{All sensory modalities} resolve to the same categorical space
\item \textbf{Dreams} are unbounded thought trajectories (no perceptual constraint)
\item \textbf{Memory} = accumulated emotional change $\int \dot{\Hfield} \, dt$ (temporal differentiation + repetition marking)
\item \textbf{Memory is prerequisite for action}---without it, repetition is indistinguishable and action meaningless
\item \textbf{Mind-body problem dissolved}---physics causes physics, no non-physical component exists
\item \textbf{Charge-naming isomorphism}---soul (charge) and consciousness (naming) are the same non-grounded circuit at different levels
\end{enumerate}

\subsection{Philosophical Resolutions}

The framework resolves three classical problems:

\textbf{Mind-Body Problem}: There is no non-physical ``mental'' substance. Thoughts are O$_2$ configurations, emotions are H$^+$ flux, consciousness is decay curve intersection. Physics causes physics throughout.

\textbf{Qualia Problem}: ``Do we see the same red?'' is malformed. Perception is never isolated---always embedded in trajectory, emotion, memory. Sufficiency, not completeness, determines perception.

\textbf{Hard Problem}: We don't need to explain why experience ``feels like'' something. We derived that thoughts are geometric objects, consciousness is intersection, time is circuit completion. The explanatory gap closes because there was never a non-physical side to bridge.

\subsection{Ready for Drug Perturbations}

With the Virtual Brain substrate established, drug effects can now be incorporated as trajectory modifiers:
\begin{equation}
\Pop_{\mathrm{drug}}(\omega): \Mstate_1 \to \Mstate_2
\end{equation}

Each drug operates as a Biological Maxwell Demon, selectively filtering oscillatory states and modifying the decay curves $\Pdecay$ and $\Tdecay$. The Virtual Categorical Spectrometer provides the measurement framework.

\subsection{The Complete Picture}

\begin{equation}
\text{Virtual Brain} = \underbrace{\rho(\mathbf{r},t)}_{\text{Soul}} + \underbrace{\Ccat}_{\text{Consciousness}} + \underbrace{M}_{\text{Memory}} = \text{Single Physical System}
\end{equation}

The framework is complete. Drug perturbations are the next layer.

\begin{acknowledgments}
This work extends the Cellular Partition Language to neural systems, building on the observational identity theorem and categorical completion dynamics.
\end{acknowledgments}

\end{document}
