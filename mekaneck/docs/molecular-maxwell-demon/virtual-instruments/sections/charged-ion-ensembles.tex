\section{Virtual Mass Spectrometer Ensembles: Orchestrated Construction}

\subsection{Ensemble Definition and Motivation}

\begin{definition}[Virtual Mass Spectrometer Ensemble]
\label{def:virtual_ensemble}
A virtual mass spectrometer ensemble $\mathcal{E}_{\text{virtual}}$ is a coordinated collection of $N$ virtual detectors operating on the same categorical state $\mathbf{S}_{\text{cat}}$:

\begin{equation}
\mathcal{E}_{\text{virtual}} = \{\mathcal{D}_{\text{inst}_i}\}_{i=1}^N \quad \text{where all } \mathcal{D}_{\text{inst}_i} \text{ read } \mathbf{S}_{\text{cat}}
\label{eq:virtual_ensemble}
\end{equation}

with instrument types: $\{\text{inst}_i\} \subseteq \{\text{TOF, Orbitrap, FT-ICR, IMS, QQQ, qTOF, Ion Trap, \ldots}\}$.

\textbf{Key property}: All ensemble members measure \textit{simultaneously and coherently}—same molecular state, same convergence node, same instant.
\end{definition}

\begin{remark}[Why Ensembles?]
Single virtual detector provides one view of categorical state. Ensemble provides:

\begin{enumerate}
    \item \textbf{Categorical completion}: Each instrument partitions molecular space differently; intersection shrinks exponentially (Corollary \ref{cor:multi_instrument_completion})

    \item \textbf{Cross-validation}: Agreement across instruments increases confidence; disagreement flags errors

    \item \textbf{Complementary information}: TOF (fast, broad range) + Orbitrap (high resolution) + IMS (structural) = comprehensive characterization

    \item \textbf{Robustness}: If one projection fails validation, others provide fallback

    \item \textbf{Novel discovery}: Unknown compounds identified via consensus across instruments
\end{enumerate}

Physical impossibility: Cannot measure same ions with multiple instruments (destructive detection). Virtual detectors make this trivial.
\end{remark}

\subsection{Ensemble Construction Algorithm}

\begin{algorithm}[H]
\caption{Virtual Mass Spectrometer Ensemble Construction}
\label{alg:ensemble_construction}
\begin{algorithmic}[1]
\Procedure{ConstructVirtualEnsemble}{spectrum\_data, instrument\_types}
    \State \textbf{Phase 1: Hardware Harvesting}
    \State $\mathcal{H} \gets$ InitializeHardwareHarvesters()  \Comment{8-scale hierarchy}
    \For{scale $\ell \in \{0, 1, \ldots, 7\}$}
        \State $\omega_{\ell} \gets \mathcal{H}_{\ell}$. MeasureOscillation()
        \State $\phi_{\ell} \gets \mathcal{H}_{\ell}$.MeasurePhase()
    \EndFor
    \State

    \State \textbf{Phase 2: Frequency Hierarchy Construction}
    \State $\mathcal{T}_{\text{freq}} \gets$ BuildFrequencyHierarchy($\{\omega_{\ell}, \phi_{\ell}\}_{\ell=0}^7$)
    \State Compute gear ratios: $r_{ij} = \omega_i / \omega_j$ for all pairs
    \State

    \State \textbf{Phase 3: Finite Observer Deployment}
    \State $\mathcal{T}_{\text{trans}} \gets$ TranscendentObserver()
    \State $\{\mathcal{O}_{\ell}\}_{\ell=0}^7 \gets \mathcal{T}_{\text{trans}}$.DeployFiniteObservers($\mathcal{T}_{\text{freq}}$)
    \State

    \State \textbf{Phase 4: Phase-Lock Detection}
    \State $\mathcal{S}_{\text{mol}} \gets$ ConvertSpectrumToSignals(spectrum\_data)
    \State $\{\Sigma_{\ell}\}_{\ell=0}^7 \gets \mathcal{T}_{\text{trans}}$.CoordinateObservations($\mathcal{S}_{\text{mol}}$)
    \State \Comment{$\Sigma_{\ell}$ = phase-lock signatures at scale $\ell$}
    \State

    \State \textbf{Phase 5: Convergence Node Identification}
    \State $\mathcal{C}_{\text{nodes}} \gets \mathcal{T}_{\text{trans}}$.IdentifyConvergenceSites($\{\Sigma_{\ell}\}$)
    \State Select primary node: $\mathcal{C}^* \gets \arg\max_{\mathcal{C} \in \mathcal{C}_{\text{nodes}}} \rho_{\text{conv}}(\mathcal{C})$
    \State

    \State \textbf{Phase 6: Categorical State Extraction}
    \State $\mathbf{S}_{\text{cat}} \gets$ ExtractCategoricalState($\mathcal{C}^*$, $\{\Sigma_{\ell}\}$)
    \State \Comment{14-dimensional S-entropy coordinates}
    \State

    \State \textbf{Phase 7: MMD Ensemble Materialization}
    \State $\mathcal{E}_{\text{virtual}} \gets \emptyset$
    \For{inst\_type in instrument\_types}
        \State $\mathcal{P}_{\text{inst}} \gets$ GetProjectionOperator(inst\_type)
        \State $\mathcal{D}_{\text{inst}} \gets$ MaterializeVirtualDetector($\mathbf{S}_{\text{cat}}$, $\mathcal{C}^*$, $\mathcal{P}_{\text{inst}}$)
        \State $\mathcal{E}_{\text{virtual}} \gets \mathcal{E}_{\text{virtual}} \cup \{\mathcal{D}_{\text{inst}}\}$
    \EndFor
    \State

    \State \textbf{Phase 8: Parallel Measurement}
    \State $\mathcal{R}_{\text{ensemble}} \gets \emptyset$
    \For{$\mathcal{D}_{\text{inst}} \in \mathcal{E}_{\text{virtual}}$} \textbf{in parallel}
        \State $\text{Spectrum}_{\text{inst}} \gets \mathcal{D}_{\text{inst}}$.Measure($\mathbf{S}_{\text{cat}}$)
        \State $\text{Valid} \gets$ ValidateHardwareCoherence($\text{Spectrum}_{\text{inst}}$)
        \If{Valid}
            \State $\mathcal{R}_{\text{ensemble}} \gets \mathcal{R}_{\text{ensemble}} \cup \{(\text{inst\_type}, \text{Spectrum}_{\text{inst}})\}$
        \EndIf
    \EndFor
    \State

    \State \textbf{Phase 9: Ensemble Dissolution}
    \For{$\mathcal{D}_{\text{inst}} \in \mathcal{E}_{\text{virtual}}$}
        \State $\mathcal{D}_{\text{inst}} \gets \emptyset$  \Comment{Detectors dissolve}
    \EndFor
    \State

    \State \Return $\mathcal{R}_{\text{ensemble}}$
\EndProcedure
\end{algorithmic}
\end{algorithm}

\subsection{Computational Complexity Analysis}

\begin{theorem}[Ensemble Construction Complexity]
\label{thm:ensemble_complexity}
Constructing and operating a virtual mass spectrometer ensemble with $N$ instrument types for spectrum with $M$ ions has total complexity:

\begin{equation}
T_{\text{total}} = O(M \cdot K) + O(N \cdot M \cdot K)
\label{eq:ensemble_complexity}
\end{equation}

where $K = 14$ is S-entropy dimensionality. Breaking down by phase:

\begin{align}
\text{Phase 1-2 (Hardware + Hierarchy):} & \quad O(L^2) \quad \text{where } L=8 \text{ scales} = 64 \text{ ops} \\
\text{Phase 3 (Observer Deployment):} & \quad O(L) = 8 \text{ ops} \\
\text{Phase 4 (Phase-Lock Detection):} & \quad O(M \cdot L) = 8M \text{ ops} \\
\text{Phase 5 (Convergence Nodes):} & \quad O(L \log L) \approx 24 \text{ ops} \\
\text{Phase 6 (Categorical Extraction):} & \quad O(M \cdot K) = 14M \text{ ops} \\
\text{Phase 7 (Materialization):} & \quad O(N) \text{ ops} \\
\text{Phase 8 (Parallel Measurement):} & \quad O(N \cdot M \cdot K) = 14NM \text{ ops (parallelizable)} \\
\text{Phase 9 (Dissolution):} & \quad O(N) \text{ ops}
\end{align}

\textbf{Dominant term}: Phase 8 ($O(N \cdot M \cdot K)$), but parallelizable across $N$ instruments → effective $O(M \cdot K)$.
\end{theorem}

\begin{proof}
\textbf{Hardware and hierarchy} (Phases 1-2):

Measuring 8 hardware oscillations: $O(L)$ where $L=8$. Computing all pairwise gear ratios: $O(L^2) = 64$ operations. Both are constant (independent of $M$).

\textbf{Finite observers} (Phases 3-4):

Deploying observers: $O(L) = 8$ (one per scale). Each observer processes $M$ molecular signals: $O(M)$ per observer. Across $L$ observers in parallel: $O(M)$ total. With sequential execution: $O(L \cdot M) = 8M$.

\textbf{Convergence identification} (Phase 5):

Computing density at each scale: $O(L)$. Sorting by density: $O(L \log L) = 8 \log 8 \approx 24$ operations. Constant.

\textbf{Categorical extraction} (Phase 6):

For each of $M$ ions, compute 14 S-entropy coordinates: $O(M \cdot K) = 14M$ operations. This is the first $M$-dependent dominant term.

\textbf{Materialization} (Phase 7):

Creating $N$ virtual detectors: $O(N)$ operations (setting function pointers, initializing filters). For typical $N = 4$ to $8$: negligible.

\textbf{Measurement} (Phase 8):

Each of $N$ detectors processes $M$ ions with $K$ S-coordinate reads: $O(M \cdot K)$ per detector. Total: $O(N \cdot M \cdot K)$. BUT: detectors operate in parallel (Theorem \ref{thm:parallel_observation}) → wall-clock time is $O(M \cdot K)$ if $N$ cores available.

\textbf{Dissolution} (Phase 9):

Nullifying $N$ detector references: $O(N)$. Negligible.

\textbf{Total sequential}: $O(M \cdot K) + O(N \cdot M \cdot K) = O((N+1) M K)$.

\textbf{Total parallel}: $O(M \cdot K) + O(M \cdot K) = O(M \cdot K)$.

For $M = 10^3$ ions, $K = 14$, $N = 4$ instruments:
\begin{itemize}
    \item Sequential: $(4+1) \times 10^3 \times 14 = 70{,}000$ operations
    \item Parallel: $2 \times 10^3 \times 14 = 28{,}000$ operations
    \item At $10^9$ ops/sec: $\sim 28$ microseconds (parallel)
\end{itemize}

Real-time performance even for complex mixtures.

$\square$
\end{proof}

\subsection{Ensemble Coordination Mechanisms}

\begin{definition}[Ensemble Coordinator]
\label{def:ensemble_coordinator}
The ensemble coordinator $\mathcal{K}_{\text{ensemble}}$ is a meta-controller that:

\begin{enumerate}
    \item \textbf{Synchronizes materialization}: Ensures all detectors materialize at same convergence node simultaneously

    \item \textbf{Manages resource allocation}: Assigns computational resources (CPU cores, memory) to virtual detectors

    \item \textbf{Orchestrates parallel measurement}: Triggers all detectors to read $\mathbf{S}_{\text{cat}}$ in parallel

    \item \textbf{Aggregates results}: Collects spectra from all instruments, performs cross-validation

    \item \textbf{Handles failures}: If any detector fails validation, removes it from ensemble without affecting others
\end{enumerate}
\end{definition}

\begin{theorem}[Ensemble Temporal Coherence]
\label{thm:ensemble_coherence}
All virtual detectors in ensemble $\mathcal{E}_{\text{virtual}}$ measure the same categorical state $\mathbf{S}_{\text{cat}}$ captured at the same instant $t_{\text{conv}}$ (convergence node formation time), guaranteeing perfect temporal coherence:

\begin{equation}
\forall \mathcal{D}_i, \mathcal{D}_j \in \mathcal{E}_{\text{virtual}}: \quad \mathbf{S}_{\text{cat}}^{(i)}(t_{\text{conv}}) = \mathbf{S}_{\text{cat}}^{(j)}(t_{\text{conv}})
\label{eq:ensemble_coherence}
\end{equation}

No temporal drift, no sample evolution, no systematic bias between instruments.
\end{theorem}

\begin{proof}
\textbf{Single categorical state extraction}:

Algorithm \ref{alg:ensemble_construction}, Phase 6 extracts $\mathbf{S}_{\text{cat}}$ \textit{once} from convergence node $\mathcal{C}^*$. This extraction occurs at time $t_{\text{conv}}$ determined by:
\begin{itemize}
    \item Hardware oscillation measurements (Phase 1): timestamp $t_{\text{hw}}$
    \item Spectrum acquisition: timestamp $t_{\text{acq}}$
    \item Phase-lock detection: timestamp $t_{\text{lock}}$
\end{itemize}

The convergence node exists at $t_{\text{conv}} = \max(t_{\text{hw}}, t_{\text{acq}}, t_{\text{lock}})$. Once formed, $\mathcal{C}^*$ is static (categorical irreversibility).

\textbf{Shared categorical state}:

All $N$ virtual detectors receive \textit{reference} to same $\mathbf{S}_{\text{cat}}$ object in Phase 7. Not copies—the same 14-dimensional vector in memory. Therefore:
\begin{equation}
\mathbf{S}_{\text{cat}}^{(1)} \equiv \mathbf{S}_{\text{cat}}^{(2)} \equiv \cdots \equiv \mathbf{S}_{\text{cat}}^{(N)} \quad \text{(identity, not just equality)}
\end{equation}

\textbf{Parallel measurement}:

Phase 8 triggers all detectors simultaneously. Each reads the same $\mathbf{S}_{\text{cat}}$ reference. The read operation is:
\begin{itemize}
    \item \textbf{Instantaneous}: $O(1)$ memory access per coordinate
    \item \textbf{Non-destructive}: Reading doesn't modify $\mathbf{S}_{\text{cat}}$
    \item \textbf{Concurrent-safe}: Multiple threads can read same memory simultaneously (no race conditions)
\end{itemize}

\textbf{Contrast with sequential physical measurements}:

Physical MS workflow:
\begin{enumerate}
    \item Run sample on TOF at time $t_1$
    \item Week later (after scheduling, prep): run on Orbitrap at time $t_2 = t_1 + 7$ days
    \item Sample has degraded: oxidation, hydrolysis, evaporation
    \item Different sample state: $\mathbf{S}_{\text{TOF}}(t_1) \neq \mathbf{S}_{\text{Orbitrap}}(t_2)$
\end{enumerate}

Temporal incoherence is inherent to physical workflow. Virtual ensemble eliminates this by measuring at single instant $t_{\text{conv}}$.

$\square$
\end{proof}

\subsection{Result Integration and Cross-Validation}

\begin{definition}[Ensemble Cross-Validation]
\label{def:ensemble_validation}
Given ensemble results $\mathcal{R}_{\text{ensemble}} = \{(\text{inst}_i, \text{Spectrum}_i)\}_{i=1}^N$, cross-validation computes agreement score:

\begin{equation}
A_{\text{ensemble}} = \frac{1}{\binom{N}{2}} \sum_{i<j} \text{Agreement}(\text{Spectrum}_i, \text{Spectrum}_j)
\label{eq:ensemble_agreement}
\end{equation}

where:
\begin{equation}
\text{Agreement}(\text{Spec}_i, \text{Spec}_j) = \frac{|\text{Peaks}_i \cap \text{Peaks}_j|}{|\text{Peaks}_i \cup \text{Peaks}_j|}
\label{eq:spectrum_agreement}
\end{equation}

is the Jaccard similarity of detected peaks (accounting for instrumental resolution).

\textbf{Threshold for validation}: $A_{\text{ensemble}} > 0.75$ (75\% agreement across instruments).
\end{definition}

\begin{theorem}[Ensemble Agreement Guarantees Identification Correctness]
\label{thm:ensemble_correctness}
If ensemble agreement score $A_{\text{ensemble}} > 0.75$ and all instruments independently identify the same molecule $M^*$, the probability of correct identification is:

\begin{equation}
p(\text{correct} | A_{\text{ensemble}} > 0.75, \text{consensus}) > 0.999
\label{eq:ensemble_correctness_prob}
\end{equation}

(99.9\% confidence with 4+ instrument consensus).
\end{theorem}

\begin{proof}
\textbf{Independent identification errors}:

Each instrument has error probability $p_{\text{error}}^{(i)}$ for misidentification. For high-quality S-entropy matching: $p_{\text{error}}^{(i)} \approx 0.05$ (5\% false positive rate).

\textbf{Consensus requirement}:

For $N$ instruments to all identify same (incorrect) molecule $M_{\text{wrong}}$:
\begin{equation}
p(\text{all wrong}) = \prod_{i=1}^N p_{\text{error}}^{(i)} \approx (0.05)^N
\end{equation}

For $N = 4$: $p(\text{all wrong}) = (0.05)^4 = 6.25 \times 10^{-6}$.

\textbf{Agreement constraint}:

The agreement score $A_{\text{ensemble}} > 0.75$ adds additional constraint: spectra from all instruments must overlap substantially. If instruments identified different molecules, peaks would be disjoint: $A_{\text{ensemble}} \approx 0$.

Achieving $A > 0.75$ with different identifications requires instruments to misidentify the \textit{same wrong molecule} whose spectrum happens to be compatible across all instrument types. This is exponentially unlikely.

\textbf{Bayesian update}:

Prior probability of correct ID (from single instrument): $p_0 = 1 - 0.05 = 0.95$.

Posterior after $N = 4$ instrument consensus:
\begin{equation}
p(\text{correct} | \text{consensus}) = \frac{p_0^N}{p_0^N + (1-p_0)^N} = \frac{(0.95)^4}{(0.95)^4 + (0.05)^4} \approx \frac{0.8145}{0.8145 + 6.25 \times 10^{-6}} > 0.99999
\end{equation}

With agreement constraint $A > 0.75$, additional factor of $\sim 10\times$ confidence boost (empirically, Section 7).

Conservative estimate: $p(\text{correct}) > 0.999$.

$\square$
\end{proof}

\subsection{Ensemble Reconfigurability}

\begin{theorem}[Dynamic Ensemble Reconfiguration]
\label{thm:dynamic_reconfiguration}
Given categorical state $\mathbf{S}_{\text{cat}}$, the ensemble instrument composition can be changed dynamically without re-measurement:

\begin{equation}
\mathcal{E}_1 = \{\text{TOF, Orbitrap, FT-ICR}\} \xrightarrow{\text{Reconfigure}} \mathcal{E}_2 = \{\text{IMS, qTOF, QQQ}\}
\end{equation}

Both ensembles measure same $\mathbf{S}_{\text{cat}}$, but provide different projections.

\textbf{Reconfiguration time}: $O(N_{\text{new}})$ for dissolving old detectors and materializing new ones.
\end{theorem}

\begin{proof}
\textbf{Categorical state persistence}:

Once extracted (Phase 6), $\mathbf{S}_{\text{cat}}$ persists in memory. It represents the condition-independent molecular information. Changing instruments doesn't require re-extracting $\mathbf{S}_{\text{cat}}$—it's already available.

\textbf{Reconfiguration procedure}:

\begin{algorithmic}[1]
\Procedure{ReconfigureEnsemble}{$\mathbf{S}_{\text{cat}}$, $\mathcal{C}^*$, new\_instruments}
    \For{$\mathcal{D}_{\text{old}} \in \mathcal{E}_{\text{current}}$}
        \State $\mathcal{D}_{\text{old}} \gets \emptyset$  \Comment{Dissolve}
    \EndFor
    \State $\mathcal{E}_{\text{new}} \gets \emptyset$
    \For{inst\_type in new\_instruments}
        \State $\mathcal{P}_{\text{inst}} \gets$ GetProjectionOperator(inst\_type)
        \State $\mathcal{D}_{\text{new}} \gets$ MaterializeVirtualDetector($\mathbf{S}_{\text{cat}}$, $\mathcal{C}^*$, $\mathcal{P}_{\text{inst}}$)
        \State $\mathcal{E}_{\text{new}} \gets \mathcal{E}_{\text{new}} \cup \{\mathcal{D}_{\text{new}}\}$
    \EndFor
    \State \Return $\mathcal{E}_{\text{new}}$
\EndProcedure
\end{algorithmic}

Time: $O(N_{\text{old}}) + O(N_{\text{new}}) = O(N)$. For typical $N \sim 4$ to $8$: microseconds.

\textbf{Use case - Adaptive ensemble}:

\begin{enumerate}
    \item Start with fast instruments (TOF, qTOF): $\mathcal{E}_1 = \{\text{TOF, qTOF}\}$
    \item Get preliminary identification
    \item If ambiguous ($A_{\text{ensemble}} \in [0.5, 0.75]$), reconfigure to high-resolution: $\mathcal{E}_2 = \{\text{Orbitrap, FT-ICR}\}$
    \item If still ambiguous, add structural: $\mathcal{E}_3 = \mathcal{E}_2 \cup \{\text{IMS}\}$
    \item Iterate until $A_{\text{ensemble}} > 0.75$ or all available instruments exhausted
\end{enumerate}

Adaptive strategy minimizes computational cost while guaranteeing identification quality.

$\square$
\end{proof}

\subsection{Practical Implementation Considerations}

\begin{remark}[Memory Management]
Each virtual detector requires minimal memory:
\begin{itemize}
    \item S-coordinate reference: 8 bytes (pointer)
    \item Projection parameters: $\sim 64$ bytes (instrument-specific constants)
    \item Output buffer: $\sim 8M$ bytes for $M$ ions × 64 bits per value
\end{itemize}

For ensemble with $N = 8$ instruments and $M = 10^3$ ions:
\begin{equation}
\text{Memory}_{\text{ensemble}} = N \times (8 + 64 + 8 \times 10^3) = 8 \times 8{,}072 \approx 64 \text{ KB}
\end{equation}

Negligible compared to modern RAM (GB scale).
\end{remark}

\begin{remark}[Parallelization Strategy]
Ensemble measurement (Phase 8) is embarrassingly parallel:
\begin{itemize}
    \item Each detector reads same $\mathbf{S}_{\text{cat}}$ (read-only, no contention)
    \item No inter-detector communication required
    \item No synchronization barriers (except initial trigger and final collection)
\end{itemize}

Ideal for:
\begin{itemize}
    \item Multi-core CPUs: Assign one detector per core
    \item GPUs: Assign detector to GPU stream, process ions in parallel
    \item Distributed systems: Deploy detectors on different nodes, aggregate results
\end{itemize}

Scaling: Linear speedup with number of cores up to $N$ (number of instruments).
\end{remark}

\begin{remark}[Error Propagation in Ensemble]
Categorical state extraction (Phase 6) has precision $\epsilon_{\text{cat}} \sim 10^{-3}$ (Theorem \ref{thm:finite_approximation}). This error propagates through projections:

\begin{equation}
\epsilon_{\text{inst}} = \|\nabla \mathcal{P}_{\text{inst}}\| \cdot \epsilon_{\text{cat}}
\end{equation}

where $\|\nabla \mathcal{P}_{\text{inst}}\|$ is the Lipschitz constant of projection operator.

For typical MS projections: $\|\nabla \mathcal{P}\| \sim 1$ to $10$ (weakly amplifying). Therefore:
\begin{equation}
\epsilon_{\text{inst}} \sim 10^{-3} \text{ to } 10^{-2}
\end{equation}

(0.1\% to 1\% relative error per instrument).

Ensemble averaging reduces this:
\begin{equation}
\epsilon_{\text{ensemble}} \approx \frac{1}{\sqrt{N}} \epsilon_{\text{inst}} \approx \frac{10^{-2}}{\sqrt{4}} = 5 \times 10^{-3}
\end{equation}

(0.5\% relative error with 4-instrument ensemble).
\end{remark}

\subsection{Ensemble Architecture Diagram}

\begin{figure}[h]
\centering
\begin{tikzpicture}[scale=0.85]
% Categorical state (center)
\node[circle, draw, ultra thick, fill=yellow!30, minimum size=2cm, font=\small] (SCAT) at (0, 0) {
\begin{tabular}{c}
$\mathbf{S}_{\text{cat}}$\\
\tiny 14D State
\end{tabular}
};

% Virtual detectors arranged in circle
\foreach \angle/\inst/\color in {
    90/TOF/blue,
    45/Orbitrap/green,
    0/FT-ICR/red,
    315/IMS/purple,
    270/qTOF/orange,
    225/QQQ/cyan,
    180/Ion Trap/magenta,
    135/QTOF/brown
} {
    \node[rectangle, draw, thick, fill=\color!20, minimum width=1.5cm, minimum height=0.6cm, font=\tiny] (D\angle) at (\angle:4) {\inst};
    \draw[->, very thick, \color] (SCAT) -- (D\angle);
}

% Coordinator above
\node[rectangle, draw, ultra thick, fill=gray!20, minimum width=3cm, minimum height=0.8cm, font=\small] (COORD) at (0, 6) {
Ensemble Coordinator $\mathcal{K}_{\text{ensemble}}$
};

% Control arrows
\foreach \angle in {90, 45, 0, 315, 270, 225, 180, 135} {
    \draw[->, dashed, thick, gray] (COORD) to[bend right=10] (D\angle);
}

% Result aggregation
\node[rectangle, draw, thick, fill=green!10, minimum width=3cm, minimum height=0.8cm, font=\small] (RESULTS) at (0, -5) {
Ensemble Results $\mathcal{R}_{\text{ensemble}}$
};

\foreach \angle in {90, 45, 0, 315, 270, 225, 180, 135} {
    \draw[->, thick, gray] (D\angle) to[bend left=10] (RESULTS);
}

% Annotations
\node[font=\tiny, text width=2cm, align=center] at (-7, 2) {
\textbf{Parallel}\\
Measurement\\
$O(M \cdot K)$
};

\node[font=\tiny, text width=2cm, align=center] at (7, 2) {
\textbf{Same} $\mathbf{S}_{\text{cat}}$\\
Perfect\\
Coherence
};

\node[font=\tiny, text width=2.5cm, align=center] at (-7, -3) {
\textbf{Cross-Validation}\\
Agreement\\
$A > 0.75$
};

\node[font=\tiny, text width=2.5cm, align=center] at (7, -3) {
\textbf{Consensus}\\
$p(\text{correct})$\\
$> 0.999$
};

% Lifecycle phases
\node[rectangle, draw, dashed, fill=blue!5, minimum width=2.5cm, minimum height=0.6cm, font=\tiny] at (-7, 6) {
Phase 7: Materialize\\
$O(N)$
};

\node[rectangle, draw, dashed, fill=green!5, minimum width=2.5cm, minimum height=0.6cm, font=\tiny] at (-7, 5) {
Phase 8: Measure\\
$O(M \cdot K)$ parallel
};

\node[rectangle, draw, dashed, fill=red!5, minimum width=2.5cm, minimum height=0.6cm, font=\tiny] at (-7, 4) {
Phase 9: Dissolve\\
$O(N)$
};

\end{tikzpicture}
\caption{Virtual mass spectrometer ensemble architecture. Eight virtual detectors (different colors) simultaneously measure the same categorical state $\mathbf{S}_{\text{cat}}$ (yellow circle center). Ensemble coordinator $\mathcal{K}_{\text{ensemble}}$ (gray box top) orchestrates materialization, triggers parallel measurement, and aggregates results. All detectors read same state → perfect temporal coherence. Results flow to validation (green box bottom) where cross-validation ensures $>99.9\%$ identification confidence. Lifecycle annotations (left) show O(1) overhead for ensemble management.}
\label{fig:ensemble_architecture}
\end{figure}

\subsection{Summary: Ensemble as Unified Measurement System}

Virtual mass spectrometer ensembles transform mass spectrometry from single-instrument measurements to unified multi-instrument systems:

\begin{enumerate}
    \item \textbf{Simultaneous measurement}: All instruments measure same $\mathbf{S}_{\text{cat}}$ at same instant (Theorem \ref{thm:ensemble_coherence})

    \item \textbf{Efficient construction}: 9-phase algorithm with $O(M \cdot K)$ parallel complexity (Algorithm \ref{alg:ensemble_construction}, Theorem \ref{thm:ensemble_complexity})

    \item \textbf{Perfect coherence}: No temporal drift, sample degradation, or systematic bias between instruments (Proof of Theorem \ref{thm:ensemble_coherence})

    \item \textbf{Cross-validation}: Agreement score $A_{\text{ensemble}}$ with consensus guarantees $>99.9\%$ identification correctness (Theorem \ref{thm:ensemble_correctness})

    \item \textbf{Dynamic reconfiguration}: Change instrument composition in $O(N)$ time without re-measurement (Theorem \ref{thm:dynamic_reconfiguration})

    \item \textbf{Minimal overhead}: $\sim 64$ KB memory, microsecond latency, embarrassingly parallel

    \item \textbf{Error reduction}: Ensemble averaging reduces error by $1/\sqrt{N}$ (Remark on error propagation)
\end{enumerate}

This ensemble architecture—enabled by virtual detectors reading shared categorical states—achieves what is physically impossible: measuring the same molecular sample with multiple mass spectrometers simultaneously, with perfect temporal coherence, at effectively zero marginal cost per additional instrument.

The ensemble is the practical realization of the categorical completion principle (Section 3): multiple independent measurements intersecting to shrink equivalence classes exponentially, converging rapidly to unique molecular identification.
