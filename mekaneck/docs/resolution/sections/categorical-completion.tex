%==============================================================================
\section{Categorical Completion in Gas Dynamics}
\label{sec:categorical}
%==============================================================================

\subsection{Categorical State Space}

We now develop the categorical framework for describing gas configurations.

\begin{definition}[Categorical State]
\label{def:categorical_state}
A \textbf{categorical state} $C \in \catspace$ specifies not only the spatial configuration of molecules but also:
\begin{enumerate}
    \item The phase-lock network topology $\phaselockgraph$
    \item The phase relationships $\{\Delta\phi_{ij}\}$ for all locked pairs
    \item The vibrational mode occupations $\{n_{\text{vib},i}\}$
    \item The rotational state quantum numbers $\{J_i, M_i\}$
    \item The electronic configuration descriptors
\end{enumerate}
\end{definition}

\begin{remark}
A categorical state contains strictly more information than a classical phase space point $(\mathbf{q}, \mathbf{p})$. Two configurations with identical positions and momenta can occupy different categorical states if their phase relationships or network topologies differ.
\end{remark}

\begin{definition}[Categorical State Space]
\label{def:categorical_state_space}
The \textbf{categorical state space} $\catspace$ is the set of all categorical states equipped with:
\begin{enumerate}
    \item A partial order $\prec$ (the \textbf{completion order})
    \item A completion operator $\mu: \catspace \times \mathbb{R}_{\geq 0} \to \{0, 1\}$
    \item A topology $\tau$ induced by $\prec$
\end{enumerate}
\end{definition}

\begin{axiom}[Categorical Irreversibility]
\label{axiom:categorical_irreversibility}
Once a categorical state $C_i$ is occupied (completed), it cannot be re-occupied. For all $C_i \in \catspace$ and times $t_1 \leq t_2$:
\begin{equation}
\mu(C_i, t_1) = 1 \implies \mu(C_i, t_2) = 1
\label{eq:irreversibility}
\end{equation}
Any process returning to a spatially identical configuration must occupy a new categorical state $C_j$ with $C_i \prec C_j$.
\end{axiom}

\begin{proposition}[Monotonic Completion]
\label{prop:monotonic_completion}
Let $\gamma(t) = \{C \in \catspace : \mu(C, t) = 1\}$ be the set of completed states at time $t$. Then:
\begin{equation}
t_1 \leq t_2 \implies \gamma(t_1) \subseteq \gamma(t_2)
\end{equation}
The completed set grows monotonically.
\end{proposition}

\begin{proof}
Immediate from Axiom~\ref{axiom:categorical_irreversibility}: once $C \in \gamma(t_1)$, we have $\mu(C, t_1) = 1$, hence $\mu(C, t_2) = 1$, so $C \in \gamma(t_2)$. \qed
\end{proof}

\begin{figure*}[htbp]
\centering
\includegraphics[width=0.95\textwidth]{figures/arg1_temporal_triviality.png}
\caption{\textbf{Argument 1: Temporal Triviality—Any Configuration Occurs Naturally Through Thermal Fluctuations.}
\textbf{(A)} Boltzmann probability landscape showing all configurations are thermally accessible. The probability distribution $P(\text{config}) = \exp(-E/k_BT)/Z$ ensures every spatial arrangement, including ``sorted'' states, occurs naturally through fluctuations.
\textbf{(B)} Poincaré recurrence times as a function of sorting degree. Higher sorting corresponds to exponentially longer recurrence times $\tau_{\text{rec}} \sim \exp(N\Delta S)$, but all states eventually recur. The horizontal dashed line indicates laboratory timescales; even highly sorted states recur within observable time for small systems.
\textbf{(C)} Configuration space flow field showing all trajectories converge to equilibrium. The flow follows $\dot{\mathbf{q}} = -\nabla_{\mathbf{q}} F(\mathbf{q})$ where $F$ is the free energy. Red squares mark ``sorted'' configurations; yellow circles mark equilibrium. All paths lead to the central attractor, demonstrating that sorted states are unstable fixed points.
\textbf{(D)} Entropy evolution over time showing fluctuations enable access to all states. The solid black line shows total entropy $S(t) = -k_B \sum_i p_i \ln p_i$ increasing monotonically toward equilibrium (horizontal dashed line). The dotted red line marks the entropy of the ``sorted'' state. Yellow triangles indicate moments when the system spontaneously visits sorted configurations through thermal fluctuations, demonstrating temporal triviality: the demon's purported action is redundant.}
\label{fig:temporal_triviality}
\end{figure*}

\subsection{Phase-Lock Degeneracy}

\begin{theorem}[Phase-Lock Degeneracy]
\label{thm:phase_lock_degeneracy}
For a spatial configuration $\mathbf{q} = \{\mathbf{r}_1, \ldots, \mathbf{r}_N\}$, there exist multiple categorical states producing identical spatial observables. The \textbf{phase-lock degeneracy} is:
\begin{equation}
\Omega_{\text{PL}}(\mathbf{q}) = |\{C \in \catspace : \pi_{\text{spatial}}(C) = \mathbf{q}\}|
\label{eq:phase_lock_degeneracy}
\end{equation}
where $\pi_{\text{spatial}}: \catspace \to \mathbb{R}^{3N}$ is the spatial projection.
\end{theorem}

\begin{proof}
Consider two molecules at fixed positions $\mathbf{r}_1$, $\mathbf{r}_2$. The same spatial configuration can be achieved through different combinations of:
\begin{itemize}
    \item Van der Waals interaction angles: $\theta_{vdW} \in [0, 2\pi]$
    \item Dipole orientations: $(\phi_1, \phi_2) \in [0, 2\pi]^2$
    \item Vibrational phase differences: $\Delta\phi_{\text{vib}} \in [0, 2\pi]$
    \item Rotational phase offsets: $\Delta\phi_{\text{rot}} \in [0, 2\pi]$
\end{itemize}

These constitute distinct categorical states (different phase relationships) with identical spatial projection.

For $N$ molecules with $\binom{N}{2}$ pairwise interactions, each having continuous phase degrees of freedom:
\begin{equation}
\Omega_{\text{PL}}(\mathbf{q}) \sim (2\pi)^{k \cdot \binom{N}{2}}
\end{equation}
where $k$ is the number of independent phase variables per pair. For typical gases, $\Omega_{\text{PL}} \sim 10^6$ to $10^{12}$ per spatial configuration. \qed
\end{proof}

\begin{definition}[Categorical Equivalence Class]
\label{def:categorical_equivalence_class}
The \textbf{categorical equivalence class} of state $C$ under spatial observation is:
\begin{equation}
[C]_{\text{spatial}} = \{C' \in \catspace : \pi_{\text{spatial}}(C') = \pi_{\text{spatial}}(C)\}
\end{equation}
States in the same equivalence class are spatially indistinguishable but categorically distinct.
\end{definition}

\begin{corollary}[Categorical Richness]
\label{cor:categorical_richness}
The \textbf{categorical richness} of a spatial configuration is:
\begin{equation}
R(\mathbf{q}) = \log \Omega_{\text{PL}}(\mathbf{q}) = \log |[C]_{\text{spatial}}|
\end{equation}
This quantifies the information content of categorical specification beyond spatial description.
\end{corollary}

\subsection{Categorical Completion Dynamics}

\begin{definition}[Completion Rate]
\label{def:completion_rate}
The \textbf{categorical completion rate} is:
\begin{equation}
\dot{C}(t) = \frac{d|\gamma(t)|}{dt}
\label{eq:completion_rate}
\end{equation}
measuring the rate at which new categorical states are completed.
\end{definition}

\begin{proposition}[Non-Negative Completion Rate]
\label{prop:nonnegative_completion}
For all times $t$:
\begin{equation}
\dot{C}(t) \geq 0
\end{equation}
with equality only when no physical processes occur.
\end{proposition}

\begin{proof}
From Proposition~\ref{prop:monotonic_completion}, $|\gamma(t)|$ is monotonically non-decreasing, hence $\dot{C}(t) = d|\gamma(t)|/dt \geq 0$. \qed
\end{proof}

\begin{theorem}[Categorical Completion as Physical Process]
\label{thm:completion_physical}
Every physical process in a gas system corresponds to categorical completion:
\begin{equation}
\text{Process: } \mathbf{q}(t_1) \to \mathbf{q}(t_2) \quad \Longleftrightarrow \quad \text{Completion: } C(t_1) \prec C(t_2)
\end{equation}
The categorical state advances along the completion order.
\end{theorem}

\begin{proof}
Consider a gas evolving from configuration $\mathbf{q}(t_1)$ to $\mathbf{q}(t_2)$.

\textbf{Case 1: $\mathbf{q}(t_2) \neq \mathbf{q}(t_1)$ (different spatial configuration).}
The new configuration occupies categorical states not accessible from $\mathbf{q}(t_1)$. By Axiom~\ref{axiom:categorical_irreversibility}, these must be new completions: $C(t_2) \in \gamma(t_2) \setminus \gamma(t_1)$.

\textbf{Case 2: $\mathbf{q}(t_2) = \mathbf{q}(t_1)$ (same spatial configuration).}
Even with identical spatial positions, the phase relationships have evolved:
\begin{equation}
\Phi_i(t_2) - \Phi_j(t_2) \neq \Phi_i(t_1) - \Phi_j(t_1)
\end{equation}
in general. The categorical state has changed despite spatial identity.

By Axiom~\ref{axiom:categorical_irreversibility}, return to $C(t_1)$ is impossible; the system occupies a new state $C(t_2)$ with $C(t_1) \prec C(t_2)$.

In both cases, categorical position advances. \qed
\end{proof}

\subsection{Categorical Distance}

\begin{definition}[Categorical Distance]
\label{def:categorical_distance}
The \textbf{categorical distance} between states $C_i, C_j \in \catspace$ is:
\begin{equation}
d_{\catspace}(C_i, C_j) = \inf_{\text{paths } C_i \to C_j} \sum_{\text{transitions}} w(C_k \to C_{k+1})
\label{eq:categorical_distance}
\end{equation}
where the infimum is over all completion paths from $C_i$ to $C_j$, and $w(C_k \to C_{k+1})$ is the transition weight.
\end{definition}

\begin{proposition}[Metric Properties]
\label{prop:metric_properties}
The categorical distance $d_{\catspace}$ satisfies:
\begin{enumerate}
    \item Non-negativity: $d_{\catspace}(C_i, C_j) \geq 0$
    \item Identity: $d_{\catspace}(C_i, C_j) = 0 \iff C_i = C_j$
    \item Symmetry: $d_{\catspace}(C_i, C_j) = d_{\catspace}(C_j, C_i)$
    \item Triangle inequality: $d_{\catspace}(C_i, C_k) \leq d_{\catspace}(C_i, C_j) + d_{\catspace}(C_j, C_k)$
\end{enumerate}
Thus $(\catspace, d_{\catspace})$ is a metric space.
\end{proposition}

\begin{theorem}[Categorical-Physical Distance Inequivalence]
\label{thm:distance_inequivalence}
Categorical distance $d_{\catspace}$ is not a function of physical distance $d_{\text{phys}}$:
\begin{equation}
\nexists f: \mathbb{R}_{\geq 0} \to \mathbb{R}_{\geq 0} \text{ such that } d_{\catspace}(C_i, C_j) = f(d_{\text{phys}}(\mathbf{r}_i, \mathbf{r}_j))
\label{eq:distance_inequivalence}
\end{equation}
\end{theorem}

\begin{proof}
We construct explicit counterexamples.

\textbf{Counterexample 1: Categorical adjacency without physical proximity.}
Consider molecules $A$ and $B$ at positions $\mathbf{r}_A = (0, 0, 0)$ and $\mathbf{r}_B = (L, 0, 0)$ with large separation $L \gg r_{\text{lock}}$.

Through a chain of intermediate molecules:
\begin{equation}
A \leftrightarrow M_1 \leftrightarrow M_2 \leftrightarrow \cdots \leftrightarrow M_n \leftrightarrow B
\end{equation}
where each pair is phase-locked, molecules $A$ and $B$ are in the same phase-lock cluster.

Categorical distance: $d_{\catspace}(C_A, C_B) = n$ (path length through network).
Physical distance: $d_{\text{phys}}(\mathbf{r}_A, \mathbf{r}_B) = L$ (arbitrarily large).

For large $L$ and small $n$, we have $d_{\catspace} \ll d_{\text{phys}}$.

\textbf{Counterexample 2: Physical proximity without categorical adjacency.}
Consider molecules $A$ and $B$ at positions $\mathbf{r}_A = (0, 0, 0)$ and $\mathbf{r}_B = (\delta, 0, 0)$ with $\delta \to 0$.

If $A$ and $B$ belong to different phase-lock clusters (e.g., different vibrational phases that prevent locking), then:
\begin{equation}
d_{\catspace}(C_A, C_B) = \infty \quad \text{(no path between clusters)}
\end{equation}
while $d_{\text{phys}}(\mathbf{r}_A, \mathbf{r}_B) = \delta \to 0$.

Here $d_{\catspace} \gg d_{\text{phys}}$.

Since both $d_{\catspace} \ll d_{\text{phys}}$ and $d_{\catspace} \gg d_{\text{phys}}$ are achievable, no function $f$ can satisfy~\eqref{eq:distance_inequivalence}. \qed
\end{proof}

\begin{corollary}[Categorical Adjacency Determines Accessibility]
\label{cor:adjacency_accessibility}
The set of states accessible from $C_i$ through single-step transitions is:
\begin{equation}
\accessible(C_i) = \{C_j \in \catspace : d_{\catspace}(C_i, C_j) = 1\}
\end{equation}
This is determined by phase-lock network topology, not physical proximity.
\end{corollary}

