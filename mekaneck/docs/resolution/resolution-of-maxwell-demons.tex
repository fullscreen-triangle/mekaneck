\documentclass[12pt,a4paper]{article}

\usepackage[utf8]{inputenc}
\usepackage[T1]{fontenc}
\usepackage{amsmath,amssymb,amsfonts,amsthm}
\usepackage{mathtools}
\usepackage{geometry}
\usepackage{graphicx}
\usepackage{float}
\usepackage{booktabs}
\usepackage{hyperref}
\usepackage{cleveref}
\DeclareMathOperator*{\argmax}{arg\,max}
\usepackage{physics}
\usepackage{natbib}
\usepackage{tikz-cd}

\geometry{margin=1in}

% Theorem environments
\newtheorem{theorem}{Theorem}[section]
\newtheorem{lemma}[theorem]{Lemma}
\newtheorem{proposition}[theorem]{Proposition}
\newtheorem{corollary}[theorem]{Corollary}
\theoremstyle{definition}
\newtheorem{definition}[theorem]{Definition}
\newtheorem{axiom}[theorem]{Axiom}
\theoremstyle{remark}
\newtheorem{remark}[theorem]{Remark}
\newtheorem{example}[theorem]{Example}

% Custom commands
\newcommand{\phaselockgraph}{\mathcal{G}}
\newcommand{\catspace}{\mathcal{C}}
\newcommand{\accessible}{\text{Acc}}

\hypersetup{
    colorlinks=true,
    linkcolor=blue,
    citecolor=blue,
    urlcolor=blue
}

\title{\textbf{On the Resolution of Maxwell's Demon: \\[0.3em] Phase-Lock Network Topology and the Dissolution of the Sorting Paradox}}

\author{Kundai Farai Sachikonye\\
\texttt{kundai.sachikonye@wzw.tum.de}}

\date{\today}

\begin{document}

\maketitle

\begin{abstract}
We present a complete resolution of Maxwell's Demon paradox through the theory of categorical phase-lock networks. The standard formulation asks how a demon can sort molecules by kinetic energy without violating the second law of thermodynamics, with proposed resolutions invoking information-theoretic costs of measurement and memory erasure. We demonstrate that this framing is fundamentally misconceived: there is no demon because there is no sorting by kinetic energy. Gas molecules exist in phase-lock networks formed through Van der Waals forces ($\sim r^{-6}$) and dipole interactions ($\sim r^{-3}$)---interactions that depend on spatial configuration and electronic structure, not molecular velocity.

We prove six independent results that collectively dissolve the paradox: (1) \textit{Temporal triviality}: any configuration the demon purportedly creates will occur naturally through thermal fluctuations, rendering the demon redundant; (2) \textit{Phase-lock temperature independence}: the same phase-lock network (spatial arrangement and categorical structure) can exist at any temperature---a ``snapshot'' of positions is velocity-blind; (3) \textit{The retrieval paradox}: velocity-based sorting is self-defeating because thermal equilibration ($\sim 10^{10}$ collisions/s) randomises velocities faster than any sorting can occur, requiring infinite retrieval operations; (4) \textit{Phase-lock kinetic independence}: $\partial \phaselockgraph / \partial E_{\text{kin}} = 0$---network topology does not depend on molecular velocities; (5) \textit{Categorical-physical distance inequivalence}: categorical adjacency does not correspond to spatial proximity or kinetic similarity; (6) \textit{Temperature emergence}: temperature is a statistical observable of phase-lock cluster structure, not a sorting criterion.

The demon dissolves: it is temporally redundant (fluctuations produce the same result), categorically misconceived (phase-lock structure is temperature-independent), and operationally self-defeating (cannot maintain velocity-sorted states). Most fundamentally, (7) \textit{information complementarity}: information has two conjugate faces (kinetic and categorical) that cannot be simultaneously observed, analogous to ammeter/voltmeter measurement incompatibility in electrical circuits. Maxwell observed only the kinetic face; the ``demon'' was not an agent but the projection of hidden categorical dynamics onto the observable face.

What appeared to require an information-processing agent is revealed as topological navigation through categorical state space, visible only from the conjugate face. This resolution requires no information-theoretic arguments, no quantum considerations, and no appeal to measurement costs---the paradox is resolved purely through the geometry of phase-lock networks, the mathematics of categorical completion, and the recognition that Maxwell was looking at one face of a two-faced information structure.

\textbf{Keywords:} Maxwell's Demon, phase-lock networks, categorical completion, thermodynamic irreversibility, Van der Waals interactions, topological entropy, thermal fluctuations, retrieval paradox
\end{abstract}

\tableofcontents
\newpage

%==============================================================================
\section{Introduction}
%==============================================================================

\subsection{The Paradox}

In 1867, James Clerk Maxwell introduced a thought experiment that has challenged thermodynamic foundations for over 150 years \citep{maxwell1871theory}. Consider two chambers A and B containing gas at thermal equilibrium, separated by a partition with a small door controlled by ``a being whose faculties are so sharpened that he can follow every molecule in its course.'' This being---the demon---observes molecules approaching the door and selectively opens it to allow fast molecules to pass from A to B and slow molecules from B to A. After sufficient operation, chamber B contains predominantly fast (hot) molecules while chamber A contains slow (cold) molecules, creating a temperature difference from equilibrium without apparent work expenditure.

The paradox is immediate: the second law of thermodynamics prohibits spontaneous heat flow from cold to hot, yet the demon appears to achieve precisely this through information alone. The total entropy of the system appears to decrease:
\begin{equation}
\Delta S_{\text{total}} = \Delta S_A + \Delta S_B < 0
\label{eq:entropy_decrease}
\end{equation}
contradicting the fundamental requirement $\Delta S \geq 0$ for isolated systems.

\subsection{Standard Resolutions and Their Limitations}

The dominant resolution, developed through contributions by Szilard \citep{szilard1929entropieverminderung}, Brillouin \citep{brillouin1951maxwells}, Landauer \citep{landauer1961irreversibility}, and Bennett \citep{bennett1982thermodynamics}, locates the entropy cost in information processing:

\begin{enumerate}
    \item \textbf{Measurement cost}: The demon must acquire information about molecular velocities, requiring interaction with the molecules that generates entropy.

    \item \textbf{Memory erasure}: The demon's memory, after accumulating sorting decisions, must eventually be erased. Landauer's principle establishes that erasing one bit of information dissipates at least $kT \ln 2$ of heat, generating entropy $\Delta S \geq k \ln 2$ per bit.

    \item \textbf{Computational irreversibility}: Bennett showed that logically irreversible operations (including measurement with finite memory) necessarily produce entropy, offsetting any decrease from sorting.
\end{enumerate}

These resolutions, while logically consistent, suffer from a fundamental limitation: they accept the demon's operation as given and locate entropy production in ancillary processes. They answer ``how does sorting avoid violating the second law?'' rather than questioning whether sorting occurs at all.

We propose a more radical resolution: \textit{there is no demon because there is no sorting by kinetic energy}. The apparent sorting is a manifestation of categorical completion through phase-lock network topology---a process requiring no information, no measurement, and no decision-making.

\subsection{The Phase-Lock Network Perspective}

Gas molecules are not independent particles moving through empty space. They exist in networks of phase-locked oscillatory relationships mediated by:
\begin{itemize}
    \item Van der Waals forces: induced dipole-dipole interactions scaling as $U_{vdW} \propto r^{-6}$
    \item Permanent dipole interactions: scaling as $U_{\text{dipole}} \propto r^{-3}$
    \item Vibrational coupling: molecular vibrations synchronised through collisions
    \item Rotational coordination: orientational correlations through multipole moments
\end{itemize}

Crucially, \textit{none of these interactions depend on molecular kinetic energy}. Van der Waals forces depend on polarisability and separation; dipole interactions depend on molecular geometry and orientation; vibrational coupling depends on normal mode frequencies. A molecule's translational velocity---the quantity the demon supposedly measures---is irrelevant to phase-lock network formation.

This observation inverts the standard picture:
\begin{center}
\begin{tabular}{l|l}
\textbf{Standard View} & \textbf{Phase-Lock View} \\
\hline
Temperature $\to$ molecular speeds & Phase-lock topology $\to$ categorical structure \\
Demon measures velocity & No measurement needed \\
Sorting creates order & Topology reveals pre-existing structure \\
Information processing required & Categorical completion sufficient
\end{tabular}
\end{center}

\subsection{Central Claims}

This paper establishes three central results:

\begin{theorem}[Phase-Lock Kinetic Independence]
\label{thm:kinetic_independence_intro}
The phase-lock network $\phaselockgraph = (V, E)$ of a gas system satisfies:
\begin{equation}
\frac{\partial \phaselockgraph}{\partial E_{\text{kin}}} = 0
\end{equation}
Network topology is determined by spatial configuration and electronic structure, independent of molecular velocities.
\end{theorem}

\begin{theorem}[Categorical-Physical Distance Inequivalence]
\label{thm:distance_inequivalence_intro}
For categorical distance $d_{\catspace}$ and physical distance $d_{\text{phys}}$:
\begin{equation}
d_{\catspace}(C_i, C_j) \neq f(d_{\text{phys}}(\mathbf{r}_i, \mathbf{r}_j))
\end{equation}
for any function $f$. Categorical adjacency does not correspond to spatial proximity.
\end{theorem}

\begin{theorem}[Temperature Emergence]
\label{thm:temperature_emergence_intro}
Temperature $T$ emerges as a statistical property of phase-lock cluster structure:
\begin{equation}
T = \mathcal{F}[\{\phaselockgraph_\alpha\}]
\end{equation}
where $\{\phaselockgraph_\alpha\}$ denotes the ensemble of phase-lock clusters. Temperature does not determine network structure; network structure determines apparent temperature.
\end{theorem}

From these results, the resolution follows: Maxwell's Demon dissolves because the ``sorting'' it supposedly performs is categorical completion through phase-lock topology. Molecules following phase-lock adjacency relations appear sorted by temperature because phase-lock clusters correlate with---but are not caused by---kinetic properties.

\subsection{Paper Structure}

Section~\ref{sec:phase_lock} establishes the mathematical framework for phase-lock networks and proves kinetic independence. Section~\ref{sec:categorical} develops categorical completion theory in the context of gas dynamics. Section~\ref{sec:selection} analyses how categorical selection opens accessibility pathways. Section~\ref{sec:temperature} proves that temperature emerges from phase-lock statistics. Section~\ref{sec:entropy} establishes the entropy mechanism through network topology. Section~\ref{sec:dissolution} presents the complete dissolution of the demon paradox. Section~\ref{sec:conclusion} concludes with implications and experimental predictions.

%==============================================================================
% Section imports
%==============================================================================

%==============================================================================
\section{Phase-Lock Networks and Kinetic Independence}
\label{sec:phase_lock}
%==============================================================================

\subsection{Intermolecular Interactions in Gas Systems}

We begin by establishing the physical basis for phase-lock networks in gas systems. Gas molecules interact through several mechanisms, each with characteristic distance dependence.


\begin{definition}[Van der Waals Interaction]
\label{def:vdw}
The Van der Waals interaction between two molecules $i$ and $j$ separated by distance $r_{ij}$ is:
\begin{equation}
U_{vdW}(r_{ij}) = -\frac{C_6^{(ij)}}{r_{ij}^6}
\label{eq:vdw_potential}
\end{equation}
where $C_6^{(ij)}$ is the dispersion coefficient determined by molecular polarisabilities:
\begin{equation}
C_6^{(ij)} = \frac{3}{2} \frac{\alpha_i \alpha_j}{(4\pi\varepsilon_0)^2} \frac{I_i I_j}{I_i + I_j}
\label{eq:c6_coefficient}
\end{equation}
with $\alpha_i$ and $\alpha_j$ being the static polarizabilities and $I_i$ and $I_j$ the ionisation energies.
\end{definition}

\begin{definition}[Dipole-Dipole Interaction]
\label{def:dipole}
For molecules with permanent dipole moments $\boldsymbol{\mu}_i$ and $\boldsymbol{\mu}_j$, the interaction is:
\begin{equation}
U_{\text{dipole}}(r_{ij}, \theta_i, \theta_j, \phi) = -\frac{\mu_i \mu_j}{4\pi\varepsilon_0 r_{ij}^3} \left( 2\cos\theta_i \cos\theta_j - \sin\theta_i \sin\theta_j \cos\phi \right)
\label{eq:dipole_potential}
\end{equation}
where $\theta_i$, $\theta_j$ are angles between dipoles and the intermolecular axis, and $\phi$ is the dihedral angle.
\end{definition}

\begin{proposition}[Kinetic Energy Independence of Interactions]
\label{prop:kinetic_independence_interactions}
The interaction potentials $U_{vdW}$ and $U_{\text{dipole}}$ satisfy:
\begin{equation}
\frac{\partial U_{vdW}}{\partial E_{\text{kin}}} = 0, \quad \frac{\partial U_{\text{dipole}}}{\partial E_{\text{kin}}} = 0
\end{equation}
where $E_{\text{kin}} = \frac{1}{2}m|\mathbf{v}|^2$ is molecular translational kinetic energy.
\end{proposition}

\begin{proof}
From Equation~\eqref{eq:vdw_potential}, $U_{vdW}$ depends only on $r_{ij}$, $\alpha_i$, $\alpha_j$, $I_i$, $I_j$. None of these quantities involve molecular velocity $\mathbf{v}$.

The polarisability $\alpha$ is an electronic property determined by:
\begin{equation}
\alpha = \sum_n \frac{2|\langle 0 | \hat{\mathbf{d}} | n \rangle|^2}{E_n - E_0}
\end{equation}
where $|n\rangle$ are electronic states and $\hat{\mathbf{d}}$ is the dipole operator. This sum over electronic transitions is independent of nuclear translational motion.

Similarly, Equation~\eqref{eq:dipole_potential} depends on $r_{ij}$, $\mu_i$, $\mu_j$, and orientational angles---none involving translational velocity.

Therefore $\partial U / \partial E_{\text{kin}} = 0$ for both interaction types. \qed
\end{proof}

\subsection{Phase-Lock Network Construction}

\begin{definition}[Molecular Phase]
\label{def:molecular_phase}
The instantaneous phase of molecule $i$ is a composite quantity:
\begin{equation}
\Phi_i(t) = \omega_{\text{vib},i} t + \phi_{\text{vib},i} + \omega_{\text{rot},i} t + \phi_{\text{rot},i} + \Phi_{\text{elec},i}(t)
\label{eq:molecular_phase}
\end{equation}
where:
\begin{itemize}
    \item $\omega_{\text{vib},i}$, $\phi_{\text{vib},i}$: vibrational frequency and initial phase
    \item $\omega_{\text{rot},i}$, $\phi_{\text{rot},i}$: rotational frequency and initial phase
    \item $\Phi_{\text{elec},i}(t)$: electronic oscillation phase
\end{itemize}
\end{definition}

\begin{definition}[Phase-Lock Condition]
\label{def:phase_lock}
Molecules $i$ and $j$ are \textbf{phase-locked} if their phase difference remains bounded:
\begin{equation}
|\Phi_i(t) - \Phi_j(t) - \Delta\phi_{ij}| < \varepsilon \quad \forall t \in [t_0, t_0 + \tau]
\label{eq:phase_lock_condition}
\end{equation}
for some constant offset $\Delta\phi_{ij}$, threshold $\varepsilon < \pi/4$, and coherence time $\tau > \tau_{\min}$.
\end{definition}

\begin{definition}[Phase-Lock Network]
\label{def:phase_lock_network}
The \textbf{phase-lock network} of a gas system is the graph $\phaselockgraph = (V, E)$ where:
\begin{itemize}
    \item $V = \{m_1, m_2, \ldots, m_N\}$ is the set of molecules
    \item $(m_i, m_j) \in E$ if and only if molecules $i$ and $j$ satisfy the phase-lock condition~\eqref{eq:phase_lock_condition}
\end{itemize}
\end{definition}

\begin{proposition}[Phase-Lock Formation Mechanism]
\label{prop:phase_lock_formation}
Phase-locking between molecules $i$ and $j$ occurs when:
\begin{equation}
|U_{int}(r_{ij})| > k_B T \cdot \eta_{\text{threshold}}
\label{eq:phase_lock_threshold}
\end{equation}
where $U_{int} = U_{vdW} + U_{\text{dipole}} + \ldots$ is the total interaction potential and $\eta_{\text{threshold}} \approx 0.1$ is a dimensionless coupling threshold.
\end{proposition}

\begin{proof}
Phase synchronisation requires coupling strength exceeding thermal fluctuations. The coupling strength scales with interaction energy $|U_{int}|$, while thermal disruption scales with $k_B T$. Standard synchronisation theory \citep{pikovsky2001synchronization, kuramoto1975self} establishes that phase-locking occurs when:
\begin{equation}
K_{ij} > K_c
\end{equation}
where $K_{ij} \propto |U_{int}(r_{ij})|$ is the coupling strength and $K_c \propto k_B T$ is the critical coupling. This yields condition~\eqref{eq:phase_lock_threshold}. \qed
\end{proof}

\subsection{The Kinetic Independence Theorem}

We now prove the central result of this section.

\begin{theorem}[Phase-Lock Kinetic Independence]
\label{thm:kinetic_independence}
The phase-lock network $\phaselockgraph = (V, E)$ is independent of molecular kinetic energies:
\begin{equation}
\frac{\partial \phaselockgraph}{\partial E_{\text{kin},i}} = 0 \quad \forall i \in V
\label{eq:network_kinetic_independence}
\end{equation}
Specifically, the edge set $E$ is determined by spatial configuration $\{\mathbf{r}_i\}$ and molecular properties $\{\alpha_i, \mu_i, \omega_{\text{vib},i}, \ldots\}$, but not by velocities $\{\mathbf{v}_i\}$.
\end{theorem}

\begin{proof}
We prove this by showing that each factor determining edge existence is kinetically independent.

\textbf{Step 1: Interaction potential independence.}
From Proposition~\ref{prop:kinetic_independence_interactions}, $U_{int}(r_{ij})$ does not depend on molecular velocities.

\textbf{Step 2: Phase-lock threshold independence.}
The threshold condition~\eqref{eq:phase_lock_threshold} involves $U_{int}$ and $T$. While temperature $T$ is related to average kinetic energy through:
\begin{equation}
\langle E_{\text{kin}} \rangle = \frac{3}{2} k_B T
\end{equation}
this is a statistical relationship. For a given instantaneous configuration, the phase-lock condition depends on:
\begin{itemize}
    \item Separation $r_{ij}$ (spatial, not velocity)
    \item Polarisabilities $\alpha_i$, $\alpha_j$ (electronic property)
    \item Dipole moments $\mu_i$, $\mu_j$ (molecular geometry)
    \item Orientational angles $\theta_i$, $\theta_j$, $\phi$ (spatial orientation)
\end{itemize}

None of these depend on translational velocity $\mathbf{v}$.

\textbf{Step 3: Phase dynamics independence.}
From Definition~\ref{def:molecular_phase}, the molecular phase $\Phi_i(t)$ involves:
\begin{itemize}
    \item Vibrational modes: determined by molecular structure, not translation
    \item Rotational modes: determined by angular momentum, which can correlate with temperature but is independent of translational velocity direction
    \item Electronic oscillations: determined by electronic structure
\end{itemize}

Translational kinetic energy $E_{\text{kin}} = \frac{1}{2}m|\mathbf{v}|^2$ does not appear in the phase equation~\eqref{eq:molecular_phase}.

\textbf{Step 4: Edge set determination.}
An edge $(m_i, m_j) \in E$ if and only if:
\begin{enumerate}
    \item Coupling exceeds threshold: $|U_{int}(r_{ij})| > k_B T \cdot \eta_{\text{threshold}}$
    \item Phase coherence is maintained: condition~\eqref{eq:phase_lock_condition} satisfied
\end{enumerate}

Both conditions are determined by spatial configuration and molecular properties, not translational velocities.

Therefore $E = E(\{\mathbf{r}_i\}, \{\alpha_i, \mu_i, \ldots\})$ with no dependence on $\{\mathbf{v}_i\}$, establishing~\eqref{eq:network_kinetic_independence}. \qed
\end{proof}

\begin{corollary}[Velocity-Invariant Network Topology]
\label{cor:velocity_invariant}
Two gas configurations with identical spatial arrangements $\{\mathbf{r}_i\}$ but different velocity distributions $\{\mathbf{v}_i\}$, $\{\mathbf{v}'_i\}$ have identical phase-lock networks:
\begin{equation}
\phaselockgraph(\{\mathbf{r}_i\}, \{\mathbf{v}_i\}) = \phaselockgraph(\{\mathbf{r}_i\}, \{\mathbf{v}'_i\})
\end{equation}
\end{corollary}

\begin{proof}
Immediate from Theorem~\ref{thm:kinetic_independence}: since $\phaselockgraph$ does not depend on velocities, changing velocities while preserving positions leaves the network unchanged. \qed
\end{proof}

\begin{figure*}[htbp]
\centering
\includegraphics[width=0.95\textwidth]{figures/arg2_temperature_independence.png}
\caption{\textbf{Argument 2: Phase-Lock Temperature Independence—Network Topology $\partial G/\partial E_{\text{kin}} = 0$.}
\textbf{(A)} Same network topology across all temperatures. Phase-lock networks at $T = 0.5, 1.0, 2.0, 5.0$ (color-coded) show identical spatial configurations. Network edges depend on intermolecular distances $r_{ij}$ through Van der Waals interactions $U_{\text{vdW}} \propto r^{-6}$, which are velocity-independent. The 3D scatter demonstrates that network structure (spatial arrangement) is preserved across temperature layers.
\textbf{(B)} Network properties versus kinetic properties showing $\partial G/\partial T = 0$. Left axis (black): network edges remain constant at $\sim 107$ across all temperatures. Right axis (red): kinetic energy $E_{\text{kin}} = \sum_i \frac{1}{2}m_i v_i^2$ scales linearly with temperature as expected from equipartition theorem. The divergence of these curves proves that network topology $G$ is independent of thermal energy.
\textbf{(C)} Maxwell-Boltzmann velocity distributions at different temperatures. The distribution $f(v) \propto v^2 \exp(-mv^2/2k_BT)$ widens with increasing temperature (blue to red colormap), but the underlying network structure (not shown) remains unchanged. This demonstrates that the same ``snapshot'' of molecular positions corresponds to vastly different kinetic states.
\textbf{(D)} Property correlation matrix showing network-kinetic decoupling. Network properties (edges, degree, clustering) exhibit strong internal correlations ($r > 0.78$, red block), while kinetic properties (kinetic energy, temperature) also correlate strongly ($r = 0.99$, dark red block). Crucially, cross-correlations between network and kinetic properties are near-zero ($|r| < 0.03$, white region), confirming $\partial G/\partial E_{\text{kin}} = 0$. This proves that phase-lock structure is velocity-blind.}
\label{fig:temperature_independence}
\end{figure*}

\subsection{Network Properties}

\begin{definition}[Phase-Lock Degree]
\label{def:phase_lock_degree}
The \textbf{phase-lock degree} of molecule $i$ is:
\begin{equation}
k_i = |\{j : (m_i, m_j) \in E\}|
\end{equation}
the number of molecules phase-locked to $i$.
\end{definition}

\begin{proposition}[Degree Distribution]
\label{prop:degree_distribution}
For a gas at uniform density $n = N/V$, the expected phase-lock degree scales as:
\begin{equation}
\langle k \rangle \sim n \cdot \frac{4\pi}{3} r_{\text{lock}}^3
\label{eq:expected_degree}
\end{equation}
where $r_{\text{lock}}$ is the characteristic distance at which phase-locking occurs, determined by:
\begin{equation}
|U_{int}(r_{\text{lock}})| = k_B T \cdot \eta_{\text{threshold}}
\end{equation}
\end{proposition}

\begin{proof}
Molecules within distance $r_{\text{lock}}$ satisfy the phase-lock condition with high probability. The expected number of neighbours within this distance is:
\begin{equation}
\langle k \rangle = n \cdot V_{\text{sphere}}(r_{\text{lock}}) = n \cdot \frac{4\pi}{3} r_{\text{lock}}^3
\end{equation}
For Van der Waals interactions, $r_{\text{lock}} \sim (C_6 / k_B T \eta)^{1/6} \sim 0.3$--$0.5$ nm for typical gases at room temperature. \qed
\end{proof}

\begin{definition}[Phase-Lock Cluster]
\label{def:phase_lock_cluster}
A \textbf{phase-lock cluster} is a connected component of $\phaselockgraph$: a maximal subset $S \subseteq V$ such that for any $i, j \in S$, there exists a path in $\phaselockgraph$ connecting $m_i$ and $m_j$.
\end{definition}

\begin{remark}[Zero Kelvin Persistence]
At absolute zero ($T = 0$), molecular translational motion ceases, but phase-lock networks persist. Electronic orbitals continue oscillating, vibrational zero-point motion persists, and intermolecular forces remain active. The phase-lock network $\phaselockgraph(T=0)$ is well-defined and nontrivial. This underscores the kinetic independence: the network exists independently of thermal motion.
\end{remark}


%==============================================================================
\section{Categorical Completion in Gas Dynamics}
\label{sec:categorical}
%==============================================================================

\subsection{Categorical State Space}

We now develop the categorical framework for describing gas configurations.

\begin{definition}[Categorical State]
\label{def:categorical_state}
A \textbf{categorical state} $C \in \catspace$ specifies not only the spatial configuration of molecules but also:
\begin{enumerate}
    \item The phase-lock network topology $\phaselockgraph$
    \item The phase relationships $\{\Delta\phi_{ij}\}$ for all locked pairs
    \item The vibrational mode occupations $\{n_{\text{vib},i}\}$
    \item The rotational state quantum numbers $\{J_i, M_i\}$
    \item The electronic configuration descriptors
\end{enumerate}
\end{definition}

\begin{remark}
A categorical state contains strictly more information than a classical phase space point $(\mathbf{q}, \mathbf{p})$. Two configurations with identical positions and momenta can occupy different categorical states if their phase relationships or network topologies differ.
\end{remark}

\begin{definition}[Categorical State Space]
\label{def:categorical_state_space}
The \textbf{categorical state space} $\catspace$ is the set of all categorical states equipped with:
\begin{enumerate}
    \item A partial order $\prec$ (the \textbf{completion order})
    \item A completion operator $\mu: \catspace \times \mathbb{R}_{\geq 0} \to \{0, 1\}$
    \item A topology $\tau$ induced by $\prec$
\end{enumerate}
\end{definition}

\begin{axiom}[Categorical Irreversibility]
\label{axiom:categorical_irreversibility}
Once a categorical state $C_i$ is occupied (completed), it cannot be re-occupied. For all $C_i \in \catspace$ and times $t_1 \leq t_2$:
\begin{equation}
\mu(C_i, t_1) = 1 \implies \mu(C_i, t_2) = 1
\label{eq:irreversibility}
\end{equation}
Any process returning to a spatially identical configuration must occupy a new categorical state $C_j$ with $C_i \prec C_j$.
\end{axiom}

\begin{proposition}[Monotonic Completion]
\label{prop:monotonic_completion}
Let $\gamma(t) = \{C \in \catspace : \mu(C, t) = 1\}$ be the set of completed states at time $t$. Then:
\begin{equation}
t_1 \leq t_2 \implies \gamma(t_1) \subseteq \gamma(t_2)
\end{equation}
The completed set grows monotonically.
\end{proposition}

\begin{proof}
Immediate from Axiom~\ref{axiom:categorical_irreversibility}: once $C \in \gamma(t_1)$, we have $\mu(C, t_1) = 1$, hence $\mu(C, t_2) = 1$, so $C \in \gamma(t_2)$. \qed
\end{proof}

\begin{figure*}[htbp]
\centering
\includegraphics[width=0.95\textwidth]{figures/arg1_temporal_triviality.png}
\caption{\textbf{Argument 1: Temporal Triviality—Any Configuration Occurs Naturally Through Thermal Fluctuations.}
\textbf{(A)} Boltzmann probability landscape showing all configurations are thermally accessible. The probability distribution $P(\text{config}) = \exp(-E/k_BT)/Z$ ensures every spatial arrangement, including ``sorted'' states, occurs naturally through fluctuations.
\textbf{(B)} Poincaré recurrence times as a function of sorting degree. Higher sorting corresponds to exponentially longer recurrence times $\tau_{\text{rec}} \sim \exp(N\Delta S)$, but all states eventually recur. The horizontal dashed line indicates laboratory timescales; even highly sorted states recur within observable time for small systems.
\textbf{(C)} Configuration space flow field showing all trajectories converge to equilibrium. The flow follows $\dot{\mathbf{q}} = -\nabla_{\mathbf{q}} F(\mathbf{q})$ where $F$ is the free energy. Red squares mark ``sorted'' configurations; yellow circles mark equilibrium. All paths lead to the central attractor, demonstrating that sorted states are unstable fixed points.
\textbf{(D)} Entropy evolution over time showing fluctuations enable access to all states. The solid black line shows total entropy $S(t) = -k_B \sum_i p_i \ln p_i$ increasing monotonically toward equilibrium (horizontal dashed line). The dotted red line marks the entropy of the ``sorted'' state. Yellow triangles indicate moments when the system spontaneously visits sorted configurations through thermal fluctuations, demonstrating temporal triviality: the demon's purported action is redundant.}
\label{fig:temporal_triviality}
\end{figure*}

\subsection{Phase-Lock Degeneracy}

\begin{theorem}[Phase-Lock Degeneracy]
\label{thm:phase_lock_degeneracy}
For a spatial configuration $\mathbf{q} = \{\mathbf{r}_1, \ldots, \mathbf{r}_N\}$, there exist multiple categorical states producing identical spatial observables. The \textbf{phase-lock degeneracy} is:
\begin{equation}
\Omega_{\text{PL}}(\mathbf{q}) = |\{C \in \catspace : \pi_{\text{spatial}}(C) = \mathbf{q}\}|
\label{eq:phase_lock_degeneracy}
\end{equation}
where $\pi_{\text{spatial}}: \catspace \to \mathbb{R}^{3N}$ is the spatial projection.
\end{theorem}

\begin{proof}
Consider two molecules at fixed positions $\mathbf{r}_1$, $\mathbf{r}_2$. The same spatial configuration can be achieved through different combinations of:
\begin{itemize}
    \item Van der Waals interaction angles: $\theta_{vdW} \in [0, 2\pi]$
    \item Dipole orientations: $(\phi_1, \phi_2) \in [0, 2\pi]^2$
    \item Vibrational phase differences: $\Delta\phi_{\text{vib}} \in [0, 2\pi]$
    \item Rotational phase offsets: $\Delta\phi_{\text{rot}} \in [0, 2\pi]$
\end{itemize}

These constitute distinct categorical states (different phase relationships) with identical spatial projection.

For $N$ molecules with $\binom{N}{2}$ pairwise interactions, each having continuous phase degrees of freedom:
\begin{equation}
\Omega_{\text{PL}}(\mathbf{q}) \sim (2\pi)^{k \cdot \binom{N}{2}}
\end{equation}
where $k$ is the number of independent phase variables per pair. For typical gases, $\Omega_{\text{PL}} \sim 10^6$ to $10^{12}$ per spatial configuration. \qed
\end{proof}

\begin{definition}[Categorical Equivalence Class]
\label{def:categorical_equivalence_class}
The \textbf{categorical equivalence class} of state $C$ under spatial observation is:
\begin{equation}
[C]_{\text{spatial}} = \{C' \in \catspace : \pi_{\text{spatial}}(C') = \pi_{\text{spatial}}(C)\}
\end{equation}
States in the same equivalence class are spatially indistinguishable but categorically distinct.
\end{definition}

\begin{corollary}[Categorical Richness]
\label{cor:categorical_richness}
The \textbf{categorical richness} of a spatial configuration is:
\begin{equation}
R(\mathbf{q}) = \log \Omega_{\text{PL}}(\mathbf{q}) = \log |[C]_{\text{spatial}}|
\end{equation}
This quantifies the information content of categorical specification beyond spatial description.
\end{corollary}

\subsection{Categorical Completion Dynamics}

\begin{definition}[Completion Rate]
\label{def:completion_rate}
The \textbf{categorical completion rate} is:
\begin{equation}
\dot{C}(t) = \frac{d|\gamma(t)|}{dt}
\label{eq:completion_rate}
\end{equation}
measuring the rate at which new categorical states are completed.
\end{definition}

\begin{proposition}[Non-Negative Completion Rate]
\label{prop:nonnegative_completion}
For all times $t$:
\begin{equation}
\dot{C}(t) \geq 0
\end{equation}
with equality only when no physical processes occur.
\end{proposition}

\begin{proof}
From Proposition~\ref{prop:monotonic_completion}, $|\gamma(t)|$ is monotonically non-decreasing, hence $\dot{C}(t) = d|\gamma(t)|/dt \geq 0$. \qed
\end{proof}

\begin{theorem}[Categorical Completion as Physical Process]
\label{thm:completion_physical}
Every physical process in a gas system corresponds to categorical completion:
\begin{equation}
\text{Process: } \mathbf{q}(t_1) \to \mathbf{q}(t_2) \quad \Longleftrightarrow \quad \text{Completion: } C(t_1) \prec C(t_2)
\end{equation}
The categorical state advances along the completion order.
\end{theorem}

\begin{proof}
Consider a gas evolving from configuration $\mathbf{q}(t_1)$ to $\mathbf{q}(t_2)$.

\textbf{Case 1: $\mathbf{q}(t_2) \neq \mathbf{q}(t_1)$ (different spatial configuration).}
The new configuration occupies categorical states not accessible from $\mathbf{q}(t_1)$. By Axiom~\ref{axiom:categorical_irreversibility}, these must be new completions: $C(t_2) \in \gamma(t_2) \setminus \gamma(t_1)$.

\textbf{Case 2: $\mathbf{q}(t_2) = \mathbf{q}(t_1)$ (same spatial configuration).}
Even with identical spatial positions, the phase relationships have evolved:
\begin{equation}
\Phi_i(t_2) - \Phi_j(t_2) \neq \Phi_i(t_1) - \Phi_j(t_1)
\end{equation}
in general. The categorical state has changed despite spatial identity.

By Axiom~\ref{axiom:categorical_irreversibility}, return to $C(t_1)$ is impossible; the system occupies a new state $C(t_2)$ with $C(t_1) \prec C(t_2)$.

In both cases, categorical position advances. \qed
\end{proof}

\subsection{Categorical Distance}

\begin{definition}[Categorical Distance]
\label{def:categorical_distance}
The \textbf{categorical distance} between states $C_i, C_j \in \catspace$ is:
\begin{equation}
d_{\catspace}(C_i, C_j) = \inf_{\text{paths } C_i \to C_j} \sum_{\text{transitions}} w(C_k \to C_{k+1})
\label{eq:categorical_distance}
\end{equation}
where the infimum is over all completion paths from $C_i$ to $C_j$, and $w(C_k \to C_{k+1})$ is the transition weight.
\end{definition}

\begin{proposition}[Metric Properties]
\label{prop:metric_properties}
The categorical distance $d_{\catspace}$ satisfies:
\begin{enumerate}
    \item Non-negativity: $d_{\catspace}(C_i, C_j) \geq 0$
    \item Identity: $d_{\catspace}(C_i, C_j) = 0 \iff C_i = C_j$
    \item Symmetry: $d_{\catspace}(C_i, C_j) = d_{\catspace}(C_j, C_i)$
    \item Triangle inequality: $d_{\catspace}(C_i, C_k) \leq d_{\catspace}(C_i, C_j) + d_{\catspace}(C_j, C_k)$
\end{enumerate}
Thus $(\catspace, d_{\catspace})$ is a metric space.
\end{proposition}

\begin{theorem}[Categorical-Physical Distance Inequivalence]
\label{thm:distance_inequivalence}
Categorical distance $d_{\catspace}$ is not a function of physical distance $d_{\text{phys}}$:
\begin{equation}
\nexists f: \mathbb{R}_{\geq 0} \to \mathbb{R}_{\geq 0} \text{ such that } d_{\catspace}(C_i, C_j) = f(d_{\text{phys}}(\mathbf{r}_i, \mathbf{r}_j))
\label{eq:distance_inequivalence}
\end{equation}
\end{theorem}

\begin{proof}
We construct explicit counterexamples.

\textbf{Counterexample 1: Categorical adjacency without physical proximity.}
Consider molecules $A$ and $B$ at positions $\mathbf{r}_A = (0, 0, 0)$ and $\mathbf{r}_B = (L, 0, 0)$ with large separation $L \gg r_{\text{lock}}$.

Through a chain of intermediate molecules:
\begin{equation}
A \leftrightarrow M_1 \leftrightarrow M_2 \leftrightarrow \cdots \leftrightarrow M_n \leftrightarrow B
\end{equation}
where each pair is phase-locked, molecules $A$ and $B$ are in the same phase-lock cluster.

Categorical distance: $d_{\catspace}(C_A, C_B) = n$ (path length through network).
Physical distance: $d_{\text{phys}}(\mathbf{r}_A, \mathbf{r}_B) = L$ (arbitrarily large).

For large $L$ and small $n$, we have $d_{\catspace} \ll d_{\text{phys}}$.

\textbf{Counterexample 2: Physical proximity without categorical adjacency.}
Consider molecules $A$ and $B$ at positions $\mathbf{r}_A = (0, 0, 0)$ and $\mathbf{r}_B = (\delta, 0, 0)$ with $\delta \to 0$.

If $A$ and $B$ belong to different phase-lock clusters (e.g., different vibrational phases that prevent locking), then:
\begin{equation}
d_{\catspace}(C_A, C_B) = \infty \quad \text{(no path between clusters)}
\end{equation}
while $d_{\text{phys}}(\mathbf{r}_A, \mathbf{r}_B) = \delta \to 0$.

Here $d_{\catspace} \gg d_{\text{phys}}$.

Since both $d_{\catspace} \ll d_{\text{phys}}$ and $d_{\catspace} \gg d_{\text{phys}}$ are achievable, no function $f$ can satisfy~\eqref{eq:distance_inequivalence}. \qed
\end{proof}

\begin{corollary}[Categorical Adjacency Determines Accessibility]
\label{cor:adjacency_accessibility}
The set of states accessible from $C_i$ through single-step transitions is:
\begin{equation}
\accessible(C_i) = \{C_j \in \catspace : d_{\catspace}(C_i, C_j) = 1\}
\end{equation}
This is determined by phase-lock network topology, not physical proximity.
\end{corollary}


%==============================================================================
\section{Categorical Selection and Accessibility Pathways}
\label{sec:selection}
%==============================================================================

\subsection{The Selection Problem}

In Maxwell's thought experiment, the demon ``selects'' fast molecules to pass through the door. We now analyse what selection means in categorical terms.

\begin{definition}[Categorical Selection]
\label{def:categorical_selection}
A \textbf{categorical selection} is the completion of a specific categorical state $C^* \in [C]_{\text{spatial}}$ from an equivalence class of spatially indistinguishable states.
\end{definition}

\begin{proposition}[Selection as Equivalence Class Reduction]
\label{prop:selection_reduction}
Categorical selection reduces the equivalence class to a singleton:
\begin{equation}
[C]_{\text{spatial}} \xrightarrow{\text{selection}} \{C^*\}
\end{equation}
This is an information-gain process: $\log |[C]_{\text{spatial}}|$ bits of categorical information are specified.
\end{proposition}

\begin{proof}
Before selection, any state in $[C]_{\text{spatial}}$ is possible. After selection, exactly one state $C^*$ is completed. The information gained is:
\begin{equation}
I_{\text{selection}} = \log_2 |[C]_{\text{spatial}}| - \log_2 1 = \log_2 |[C]_{\text{spatial}}|
\end{equation}
For typical gas systems with $|[C]_{\text{spatial}}| \sim 10^6$, this is $\sim 20$ bits per selection. \qed
\end{proof}

\subsection{Accessibility Through Phase-Lock Networks}

\begin{theorem}[Phase-Lock Accessibility]
\label{thm:phase_lock_accessibility}
When categorical state $C_i$ is completed, the accessible states for subsequent completion are determined by phase-lock adjacency:
\begin{equation}
\accessible(C_i) = \{C_j \in \catspace : \exists (m_k, m_l) \in E(\phaselockgraph) \text{ connecting } C_i \text{ to } C_j\}
\label{eq:accessible_states}
\end{equation}
\end{theorem}

\begin{proof}
Categorical transitions require physical mechanism. The mechanisms available are:
\begin{enumerate}
    \item Molecular collisions: transfer energy and phase information
    \item Phase-lock coupling: synchronise oscillatory states
    \item Electromagnetic interaction: modify electronic configurations
\end{enumerate}

All these mechanisms operate through intermolecular interactions. Molecules not connected in $\phaselockgraph$ have negligible interaction strength (below threshold~\eqref{eq:phase_lock_threshold}), hence cannot mediate transitions.

Therefore, accessible states are precisely those reachable through phase-lock network edges. \qed
\end{proof}

\begin{corollary}[Pathway Opening]
\label{cor:pathway_opening}
Completing categorical state $C_i$ ``opens'' pathways to all states in:
\begin{equation}
\text{Pathways}(C_i) = \{C_j : d_{\catspace}(C_i, C_j) < \infty\}
\end{equation}
the connected component of $\catspace$ containing $C_i$.
\end{corollary}

\subsection{The Cascade Effect}

\begin{theorem}[Categorical Cascade]
\label{thm:categorical_cascade}
Selection of a single categorical state $C_1$ initiates a cascade of accessible completions:
\begin{align}
C_1 &\to \accessible(C_1) = \{C_2^{(1)}, C_2^{(2)}, \ldots\} \\
C_2^{(k)} &\to \accessible(C_2^{(k)}) = \{C_3^{(k,1)}, C_3^{(k,2)}, \ldots\} \\
&\vdots
\end{align}
The cascade propagates through the phase-lock network.
\end{theorem}

\begin{proof}
Each completed state makes adjacent states accessible (Theorem~\ref{thm:phase_lock_accessibility}). Completing any accessible state makes its neighbours accessible. This propagation continues until:
\begin{enumerate}
    \item The entire connected component is exhausted, or
    \item Energy/entropy constraints halt the cascade
\end{enumerate}

The structure of the cascade is determined by $\phaselockgraph$ topology. \qed
\end{proof}

\begin{definition}[Cascade Wavefront]
\label{def:cascade_wavefront}
The \textbf{cascade wavefront} at step $n$ is:
\begin{equation}
W_n = \{C \in \catspace : d_{\catspace}(C_1, C) = n\}
\end{equation}
the set of states at categorical distance $n$ from the initial selection.
\end{definition}

\begin{proposition}[Wavefront Propagation]
\label{prop:wavefront_propagation}
The wavefront size evolves according to:
\begin{equation}
|W_{n+1}| = \sum_{C \in W_n} |\accessible(C) \setminus \gamma(t_n)|
\end{equation}
where $\gamma(t_n)$ is the set of already-completed states.
\end{proposition}

\begin{figure*}[htbp]
\centering
\includegraphics[width=0.95\textwidth]{figures/panel_arg5_dissolution_decision.png}
\caption{\textbf{Argument 5: Dissolution of Decision—Categorical Completion is Automatic, Not Deliberative.}
\textbf{(A)} No decision tree exists in phase-lock network topology. The schematic shows a molecule (top teal node) with multiple potential paths (gray dashed lines indicate non-existent alternatives). Only one path (solid green line through teal nodes) exists in the network topology, determined by categorical adjacency. Red lines with crosses mark paths that are topologically forbidden. The system has no choice points: navigation is deterministic. The caption ``Only ONE path exists in topology'' emphasizes that categorical completion requires no decision-making.
\textbf{(B)} Automatic path following through configuration space. The trajectory (dark teal curve) shows completion progress from start (green circle, configuration $\approx 0.5$) to end (red star, categorical progress $= 1.0$). The smooth, deterministic path follows the gradient of categorical distance $\nabla d_{\text{cat}}(\mathbf{q})$ with no branching points. The system automatically follows network structure without decisions, as indicated by the annotation ``Automatic following / No decisions required.'' The configuration parameter represents position in categorical state space, not physical space.
\textbf{(C)} No branching implies no decision. Bar plot showing the number of available options at each completion step. All bars are exactly 1.0 (marked by red dashed line), confirming ``Always exactly 1 option'' at every step. If decision-making were required, we would observe $n_{\text{options}} > 1$ at branch points. The constant value $n = 1$ proves that categorical completion is a deterministic flow, not a stochastic or deliberative process. This directly contradicts the demon's purported role as a decision-maker.
\textbf{(D)} Deterministic reproducibility across 10 independent runs. The completion curve (dark teal) shows identical trajectories for all 10 runs, with completion increasing from 0 to 1.0 following $C(t) = 1 - \exp(-t/\tau_{\text{cat}})$ where $\tau_{\text{cat}}$ is the categorical completion timescale. Perfect overlap of all runs confirms deterministic dynamics: given the same initial categorical state, the system always follows the same path. This demonstrates that categorical completion is automatic and reproducible, requiring no agent, no information processing, and no decisions. The demon's ``choice'' to open the door is revealed as automatic topological navigation.}
\label{fig:dissolution_decision}
\end{figure*}

\subsection{Selection Without Information}

We now prove the central result: categorical selection requires no external information.

\begin{theorem}[Information-Free Selection]
\label{thm:information_free}
Categorical selection from equivalence class $[C]_{\text{spatial}}$ is determined by phase-lock network topology without external information input. Specifically:
\begin{equation}
C^* = \argmin_{C \in [C]_{\text{spatial}}} d_{\catspace}(C, C_{\text{prev}})
\end{equation}
where $C_{\text{prev}}$ is the previously completed state.
\end{theorem}

\begin{proof}
Consider a system at categorical state $C_{\text{prev}}$ transitioning to spatial configuration $\mathbf{q}_{\text{new}}$.

The accessible categorical states at $\mathbf{q}_{\text{new}}$ are:
\begin{equation}
\accessible(C_{\text{prev}}) \cap [C]_{\text{spatial}}(\mathbf{q}_{\text{new}})
\end{equation}
the intersection of phase-lock accessible states and spatially compatible states.

\textbf{Case 1: Unique accessible state.}
If $|\accessible(C_{\text{prev}}) \cap [C]_{\text{spatial}}| = 1$, selection is deterministic---only one categorical state is reachable.

\textbf{Case 2: Multiple accessible states.}
If multiple states are accessible, physical dynamics (minimising action, maximising entropy production rate) select among them. This selection is governed by:
\begin{equation}
C^* = \argmax_{C \in \accessible(C_{\text{prev}}) \cap [C]_{\text{spatial}}}\frac{d\alpha}{dt}
\end{equation}
where $\alpha$ is the oscillatory termination probability, favouring states with shorter paths to equilibrium.

In both cases, selection follows from network topology and physical dynamics---no external ``measurement'' or ``decision'' is required.

The ``information'' specifying which categorical state to occupy is structural (encoded in $\phaselockgraph$) rather than acquired through measurement. \qed
\end{proof}

\begin{corollary}[No Demon Required]
\label{cor:no_demon}
The selection process attributed to Maxwell's Demon is categorical completion through phase-lock topology. No agent is required because:
\begin{enumerate}
    \item Selection is determined by network structure (Theorem~\ref{thm:information_free})
    \item Accessibility follows from phase-lock adjacency (Theorem~\ref{thm:phase_lock_accessibility})
    \item Cascade propagation is automatic (Theorem~\ref{thm:categorical_cascade})
\end{enumerate}
\end{corollary}

\subsection{Apparent Sorting Through Categorical Pathways}

\begin{theorem}[Apparent Temperature Sorting]
\label{thm:apparent_sorting}
Molecules following categorical pathways appear sorted by temperature because phase-lock clusters correlate with kinetic properties. Specifically, for molecules $i, j$ in the same phase-lock cluster:
\begin{equation}
\text{Cov}(E_{\text{kin},i}, E_{\text{kin},j}) > 0
\label{eq:kinetic_correlation}
\end{equation}
despite phase-lock formation being kinetically independent.
\end{theorem}

\begin{proof}
Phase-lock clusters form based on molecular properties: polarisability, dipole moment, vibrational frequencies. These properties correlate with molecular mass and structure, which in turn correlate with kinetic energy distribution at thermal equilibrium.

Consider the Maxwell-Boltzmann distribution:
\begin{equation}
f(v) = 4\pi \left(\frac{m}{2\pi k_B T}\right)^{3/2} v^2 \exp\left(-\frac{mv^2}{2k_B T}\right)
\end{equation}

Molecules with similar mass $m$ have similar most-probable velocities $v_p = \sqrt{2k_B T/m}$. Since phase-lock clusters tend to contain molecules with similar properties (similar polarisabilities, similar dipole moments), they tend to contain molecules with similar masses, hence similar kinetic energies.

The correlation~\eqref{eq:kinetic_correlation} arises from shared molecular properties, not from kinetic energy determining phase-lock formation.

\textbf{Causal structure:}
\begin{equation}
\text{Molecular properties} \to \begin{cases} \text{Phase-lock clustering} \\ \text{Kinetic energy distribution} \end{cases}
\end{equation}

Both phase-lock structure and kinetic properties are downstream of molecular properties. The correlation is non-causal: neither determines the other. \qed
\end{proof}

\begin{corollary}[Sorting Is Correlation, Not Causation]
\label{cor:correlation_not_causation}
When molecules ``sorted'' by categorical pathways appear to separate by temperature, this reflects:
\begin{enumerate}
    \item Pre-existing phase-lock cluster structure
    \item Correlation between cluster membership and kinetic properties
    \item NOT measurement of velocity followed by sorting decision
\end{enumerate}
\end{corollary}


%==============================================================================
\section{Temperature as Emergent from Phase-Lock Statistics}
\label{sec:temperature}
%==============================================================================

\subsection{The Standard View of Temperature}

In classical thermodynamics, temperature is a fundamental quantity defined through:
\begin{equation}
\frac{1}{T} = \left(\frac{\partial S}{\partial E}\right)_{V,N}
\label{eq:temperature_standard}
\end{equation}
or operationally through thermal equilibrium. For ideal gases:
\begin{equation}
\langle E_{\text{kin}} \rangle = \frac{3}{2} N k_B T
\label{eq:equipartition}
\end{equation}

This framing suggests temperature determines molecular behaviour: higher $T$ means faster molecules.

\subsection{The Categorical View: Temperature as Emergent}

We now prove that temperature emerges from phase-lock cluster statistics rather than determining them.

\begin{definition}[Cluster Kinetic Distribution]
\label{def:cluster_kinetic}
For phase-lock cluster $\mathcal{K}_\alpha \subset V$, define the cluster kinetic energy:
\begin{equation}
E_{\alpha} = \sum_{i \in \mathcal{K}_\alpha} \frac{1}{2} m_i |\mathbf{v}_i|^2
\end{equation}
and cluster temperature:
\begin{equation}
T_\alpha = \frac{2 E_\alpha}{3 |\mathcal{K}_\alpha| k_B}
\end{equation}
\end{definition}

\begin{theorem}[Temperature Emergence]
\label{thm:temperature_emergence}
The macroscopic temperature $T$ of a gas is a statistical functional of phase-lock cluster structure:
\begin{equation}
T = \mathcal{F}[\{(\mathcal{K}_\alpha, T_\alpha, |\mathcal{K}_\alpha|)\}_{\alpha=1}^{N_c}]
\label{eq:temperature_functional}
\end{equation}
where $N_c$ is the number of clusters. Specifically:
\begin{equation}
T = \frac{\sum_\alpha |\mathcal{K}_\alpha| T_\alpha}{\sum_\alpha |\mathcal{K}_\alpha|} = \frac{\sum_\alpha |\mathcal{K}_\alpha| T_\alpha}{N}
\label{eq:temperature_average}
\end{equation}
\end{theorem}

\begin{proof}
The total kinetic energy is:
\begin{equation}
E_{\text{total}} = \sum_{i=1}^N \frac{1}{2} m_i |\mathbf{v}_i|^2 = \sum_\alpha E_\alpha = \sum_\alpha \frac{3}{2} |\mathcal{K}_\alpha| k_B T_\alpha
\end{equation}

The macroscopic temperature satisfies:
\begin{equation}
E_{\text{total}} = \frac{3}{2} N k_B T
\end{equation}

Equating:
\begin{equation}
\frac{3}{2} N k_B T = \sum_\alpha \frac{3}{2} |\mathcal{K}_\alpha| k_B T_\alpha
\end{equation}
\begin{equation}
T = \frac{\sum_\alpha |\mathcal{K}_\alpha| T_\alpha}{N}
\end{equation}

This expresses $T$ as a weighted average over cluster temperatures, proving~\eqref{eq:temperature_functional}. \qed
\end{proof}

\begin{corollary}[Temperature Does Not Determine Clusters]
\label{cor:temperature_not_causal}
The cluster structure $\{\mathcal{K}_\alpha\}$ is determined by phase-lock network topology (Theorem~\ref{thm:kinetic_independence}), which is independent of kinetic energy. Therefore:
\begin{equation}
\frac{\partial \{\mathcal{K}_\alpha\}}{\partial T} = 0
\end{equation}
Changing temperature does not change which molecules belong to which clusters (at fixed spatial configuration).
\end{corollary}


\begin{figure*}[htbp]
\centering
\includegraphics[width=0.95\textwidth]{figures/panel_arg4_dissolution_observation.png}
\caption{\textbf{Argument 4: Dissolution of Observation—Navigation Follows Topology, Not Velocity Measurement.}
\textbf{(A)} Topology determines path without velocity information. Molecules arranged in a phase-lock network (teal nodes) follow paths determined purely by network adjacency. The path from one side to the other (red nodes indicating transition region) is determined by categorical distance $d_{\text{cat}}(i,j)$ in the network, not by molecular velocities. Navigation occurs through shortest paths in the network graph, requiring no knowledge of kinetic properties.
\textbf{(B)} Observation not required for navigation. The diagram shows two information channels: velocity measurement (red, crossed out) and topological adjacency (green, active). Navigation proceeds through the green channel alone. The system follows network structure without any measurement of velocities, demonstrating that the demon's ``observation'' is unnecessary. Path completion is automatic through categorical structure.
\textbf{(C)} Velocity is uncorrelated with network position. Scatter plot of molecular velocity versus network position shows near-zero correlation ($r = -0.150$, dashed red line). The random scatter demonstrates that knowing a molecule's position in the phase-lock network provides no information about its velocity, and vice versa. This confirms that categorical distance $d_{\text{cat}}$ and kinetic distance $d_{\text{kin}}$ are inequivalent metrics, as stated in Section 3.4.
\textbf{(D)} Topological gate operates on adjacency, not velocity. Schematic of the demon's door showing two molecules (red circles) adjacent to the door and two molecules (blue squares) far from the door. The door opens based purely on topological adjacency in the phase-lock network: any adjacent molecule passes, regardless of velocity. The gate is velocity-blind, operating on categorical structure alone. This dissolves the paradox: there is no velocity measurement, no decision based on kinetic energy, and therefore no violation of the second law. The apparent ``sorting'' is categorical completion through network topology.}
\label{fig:dissolution_observation}
\end{figure*}

\subsection{Cluster Temperature Distribution}

\begin{proposition}[Cluster Temperature Variance]
\label{prop:cluster_variance}
At thermal equilibrium, the variance of cluster temperatures satisfies:
\begin{equation}
\text{Var}(T_\alpha) = \frac{2 T^2}{3 \langle |\mathcal{K}_\alpha| \rangle}
\label{eq:cluster_variance}
\end{equation}
Smaller clusters have larger temperature fluctuations.
\end{proposition}

\begin{proof}
For a cluster of $n$ molecules at equilibrium, the kinetic energy follows:
\begin{equation}
E_\alpha \sim \text{Gamma}\left(\frac{3n}{2}, k_B T\right)
\end{equation}

The variance of $E_\alpha$ is:
\begin{equation}
\text{Var}(E_\alpha) = \frac{3n}{2} (k_B T)^2
\end{equation}

Since $T_\alpha = 2E_\alpha / (3n k_B)$:
\begin{equation}
\text{Var}(T_\alpha) = \frac{4}{9n^2 k_B^2} \text{Var}(E_\alpha) = \frac{4}{9n^2 k_B^2} \cdot \frac{3n}{2} (k_B T)^2 = \frac{2T^2}{3n}
\end{equation}

Taking the expectation over cluster sizes gives~\eqref{eq:cluster_variance}. \qed
\end{proof}

\begin{corollary}[Hot and Cold Clusters at Equilibrium]
\label{cor:hot_cold_clusters}
Even at thermal equilibrium (uniform macroscopic $T$), individual clusters have different instantaneous temperatures. Some clusters are ``hot'' ($T_\alpha > T$) and others ``cold'' ($T_\alpha < T$) at any given moment.
\end{corollary}

\subsection{The Inversion of Causality}

\begin{theorem}[Causal Structure of Temperature]
\label{thm:causal_structure}
The causal relationships between phase-lock network, cluster structure, and temperature are:
\begin{equation}
\begin{tikzcd}[row sep=small, column sep=small]
& \text{Molecular Properties} \arrow[dl] \arrow[dr] & \\
\text{Phase-Lock Network } \phaselockgraph \arrow[d] & & \text{Velocity Distribution} \arrow[d] \\
\text{Cluster Structure } \{\mathcal{K}_\alpha\} \arrow[dr] & & \text{Individual Kinetic Energies} \arrow[dl] \\
& \text{Temperature } T &
\end{tikzcd}
\end{equation}
Temperature is downstream of both network structure and kinetic distribution; it determines neither.
\end{theorem}

\begin{proof}
\textbf{Phase-lock network from molecular properties:}
From Section~\ref{sec:phase_lock}, $\phaselockgraph$ is determined by polarisabilities, dipole moments, vibrational frequencies---all molecular properties independent of velocity.

\textbf{Cluster structure from network:}
Clusters $\{\mathcal{K}_\alpha\}$ are connected components of $\phaselockgraph$---pure graph-theoretic construction.

\textbf{Velocity distribution from molecular properties:}
The Maxwell-Boltzmann distribution depends on molecular mass $m$, a molecular property.

\textbf{Temperature from cluster structure and velocities:}
From Theorem~\ref{thm:temperature_emergence}, $T$ is computed from $\{T_\alpha\}$, which require both cluster membership (from network) and velocities.

No arrow points from $T$ to $\phaselockgraph$ or $\{\mathcal{K}_\alpha\}$. Temperature is emergent, not fundamental. \qed
\end{proof}

\subsection{Implications for Maxwell's Demon}

\begin{theorem}[Demon Cannot Sort by Temperature]
\label{thm:demon_cannot_sort}
A hypothetical Maxwell's Demon cannot sort molecules ``by temperature'' because:
\begin{enumerate}
    \item Temperature is a macroscopic emergent property, not a molecular attribute
    \item Individual molecules have kinetic energies, not temperatures
    \item Kinetic energy does not determine categorical accessibility
\end{enumerate}
\end{theorem}

\begin{proof}
\textbf{(1) Temperature is macroscopic:}
From Definition~\ref{def:cluster_kinetic}, even cluster temperature requires multiple molecules. Single-molecule temperature is undefined.

\textbf{(2) Kinetic energy vs. temperature:}
A molecule has kinetic energy $E_i = \frac{1}{2}m_i|\mathbf{v}_i|^2$, an instantaneous mechanical quantity. Temperature $T$ is a statistical property of ensembles. The demon supposedly measures $E_i$ and infers ``hot'' or ``cold,'' but this conflates distinct concepts.

\textbf{(3) Kinetic energy does not determine accessibility:}
From Theorem~\ref{thm:kinetic_independence}, phase-lock networks are kinetically independent. Categorical accessibility (which states a molecule can transition to) is determined by network topology, not kinetic energy.

A ``fast'' molecule (high $E_i$) has the same categorical accessibility as a ``slow'' molecule in the same phase-lock cluster. The demon cannot use kinetic energy to predict or control categorical transitions.

Therefore, ``sorting by temperature'' is categorically meaningless. \qed
\end{proof}

\begin{corollary}[What the Demon Actually Does]
\label{cor:demon_actual}
If we reinterpret the demon's operation categorically:
\begin{enumerate}
    \item \textbf{``Observing'' a molecule}: Completing a categorical state, making adjacent states accessible
    \item \textbf{``Opening the door''}: Following phase-lock pathways to the next cluster
    \item \textbf{``Sorting''}: Revealing pre-existing cluster structure
\end{enumerate}
The demon is not sorting by temperature but navigating categorical space along phase-lock topology.
\end{corollary}


%==============================================================================
\section{Entropy Mechanism Through Network Topology}
\label{sec:entropy}
%==============================================================================

\subsection{Topological Origin of Entropy}

We now establish that entropy arises from phase-lock network topology, providing the mechanism by which the second law is preserved without invoking information-theoretic arguments.

\begin{definition}[Network Entropy]
\label{def:network_entropy}
The \textbf{network entropy} of a phase-lock configuration is:
\begin{equation}
S_{\phaselockgraph} = k_B \log \Omega_{\text{PL}}(\phaselockgraph)
\label{eq:network_entropy}
\end{equation}
where $\Omega_{\text{PL}}(\phaselockgraph)$ is the number of categorical states compatible with network topology $\phaselockgraph$.
\end{definition}

\begin{proposition}[Entropy and Edge Density]
\label{prop:entropy_edge_density}
Network entropy is related to edge density:
\begin{equation}
S_{\phaselockgraph} \propto k_B |E(\phaselockgraph)|
\label{eq:entropy_edges}
\end{equation}
More edges (constraints) correspond to higher entropy.
\end{proposition}

\begin{proof}
Each edge $(m_i, m_j) \in E$ represents a phase-lock constraint: the phase difference $\Phi_i - \Phi_j$ must remain within bounds. More constraints reduce the volume of accessible phase space but increase the categorical richness:
\begin{equation}
\Omega_{\text{PL}} \propto \exp(c \cdot |E|)
\end{equation}
for some constant $c > 0$. This counterintuitive result arises because constraints create categorical structure: each constraint defines equivalence classes, and categorical states enumerate these classes.

Taking logarithms: $S_{\phaselockgraph} = k_B \log \Omega_{\text{PL}} \propto k_B |E|$. \qed
\end{proof}

\subsection{Entropy Increase Through Network Densification}

\begin{theorem}[Categorical Mixing Increases Entropy]
\label{thm:mixing_entropy}
When two previously separated gas volumes mix, entropy increases due to phase-lock network densification:
\begin{equation}
\Delta S_{\text{mix}} = S_{\phaselockgraph_{\text{mixed}}} - S_{\phaselockgraph_{\text{separated}}} = k_B \log \frac{\Omega_{\text{mixed}}}{\Omega_{\text{separated}}} > 0
\label{eq:mixing_entropy}
\end{equation}
\end{theorem}

\begin{proof}
\textbf{Initial (separated) state:}
Two volumes A and B contain phase-lock networks $\phaselockgraph_A = (V_A, E_A)$ and $\phaselockgraph_B = (V_B, E_B)$ with no edges between them:
\begin{equation}
\phaselockgraph_{\text{separated}} = \phaselockgraph_A \sqcup \phaselockgraph_B, \quad |E_{\text{separated}}| = |E_A| + |E_B|
\end{equation}

\textbf{Mixed state:}
After mixing, molecules from A interact with molecules from B, creating new edges:
\begin{equation}
E_{\text{mixed}} = E_A \cup E_B \cup E_{A \leftrightarrow B}
\end{equation}
where $E_{A \leftrightarrow B}$ contains edges between A-molecules and B-molecules.

The number of new edges:
\begin{equation}
|E_{A \leftrightarrow B}| \approx |V_A| \cdot |V_B| \cdot P_{\text{lock}}
\end{equation}
where $P_{\text{lock}}$ is the probability that a random A-B pair satisfies the phase-lock condition.

For typical gases at standard conditions, $P_{\text{lock}} \sim 0.1$ to $0.5$, giving:
\begin{equation}
|E_{\text{mixed}}| = |E_{\text{separated}}| + |E_{A \leftrightarrow B}| > |E_{\text{separated}}|
\end{equation}

From Proposition~\ref{prop:entropy_edge_density}:
\begin{equation}
S_{\text{mixed}} \propto k_B |E_{\text{mixed}}| > k_B |E_{\text{separated}}| \propto S_{\text{separated}}
\end{equation}

Therefore $\Delta S_{\text{mix}} > 0$. \qed
\end{proof}

\begin{corollary}[No Entropy Paradox]
\label{cor:no_entropy_paradox}
The apparent ``sorting'' in Maxwell's thought experiment does not decrease entropy because:
\begin{enumerate}
    \item Sorting is categorical completion, not physical rearrangement
    \item Categorical completion always increases network density
    \item Increased network density increases entropy
\end{enumerate}
\end{corollary}

\begin{figure*}[htbp]
\centering
\includegraphics[width=0.95\textwidth]{figures/panel_arg6_dissolution_second_law.png}
\caption{\textbf{Argument 6: Dissolution of Second Law Violation—Categorical Entropy Increase Compensates.}
\textbf{(A)} Two entropy components with opposite trends. Spatial entropy $S_{\text{spatial}}$ (red line) decreases during apparent sorting as molecules become spatially segregated, following $S_{\text{spatial}} = -k_B \sum_i p_i^{\text{spatial}} \ln p_i^{\text{spatial}}$. Categorical entropy $S_{\text{categorical}}$ (green line) increases as the phase-lock network densifies, with $S_{\text{categorical}} = -k_B \sum_{\alpha} p_{\alpha}^{\text{cat}} \ln p_{\alpha}^{\text{cat}}$ where $\alpha$ indexes categorical states. Total entropy $S_{\text{total}} = S_{\text{spatial}} + S_{\text{categorical}}$ (dark teal dashed line) always increases, satisfying the second law. The gray dotted line at $S = 2.0$ marks the initial equilibrium value. The divergence of spatial and categorical components reveals the hidden entropy production.
\textbf{(B)} Network densification produces categorical entropy. The number of network edges increases from $\sim 100$ to $264$ over 50 sorting attempts, representing a gain of $+164$ edges (marked in red). Network density $\rho = 2|E|/(|V|(|V|-1))$ increases as molecules form more phase-lock connections. This densification corresponds to increased categorical entropy: $\Delta S_{\text{categorical}} = k_B \ln(\Omega_{\text{final}}/\Omega_{\text{initial}})$ where $\Omega$ is the number of accessible categorical states. The filled green area under the curve represents accumulated categorical entropy. The steep increase demonstrates that apparent sorting creates extensive network structure.
\textbf{(C)} Total entropy change distribution confirms $\Delta S_{\text{total}} > 0$. Histograms show the distribution of entropy changes across many trials. Spatial entropy changes (red) are predominantly negative ($\Delta S_{\text{spatial}} < 0$, left of dashed line at $\Delta S = 0$), confirming apparent sorting. Categorical entropy changes (green) are predominantly positive ($\Delta S_{\text{categorical}} > 0$, right of line). Crucially, total entropy changes (dark teal) are always positive, with the distribution centered at $\Delta S_{\text{total}} \approx +0.2$ (marked by green text ``$\Delta S > 0$''). The vertical dashed line at $\Delta S = 0$ separates second law violations (left, forbidden) from allowed processes (right). No trials violate the second law.
\textbf{(D)} Second law accounting shows net entropy increase. Bar chart quantifying entropy changes: spatial entropy decreases by $\Delta S_{\text{spatial}} = -0.3$ (red bar, apparent violation), categorical entropy increases by $\Delta S_{\text{categorical}} = +0.5$ (green bar, hidden compensation), yielding total entropy increase $\Delta S_{\text{total}} = +0.2 > 0$ (dark teal bar, second law satisfied). The numerical values demonstrate that categorical entropy production exceeds spatial entropy reduction by a factor of $\sim 1.7$, providing a comfortable margin. The second law is never violated; the demon's apparent sorting is compensated by hidden network entropy. This resolves the paradox: there is no thermodynamic violation because categorical completion increases total entropy.}
\label{fig:dissolution_second_law}
\end{figure*}

\subsection{Entropy as Shortest Path}

\begin{definition}[Oscillatory Termination Probability]
\label{def:termination_probability}
For a system in categorical state $C$, the \textbf{oscillatory termination probability} $\alpha(C)$ is the probability that oscillatory dynamics reach equilibrium (terminate) at state $C$.
\end{definition}

\begin{theorem}[Entropy as Path Length]
\label{thm:entropy_path}
Entropy is inversely related to the shortest path length to oscillatory termination:
\begin{equation}
S(C) = -k_B \log \ell_{\text{term}}(C)
\label{eq:entropy_path}
\end{equation}
where $\ell_{\text{term}}(C)$ is the shortest path from $C$ to any termination state in $\catspace$.
\end{theorem}

\begin{proof}
The termination probability scales inversely with path length:
\begin{equation}
\alpha(C) \propto \frac{1}{\ell_{\text{term}}(C)}
\end{equation}
Systems with longer paths to termination have lower probability of terminating at the current state.

Define entropy through termination probability:
\begin{equation}
S(C) = k_B \log \alpha(C) = k_B \log \frac{1}{\ell_{\text{term}}(C)} = -k_B \log \ell_{\text{term}}(C)
\end{equation}

Higher entropy corresponds to shorter paths to termination---the system is ``closer'' to equilibrium in categorical space. \qed
\end{proof}

\begin{corollary}[Entropy Increase as Path Optimisation]
\label{cor:path_optimisation}
The second law of thermodynamics states that entropy increases:
\begin{equation}
\frac{dS}{dt} \geq 0
\end{equation}

In path-length terms, this becomes:
\begin{equation}
\frac{d\ell_{\text{term}}}{dt} \leq 0
\end{equation}

Systems evolve toward shorter paths to termination---they optimise their route to equilibrium through categorical space.
\end{corollary}

\subsection{Why ``Sorting'' Increases Entropy}

\begin{theorem}[Sorting Increases Network Density]
\label{thm:sorting_density}
The operation attributed to Maxwell's Demon---categorical selection and pathway following---increases phase-lock network density, hence increases entropy.
\end{theorem}

\begin{proof}
Consider the ``demon operation'' as categorical completion:

\textbf{Step 1: Initial selection.}
Selecting a categorical state $C_1$ from equivalence class $[C]_{\text{spatial}}$ completes that state, making adjacent states accessible.

\textbf{Step 2: Cascade propagation.}
From Theorem~\ref{thm:categorical_cascade}, selection initiates a cascade through phase-lock network. Each step completes new categorical states.

\textbf{Step 3: Network densification.}
As more categorical states are completed, the effective phase-lock network densifies:
\begin{equation}
|E(\gamma(t_2))| > |E(\gamma(t_1))| \quad \text{for } t_2 > t_1
\end{equation}
where $\gamma(t)$ is the completed state set.

This occurs because:
\begin{itemize}
    \item New phase relationships are established as states are completed
    \item Completed states cannot be un-completed (Axiom~\ref{axiom:categorical_irreversibility})
    \item Each completion adds constraints (edges) to the effective network
\end{itemize}

\textbf{Step 4: Entropy increase.}
From Proposition~\ref{prop:entropy_edge_density}:
\begin{equation}
\Delta S = k_B \Delta |E| > 0
\end{equation}

The ``demon operation'' increases entropy. \qed
\end{proof}

\begin{corollary}[Second Law Preserved]
\label{cor:second_law}
Maxwell's Demon, reinterpreted as categorical completion through phase-lock topology, does not violate the second law. The apparent paradox arose from:
\begin{enumerate}
    \item Misidentifying the demon's operation (sorting by kinetic energy vs. categorical completion)
    \item Ignoring categorical degrees of freedom (phase-lock structure)
    \item Focusing on spatial entropy while ignoring categorical entropy
\end{enumerate}

When categorical structure is properly accounted for, entropy increases monotonically:
\begin{equation}
\frac{dS_{\text{total}}}{dt} = \frac{dS_{\text{spatial}}}{dt} + \frac{dS_{\text{categorical}}}{dt} \geq 0
\end{equation}
even if $dS_{\text{spatial}}/dt < 0$ (apparent ordering), because $dS_{\text{categorical}}/dt > 0$ (network densification) dominates.
\end{corollary}


%==============================================================================
\section{The Dissolution of Maxwell's Demon}
\label{sec:dissolution}
%==============================================================================

\subsection{Restatement of the Paradox}

Maxwell's thought experiment posits a being that:
\begin{enumerate}
    \item Observes molecules approaching a door between two chambers
    \item Measures their velocities to classify them as ``fast'' or ``slow''
    \item Opens the door selectively to allow fast molecules to pass one way, slow molecules the other
    \item Creates a temperature difference from thermal equilibrium without doing work
\end{enumerate}

The paradox: this appears to violate the second law, $\Delta S \geq 0$.

\subsection{Three Decisive Insights}

Before analysing the demon's purported operations, we establish three fundamental results that independently dissolve the paradox.

\begin{theorem}[Temporal Triviality of the Demon]
\label{thm:temporal_triviality}
The demon is temporally redundant: any configuration it purportedly creates will occur naturally through thermal fluctuations given sufficient time.
\end{theorem}

\begin{proof}
In statistical mechanics, the probability of any configuration $\Gamma$ is given by the Boltzmann distribution:
\begin{equation}
P(\Gamma) = \frac{e^{-E(\Gamma)/k_B T}}{Z} > 0 \quad \forall \text{ configurations } \Gamma
\end{equation}
where $Z = \sum_\Gamma e^{-E(\Gamma)/k_B T}$ is the partition function.

Crucially, $P(\Gamma) > 0$ for \textit{every} configuration, including the ``sorted'' state the demon supposedly creates. The Poincaré recurrence theorem guarantees that an isolated system will return arbitrarily close to any configuration in finite (though possibly astronomically long) time:
\begin{equation}
\forall \epsilon > 0, \exists T_{\text{rec}} < \infty : |\Gamma(T_{\text{rec}}) - \Gamma_{\text{sorted}}| < \epsilon
\end{equation}

\begin{figure*}[htbp]
\centering
\includegraphics[width=0.95\textwidth]{figures/arg3_retrieval_paradox.png}
\caption{\textbf{Argument 3: The Retrieval Paradox—Velocity-Based Sorting is Self-Defeating.}
\textbf{(A)} Timescale hierarchy showing collisions happen first. Molecular collision timescale $\tau_{\text{coll}} \sim 10^{-10}$ s (red bar) is orders of magnitude faster than measurement ($\sim 10^{-8}$ s), gate operation ($\sim 10^{-6}$ s), sorting ($\sim 10^{-3}$ s), and demon decision-making ($\sim 10^{-1}$ s). The collision rate $\nu_{\text{coll}} \sim 10^{10}$ collisions/s in gases at STP ensures velocities randomize before any sorting operation can complete. The label ``TOO SLOW!'' emphasizes that sorting timescale exceeds thermalization timescale by $10^7$, making velocity-based sorting operationally impossible.
\textbf{(B)} Phase space scrambling showing sorted states randomize in $\tau_{\text{coll}}$. Initially sorted molecules (blue points, $t=0$) with velocities clustered in one region of phase space become completely randomized (red points, $t > \tau_{\text{coll}}$) after a single collision time. The velocity distribution returns to Maxwell-Boltzmann, erasing any sorting. This demonstrates that maintaining sorted states requires infinite retrieval operations.
\textbf{(C)} Sorting versus thermalization dynamics. The sorting signal strength $S(t) = |\langle v_A \rangle - \langle v_B \rangle|/\sigma_v$ (teal curve) decays exponentially as $S(t) = S_0 \exp(-t/\tau_{\text{coll}})$, while thermal randomization (red shaded region) dominates. The system relaxes to equilibrium ($S \to 0$) within $\sim 2\tau_{\text{coll}}$. The demon's sorting attempt (starting from yellow circle at $S_0 \approx 0.9$) is overwhelmed by thermalization.
\textbf{(D)} Long-term sorting attempts always return to 50/50 equilibrium. Three independent sorting attempts (colored traces) show fast/total molecule ratio fluctuating around equilibrium value 0.5 (red dashed line). Despite initial deviations, all attempts converge to the equilibrium distribution within $\sim 50$ time steps. The red shaded band indicates $\pm 2\sigma$ fluctuations. This confirms that velocity-sorted states cannot be maintained: the retrieval paradox makes the demon's operation self-defeating.}
\label{fig:retrieval_paradox}
\end{figure*}


Therefore, the demon does not create anything that would not occur naturally. It merely (supposedly) accelerates what statistical mechanics already predicts. But acceleration is not violation---the second law concerns what \textit{can} happen, not how \textit{quickly} it happens.

The demon is temporally trivial: it is redundant with natural fluctuations. \qed
\end{proof}

\begin{theorem}[Phase-Lock Temperature Independence]
\label{thm:phase_lock_temperature_independence}
The same phase-lock network topology (spatial arrangement and categorical structure) can exist at any temperature. A ``snapshot'' of the system---frozen positions and phase relationships---is temperature-independent.
\end{theorem}

\begin{proof}
Consider a snapshot of the system at time $t$: a frozen configuration with definite molecular positions $\{\mathbf{r}_i\}$ and phase-lock relationships $\phaselockgraph$.

In this snapshot, molecules have:
\begin{enumerate}
    \item Positions $\mathbf{r}_i$ (which determine phase-lock structure)
    \item Velocities $\mathbf{v}_i$ (which contribute to kinetic energy and temperature)
\end{enumerate}

These are \textbf{independent variables}. The phase-lock network depends only on positions:
\begin{equation}
\phaselockgraph = \phaselockgraph(\{\mathbf{r}_i\}) \neq \phaselockgraph(\{\mathbf{v}_i\})
\end{equation}

The same spatial arrangement can occur with:
\begin{itemize}
    \item Low velocities (low temperature)
    \item High velocities (high temperature)
    \item Any velocity distribution consistent with the positions
\end{itemize}

Temperature is a statistical average over velocity distributions:
\begin{equation}
T = \frac{2}{3 N k_B} \sum_i \frac{1}{2} m_i |\mathbf{v}_i|^2
\end{equation}

Since velocities are independent of positions in a snapshot, the \textit{same} phase-lock graph $\phaselockgraph$ can correspond to \textit{any} temperature. The categorical structure is temperature-independent.

This means: rearrangement of molecules according to phase-lock topology (categorical completion) is not ``sorting by temperature.'' The same categorical pathway exists whether the system is at 100 K or 1000 K. \qed
\end{proof}

\begin{corollary}[The Snapshot Principle]
\label{cor:snapshot}
In any snapshot (frozen instant), molecular positions can be rearranged along phase-lock adjacency pathways without reference to velocity. The ``sorting'' attributed to the demon is rearrangement by phase-lock structure, which is velocity-blind.
\end{corollary}

\begin{theorem}[The Retrieval Paradox]
\label{thm:retrieval_paradox}
A demon that sorts by molecular velocity is self-defeating: thermal equilibration continuously randomises velocities, requiring infinite retrieval operations.
\end{theorem}

\begin{proof}
Suppose the demon successfully ``sorts'' molecule $A$ into the hot chamber based on its velocity $v_A > v_{\text{threshold}}$ at time $t_0$.

After sorting, molecule $A$ undergoes collisions with other molecules. The collision frequency in an ideal gas is:
\begin{equation}
\nu_{\text{collision}} = n \sigma \langle v \rangle \approx 10^{10} \text{ s}^{-1}
\end{equation}
for standard conditions, where $n$ is number density, $\sigma$ is collision cross-section, and $\langle v \rangle$ is mean velocity.

After a collision at time $t_1 = t_0 + \tau_{\text{collision}}$, molecule $A$ has new velocity $v_A'$, which may be less than $v_{\text{threshold}}$. If $v_A' < v_{\text{threshold}}$, molecule $A$ is now ``slow'' and in the wrong chamber.

The demon must:
\begin{enumerate}
    \item Detect that $A$ has become ``slow''
    \item Retrieve $A$ from the hot chamber
    \item Return $A$ to the cold chamber
\end{enumerate}

But during retrieval, $A$ collides again and may become ``fast.'' The demon enters an infinite loop of sorting and retrieval.

For $N$ molecules with collision frequency $\nu$, the demon must process:
\begin{equation}
\text{Operations per second} \sim N \cdot \nu \sim 10^{23} \times 10^{10} = 10^{33} \text{ s}^{-1}
\end{equation}

This exceeds any physical limit. More fundamentally, the demon cannot ``keep up'' with thermal equilibration. Velocity-based sorting is self-defeating because:
\begin{equation}
\tau_{\text{sorting}} \gg \tau_{\text{equilibration}} \implies \text{Sorting is futile}
\end{equation}

The demon cannot maintain a velocity-sorted state against thermal relaxation. \qed
\end{proof}

\begin{corollary}[Velocity Is the Wrong Criterion]
\label{cor:wrong_criterion}
The demon's failure is not due to information costs or measurement disturbance. It fails because velocity is not a stable molecular property---it changes on the collision timescale. Sorting by velocity is like sorting waves by their instantaneous height: the criterion changes faster than sorting can occur.
\end{corollary}

\subsection{The Dissolution}

With Theorems~\ref{thm:temporal_triviality}--\ref{thm:retrieval_paradox} established, we now show that each step of the demon's operation is either unnecessary, misconceived, or automatically entropy-increasing.

\begin{theorem}[Dissolution of Observation]
\label{thm:dissolution_observation}
The demon's ``observation'' of molecular velocities is unnecessary because phase-lock network topology encodes categorical structure without measurement.
\end{theorem}

\begin{proof}
From Theorem~\ref{thm:kinetic_independence}, the phase-lock network $\phaselockgraph$ is determined by spatial configuration and molecular properties, not velocities.

From Theorem~\ref{thm:phase_lock_accessibility}, categorical accessibility is determined by network topology.

Therefore, the system's categorical structure---which states are accessible from which---is fully determined without any velocity measurement. The ``information'' about molecular arrangement is structural, encoded in $\phaselockgraph$, not acquired through observation.

The demon need not observe velocities because categorical dynamics do not depend on them. \qed
\end{proof}

\begin{theorem}[Dissolution of Decision]
\label{thm:dissolution_decision}
The demon's ``decision'' to open or close the door is unnecessary because categorical completion follows network topology deterministically.
\end{theorem}

\begin{proof}
From Theorem~\ref{thm:information_free}, categorical selection is determined by:
\begin{equation}
C^* = \argmin_{C \in \accessible(C_{\text{prev}}) \cap [C]_{\text{spatial}}} d_{\catspace}(C, C_{\text{prev}})
\end{equation}

This selection follows from:
\begin{enumerate}
    \item Previous categorical state $C_{\text{prev}}$ (given)
    \item Network topology determining $\accessible(C_{\text{prev}})$ (structural)
    \item Physical dynamics selecting among accessible states (deterministic or stochastic but not deliberative)
\end{enumerate}

No ``decision'' by an agent is required. The categorical dynamics are self-executing. \qed
\end{proof}

\begin{theorem}[Dissolution of Sorting]
\label{thm:dissolution_sorting}
The demon's ``sorting'' by temperature is a misinterpretation of categorical completion through phase-lock pathways.
\end{theorem}

\begin{proof}
From Theorem~\ref{thm:demon_cannot_sort}, temperature is not a molecular attribute but an emergent macroscopic property. Molecules have kinetic energies, not temperatures.

From Theorem~\ref{thm:kinetic_independence}, kinetic energy does not determine phase-lock network topology.

From Theorem~\ref{thm:apparent_sorting}, molecules in the same phase-lock cluster have correlated kinetic energies due to shared molecular properties---not because kinetic energy determines clustering.

When molecules appear ``sorted by temperature,'' they are actually:
\begin{enumerate}
    \item Following categorical pathways determined by phase-lock topology
    \item Clustering by phase-lock adjacency, not kinetic similarity
    \item Exhibiting kinetic correlations that are consequences, not causes, of clustering
\end{enumerate}

The ``sorting'' reveals pre-existing categorical structure rather than creating order from measurement. \qed
\end{proof}

\begin{theorem}[Dissolution of Second Law Violation]
\label{thm:dissolution_second_law}
The apparent decrease in entropy is an artefact of ignoring categorical degrees of freedom. Total entropy increases.
\end{theorem}

\begin{proof}
From Theorem~\ref{thm:sorting_density}, the demon operation---categorical completion through phase-lock pathways---increases network density:
\begin{equation}
|E(\gamma(t_{\text{final}}))| > |E(\gamma(t_{\text{initial}}))|
\end{equation}

From Corollary~\ref{cor:second_law}, total entropy satisfies:
\begin{equation}
\Delta S_{\text{total}} = \Delta S_{\text{spatial}} + \Delta S_{\text{categorical}} \geq 0
\end{equation}

Even if spatial entropy appears to decrease (molecules ``sorted'' into hot and cold chambers), categorical entropy increases due to network densification.

The second law is not violated; it was never threatened. The paradox arose from incomplete entropy accounting. \qed
\end{proof}

\subsection{The Demon as Categorical Completion}

\begin{theorem}[Identity Theorem]
\label{thm:identity}
Maxwell's Demon is identical to categorical completion through phase-lock network topology:
\begin{equation}
\boxed{\text{``Maxwell's Demon''} \equiv \text{Categorical Completion}(\phaselockgraph)}
\end{equation}
\end{theorem}

\begin{proof}
We establish a complete correspondence:

\begin{center}
\begin{tabular}{l|l}
\textbf{Demon Operation} & \textbf{Categorical Process} \\
\hline
Observe molecule & Complete categorical state $C_i$ \\
Measure velocity & (Unnecessary---topology determines accessibility) \\
Classify fast/slow & Identify phase-lock cluster membership \\
Open door & Make adjacent states accessible \\
Close door & Categorical irreversibility prevents return \\
Sort molecules & Follow phase-lock pathways \\
Create $\Delta T$ & Reveal cluster structure (correlated with $T$) \\
\end{tabular}
\end{center}

Every demon operation has a categorical counterpart that:
\begin{itemize}
    \item Requires no external agent
    \item Requires no information acquisition
    \item Follows automatically from network topology
    \item Increases entropy rather than decreasing it
\end{itemize}

The demon is not needed because categorical completion through phase-lock topology accomplishes the same apparent effect. But this is not a demon ``in disguise''---it is the recognition that no demon was ever required. The physical process is categorical completion, which was always entropy-increasing. \qed
\end{proof}

\subsection{Why Maxwell Saw a Demon: Information Complementarity}

\begin{theorem}[Information Complementarity]
\label{thm:information_complementarity}
Information has two conjugate faces that cannot be simultaneously observed. Maxwell saw a ``demon'' because he was observing one face of information while the dynamics of the conjugate face remained hidden.
\end{theorem}

\begin{proof}
Every categorical state has a conjugate representation:
\begin{align}
\mathbf{S}_{\text{front}} &= (S_{k,f}, S_{t,f}, S_{e,f}) \quad \text{(observable face)} \\
\mathbf{S}_{\text{back}} &= (S_{k,b}, S_{t,b}, S_{e,b}) \quad \text{(hidden face)}
\end{align}
related by a conjugate transformation $T$:
\begin{equation}
\mathbf{S}_{\text{back}} = T(\mathbf{S}_{\text{front}})
\end{equation}

This is not a quantum effect but a classical measurement constraint, analogous to the ammeter/voltmeter complementarity in electrical circuits:
\begin{center}
\begin{tabular}{l|l}
\textbf{Electrical Circuit} & \textbf{Categorical Information} \\
\hline
Ammeter measures current $I$ directly & Observer sees kinetic energy face \\
Voltmeter measures voltage $V$ directly & Observer sees categorical structure face \\
Cannot measure both at same point & Cannot observe both faces simultaneously \\
$V$ derived from $I$ via Ohm's law & Hidden face derived from observable via $T$
\end{tabular}
\end{center}

Maxwell observed the \textit{kinetic energy face}: molecules with velocities, temperatures, and apparent ``sorting.'' The \textit{categorical structure face}---the phase-lock network topology and categorical completion dynamics---was hidden from his view.

When you observe only one face, the dynamics of the conjugate face appear as \textit{external intervention}. Maxwell attributed these hidden dynamics to an intelligent agent: the demon.

But the ``demon'' was not an agent. It was the conjugate face of information completing categorical states according to phase-lock topology. The ``sorting'' Maxwell observed was the projection of categorical completion onto the kinetic energy face. \qed
\end{proof}

\begin{figure*}[htbp]
\centering
\includegraphics[width=0.95\textwidth]{figures/panel_arg7_information_complementarity.png}
\caption{\textbf{Argument 7: Information Complementarity—The Demon is a Projection Artifact.}
\textbf{(A)} Two complementary faces of information. Venn diagram showing the kinetic face (red circle: velocities, energy, temperature) and categorical face (purple circle: network topology, phase-lock structure) with minimal overlap (small purple square in center). The two faces are complementary: observing one face renders the other hidden, analogous to conjugate observables in quantum mechanics. The annotation ``Complementary: cannot observe both simultaneously'' emphasizes measurement incompatibility. Maxwell observed only the kinetic face; the categorical face remained hidden, creating the illusion of a demon.
\textbf{(B)} Ammeter-voltmeter analogy. Schematic of an electrical component with ammeter (A, red) measuring current and voltmeter (V, purple) measuring voltage. The fundamental constraint ``Cannot use both meters simultaneously on same element'' illustrates complementarity: inserting an ammeter (low resistance) changes the circuit, making voltage measurement impossible, and vice versa. Similarly, observing molecular velocities (kinetic face) obscures phase-lock network structure (categorical face). The demon paradox arises from observing only one meter while the other remains hidden.
\textbf{(C)} Demon as projection artifact. Schematic showing categorical dynamics (purple box, hidden) projecting onto the kinetic face (red box, observed). The demon (yellow box with annotation ``DEMON = Shadow of hidden dynamics'') is not an agent but a projection artifact—the shadow cast by categorical completion onto the observable kinetic face. Three downward arrows represent multiple projection paths from hidden categorical dynamics to observed kinetic behavior. What Maxwell interpreted as intelligent sorting is actually the visible manifestation of automatic topological navigation occurring on the hidden face. The demon is an epiphenomenon, not a causal agent.
\textbf{(D)} Complete picture resolves the paradox. Two-column comparison showing Maxwell's incomplete view versus the complete picture. \textit{Left column (Maxwell's View)}: Observing kinetic properties only (``Kinetic only $\to$'') leads to the interpretation of ``Demon sorting'' (red text)—an apparent agent performing intelligent operations. \textit{Right column (Complete View)}: Observing both faces (``Both faces $\to$'') reveals ``Automatic topology'' (green text)—deterministic categorical completion through phase-lock networks. The yellow box at bottom states the resolution: ``NO DEMON EXISTS / Only categorical completion / along network topology.'' The vertical dashed line separates incomplete from complete understanding. The paradox dissolves when both faces are visible: what appeared to require an information-processing demon is revealed as automatic navigation through categorical state space, visible only from the complementary face. This is the deepest resolution: the demon was never real, only a shadow of hidden dynamics.}
\label{fig:information_complementarity}
\end{figure*}


\begin{corollary}[The Demon as Projection]
\label{cor:demon_projection}
Maxwell's Demon is the projection of hidden categorical dynamics onto the observable kinetic face:
\begin{equation}
\text{``Demon''} = \Pi_{\text{kinetic}}\left(\frac{d\mathbf{S}_{\text{categorical}}}{dt}\right)
\end{equation}
where $\Pi_{\text{kinetic}}$ is the projection operator onto the observable (kinetic) face.
\end{corollary}

\begin{remark}[Why the Demon Appeared Intelligent]
The demon appeared to make ``decisions'' because categorical completion follows network topology---a structured, non-random process. When this structured process is projected onto the kinetic face, it appears as purposeful selection. But the ``purpose'' is topological, not intentional. The phase-lock network already encodes which states are adjacent; ``opening the door'' is following adjacency, not deciding.
\end{remark}

\begin{theorem}[Face-Switching Dissolves the Demon]
\label{thm:face_switching}
If Maxwell had been able to observe the categorical face instead of the kinetic face, no demon would have appeared. The ``sorting'' would be revealed as categorical completion through phase-lock pathways.
\end{theorem}

\begin{proof}
On the kinetic face, molecules appear to be sorted by velocity. An agent seems required to select which molecules pass.

On the categorical face, molecules are nodes in a phase-lock network. Categorical completion follows network adjacency. No selection occurs; the system follows topological pathways.

The same physical process appears differently on different faces:
\begin{center}
\begin{tabular}{l|l}
\textbf{Kinetic Face (Maxwell's View)} & \textbf{Categorical Face (Phase-Lock View)} \\
\hline
Molecules moving with velocities & Nodes in phase-lock network \\
``Fast'' and ``slow'' classification & Phase-lock cluster membership \\
Door opening/closing & Adjacent states becoming accessible \\
Agent making decisions & Topological navigation \\
Apparent entropy decrease & Categorical entropy increase \\
Demon required & No agent required
\end{tabular}
\end{center}

The demon is an artefact of the observable face, not a feature of the physical process. \qed
\end{proof}

\subsection{Why the Paradox Persisted}

\begin{proposition}[Source of the Paradox]
\label{prop:paradox_source}
Maxwell's Demon paradox persisted for 150 years due to four conceptual errors:
\begin{enumerate}
    \item \textbf{Observing only one face of information}: Privileging the kinetic (velocity/temperature) face while the categorical (phase-lock structure) face remained hidden
    \item \textbf{Treating molecules as independent}: Ignoring phase-lock network structure
    \item \textbf{Privileging kinetic energy}: Assuming velocity determines categorical behaviour
    \item \textbf{Incomplete entropy accounting}: Ignoring categorical degrees of freedom
\end{enumerate}
\end{proposition}

\begin{proof}
\textbf{(1) Single-face observation:}
Maxwell and subsequent analysts observed molecular systems through the kinetic face: velocities, temperatures, and configurational entropy. The conjugate categorical face---phase-lock networks and categorical completion---was not accessible to their theoretical framework.

When dynamics occur on the hidden face, they must be explained through the observable face. The most parsimonious explanation for structured, non-random ``sorting'' on the kinetic face was an intelligent agent. Hence, the demon.

\textbf{(2) Independent particle assumption:}
Classical statistical mechanics treats molecules as independent particles whose only interactions are collisions. This ignores the persistent phase-lock relationships through Van der Waals and dipole forces that create network structure.

With independent particles, ``sorting'' would require external information to distinguish molecules. With networked particles, categorical structure already distinguishes them.

\textbf{(3) Kinetic energy privilege:}
The thought experiment assumes the demon sorts by velocity---a kinetic property. But Theorem~\ref{thm:kinetic_independence} establishes that phase-lock networks are kinetically independent. The demon cannot sort by a property that does not determine categorical structure.

\textbf{(4) Incomplete entropy:}
Traditional analysis computes $\Delta S_{\text{spatial}}$ (configurational entropy from particle positions) while ignoring $\Delta S_{\text{categorical}}$ (entropy from phase-lock structure). Since categorical completion always increases $S_{\text{categorical}}$, accounting for it dissolves the paradox.

These errors led to positing an information-processing demon where none was needed, then searching for where it ``hides'' entropy (in measurement, in memory, in erasure) when the entropy was always increasing through network densification---visible only on the categorical face. \qed
\end{proof}

\subsection{Final Statement}

\begin{theorem}[Non-Existence of the Demon]
\label{thm:nonexistence}
Maxwell's Demon does not exist. The thought experiment describes categorical completion through phase-lock network topology---a physical process requiring:
\begin{enumerate}
    \item No intelligent agent
    \item No information acquisition or processing
    \item No violation of the second law
\end{enumerate}

The demon is the null set:
\begin{equation}
\text{``Maxwell's Demon''} = \varnothing
\end{equation}
\end{theorem}

\begin{proof}
From Theorems~\ref{thm:dissolution_observation}--\ref{thm:dissolution_second_law}, every aspect of the demon's purported operation is either unnecessary or misconceived.

From Theorem~\ref{thm:identity}, the physical process attributed to the demon is categorical completion through phase-lock topology.

Categorical completion is a physical process, not an agent. It has no intentionality, no information processing, no decision-making. To call it a ``demon'' is a category error.

Therefore, Maxwell's Demon---as an information-processing agent that sorts molecules by temperature---does not exist. What exists is phase-lock network topology and categorical completion dynamics. These are not a demon; they are physics. \qed
\end{proof}

\begin{remark}[The Resolution Complete]
We have shown that Maxwell's Demon dissolves under categorical analysis through seven independent arguments:

\textbf{(1) The demon is temporally redundant} (Theorem~\ref{thm:temporal_triviality}): Any configuration it creates will occur naturally through fluctuations.

\textbf{(2) The demon sorts the wrong property} (Theorem~\ref{thm:phase_lock_temperature_independence}): Phase-lock structure is temperature-independent; the same categorical arrangement exists at any temperature.

\textbf{(3) The demon is self-defeating} (Theorem~\ref{thm:retrieval_paradox}): Velocity-based sorting requires infinite retrieval operations against thermal equilibration.

\textbf{(4) The demon cannot sort by temperature} (Theorem~\ref{thm:dissolution_sorting}): Phase-lock networks are kinetically independent.

\textbf{(5) The demon needs no information} (Theorem~\ref{thm:dissolution_observation}): Categorical structure is topological, not acquired.

\textbf{(6) The demon violates no laws} (Theorem~\ref{thm:dissolution_second_law}): Categorical completion increases entropy.

\textbf{(7) The demon is a projection artefact} (Theorem~\ref{thm:information_complementarity}): Information has two conjugate faces; There is no demon.

There is only the phase-lock network, completing categorical states according to topology, revealing structure that was always present and increasing entropy as the second law demands. Maxwell saw a demon because he was looking at one face of information; the ``demon'' was the hidden face doing what physics demands. Maxwell saw a demon because he was looking at one face of information; the ``''
\end{remark}

\subsection{Summary of the Seven-Fold Dissolution}

\begin{table}[h]
\centering
\begin{tabular}{l|l|l}
\textbf{Demon Claim} & \textbf{Dissolution} & \textbf{Theorem} \\
\hline
Creates special configuration & Natural fluctuations produce same & \ref{thm:temporal_triviality} \\
Sorts by temperature & Same arrangement at any $T$ & \ref{thm:phase_lock_temperature_independence} \\
Maintains sorted state & Cannot outpace equilibration & \ref{thm:retrieval_paradox} \\
Measures velocity & Topology doesn't depend on $v$ & \ref{thm:dissolution_observation} \\
Makes sorting decisions & Categorical pathways automatic & \ref{thm:dissolution_decision} \\
Decreases entropy & Categorical entropy increases & \ref{thm:dissolution_second_law} \\
Exists as agent & Projection of hidden face dynamics & \ref{thm:information_complementarity}
\end{tabular}
\caption{The seven-fold dissolution of Maxwell's Demon.}
\label{tab:dissolution_summary}
\end{table}

\begin{remark}[The Deepest Resolution]
The seventh argument---information complementarity---explains not only why the demon does not exist, but \textit{why Maxwell and others saw a demon in the first place}. The demon was not a failure of imagination or a deliberate puzzle; it was the inevitable consequence of observing one face of a two-faced information structure. Any observer confined to the kinetic face will see ``sorting'' and require an agent to explain it. The agent dissolves the moment the observer gains access to the categorical face.
\end{remark}



%==============================================================================
\section{Conclusion}
\label{sec:conclusion}
%==============================================================================

\subsection{Summary of Results}

We have presented a complete resolution of Maxwell's Demon paradox through the theory of categorical phase-lock networks. The resolution rests on six independent pillars:

\textbf{(1) Temporal triviality.} The demon is redundant. Any configuration it purportedly creates will occur naturally through thermal fluctuations (Poincaré recurrence). The demon accelerates what statistical mechanics already predicts will happen---but acceleration is not violation.

\textbf{(2) Phase-lock temperature independence.} A ``snapshot'' of the system---frozen molecular positions and phase-lock relationships---can exist at any temperature. The same spatial arrangement is compatible with 100 K or 1000 K. The demon's ``sorting'' is rearrangement by phase-lock structure, which is velocity-blind.

\textbf{(3) The retrieval paradox.} Velocity-based sorting is self-defeating. Thermal equilibration occurs on the collision timescale ($\sim 10^{-10}$ s), randomising velocities continuously. A demon sorting by velocity must retrieve molecules that change speed after sorting---requiring $\sim 10^{33}$ operations per second, an infinite loop of sorting and retrieval.

\textbf{(4) Phase-lock kinetic independence.} The interactions forming phase-lock relationships---Van der Waals forces, dipole couplings, vibrational synchronisation---depend on spatial configuration and electronic structure, not molecular velocity. Theorem~\ref{thm:kinetic_independence_intro} establishes $\partial \phaselockgraph / \partial E_{\text{kin}} = 0$: network topology is blind to kinetic energy.

\textbf{(5) Categorical-physical distance inequivalence.} Molecules can be categorically adjacent (phase-locked) while physically distant, and physically proximate while categorically separated. The categorical state space $\catspace$ has geometry determined by phase-lock topology, not Euclidean metrics.

\textbf{(6) Temperature emergence.} Temperature is a macroscopic observable that emerges from the statistical properties of phase-lock clusters, not a sorting criterion. The correlation between phase-lock structure and kinetic energy is real but not causal.

\textbf{(7) Information complementarity.} Information has two conjugate faces---the kinetic face (velocities, temperatures) and the categorical face (phase-lock networks, categorical completion)---that cannot be simultaneously observed, analogous to ammeter/voltmeter complementarity in electrical circuits. Maxwell observed only the kinetic face; the ``demon'' was the projection of hidden categorical dynamics onto his observable face. The demon appeared intelligent because categorical completion follows structured (topological) pathways, which look like purposeful selection when projected onto the kinetic face.

\subsection{The Dissolution}

Maxwell's Demon does not violate the second law because there is no demon. The thought experiment posits an agent that:
\begin{enumerate}
    \item Measures molecular velocities
    \item Makes decisions based on measurements
    \item Controls a door to sort molecules
    \item Creates temperature differences without work
    \item Maintains the sorted state
\end{enumerate}

Our analysis reveals that each step is either unnecessary, misconceived, or impossible:

\begin{enumerate}
    \item \textbf{No measurement needed}: Phase-lock network topology encodes categorical structure without any measurement. The ``information'' about which molecules belong together is structural, not acquired.

    \item \textbf{No decisions required}: Categorical completion follows network topology deterministically. Accessible states are determined by phase-lock adjacency, not by deliberation.

    \item \textbf{No door operation}: The partition between categorical clusters is topological, not physical. ``Opening the door'' is selecting a categorical state, which makes phase-lock adjacent states accessible.

    \item \textbf{No sorting by temperature}: Phase-lock structure is temperature-independent. The same categorical arrangement exists at any temperature---a snapshot of positions is velocity-blind. The demon sorts the wrong property.

    \item \textbf{No maintenance possible}: Even if sorting occurred, the demon cannot maintain it. Thermal equilibration randomises velocities on the collision timescale ($10^{-10}$ s). The demon would require infinite retrieval operations, defeating itself.

    \item \textbf{No special outcome}: The ``sorted'' configuration will occur naturally through fluctuations. The demon is temporally redundant---it creates nothing that wouldn't happen anyway.
\end{enumerate}

The demon dissolves into categorical completion:
\begin{equation}
\boxed{\text{``Maxwell's Demon''} \equiv \text{Categorical Completion through Phase-Lock Topology}}
\end{equation}

\subsection{Relationship to Information-Theoretic Resolutions}

Our resolution does not contradict Landauer-Bennett but renders it unnecessary for the core paradox. Information-theoretic resolutions correctly identify entropy costs of measurement and erasure---these costs are real. However, they address a demon that need not exist.

If one insists on constructing a physical demon (an actual device that measures and sorts), then information-theoretic constraints apply. But Maxwell's original thought experiment---and the thermodynamic puzzle it poses---dissolves once we recognise that phase-lock topology does the ``sorting'' without any agent.

\subsection{Implications}

The resolution has several implications:

\textbf{For thermodynamics}: The second law is preserved not through information costs but through categorical irreversibility. Entropy increases because categorical completion densifies phase-lock networks, regardless of apparent ``sorting.''

\textbf{For statistical mechanics}: Temperature is properly understood as emergent from categorical structure, not as a primitive quantity that determines molecular behaviour.

\textbf{For information theory}: The information content of a physical system resides in its categorical structure (phase-lock topology), not in externally acquired measurements.

\textbf{For the foundations of physics}: The demon paradox arose from treating molecules as independent particles with properties (velocity) to be measured. Recognising molecules as nodes in phase-lock networks dissolves the paradox and suggests a more relational ontology.

\subsection{Experimental Predictions}

The resolution makes testable predictions:

\begin{enumerate}
    \item \textbf{Phase-lock correlation spectroscopy}: Categorical structure should be detectable through correlation measurements independent of temperature.

    \item \textbf{Isothermal categorical separation}: Under isothermal conditions, categorical clusters should remain distinguishable while temperature-based ``sorting'' is impossible.

    \item \textbf{Residual phase correlations}: After physical separation, molecules from the same categorical cluster should exhibit residual phase correlations detectable through interference measurements.

    \item \textbf{Network topology determines dynamics}: Molecular dynamics should follow phase-lock adjacency rather than kinetic energy similarity, testable through trajectory analysis.
\end{enumerate}

\subsection{Final Statement}

Maxwell's Demon has haunted thermodynamics for over 150 years, spawning profound insights into the relationships between information, entropy, and physical law. We have shown that the demon was never there---and could never have been there.

The demon fails on seven independent counts:
\begin{enumerate}
    \item It is \textbf{redundant}: fluctuations produce the same configurations naturally.
    \item It is \textbf{misconceived}: phase-lock structure is temperature-independent.
    \item It is \textbf{self-defeating}: velocity-based sorting cannot outpace thermal equilibration.
    \item It is \textbf{unnecessary}: categorical structure requires no measurement.
    \item It is \textbf{automatic}: categorical completion follows topology without decisions.
    \item It is \textbf{entropy-increasing}: network densification increases total entropy.
    \item It is \textbf{a projection artefact}: the ``demon'' is how hidden categorical dynamics appear when projected onto the observable kinetic face.
\end{enumerate}

The ``sorting'' that appeared to require an intelligent agent is the natural dynamics of categorical completion through phase-lock network topology. The paradox dissolves not through finding where entropy is produced, but through recognising that the sorting operation was always a manifestation of pre-existing categorical structure---and that any attempt to sort by velocity would be defeated by thermal equilibration before it could succeed.

Most profoundly, we now understand \textit{why Maxwell saw a demon}. Information has two conjugate faces that cannot be simultaneously observed. Maxwell, confined to the kinetic face of information (velocities, temperatures, molecular speeds), saw structured ``sorting'' that appeared to require intelligent intervention. But the ``demon'' was simply the categorical face---the phase-lock network completing states according to topology---projected onto his observable face. Just as an ammeter cannot see voltage directly, Maxwell's theoretical apparatus could not see categorical dynamics directly. The demon was born from this observational constraint.

Any observer confined to one face of information will, when dynamics occur on the conjugate face, necessarily perceive those dynamics as external intervention. The demon is universal in this sense: it will appear whenever an observer sees only half of a two-faced information structure. The demon dissolves the moment the observer gains access to the conjugate face.

There is no demon. There is only the phase-lock network, completing its categorical states according to topology, indifferent to the velocities that Maxwell's thought experiment privileged but that physics does not. And there is the observer, looking at one face of information, inventing agents to explain what they cannot directly see.

%==============================================================================
% Bibliography
%==============================================================================

\bibliographystyle{plainnat}
\bibliography{references}

\end{document}

