

\section{Categorical Structure of Physical Processes}

\subsection{Motivation: Beyond Continuous State Spaces}

The oscillatory framework established in Section 1 reveals physical reality as fundamentally oscillatory. However, oscillations alone do not explain temporal directionality or process irreversibility. Classical dynamical systems—whether oscillatory or not—admit time-reversal symmetry: solutions $\psi(t)$ and $\psi(-t)$ are equally valid under microscopic physical laws.

Yet macroscopic processes exhibit manifest directionality. Chemical reactions proceed spontaneously in one direction. Gases mix but do not spontaneously unmix. Oscillatory patterns decay to equilibrium but do not spontaneously regenerate from equilibrium. This asymmetry requires explanation beyond oscillatory dynamics themselves.

We resolve this through \textit{categorical topology}: a mathematical framework where physical processes occur not as continuous trajectories through state space but as discrete, irreversible completion of categorical states arranged in partial order. Time emerges from the sequential structure of categorical completion rather than being imposed as an external parameter.

\subsection{Categorical Spaces}

\begin{definition}[Categorical Space]
\label{def:categorical_space}
A \textbf{categorical space} is a structure $(\mathcal{C}, \prec, \mu, \tau)$ where:
\begin{enumerate}[(i)]
\item $\mathcal{C}$ is a set of \textbf{categorical states}
\item $\prec$ is a partial order on $\mathcal{C}$ (the \textbf{completion order})
\item $\mu: \mathcal{C} \times \mathbb{R}_{\geq 0} \to \{0, 1\}$ is the \textbf{completion operator}
\item $\tau$ is the \textbf{specialization topology} induced by $\prec$
\end{enumerate}
\end{definition}

The completion order $\prec$ represents precedence: $C_i \prec C_j$ means categorical state $C_i$ must be completed before $C_j$ can be completed. This ordering is not temporal (time has not yet been defined) but logical—it represents causal or dependency structure.

\begin{axiom}[Irreversibility Axiom]
\label{axiom:irreversibility}
For all $C \in \mathcal{C}$ and all $t_1 \leq t_2$:
\begin{equation}
\mu(C, t_1) = 1 \implies \mu(C, t_2) = 1
\end{equation}
Once a categorical state is completed ($\mu(C, t) = 1$), it remains completed for all future $t$.
\end{axiom}

This axiom introduces fundamental irreversibility without invoking statistical mechanics or entropy maximisation. Categorical completion is a one-way process.

\begin{axiom}[Order Compatibility]
\label{axiom:order_compatibility}
If $C_i \prec C_j$ and $\mu(C_j, t) = 1$, then there exists $t' \leq t$ such that $\mu(C_i, t') = 1$.

Predecessors must be completed before successors.
\end{axiom}

\subsection{Completion Trajectories}

\begin{definition}[Completion Trajectory]
\label{def:completion_trajectory}
A \textbf{completion trajectory} is a function $\gamma: \mathbb{R}_{\geq 0} \to \mathcal{P}(\mathcal{C})$ satisfying:
\begin{enumerate}[(i)]
\item $\gamma(t) = \{C \in \mathcal{C} : \mu(C, t) = 1\}$ (completed states at time $t$)
\item $t_1 \leq t_2 \implies \gamma(t_1) \subseteq \gamma(t_2)$ (monotonicity from Axiom \ref{axiom:irreversibility})
\item $\gamma(t)$ is downward-closed: $C \in \gamma(t), C' \prec C \implies C' \in \gamma(t)$
\end{enumerate}
\end{definition}

The trajectory $\gamma(t)$ describes the cumulative set of completed categorical states as the parameter $t$ evolves. Note that $t$ here is introduced as a parameter indexing completion events, not yet as physical time.

\begin{definition}[Categorical Completion Rate]
\label{def:completion_rate}
The \textbf{categorical completion rate} at the parameter value $t$ is:
\begin{equation}
\dot{C}(t) = \frac{d|\gamma(t)|}{dt}
\end{equation}
where $|\gamma(t)|$ denotes the measure of completed states.
\end{definition}

\begin{proposition}[Non-Negative Completion Rate]
\label{prop:nonnegative_rate}
For any completion trajectory:
\begin{equation}
\dot{C}(t) \geq 0 \quad \forall t \geq 0
\end{equation}
\end{proposition}

\begin{proof}
Direct consequence of monotonicity (Definition \ref{def:completion_trajectory}(ii)). \qed
\end{proof}

\section{Temporal Emergence from Categorical Sequencing}

\subsection{Observer-Driven Approximation}

Physical reality—as established in Section 1—consists of continuous oscillatory patterns spanning infinite dimensional phase space. Finite observers cannot process this continuity directly.

\begin{definition}[Categorical Assignment Function]
\label{def:categorical_assignment}
A \textbf{categorical assignment function} is a map $\mathcal{A}: \mathcal{S}_{\text{osc}} \to \mathcal{C}$ from the space of oscillatory configurations $\mathcal{S}_{\text{osc}}$ to categorical states $\mathcal{C}$, satisfying:
\begin{equation}
|\mathcal{C}| \ll |\mathcal{S}_{\text{osc}}|
\end{equation}
The assignment drastically reduces dimensionality.
\end{definition}

\begin{theorem}[Approximation Necessity]
\label{thm:approximation_necessity}
Observation of continuous oscillatory reality requires categorical approximation. Without approximation—partitioning continuous oscillatory flux into discrete distinguishable configurations—no objects exist to observe.
\end{theorem}

\begin{proof}
Continuous oscillatory reality has no natural boundaries. Between any two oscillatory configurations $\psi_1(x,t)$ and $\psi_2(x,t)$, there exist infinitely many intermediate configurations:
\begin{equation}
\psi_\lambda(x,t) = (1-\lambda)\psi_1(x,t) + \lambda\psi_2(x,t), \quad \lambda \in [0,1]
\end{equation}

Without discrete categories imposing boundaries, the space is undifferentiated flux. Observation requires distinguishing objects—identifying configuration A as distinct from configuration B. This distinction is not inherent in continuous space but must be imposed through categorical assignment.

Finite observers with bounded information capacity $I_{\text{max}}$ can distinguish at most:
\begin{equation}
|\mathcal{C}| \leq 2^{I_{\text{max}}} < \infty
\end{equation}
categorical states, forcing approximation of infinite-dimensional oscillatory space to finite categorical space. \qed
\end{proof}

\subsection{Temporal Coordinates from Sequential Completion}

\begin{definition}[Temporal Coordinate]
\label{def:temporal_coordinate}
The \textbf{temporal coordinate} $T$ emerges as the indexing structure for categorical completion sequence:
\begin{equation}
T: \mathcal{C} \to \mathbb{R}_{\geq 0}, \quad T(C_i) = \inf\{t : \mu(C_i, t) = 1\}
\end{equation}
$T(C_i)$ is the parameter value at which state $C_i$ first becomes completed.
\end{definition}

\begin{theorem}[Temporal Emergence]
\label{thm:temporal_emergence}
The partial order $\prec$ on categorical space induces temporal ordering:
\begin{equation}
C_i \prec C_j \implies T(C_i) \leq T(C_j)
\end{equation}
Time emerges as the real-valued representation of categorical precedence.
\end{theorem}

\begin{proof}
By Axiom \ref{axiom:order_compatibility}, if $C_i \prec C_j$ and $C_j$ is completed at time $T(C_j)$, then $C_i$ must have been completed at some earlier time $T(C_i) \leq T(C_j)$.

The partial order $\prec$ provides the discrete structure (which states precede which). The temporal coordinate $T$ embeds this discrete structure into the real line $\mathbb{R}_{\geq 0}$, creating continuous time from discrete categorical order. \qed
\end{proof}

\begin{corollary}[Time Without External Parameter]
Time is not an external parameter imposed on physical processes but an emergent structure arising from a categorical completion sequence. The directionality of time (forward arrow) is identical to the irreversibility of categorical completion (Axiom \ref{axiom:irreversibility}).
\end{corollary}

\subsection{Completion Rate as Temporal Perception}

\begin{proposition}[Perceived Temporal Flow]
\label{prop:temporal_flow}
The rate of perceived temporal flow is proportional to the categorical completion rate:
\begin{equation}
\frac{dT_{\text{perceived}}}{dt_{\text{physical}}} \propto \dot{C}(t)
\end{equation}
\end{proposition}

When categorical states complete rapidly ($\dot{C}$ large), subjective time flows quickly. When completion stagnates ($\dot{C}$ small), subjective time slows. At equilibrium, where no new states complete ($\dot{C} = 0$), subjective time ceases despite continued physical oscillation.

This provides a mechanism for temporal perception variations observed in biological systems (detailed in later sections).

\section{Entropy from Categorical Completion}

\subsection{Equivalence Class Structure}

Physical measurements do not resolve individual categorical states but aggregate over equivalence classes—sets of categorical states producing identical observable outcomes.

\begin{definition}[Observable Projection]
\label{def:observable}
An \textbf{observable} is a continuous function $\mathcal{O}: \mathcal{C} \to \mathcal{M}$ where $\mathcal{M}$ is the observation space (typically low-dimensional compared to $\mathcal{C}$).
\end{definition}

\begin{definition}[Categorical Equivalence]
\label{def:equivalence}
States $C_i, C_j \in \mathcal{C}$ are \textbf{categorically equivalent} under observable $\mathcal{O}$ if:
\begin{equation}
C_i \sim_{\mathcal{O}} C_j \iff \mathcal{O}(C_i) = \mathcal{O}(C_j)
\end{equation}
\end{definition}

\begin{definition}[Equivalence Class]
The equivalence class of state $C$ is:
\begin{equation}
[C]_{\mathcal{O}} = \{C' \in \mathcal{C} : C' \sim_{\mathcal{O}} C\} = \mathcal{O}^{-1}(\mathcal{O}(C))
\end{equation}
\end{definition}

\begin{definition}[Degeneracy]
\label{def:degeneracy}
The \textbf{degeneracy} of categorical state $C$ under observable $\mathcal{O}$ is:
\begin{equation}
\delta_{\mathcal{O}}(C) = |[C]_{\mathcal{O}}|
\end{equation}
the cardinality of its equivalence class.
\end{definition}

\subsection{Categorical Entropy}

\begin{definition}[Categorical Completion Probability]
\label{def:completion_probability}
For a system in the categorical state $C$, let $\alpha(C)$ denote the probability that the categorical sequence terminates (reaches final completion) at or before state $C$:
\begin{equation}
\alpha(C) = P(\text{sequence terminates} \mid \text{currently at } C)
\end{equation}
where $0 < \alpha(C) \leq 1$.
\end{definition}

\begin{definition}[Categorical Entropy]
\label{def:categorical_entropy}
The \textbf{categorical entropy} of state $C$ is:
\begin{equation}
S_{\text{cat}}(C) = k \log \alpha(C)
\label{eq:categorical_entropy}
\end{equation}
where $k$ is Boltzmann's constant.
\end{definition}

\begin{remark}
Since $\alpha \leq 1$, we have $S_{\text{cat}} \leq 0$. Equivalently, define $S'_{\text{cat}} = k \log(1/\alpha) \geq 0$ for conventional sign convention. The formulation in Eq.~\eqref{eq:categorical_entropy} emphasizes connection to termination probability.
\end{remark}

\begin{proposition}[Categorical Entropy and Degeneracy]
\label{prop:entropy_degeneracy}
Categorical entropy relates to equivalence class structure:
\begin{equation}
S_{\text{cat}}(C) = k \log \left( \frac{\delta_{\mathcal{O}}(C) \cdot N_{\text{accessible}}(C)}{N_{\text{total}}} \right)
\end{equation}
where $N_{\text{accessible}}(C)$ is the number of categorical states accessible from $C$, and $N_{\text{total}}$ is the total number of categorical states.
\end{proposition}

\begin{proof}
The termination probability $\alpha(C)$ depends on:
\begin{itemize}
\item How many microstates (equivalence class members) realize the macroscopic configuration: $\delta_{\mathcal{O}}(C)$
\item How many downstream states remain to be explored: $N_{\text{accessible}}(C)$
\end{itemize}

Larger degeneracy increases termination probability (more ways to reach termination from this state). More accessible downstream states decrease termination probability (more exploration required before termination).

Combining these:
\begin{equation}
\alpha(C) \propto \frac{\delta_{\mathcal{O}}(C) \cdot N_{\text{accessible}}(C)}{N_{\text{total}}}
\end{equation}

Taking logarithm yields the stated result. \qed
\end{proof}

\section{Oscillatory Entropy and Categorical Equivalence}

\subsection{Oscillatory Termination as Categorical Completion}

We now establish the central equivalence: \textit{oscillatory termination and categorical completion are identical processes viewed from different perspectives}.

\begin{definition}[Oscillatory Termination]
\label{def:oscillatory_termination}
An oscillatory pattern $\psi(t)$ \textbf{terminates} at time $t_{\text{term}}$ if:
\begin{equation}
\|\psi(t) - \psi_{\text{eq}}\| < \epsilon \quad \forall t > t_{\text{term}}
\end{equation}
for some equilibrium configuration $\psi_{\text{eq}}$ and threshold $\epsilon > 0$.
\end{definition}

Termination means the oscillatory system has settled into stable equilibrium—no further exploration of phase space occurs.

\begin{definition}[Oscillatory Entropy]
\label{def:oscillatory_entropy}
For oscillatory configuration $\psi$, the \textbf{oscillatory entropy} is:
\begin{equation}
S_{\text{osc}}(\psi) = k \log \beta(\psi)
\label{eq:oscillatory_entropy}
\end{equation}
where $\beta(\psi)$ is the probability that oscillatory dynamics terminate at configuration $\psi$.
\end{definition}

\begin{theorem}[Oscillatory-Categorical Equivalence]
\label{thm:oscillatory_categorical_equivalence}
Oscillatory termination and categorical completion are isomorphic processes. Specifically, there exists a bijection $\Phi: \mathcal{S}_{\text{osc}} \to \mathcal{C}$ such that:
\begin{enumerate}[(i)]
\item Oscillatory configuration $\psi$ terminates $\iff$ categorical state $\Phi(\psi)$ completes
\item Oscillatory termination probability equals categorical completion probability:
\begin{equation}
\beta(\psi) = \alpha(\Phi(\psi))
\end{equation}
\item Oscillatory entropy equals categorical entropy:
\begin{equation}
S_{\text{osc}}(\psi) = S_{\text{cat}}(\Phi(\psi))
\end{equation}
\end{enumerate}
\end{theorem}

\begin{proof}
\textbf{Construction of $\Phi$}:

Define $\Phi: \mathcal{S}_{\text{osc}} \to \mathcal{C}$ by categorical assignment: oscillatory configuration $\psi$ is mapped to the categorical state representing the equivalence class of configurations observationally indistinguishable from $\psi$.

\textbf{Part (i)}: Termination-Completion Correspondence

($\Rightarrow$) Suppose oscillatory pattern $\psi(t)$ terminates at $t_{\text{term}}$. Then for $t > t_{\text{term}}$:
\begin{equation}
\psi(t) \approx \psi_{\text{eq}} \quad \text{(equilibrium)}
\end{equation}

No further oscillatory exploration occurs—the system has completed its dynamics. In categorical language, this means categorical state $C = \Phi(\psi_{\text{eq}})$ has been reached and no further categorical states will be occupied. Thus $\mu(C, t) = 1$ for $t \geq t_{\text{term}}$—the categorical state is completed.

($\Leftarrow$) Suppose categorical state $C = \Phi(\psi)$ completes at time $t_{\text{comp}}$. By definition of completion, no further categorical states will be occupied after $t_{\text{comp}}$. In oscillatory language, this means the system remains within the equivalence class $[C]_{\mathcal{O}}$—all subsequent oscillatory configurations $\psi(t>t_{\text{comp}})$ map to the same categorical state $C$. This is precisely oscillatory termination: the system has settled into the basin $\mathcal{O}^{-1}(C)$ and no longer explores new regions of phase space.

\textbf{Part (ii)}: Probability Equivalence

The probability that oscillatory pattern terminates at $\psi$ equals the fraction of phase space volume that flows to the basin containing $\psi$:
\begin{equation}
\beta(\psi) = \frac{\text{Vol}(\text{basin of } \psi)}{\text{Vol}(\text{total accessible phase space})}
\end{equation}

The probability that categorical sequence completes at $C = \Phi(\psi)$ equals the fraction of categorical states that lead to $C$:
\begin{equation}
\alpha(C) = \frac{|\{C' : C' \text{ leads to } C\}|}{|\mathcal{C}_{\text{total}}|}
\end{equation}

By construction of $\Phi$, the basin of $\psi$ in oscillatory space corresponds precisely to the set of categorical states leading to $C$ in categorical space. The equivalence class structure ensures:
\begin{equation}
\beta(\psi) = \alpha(\Phi(\psi))
\end{equation}

\textbf{Part (iii)}: Entropy Equivalence

From Eqs.~\eqref{eq:oscillatory_entropy} and \eqref{eq:categorical_entropy}:
\begin{align}
S_{\text{osc}}(\psi) &= k \log \beta(\psi) \\
S_{\text{cat}}(\Phi(\psi)) &= k \log \alpha(\Phi(\psi))
\end{align}

By part (ii), $\beta(\psi) = \alpha(\Phi(\psi))$, therefore:
\begin{equation}
S_{\text{osc}}(\psi) = S_{\text{cat}}(\Phi(\psi))
\end{equation}

The two entropy formulations are identical. \qed
\end{proof}

\begin{corollary}[Unified Entropy Framework]
\label{cor:unified_entropy}
Oscillatory entropy and categorical entropy are not merely analogous but mathematically identical:
\begin{equation}
S = k \log P(\text{termination}) = k \log P(\text{completion})
\end{equation}

Whether one speaks of ``oscillatory termination'' or ``categorical completion'' is a matter of perspective, not substance. The physics is the same.
\end{corollary}

\subsection{Physical Interpretation}

The equivalence established in Theorem \ref{thm:oscillatory_categorical_equivalence} reveals deep unity:

\textbf{Oscillatory perspective}: Physical systems consist of oscillatory patterns that explore phase space until finding equilibrium configurations where oscillations terminate.

\textbf{Categorical perspective}: Physical processes consist of sequential completion of categorical states ordered by precedence, with process termination occurring when no further categorical states remain accessible.

These are identical processes. The oscillatory framework emphasizes continuous dynamics; the categorical framework emphasizes discrete state transitions. Both describe the same underlying reality.

\begin{proposition}[Entropy Increase from Both Perspectives]
\label{prop:entropy_increase_unified}
Process irreversibility appears naturally in both frameworks:

\textbf{Oscillatory}: Once oscillatory pattern terminates at equilibrium $\psi_{\text{eq}}$, it cannot spontaneously regenerate non-equilibrium oscillations. Entropy increases:
\begin{equation}
S_{\text{osc}}(\psi_{\text{non-eq}}) < S_{\text{osc}}(\psi_{\text{eq}})
\end{equation}

\textbf{Categorical}: Once categorical state $C$ is completed, it cannot be uncompleted (Axiom \ref{axiom:irreversibility}). Entropy increases:
\begin{equation}
S_{\text{cat}}(C_{\text{early}}) < S_{\text{cat}}(C_{\text{late}})
\end{equation}

By Theorem \ref{thm:oscillatory_categorical_equivalence}, these are identical statements.
\end{proposition}

\section{Categorical Richness and Asymmetry}

\subsection{Topological Invariants}

The categorical framework introduces topological quantities that determine system behavior.

\begin{definition}[Categorical Richness]
\label{def:richness}
The \textbf{categorical richness} of state $C$ is:
\begin{equation}
R(C) = \log \delta_{\mathcal{O}}(C) + \log N_{\text{down}}(C)
\end{equation}
where $\delta_{\mathcal{O}}(C) = |[C]_{\mathcal{O}}|$ is degeneracy and $N_{\text{down}}(C) = |\{C' : C \prec C'\}|$ counts downstream accessible states.
\end{definition}

Richness combines horizontal structure (equivalence class size) with vertical structure (downstream connectivity). High richness indicates many equivalent microstates and many possible future trajectories.

\begin{proposition}[Richness and Entropy]
\label{prop:richness_entropy}
Categorical richness relates directly to entropy:
\begin{equation}
S_{\text{cat}}(C) \propto R(C)
\end{equation}

States with high richness have high entropy; states with low richness have low entropy.
\end{proposition}

\begin{definition}[Categorical Asymmetry]
\label{def:asymmetry}
For competing processes represented by state sets $A, B \subset \mathcal{C}$, the \textbf{categorical asymmetry} is:
\begin{equation}
\mathcal{A}(A, B) = \frac{R(A) - R(B)}{R(A) + R(B)}
\end{equation}
where $R(A) = \log \sum_{C \in A} e^{R(C)}$ is aggregate richness.
\end{definition}

\begin{theorem}[Asymmetry Determines Process Direction]
\label{thm:asymmetry_direction}
For system with competing forward process $A$ and reverse process $B$:
\begin{enumerate}[(i)]
\item If $|\mathcal{A}(A,B)| \ll 1$: Bidirectional—both processes occur with comparable rates
\item If $\mathcal{A}(A,B) \gg 0$: Forward-dominant—process $A$ strongly preferred
\item If $\mathcal{A}(A,B) \ll 0$: Reverse-dominant—process $B$ strongly preferred
\end{enumerate}
\end{theorem}

\begin{proof}[Proof Sketch]
Transition probability between categorical states is proportional to target state richness. For forward transition $A \to B$:
\begin{equation}
P(A \to B) \propto e^{R(B)}
\end{equation}

For reverse transition $B \to A$:
\begin{equation}
P(B \to A) \propto e^{R(A)}
\end{equation}

The ratio:
\begin{equation}
\frac{P(A \to B)}{P(B \to A)} = e^{R(B) - R(A)} = e^{-\Delta R}
\end{equation}

When $R(A) \approx R(B)$: $\mathcal{A} \approx 0$ and both directions comparable.
When $R(A) \gg R(B)$: $\mathcal{A} \to 1$ and reverse strongly preferred.
When $R(B) \gg R(A)$: $\mathcal{A} \to -1$ and forward strongly preferred. \qed
\end{proof}

\subsection{Connection to Oscillatory Dynamics}

Categorical asymmetry translates directly to oscillatory language:

\begin{proposition}[Oscillatory Interpretation of Asymmetry]
\label{prop:oscillatory_asymmetry}
For oscillatory patterns $\psi_A$ and $\psi_B$ with $\Phi(\psi_A) = A$ and $\Phi(\psi_B) = B$:
\begin{equation}
\mathcal{A}(A, B) = \frac{\log \beta(\psi_B) - \log \beta(\psi_A)}{\log \beta(\psi_B) + \log \beta(\psi_A)}
\end{equation}

Asymmetry measures the relative termination probabilities of competing oscillatory configurations.
\end{proposition}

States where oscillations readily terminate (high $\beta$) have high categorical richness (high $R$). States where oscillations rarely terminate (low $\beta$) have low categorical richness (low $R$). The categorical framework provides discrete topological structure; the oscillatory framework provides continuous dynamical mechanism. Both describe identical physics.

\section{Finite Systems and Computational Constraints}

\subsection{Bounded Categorical Spaces}

Physical systems with finite energy and finite volume admit only finite categorical structure.

\begin{theorem}[Finite Categorical Bound]
\label{thm:finite_categorical}
For physical system with total energy $E_{\text{max}}$, volume $V$, and information capacity $I_{\text{max}}$:
\begin{equation}
|\mathcal{C}| \leq \min\left\{ \frac{E_{\text{max}}}{\epsilon_{\text{min}}}, \left(\frac{V}{\ell_P^3}\right), 2^{I_{\text{max}}} \right\}
\end{equation}
where $\epsilon_{\text{min}}$ is minimum energy quantum and $\ell_P$ is Planck length.
\end{theorem}

\begin{proof}
\textbf{Energy constraint}: Each categorical state requires minimum energy $\epsilon_{\text{min}}$ (e.g., zero-point energy, thermal energy $kT$). With finite total energy $E_{\text{max}}$, the number of accessible states is bounded:
\begin{equation}
|\mathcal{C}|_{\text{energy}} \leq \frac{E_{\text{max}}}{\epsilon_{\text{min}}}
\end{equation}

\textbf{Volume constraint}: Spatial resolution limited by Planck length. Maximum number of distinguishable spatial regions:
\begin{equation}
|\mathcal{C}|_{\text{volume}} \leq \frac{V}{\ell_P^3}
\end{equation}

\textbf{Information constraint}: Holographic bound limits information content:
\begin{equation}
I_{\text{max}} \leq \frac{A}{4\ell_P^2}
\end{equation}
where $A$ is surface area. Maximum distinguishable states:
\begin{equation}
|\mathcal{C}|_{\text{information}} \leq 2^{I_{\text{max}}}
\end{equation}

The effective bound is the minimum of these three constraints. \qed
\end{proof}

\begin{corollary}[Eventual Completion]
\label{cor:eventual_completion}
For finite categorical space $|\mathcal{C}| < \infty$ with $\dot{C}(t) > 0$:
\begin{equation}
\exists T < \infty: \gamma(T) = \mathcal{C}
\end{equation}
All categorical states eventually complete in finite time.
\end{corollary}

\begin{proof}
With $|\mathcal{C}| = N < \infty$ and positive completion rate $\dot{C}(t) > \epsilon > 0$:
\begin{equation}
|\gamma(t)| = \int_0^t \dot{C}(s) ds > \epsilon t
\end{equation}
Setting $\epsilon T = N$ gives $T = N/\epsilon < \infty$. \qed
\end{proof}

\subsection{Computational Implications}

\begin{theorem}[Categorical Processing Efficiency]
\label{thm:categorical_efficiency}
Categorical approximation reduces computational complexity from infinite-dimensional oscillatory dynamics to finite-dimensional categorical dynamics:
\begin{equation}
\text{Complexity}(\text{oscillatory}) = \mathcal{O}(2^{N_{\text{osc}}}) \to \text{Complexity}(\text{categorical}) = \mathcal{O}(|\mathcal{C}|)
\end{equation}
where typically $|\mathcal{C}| \ll 2^{N_{\text{osc}}}$.
\end{theorem}

This explains why finite observers—including biological systems—necessarily operate through categorical approximation rather than tracking continuous oscillatory dynamics exactly. The computational cost of perfect oscillatory resolution exceeds available resources (as proven in Section 1, Computational Impossibility Theorem).


