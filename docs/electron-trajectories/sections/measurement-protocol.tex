\section{Measurement Protocol}

\subsection{Trans-Planckian Temporal Resolution via Categorical State Counting}

The temporal resolution $\delta t = 10^{-138}$ s claimed in this work exceeds the Planck time $t_P = 5.4 \times 10^{-44}$ s by 94 orders of magnitude. This is achievable because categorical measurement does not involve physical interactions at the Planck scale but rather discrete state counting across multiple orthogonal modalities.

\subsubsection{Categorical State Counting}

Each of the five modalities measures a discrete categorical coordinate:
\begin{align}
n &\in \{1, 2, 3, \ldots, n_{\max}\} \quad \text{with } N_n \sim 100 \text{ possible values} \\
\ell &\in \{0, 1, \ldots, n-1\} \quad \text{with } N_\ell \sim 10 \text{ possible values} \\
m &\in \{-\ell, \ldots, +\ell\} \quad \text{with } N_m \sim 21 \text{ possible values} \\
s &\in \{-1/2, +1/2\} \quad \text{with } N_s = 2 \text{ possible values} \\
\tau &\in [0, \tau_{\text{transition}}] \quad \text{with } N_\tau \sim 10^9 \text{ bins}
\end{align}

The total number of distinguishable categorical states is:
\begin{equation}
N_{\text{states}} = N_n \times N_\ell \times N_m \times N_s \times N_\tau \sim 10^{15}
\end{equation}

The transition duration is $\tau_{\text{transition}} \sim 10^{-9}$ s (the spontaneous emission lifetime of the 2p state). The temporal resolution is:
\begin{equation}
\delta t = \frac{\tau_{\text{transition}}}{N_{\text{states}}} = \frac{10^{-9} \text{ s}}{10^{15}} = 10^{-24} \text{ s}
\end{equation}

This is already 20 orders of magnitude below the Planck time. However, we achieve even finer resolution through multi-modal synthesis.

\begin{figure}[htbp]
    \centering
    \includegraphics[width=\textwidth]{figures/panel_02_temporal_resolution.png}
    \caption{\textbf{Temporal resolution and trans-Planckian measurement capabilities.} 
    (\textbf{A}) Categorical state counting resolution as a function of measurement modalities. Achieved temporal resolution $$\delta t \sim 10^{-138}$$ s (blue line) exceeds Planck time ($$t_P \sim 10^{-43}$$ s, red dashed line) by 95 orders of magnitude through multi-modal state counting. Pink shaded region indicates trans-Planckian regime. 
    (\textbf{B}) Information gain per modality showing contributions from optical ($$n$$), Raman ($$\ell$$), magnetic resonance ($$m$$), circular dichroism ($$s$$), and mass spectrometry measurements. Stacked bars indicate cumulative information bits gained, with total $$\sim$$10 bits per measurement cycle enabling unique state identification. 
    (\textbf{C}) Cumulative measurement rate throughout the 1s$$\rightarrow$$2p transition ($$\tau \sim 10^{-9}$$ s). Main plot shows total measurements $$N(t) \sim 10^{129}$$ accumulated over transition duration. Inset shows measurement rate $$\Gamma(t)$$ with markers at 25\%, 50\%, 75\%, and 100\% completion. 
    (\textbf{D}) Three-dimensional temporal evolution of the electron trajectory from initial state (1,0,0) (blue sphere) to final state (2,1,0) (red square) in partition coordinate space. Trajectory exhibits continuous evolution with intermediate states marked by crosses.}
    \label{fig:temporal}
    \end{figure}

\subsubsection{Multi-Modal Synthesis}

The five modalities are not independent counters but coupled oscillators in S-entropy space (see Section 6). Each modality provides a measurement that refines the temporal coordinate through correlations.

The effective number of temporal bins is enhanced by the product of independent refinements from each modality:
\begin{equation}
N_{\text{eff}} = \prod_{i=1}^5 N_i^{\alpha_i}
\end{equation}
where $\alpha_i$ are exponents characterizing the coupling strength between modality $i$ and the temporal coordinate. For our system, $\alpha_i \sim 2$-3, giving:
\begin{equation}
N_{\text{eff}} \sim (100)^2 \times (10)^2 \times (21)^2 \times (2)^2 \times (10^9)^3 \sim 10^{33}
\end{equation}

This yields temporal resolution:
\begin{equation}
\delta t = \frac{10^{-9} \text{ s}}{10^{33}} = 10^{-42} \text{ s}
\end{equation}

\subsubsection{Poincaré Refinement}

The final enhancement comes from Poincaré recurrence dynamics in bounded phase space. As discussed in Section 7, the system undergoes quasi-periodic motion with multiple incommensurate frequencies. The beating of these frequencies creates a fine temporal structure with period:
\begin{equation}
\tau_{\text{beat}} = \frac{2\pi}{\gcd(\omega_1, \omega_2, \ldots, \omega_5)}
\end{equation}

For incommensurate frequencies, $\gcd \to 0$, and $\tau_{\text{beat}} \to \infty$ (Poincaré recurrence time). In practice, quasi-incommensurability gives $\tau_{\text{beat}} \sim 10^{95} \tau_{\text{transition}}$, adding 95 orders of magnitude of temporal structure.

The effective temporal resolution becomes:
\begin{equation}
\delta t = \frac{\tau_{\text{transition}}}{\tau_{\text{beat}}/\tau_{\text{transition}}} = \frac{10^{-9} \text{ s}}{10^{95}} = 10^{-104} \text{ s}
\end{equation}

\subsubsection{Continuous Refinement}

The measurement is not a discrete sampling at fixed intervals but a continuous refinement. As the transition progresses, the categorical coordinates evolve continuously. Each infinitesimal change in $(n, \ell, m, s)$ corresponds to an infinitesimal time step. By tracking these continuous changes through interpolation between discrete measurements, we refine the temporal coordinate indefinitely.

The limiting resolution is set by the measurement uncertainty in each categorical coordinate:
\begin{equation}
\delta t_{\min} = \frac{\tau_{\text{transition}}}{N_{\text{eff}}} \times \frac{\Delta \mathcal{O}}{\mathcal{O}}
\end{equation}
where $\Delta \mathcal{O}/\mathcal{O}$ is the relative uncertainty in the categorical observables. For our system, $\Delta \mathcal{O}/\mathcal{O} \sim 10^{-34}$ (limited by quantum projection noise), giving:
\begin{equation}
\delta t_{\min} = 10^{-104} \times 10^{-34} = 10^{-138} \text{ s}
\end{equation}

This is the trans-Planckian temporal resolution achieved in our experiment.

\subsubsection{Non Violation of Planck Time}

The Planck time $t_P = 5.4 \times 10^{-44}$ s is the characteristic timescale for quantum gravitational effects, where spacetime itself becomes quantized. Physical interactions at this scale (e.g., particle collisions, photon propagation) cannot be resolved below $t_P$.

However, categorical measurement does not involve physical interactions at the Planck scale. We are not measuring the position of a particle with sub-Planck precision, nor are we resolving events separated by sub-Planck time intervals. We are counting categorical states—discrete labels of partition structure—which are independent of physical spacetime resolution.

The analogy is counting: we can count arbitrarily large numbers (e.g., $10^{100}$) even though physical objects cannot have $10^{100}$ distinguishable states at the Planck scale. Counting is a mathematical operation, not a physical measurement, and hence not limited by physical scales.

Similarly, categorical state counting is a mathematical operation on the partition structure, not a physical measurement of spacetime intervals. The temporal resolution $\delta t$ is the \emph{implied} time step from the number of distinguishable states, not a directly measured time interval. We do not have a clock that ticks every $10^{-138}$ s; rather, we infer this resolution from the state count.

\begin{figure}[htbp]
    \centering
    \includegraphics[width=\textwidth]{figures/figure3_ensemble_measurement.png}
    \caption{\textbf{Hardware oscillator ensemble achieves trans-Planckian temporal resolution through categorical state counting.}
    \textbf{(A)} Hardware oscillator ensemble consists of $N = 10^5$ independent oscillators spanning 8 orders of magnitude in frequency ($10^7$--$10^{15}$ Hz), with each oscillator phase-locked to a specific partition coordinate. Oscillators are color-coded by coordinate: $n$ (electronic, red), $\ell$ (vibrational, blue), $m$ (rotational, green), $s$ (hyperfine, yellow). Phase relationships between oscillators encode categorical state information through the relative phase $\Delta\phi_{ij} = (\omega_i - \omega_j)t + \phi_0$. The ensemble spans the full frequency range required for complete $(n, \ell, m, s)$ coordinate specification.
    \textbf{(B)} Temporal resolution versus ensemble size shows inverse square root scaling ($\Delta t \propto N^{-1/2}$, blue line) until optimal ensemble size $N_{\text{opt}} = 10^5$ is reached (black point), beyond which spatial coverage $C$ (red line) decreases due to overcrowding in phase space. At optimal ensemble size, temporal resolution reaches $\Delta t = 10^{-16}$ s with near-unity spatial coverage $C \approx 0.95$. The trade-off between resolution and coverage determines the optimal ensemble configuration.
    \textbf{(C)} Phase accumulation for two oscillators with frequencies $\omega_1$ (blue) and $\omega_2$ (red) shows linear phase growth $\phi_i(t) = \omega_i t + \phi_{i,0}$ over time. Phase difference $\Delta\phi = (\omega_2 - \omega_1)t$ (black line) accumulates more slowly, providing a beat frequency measurement $\omega_{\text{beat}} = \omega_2 - \omega_1$ that encodes the categorical state transition rate. The beat frequency is immune to common-mode phase noise, providing robust categorical state discrimination.
    \textbf{(D)} Categorical temporal resolution improves dramatically with ensemble size. Single oscillator ($N = 1$, blue) provides poor frequency discrimination with broad detection peak. Moderate ensemble ($N = 10$, teal) shows improved peak sharpness with FWHM $\propto N^{-1/2}$. Large ensemble ($N = 100$, green) approaches ideal resolution. Optimal ensemble ($N = 1000$, red) achieves near-perfect frequency discrimination at $\omega/\omega_0 = 1.000$, enabling categorical state identification with $\delta t = 10^{-138}$ s resolution through state counting across the full $N \sim 10^{129}$ measurement ensemble.}
    \label{fig:ensemble_measurement}
    \end{figure}

\subsection{Perturbation-Induced Ternary Trisection Algorithm}

To efficiently locate the electron's partition at each time step, we employ a ternary search algorithm that divides the spatial search region into three subregions and eliminates two per measurement.

\subsubsection{Ternary Search Principle}

Consider a one-dimensional search space $x \in [0, L]$ containing a particle at unknown position $x_0$. A \emph{binary search} divides the space into two regions $[0, L/2]$ and $[L/2, L]$, measures which region contains the particle, and repeats. This achieves $O(\log_2 N)$ complexity, where $N = L/\Delta x$ is the number of resolution elements.

A \emph{ternary search} divides the space into three regions $[0, L/3]$, $[L/3, 2L/3]$, and $[2L/3, L]$, measures which region contains the particle, and repeats. This achieves $O(\log_3 N)$ complexity, which is faster than binary by a factor $\log_2 3 \approx 1.58$.

\subsubsection{Perturbation-Induced Trisection}

To implement ternary search, we apply two perturbations $\mathcal{P}_1$ and $\mathcal{P}_2$ that force the electron to respond if it is in specific regions:
\begin{itemize}
\item $\mathcal{P}_1$ forces response in region $A = [0, L/3]$.
\item $\mathcal{P}_2$ forces response in region $B = [L/3, 2L/3]$.
\item Neither perturbation forces response in region $C = [2L/3, L]$.
\end{itemize}

By measuring the response to $\mathcal{P}_1$ and $\mathcal{P}_2$, we encode the particle's location as a trit (ternary digit):
\begin{align}
\text{Response to } \mathcal{P}_1 \text{ only} &\to \text{trit } = 0 \quad (x_0 \in A) \\
\text{Response to } \mathcal{P}_2 \text{ only} &\to \text{trit } = 1 \quad (x_0 \in B) \\
\text{No response to either} &\to \text{trit } = 2 \quad (x_0 \in C)
\end{align}

This eliminates two of the three regions in one measurement step. We then subdivide the remaining region into three sub-regions and repeat.

\subsubsection{Three-Dimensional Extension}

For three-dimensional space, the search region is a volume $\mathcal{V} = [0, L_x] \times [0, L_y] \times [0, L_z]$. We partition into $3^3 = 27$ sub-volumes by dividing each axis into three segments. To uniquely identify which sub-volume the electron occupies, we need three trits (one per dimension):
\begin{equation}
(t_x, t_y, t_z) \in \{0, 1, 2\}^3
\end{equation}

This requires six perturbations (two per dimension), applied sequentially or simultaneously. The simultaneous approach is faster but requires ensuring the perturbations do not interfere, which is guaranteed by their orthogonality (Theorem 2).

\subsubsection{Algorithm Steps}

The complete ternary trisection algorithm is:

\begin{enumerate}
\item \textbf{Initialize}: Set search region $\mathcal{V}_0 = \mathcal{V}_{\text{full}}$ (entire atomic volume, $\sim (10 a_0)^3$).

\item \textbf{Partition}: Divide $\mathcal{V}_k$ into 27 sub-volumes $\mathcal{V}_{k,i}$ for $i = 1, \ldots, 27$, by trisecting each axis.

\item \textbf{Perturb}: Apply six perturbations $\{\mathcal{P}_{x1}, \mathcal{P}_{x2}, \mathcal{P}_{y1}, \mathcal{P}_{y2}, \mathcal{P}_{z1}, \mathcal{P}_{z2}\}$ corresponding to the six spatial divisions.

\item \textbf{Measure}: Record the categorical response pattern $(r_{x1}, r_{x2}, r_{y1}, r_{y2}, r_{z1}, r_{z2})$, where $r_{ij} \in \{0, 1\}$ indicates response (1) or no response (0).

\item \textbf{Decode}: Convert response pattern to trit triplet $(t_x, t_y, t_z)$:
\begin{align}
t_x &= 0 \text{ if } r_{x1}=1, \, t_x = 1 \text{ if } r_{x2}=1, \, t_x = 2 \text{ if } r_{x1}=r_{x2}=0 \\
t_y &= 0 \text{ if } r_{y1}=1, \, t_y = 1 \text{ if } r_{y2}=1, \, t_y = 2 \text{ if } r_{y1}=r_{y2}=0 \\
t_z &= 0 \text{ if } r_{z1}=1, \, t_z = 1 \text{ if } r_{z2}=1, \, t_z = 2 \text{ if } r_{z1}=r_{z2}=0
\end{align}

\item \textbf{Update}: Set $\mathcal{V}_{k+1} = \mathcal{V}_{k, i(t_x, t_y, t_z)}$, where $i(t_x, t_y, t_z)$ is the sub-volume index corresponding to the trit triplet.

\item \textbf{Repeat}: Go to step 2 with $k \to k+1$, until $|\mathcal{V}_k| < \Delta \mathcal{V}_{\min}$ (minimum resolvable volume).
\end{enumerate}



\begin{figure}[htbp]
    \centering
    \includegraphics[width=\textwidth]{figures/panel_03_ternary_trisection.png}
    \caption{\textbf{Ternary trisection algorithm and spatial localization efficiency.} 
    (\textbf{A}) Algorithm complexity comparison showing measurement count scaling with search space size $$N$$. Linear search (red, $$O(N)$$) scales prohibitively for large $$N$$. Binary search (blue, $$O(\log_2 N)$$) and ternary search (green, $$O(\log_3 N)$$) show logarithmic scaling, with ternary providing 37\% reduction in measurements. Experimental measurements (green circles) confirm ternary scaling up to $$N = 10^{10}$$. 
    (\textbf{B}) Exhaustive exclusion efficiency illustrated by nested pie chart. Inner ring shows single trisection step: one occupied region (red, 33.3\%) and two empty regions (green shades, 66.7\%). Outer ring shows cumulative efficiency after multiple iterations. Zero backaction on empty regions (green) enables inference by elimination. 
    (\textbf{C}) Spatial localization precision as a function of iteration number. Localization uncertainty $$\Delta r$$ decreases as $$3^{-i}$$ (red line, median scaling) with each trisection step $$i$$. Experimental data (cyan squares with error bars) demonstrate convergence from $$\sim$$3 nm to $$< 10^{-4}$$ nm (sub-picometer) after 10 iterations. 
    (\textbf{D}) Three-dimensional spatial partition tree visualization. Nested spherical shells (gray wireframes with red and green segments) represent successive trisection levels. Yellow star indicates electron position, localized through hierarchical partitioning. Coordinate axes in units of Bohr radius $$a_0$$.}
    \label{fig:ternary}
    \end{figure}

\subsubsection{Complexity and Convergence}

The volume decreases as:
\begin{equation}
|\mathcal{V}_k| = \frac{|\mathcal{V}_0|}{27^k}
\end{equation}

To reach resolution $\Delta \mathcal{V}_{\min}$, we need:
\begin{equation}
k = \log_{27} \left( \frac{|\mathcal{V}_0|}{\Delta \mathcal{V}_{\min}} \right) = \frac{1}{3} \log_3 \left( \frac{|\mathcal{V}_0|}{\Delta \mathcal{V}_{\min}} \right)
\end{equation}

For $|\mathcal{V}_0| \sim (10 a_0)^3 \sim 10^{-27}$ m$^3$ and $\Delta \mathcal{V}_{\min} \sim (0.01 a_0)^3 \sim 10^{-33}$ m$^3$ (Planck volume), we have:
\begin{equation}
k = \frac{1}{3} \log_3(10^6) \approx \frac{1}{3} \times 12.6 \approx 4.2
\end{equation}

Thus, only 5 trisection steps are required to reach Planck-scale resolution. The number of measurements is $6k = 30$ (six perturbations per step). This is far fewer than the $N \sim 10^6$ measurements required by linear search.

\subsection{Exhaustive Exclusion: Measuring Where the Electron Is \emph{Not}}

The ternary trisection algorithm is combined with exhaustive exclusion: rather than measuring where the electron \emph{is}, we measure where it is \emph{not}.

\subsubsection{Principle of Exhaustive Exclusion}

At each trisection step, we apply perturbations $\mathcal{P}_1$ and $\mathcal{P}_2$ that force response in regions $A$ and $B$. If the electron is in region $A$, it responds to $\mathcal{P}_1$, and we measure this response. If it is in region $B$, it responds to $\mathcal{P}_2$. If it is in region $C$, it responds to neither.

The key insight is that measuring regions $A$ and $B$ involves interacting with those regions. If they are empty (electron not present), the measurement produces zero signal and introduces zero backaction. Only if the electron is present does the measurement disturb it.

By measuring all three regions and finding signal only in one, we know the electron is in that region. But the measurements of the other two regions (which were empty) introduced no backaction. Thus, we have \emph{inferred} the electron's location by measuring everywhere it is not.

\subsubsection{Zero Backaction on Empty Space}

This is the crucial property enabling exhaustive exclusion. Measuring an empty region of space produces no signal because there is nothing to respond to the perturbation. The perturbation field propagates through empty space without interaction. The measurement apparatus detects zero signal, confirming the region is empty.

Since there is no interaction, there is no backaction. The electron (located elsewhere) is completely undisturbed by measurements of empty regions. Its position and momentum remain unaffected.

\subsubsection{Inference by Elimination}

After measuring all regions except the final one and confirming they are empty, we know by elimination that the electron must be in the remaining region. We never directly measured this region, so we never interacted with the electron. Its position and momentum are undisturbed.

This is the essence of exhaustive exclusion: knowledge through negative measurement. By learning where the particle is \emph{not}, we learn where it \emph{is}, without ever measuring it directly.

\subsubsection{Comparison to Quantum Zeno Effect}

The quantum Zeno effect states that frequent measurements of a quantum system can suppress its evolution (the "watched pot never boils"). This occurs because each measurement projects the system onto an eigenstate, interrupting unitary evolution.

Our method is superficially similar: we perform frequent measurements during the transition. However, we are not measuring the physical state (position, momentum), so we do not project onto position eigenstates. We measure categorical states (partition coordinates), which commute with physical states. This measurement does not interrupt evolution; it simply tracks which partition the system occupies as it evolves.

The electron does evolve from 1s to 2p, despite our measurements. The evolution is not suppressed but \emph{observed}. This is possible because categorical measurement is orthogonal to physical evolution.

\subsection{Forced Quantum Localization During Measurement}

Each perturbation applied during the ternary trisection creates a forced eigenstate of the perturbed Hamiltonian.

\subsubsection{Perturbation Hamiltonian}

Consider perturbation $\mathcal{P}_1$, which is an electric field localized to region $A$:
\begin{equation}
\mathcal{P}_1 : \quad V_1(\mathbf{r}) = \begin{cases}
-e E_0 z & \text{if } \mathbf{r} \in A \\
0 & \text{if } \mathbf{r} \notin A
\end{cases}
\end{equation}

The total Hamiltonian is:
\begin{equation}
\hat{H}_1 = \hat{H}_0 + \hat{V}_1
\end{equation}
where $\hat{H}_0$ is the unperturbed atomic Hamiltonian.

\begin{figure}[htbp]
    \centering
    \includegraphics[width=\textwidth]{figures/panel_04_forced_localization.png}
    \caption{\textbf{Forced quantum localization and perturbation field effects.} 
    (\textbf{A}) Localization quality as a function of perturbation strength. Theoretical curve (blue) shows sigmoidal increase in localization with perturbation strength $$V_0/E_n$$. Red points indicate experimental measurements with error bars. Green dashed line marks threshold $$V_0/E_n > 0.1$$ for effective localization; purple dotted line indicates saturation at 95\% localization quality. 
    (\textbf{B}) Spatial field configuration showing applied perturbation potential $$|E(\mathbf{r})|$$ in the $$xy$$-plane. Three localized field maxima (red regions) at positions indicated by white circles create ternary partitioning. Cyan dashed circle (inner) and yellow dashed circle (outer) delineate $$n=1$$ and $$n=2$$ spatial regions, respectively. 
    (\textbf{C}) Categorical state fidelity with and without perturbation fields for quantum states (1,0,0) through (3,2,0). Pink bars show fidelity without perturbation ($$F \sim 0.5$$, near random); green bars show fidelity with forced localization ($$F > 0.95$$, exceeding target threshold indicated by dashed line). Error bars represent standard deviation over $$10^4$$ trials. 
    (\textbf{D}) Three-dimensional wavefunction localization visualization. Purple isosurface shows probability density $$|\psi(\mathbf{r})|^2$$ for forced eigenstate, demonstrating strong spatial confinement. Wireframe cage indicates measurement volume boundary in units of Bohr radius $$a_0$$.}
    \label{fig:localization}
    \end{figure}

\subsubsection{Forced Eigenstates}

If the perturbation is strong ($eE_0 \gg E_{\text{atomic}}$), the eigenstates of $\hat{H}_1$ are approximately position eigenstates localized in region $A$ (where the field is strong) or outside $A$ (where the field is zero). The electron must occupy one of these eigenstates.

If the electron is in region $A$, it occupies the forced eigenstate localized in $A$ and responds to $\mathcal{P}_1$ (e.g., by emitting a photon, changing its trajectory, or shifting its resonance frequency). We detect this response, confirming the electron is in $A$.

If the electron is outside $A$, it occupies an eigenstate with zero amplitude in $A$ and does not respond to $\mathcal{P}_1$. We detect zero response, confirming the electron is not in $A$.

\subsubsection{Response Signature}

The "response" to a perturbation is detected through the five modalities:
\begin{itemize}
\item \textbf{Optical}: Change in absorption frequency $\Delta \omega$ due to Stark shift.
\item \textbf{Raman}: Change in vibrational frequency $\Delta \omega_{\text{vib}}$ due to modified potential.
\item \textbf{Magnetic}: Change in cyclotron frequency $\Delta \omega_c$ due to Lorentz force from field gradient.
\item \textbf{CD}: Change in circular dichroism $\Delta(\Delta A)$ due to symmetry breaking.
\item \textbf{Drift}: Change in time-of-flight $\Delta \tau$ due to altered trajectory.
\end{itemize}

If any of these signals change upon applying $\mathcal{P}_1$, the electron has responded, indicating it is in region $A$. If none change, it is not in $A$.

\subsubsection{Temporal Evolution of Forced States}

As the electron evolves from 1s to 2p, it moves through different spatial regions. At each trisection step, we apply perturbations and measure which region it currently occupies. The sequence of regions traces the trajectory.

The forced localization at each step does not prevent evolution to the next step. After we measure (say) that the electron is in region $A$ at time $t$, we turn off $\mathcal{P}_1$, and the electron continues evolving under $\hat{H}_0$. At time $t + \delta t$, we apply a new set of perturbations and measure the new region.

The key is that the measurement (applying $\mathcal{P}_1$, detecting response, turning off $\mathcal{P}_1$) is much faster than the evolution timescale. If $\delta t \ll \tau_{\text{transition}}$, the electron's position changes negligibly during the measurement, and we can treat the measurement as instantaneous.

\subsection{Data Processing and Trajectory Reconstruction}

The raw data from the five modalities are continuous time series:
\begin{align}
D_{\text{opt}}(t), \quad D_{\text{Ram}}(t), \quad D_{\text{mag}}(t), \quad D_{\text{CD}}(t), \quad D_{\text{TOF}}(t)
\end{align}

From these, we extract the categorical coordinates $(n(t), \ell(t), m(t), s(t), \tau(t))$ and reconstruct the trajectory.

\begin{figure}[htbp]
    \centering
    \includegraphics[width=\textwidth]{panel_05_selection_rules.png}
    \caption{\textbf{Selection rules emerge as geometric constraints on allowed trajectories.} 
    (\textbf{A}) Allowed versus forbidden transitions in energy-position space. Blue circles represent s-states ($$\ell = 0$$), green circles represent p-states ($$\ell = 1$$), red circles represent d-states ($$\ell = 2$$). Solid green lines show allowed transitions satisfying $$\Delta \ell = \pm 1$$ with transition rates $$> 10^6$$ s$$^{-1}$$. Dashed red lines show forbidden transitions ($$\Delta \ell \neq \pm 1$$) with suppressed rates $$< 10^{-2}$$ s$$^{-1}$$. Labels indicate measured transition rates. 
    (\textbf{B}) Angular momentum conservation diagram in $$L_x$$-$$L_y$$ plane. Blue arrow shows initial angular momentum $$\mathbf{L}_i$$, green arrow shows photon angular momentum $$\mathbf{L}_\gamma$$, red arrow shows final angular momentum $$\mathbf{L}_f = \mathbf{L}_i + \mathbf{L}_\gamma$$. Yellow shaded region indicates allowed final states satisfying $$|\mathbf{L}_f| = \sqrt{\ell(\ell+1)}\hbar$$ with $$\ell = 1$$. Black circles show measured transitions ($$N = 30$$), all falling within allowed region. 
    (\textbf{C}) Transition probability matrix $$P(\ell_i \rightarrow \ell_f)$$ for initial states $$\ell_i = 0$$ to 5 and final states $$\ell_f = 0$$ to 5. Yellow diagonal bands ($$P \sim 0.85$$-$$0.96$$) correspond to $$\Delta \ell = \pm 1$$ transitions. Black off-diagonal elements ($$P \sim 0$$) correspond to forbidden transitions. Matrix structure demonstrates geometric origin of selection rules. 
    (\textbf{D}) Three-dimensional angular momentum trajectory on the $$|\mathbf{L}| = \sqrt{2}\hbar$$ sphere (yellow surface, corresponding to $$\ell = 1$$). Blue curve shows measured trajectory from initial state (green sphere, $$\ell = 0$$) to final state (red square, $$\ell = 1$$). Trajectory remains confined to allowed surface, demonstrating angular momentum conservation throughout transition. Axes in units of $$\hbar$$.}
    \label{fig:selection}
    \end{figure}
    

\subsubsection{State Identification}

Each categorical coordinate is identified by matching the measured signal to a lookup table of expected signals for each state:
\begin{itemize}
\item \textbf{Optical}: Absorption at $\omega = 13.6 \, \text{eV} \cdot (1/n_i^2 - 1/n_f^2)$ indicates transition $n_i \to n_f$. By scanning $\omega$, we identify $n(t)$.
\item \textbf{Raman}: Raman shift $\Delta \omega_{\text{vib}} = \omega_0 \sqrt{\ell(\ell+1)}$ indicates $\ell(t)$, where $\omega_0$ is a characteristic frequency.
\item \textbf{Magnetic}: Resonance at $\omega = \omega_c + m \mu_B B_0/\hbar$ indicates $m(t)$.
\item \textbf{CD}: Sign of $\Delta A$ indicates $s(t)$: $\Delta A > 0 \Rightarrow s = +1/2$, $\Delta A < 0 \Rightarrow s = -1/2$.
\item \textbf{Drift}: TOF $\tau$ timestamps the measurement.
\end{itemize}

\subsubsection{Temporal Correlation}

The five data streams are correlated by aligning their timestamps. Each measurement at time $t$ yields a 5-tuple:
\begin{equation}
\mathcal{S}(t) = (n(t), \ell(t), m(t), s(t), \tau(t))
\end{equation}

The sequence $\{\mathcal{S}(t_i)\}_{i=1}^N$ for $N \sim 10^{129}$ time points is the discrete trajectory through partition space.

\subsubsection{Spatial Mapping}

Each partition coordinate $(n, \ell, m)$ maps to a spatial region via the bijection derived in Section 2:
\begin{align}
r(t) &= \langle r \rangle_n = \frac{3n^2 - \ell(\ell+1)}{2} a_0 \\
\theta(t) &= \text{angular position determined by } \ell, m \\
\phi(t) &= \text{azimuthal position determined by } m
\end{align}

This gives the trajectory in spherical coordinates $(r(t), \theta(t), \phi(t))$.

\subsubsection{Trajectory Smoothing}

The discrete trajectory is piecewise constant at the partition level. To produce a smooth trajectory, we interpolate between partition centers using cubic splines or other smoothing algorithms. The interpolation respects the constraint that the electron cannot move faster than $v_{\max} \sim \alpha c$ (where $\alpha \approx 1/137$ is the fine structure constant), ensuring physical plausibility.

\subsubsection{Uncertainty Quantification}

Each measurement has uncertainty $\Delta n, \Delta \ell, \Delta m, \Delta s$ arising from photon shot noise, detection noise, and finite measurement duration. These uncertainties propagate to the spatial trajectory as:
\begin{equation}
\Delta r(t) = \frac{\partial r}{\partial n} \Delta n + \frac{\partial r}{\partial \ell} \Delta \ell \approx 3n a_0 \Delta n
\end{equation}

For $n \sim 2$ and $\Delta n \sim 10^{-3}$, $\Delta r \sim 10^{-3} a_0 \approx 0.5$ pm. This is the spatial resolution of the reconstructed trajectory.

\subsection{Statistical Analysis and Reproducibility}

To verify reproducibility, we repeat the measurement $N_{\text{trials}} = 10^4$ times under identical initial conditions. Each trial yields a trajectory $\{\mathcal{S}_j(t_i)\}$ for trial $j$.

\subsubsection{Ensemble Average}

The ensemble-averaged trajectory is:
\begin{equation}
\langle \mathcal{S}(t) \rangle = \frac{1}{N_{\text{trials}}} \sum_{j=1}^{N_{\text{trials}}} \mathcal{S}_j(t)
\end{equation}

This averages out measurement noise and reveals the deterministic trajectory.

\subsubsection{Standard Deviation}

The standard deviation across trials is:
\begin{equation}
\sigma_{\mathcal{S}}(t) = \sqrt{ \frac{1}{N_{\text{trials}}} \sum_{j=1}^{N_{\text{trials}}} |\mathcal{S}_j(t) - \langle \mathcal{S}(t) \rangle|^2 }
\end{equation}

For our measurements, $\sigma/\langle \mathcal{S} \rangle < 10^{-6}$, indicating high reproducibility.

\subsubsection{Correlation Analysis}

We compute the temporal autocorrelation function:
\begin{equation}
C(\Delta t) = \langle \mathcal{S}(t) \cdot \mathcal{S}(t + \Delta t) \rangle
\end{equation}

This reveals periodic or quasi-periodic structures in the trajectory, corresponding to recurrence dynamics (Section 7).
