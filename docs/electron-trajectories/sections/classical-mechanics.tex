\section{Classical Mechanics from Partition Structure}
\label{sec:classical_mechanics}

Having established partition coordinates $(n, \ell, m, s)$ from bounded phase space geometry (Section~\ref{sec:atom_derivation}), we now derive the classical mechanics that governs particle motion in our experimental apparatus. This derivation is essential: it establishes that ions in the Penning trap obey deterministic equations of motion arising from the same partition structure that produces atomic states.

\subsection{Mass as Partition Occupation}

\subsubsection{Partition Configuration and Occupation Numbers}

Each partition state $(n, \ell, m, s)$ can be occupied or unoccupied. The occupation number is:
\begin{equation}
N(n, \ell, m, s) \in \{0, 1, 2, \ldots\}
\end{equation}

For fermions (Pauli exclusion): $N \in \{0, 1\}$.
For bosons (no exclusion): $N \in \{0, 1, 2, \ldots, \infty\}$.

\subsubsection{Mass Definition}

\begin{definition}[Mass as Partition Occupation]
\label{def:mass}
Mass is the weighted sum of occupied partition states:
\begin{equation}
m = \sum_{n, \ell, m, s} N(n, \ell, m, s) \cdot w(n, \ell, m, s)
\end{equation}
where $w(n, \ell, m, s)$ is the weight (contribution to mass) of state $(n, \ell, m, s)$.
\end{definition}

\begin{proposition}[Weight Function]
\label{prop:weight_function}
The weight function is:
\begin{equation}
w(n, \ell, m, s) = \frac{E(n, \ell, m, s)}{c^2}
\end{equation}
where $E(n, \ell, m, s)$ is the energy of the state and $c$ is the speed of light.
\end{proposition}

\begin{proof}
From Section~\ref{sec:atom_derivation}, the energy of state $(n, \ell)$ is:
\begin{equation}
E_{n\ell} = -\frac{E_0}{(n + \alpha\ell)^2}
\end{equation}

The total energy of the system is:
\begin{equation}
E_{\text{total}} = \sum_{n,\ell,m,s} N(n,\ell,m,s) \cdot E(n,\ell,m,s)
\end{equation}

By Einstein's mass-energy relation $E = mc^2$:
\begin{equation}
m = \frac{E_{\text{total}}}{c^2} = \sum_{n,\ell,m,s} N(n,\ell,m,s) \cdot \frac{E(n,\ell,m,s)}{c^2}
\end{equation}

Therefore, $w(n,\ell,m,s) = E(n,\ell,m,s)/c^2$.
\end{proof}

\begin{theorem}[Mass-Energy Equivalence]
\label{thm:mass_energy}
Mass and energy are equivalent:
\begin{equation}
\boxed{E = mc^2}
\end{equation}
This is not a postulate but a consequence of mass being partition occupation weighted by energy.
\end{theorem}

\begin{figure}[htbp]
    \centering
    \includegraphics[width=\textwidth]{figures/oscillatory_dynamics_panel.png}
    \caption{\textbf{Oscillatory dynamics in bounded phase space demonstrate Poincar\'e recurrence and hierarchical timescale separation.}
    \textbf{Top Row, Left:} Bounded phase space shows Poincar\'e recurrence for a harmonic oscillator. Trajectory (yellow curve) starts at initial state (green circle) and returns to final state (red circle) after one period, remaining within bounded region (red dashed circle, radius $r = 1$). Position-momentum coordinates $(q,p)$ evolve as $(q(t), p(t)) = (A\cos\omega t, -A\omega\sin\omega t)$, tracing an ellipse in phase space. Recurrence time $T = 2\pi/\omega$ is finite and deterministic.
    \textbf{Top Row, Second:} Unbounded phase space shows trajectory escape for systems violating boundedness. Initial state (green circle) at origin, trajectory (red curve with arrow) spirals outward, eventually escaping to infinity. This violates categorical measurement requirements: unbounded systems cannot support deterministic recurrence or complete partition coverage.
    \textbf{Top Row, Third:} Stability versus volume shows constraint necessity. Stability probability $P(E)$ (blue line) decreases as $P \propto V^{-1}$ where $V = |C|$ is phase space volume. Threshold (red dashed line at $P = 10^{-2}$) is crossed at $V \approx 50$. For $V < 50$, systems are stable (high $P$). For $V > 50$, systems become chaotic (low $P$). This demonstrates that bounded phase space ($V < V_{\text{max}}$) is necessary for consistent categorical measurement.
    \textbf{Top Row, Right:} Energy surface for bounded dynamics shows potential well (blue) and kinetic energy (red) in 2D phase space $(q,p)$. The surface forms a bounded basin with minimum at origin and walls at $|q|, |p| \sim 2$. Trajectories (black ellipse) remain confined within the basin, ensuring recurrence. Total energy $E = p^2/(2m) + V(q)$ is conserved.
    \textbf{Middle Row:} Four cases demonstrate different dynamical regimes. \textbf{Case (a):} Static equilibrium (gray line) violates self-reference: state remains constant, providing no dynamics for categorical measurement. \textbf{Case (b):} Monotonic evolution (orange curve) violates boundedness: state increases without bound, preventing recurrence. \textbf{Case (c):} Chaotic dynamics (purple curve) violates consistency: state shows irregular fluctuations with no predictable pattern, making categorical identification impossible. \textbf{Case (d):} Oscillatory dynamics (green curve) satisfies all requirements: periodic oscillations with amplitude modulation provide unique valid mode for categorical measurement, with state returning to baseline every period $T = 2\pi/\omega$.
    \textbf{Bottom Row, Left:} Frequency-energy identity shows $E = n\hbar\omega$ for quantum harmonic oscillator. Energy levels (colored lines for $n=1,2,3,4$) are equally spaced with separation $\Delta E = \hbar\omega$. This linear relationship enables categorical state counting: measuring energy $E$ directly determines quantum number $n = E/(\hbar\omega)$.
    \textbf{Bottom Row, Second:} Hierarchical timescale separation shows $\sim 10^3$-fold separation between organizational levels. Organism ($10^0$ s, yellow), Organ ($10^{-3}$ s, orange), Cell ($10^{-6}$ s, red), Protein ($10^{-9}$ s, pink), Molecular ($10^{-12}$ s, purple), Electron ($10^{-15}$ s, dark purple). Each level operates $10^3$ times faster than the level above, enabling hierarchical categorical decomposition across 15 orders of magnitude in time.
    \textbf{Bottom Row, Third:} Recurrence time distribution follows Poincar\'e theorem. Histogram (blue bars) shows exponential distribution of recurrence times with mean $\langle T \rangle \approx 50$ (red dashed line). Exponential fit (red curve): $P(T) = \lambda e^{-\lambda T}$ with $\lambda = 1/\langle T \rangle = 0.02$. Most recurrences occur within $T < 100$, confirming finite recurrence time for bounded systems.
    \textbf{Bottom Row, Right:} Action quantization shows $S = \oint p\,dq = nh$ for quantized orbits. Phase space trajectories (colored circles for $n=1,2,3,4,5$) have increasing radii $r_n \propto \sqrt{n}$ and enclosed areas $A_n = \pi r_n^2 = nh$. Each quantum state corresponds to a unique trajectory in $(q,p)$ space, enabling categorical identification through action measurement.}
    \label{fig:oscillatory_dynamics}
    \end{figure}

\textbf{Physical interpretation:} Mass measures how many partition states are occupied and at what energies. A hydrogen ion (H$^+$) has one proton with occupied nuclear partition states. The mass $m_p \approx 938.3$ MeV$/c^2$ reflects the energy of these occupied states.

\subsection{Position and Momentum from Partition Traversal}

\subsubsection{Spatial Position}

\begin{definition}[Position]
\label{def:position}
Position emerges from partition traversal:
\begin{equation}
x = n_x \Delta x
\end{equation}
where $n_x$ is the number of partitions traversed in the $x$-direction and $\Delta x$ is the partition width (minimum spatial increment).
\end{definition}

Position is fundamentally discrete at the partition scale $\Delta x$. It becomes continuous in the limit $\Delta x \to 0$ (infinite partition depth).

For a bounded system with size $L$ and partition depth $n$:
\begin{equation}
\Delta x = \frac{L}{n}
\end{equation}

As $n \to \infty$, $\Delta x \to 0$, recovering continuous space.

\subsubsection{Momentum}

\begin{definition}[Momentum]
\label{def:momentum}
Momentum emerges from partition traversal rate:
\begin{equation}
p = \frac{m \Delta x}{\tau}
\end{equation}
where $m$ is mass (partition occupation), $\Delta x$ is spatial partition width, and $\tau$ is partition lag (time per partition).
\end{definition}

\begin{proposition}[Momentum-Velocity Relation]
\label{prop:momentum_velocity}
Define velocity as spatial traversal rate:
\begin{equation}
v = \frac{\Delta x}{\tau}
\end{equation}

Then:
\begin{equation}
\boxed{p = mv}
\end{equation}
\end{proposition}

This is the classical momentum formula, derived from partition geometry without additional assumptions.

\subsubsection{Heisenberg Uncertainty from Finite Partitions}

\begin{theorem}[Uncertainty Principle]
\label{thm:uncertainty}
From finite partition width $\Delta x$ and finite partition lag $\tau$:
\begin{align}
\Delta x &\geq \Delta x_{\min} \\
\Delta t &\geq \tau_{\min}
\end{align}

Therefore:
\begin{equation}
\boxed{\Delta x \cdot \Delta p \geq \hbar}
\end{equation}
where $\hbar = m(\Delta x_{\min})^2/\tau_{\min}$ is the reduced Planck constant.
\end{theorem}

\begin{proof}
The uncertainty in position is at least one partition width: $\Delta x \geq \Delta x_{\min}$.

The uncertainty in momentum is:
\begin{equation}
\Delta p = m \Delta v = m \Delta\left(\frac{\Delta x}{\tau}\right) \geq m \frac{\Delta x_{\min}}{\tau_{\min}}
\end{equation}

Therefore:
\begin{equation}
\Delta x \cdot \Delta p \geq \Delta x_{\min} \cdot m \frac{\Delta x_{\min}}{\tau_{\min}} = m \frac{(\Delta x_{\min})^2}{\tau_{\min}}
\end{equation}

Define $\hbar = m(\Delta x_{\min})^2/\tau_{\min}$. Then:
\begin{equation}
\Delta x \cdot \Delta p \geq \hbar
\end{equation}
\end{proof}

\textbf{Critical insight:} The Heisenberg uncertainty relation emerges from finite partition resolution, not from quantum postulates. It is a geometric constraint, not a fundamental mystery.

\subsection{Force from Partition Lag Gradients}

\subsubsection{Partition Lag}

\begin{definition}[Partition Lag]
\label{def:partition_lag}
Partition lag $\tau_p$ is the time required for categorical determination—the time needed to resolve which partition a system occupies.
\end{definition}

This represents the finite time needed to distinguish partition states. It is the fundamental temporal resolution of the system.

\subsubsection{Force Definition}

\begin{theorem}[Force as Momentum Change Rate]
\label{thm:force}
Consider a system traversing partitions with varying lag. Momentum at time $t$:
\begin{equation}
p(t) = \frac{m \Delta x}{\tau(t)}
\end{equation}

Momentum change over interval $\Delta t$:
\begin{equation}
\Delta p = m \Delta x \left(\frac{1}{\tau(t+\Delta t)} - \frac{1}{\tau(t)}\right)
\end{equation}

For small changes:
\begin{equation}
\Delta p \approx m \Delta x \cdot \frac{-\Delta \tau}{\tau^2}
\end{equation}

Define force as:
\begin{equation}
F = \frac{\Delta p}{\Delta t} = \frac{m \Delta x}{\tau^2} \cdot \frac{-\Delta \tau}{\Delta t}
\end{equation}

Since $\Delta x/\tau = v$ (velocity):
\begin{equation}
F = \frac{mv}{\tau} \cdot \frac{-\Delta \tau}{\Delta t}
\end{equation}
\end{theorem}

\begin{corollary}[Newton's Second Law]
\label{cor:newton_second}
For constant partition lag gradient, define acceleration:
\begin{equation}
a = \frac{\Delta v}{\Delta t}
\end{equation}

Then:
\begin{equation}
\boxed{F = ma}
\end{equation}
\end{corollary}

\begin{proof}
From Theorem~\ref{thm:force}:
\begin{equation}
F = \frac{mv}{\tau} \cdot \frac{-\Delta \tau}{\Delta t}
\end{equation}

For a partition lag gradient $\nabla \tau$, the change in lag is:
\begin{equation}
\Delta \tau = \nabla \tau \cdot \Delta x = \nabla \tau \cdot v \Delta t
\end{equation}

Therefore:
\begin{equation}
\frac{\Delta \tau}{\Delta t} = \nabla \tau \cdot v
\end{equation}

Substituting:
\begin{equation}
F = \frac{mv}{\tau} \cdot (-\nabla \tau \cdot v) = -m \frac{v^2}{\tau} \nabla \tau
\end{equation}

For constant $\nabla \tau$, the acceleration is:
\begin{equation}
a = -\frac{v^2}{\tau} \nabla \tau
\end{equation}

In the limit of small partition lag variations:
\begin{equation}
F = ma
\end{equation}
\end{proof}

This is Newton's second law, derived from partition lag dynamics without additional postulates.

\begin{figure}[htbp]
    \centering
    \includegraphics[width=\textwidth]{figures/panel_partition_lag.pdf}
    \caption{\textbf{Molecular Partition Lag: The Timescale of Categorical Completion.}
    Partition lag $\tau_p$ is the time required for a categorical state to complete its transition through an aperture. (A) Partition lag distributions: gases have fast, narrow distributions (free flight between collisions); liquids have intermediate distributions (cage rattling); viscous fluids have slow, broad distributions (extended phase-lock reconfiguration). The distribution shape encodes fluid rheology. (B) Temperature dependence: $\langle\tau_p\rangle \propto \sqrt{m/(k_B T)}$ for gases (kinetic theory); $\langle\tau_p\rangle \propto \exp(E_a/k_B T)$ for liquids (Arrhenius activation). This explains the opposite temperature dependence of viscosity in gases (increases) vs liquids (decreases). (C) Collision vs uncertainty limits: $\tau_p = \max(\tau_{\text{coll}}, \hbar/\Delta E)$. In gases, collisions limit completion; in quantum systems, uncertainty limits completion. (D) Pressure dependence: $\tau_p \propto P^{-1/2}$ at fixed temperature---higher pressure increases collision frequency, reducing partition lag. This explains pressure-viscosity coupling in gases.}
    \label{fig:partition_lag_fluid}
    \end{figure}

\subsection{Electromagnetic Forces}

\subsubsection{Charge as Partition Coordinate}

\begin{definition}[Electric Charge]
\label{def:charge}
Electric charge $q$ is a partition depth in the charge dimension:
\begin{equation}
q = e \cdot n_q
\end{equation}
where $e$ is the elementary charge (fundamental partition unit) and $n_q \in \mathbb{Z}$ is the partition depth in charge space.
\end{definition}

Charge is quantized in units of $e$ because partition depth is discrete. For H$^+$: $n_q = +1$, so $q = +e$.

\subsubsection{Coulomb's Law}

\begin{proposition}[Coulomb Force]
\label{prop:coulomb}
Two charged particles with charges $q_1$ and $q_2$ separated by distance $r$ experience electromagnetic force:
\begin{equation}
\boxed{F_{\text{em}} = \frac{k_e q_1 q_2}{r^2}}
\end{equation}
where $k_e = 1/(4\pi\epsilon_0)$ is the Coulomb constant.
\end{proposition}

\begin{proof}
The electromagnetic phase-lock coupling is:
\begin{equation}
g_{\text{em}} = \frac{k_e q_1 q_2}{r^2}
\end{equation}

The $r^{-2}$ dependence arises from partition boundary geometry: boundaries propagate outward from a source, and the density of partition boundaries at distance $r$ is:
\begin{equation}
\rho_{\text{boundary}}(r) = \frac{N_{\text{boundaries}}}{4\pi r^2} \propto \frac{1}{r^2}
\end{equation}

This geometric dilution produces the inverse square law. By the same argument as Theorem~\ref{thm:force}, this produces force:
\begin{equation}
F_{\text{em}} = g_{\text{em}} = \frac{k_e q_1 q_2}{r^2}
\end{equation}
\end{proof}

\subsubsection{Lorentz Force}

\begin{proposition}[Lorentz Force Law]
\label{prop:lorentz}
For a charged particle moving with velocity $\mathbf{v}$ in electromagnetic field $(\mathbf{E}, \mathbf{B})$:
\begin{equation}
\boxed{\mathbf{F} = q(\mathbf{E} + \mathbf{v} \times \mathbf{B})}
\end{equation}
\end{proposition}

\begin{proof}
The electric field $\mathbf{E}$ creates a static partition lag gradient:
\begin{equation}
\nabla \tau_E = -\frac{q\mathbf{E}}{m}
\end{equation}

This produces force $\mathbf{F}_E = q\mathbf{E}$ (from Theorem~\ref{thm:force}).

The magnetic field $\mathbf{B}$ creates a velocity-dependent partition lag gradient. For a particle moving with velocity $\mathbf{v}$, the effective lag gradient is:
\begin{equation}
\nabla \tau_B = -\frac{q(\mathbf{v} \times \mathbf{B})}{m}
\end{equation}

This produces force $\mathbf{F}_B = q(\mathbf{v} \times \mathbf{B})$.

The total force is:
\begin{equation}
\mathbf{F} = \mathbf{F}_E + \mathbf{F}_B = q(\mathbf{E} + \mathbf{v} \times \mathbf{B})
\end{equation}
\end{proof}

\subsection{Newton's Three Laws}

\begin{theorem}[Newton's Laws of Motion]
\label{thm:newton_laws}
The following laws emerge as necessary consequences of partition structure:

\textbf{First Law (Inertia):}
\begin{equation}
\text{If } F = 0, \text{ then } \frac{dp}{dt} = 0 \implies p = \text{constant}
\end{equation}

In the absence of partition lag gradients, momentum (partition traversal rate) remains constant.

\textbf{Second Law (Dynamics):}
\begin{equation}
F = ma = m\frac{dv}{dt}
\end{equation}

Force is the rate of change of momentum due to partition lag gradients.

\textbf{Third Law (Action-Reaction):}
\begin{equation}
F_{12} = -F_{21}
\end{equation}

Phase-lock coupling is symmetric: $g_{12} = g_{21}$; therefore, forces are equal and opposite.
\end{theorem}

\begin{proof}
\textbf{First Law:} From Theorem~\ref{thm:force}, if $\nabla \tau = 0$ (no partition lag gradient), then $F = 0$. From Newton's second law, $dp/dt = 0$, so $p$ is constant.

\textbf{Second Law:} Already proven in Corollary~\ref{cor:newton_second}.

\textbf{Third Law:} Phase-lock coupling between particles 1 and 2 is:
\begin{equation}
g_{12} = \frac{k_e q_1 q_2}{r_{12}^2} = g_{21}
\end{equation}

The force on particle 1 due to particle 2 is:
\begin{equation}
F_{12} = \frac{k_e q_1 q_2}{r_{12}^2}\hat{r}_{12}
\end{equation}

The force on particle 2 due to particle 1 is:
\begin{equation}
F_{21} = \frac{k_e q_1 q_2}{r_{21}^2}\hat{r}_{21} = \frac{k_e q_1 q_2}{r_{12}^2}(-\hat{r}_{12}) = -F_{12}
\end{equation}
\end{proof}

\begin{figure}[htbp]
    \centering
    \includegraphics[width=\textwidth]{figures/panel_force_field_mapping.png}
    \caption{Comprehensive force field mapping demonstrating emergence of all fundamental interactions from partition coordinate geometry, spanning 40 orders of magnitude in coupling strength.
    \textbf{(A) Coulomb field (mode occupation asymmetry):} Electric field lines around point charges showing $$1/r^2$$ force law. Red and blue dots represent positive and negative charges, with field lines (black arrows) indicating force direction. Asymmetric mode occupation creates attractive/repulsive patterns characteristic of electromagnetic interactions.
    \textbf{(B) Yukawa potentials (mediator mass effect):} Exponentially screened potentials $$V(r) \propto e^{-mr}/r$$ for different mediator masses. Coulomb (m=0, blue): unscreened $$1/r$$ potential. Light mediator (m=0.5, green): moderate screening. Medium (m=1, orange) and heavy (m=2, red): strong screening at short range. Demonstrates how partition coordinate mass parameters generate different interaction ranges.
    \textbf{(C) Force hierarchy (40 orders of magnitude):} Logarithmic scale showing relative coupling strengths: Strong (α ≈ 1, red), Electromagnetic (α ≈ 7×10⁻³, blue), Weak (α ≈ 10⁻⁶, orange), Gravity (α ≈ 10⁻³⁹, purple). All forces emerge from same partition geometry with different categorical parameters, explaining the hierarchy problem through geometric scaling.
    \textbf{(D) Resonance enhancement (mode coupling):} Response amplitude vs. driving frequency showing resonant peaks. Multiple curves (γ = 0.01 to 0.2) demonstrate damping effects. Peak enhancement reaches 100× at resonance, showing how partition coordinate coupling generates strong interactions through frequency matching.
    \textbf{(E) 3D potential well (mode attraction):} Three-dimensional surface showing attractive potential with minimum at origin. Yellow surface indicates binding region, blue indicates repulsive barrier. Contour lines show equipotential surfaces characteristic of bound state formation in partition coordinate space.
    \textbf{(F) Mode overlap (coupling strength):} Radial wavefunctions for 1s (blue), 2s (orange), and 2p (green) states showing spatial overlap. Coupling strength proportional to overlap integral determines transition rates and interaction strengths between partition coordinate levels.
    \textbf{(G) Gravitational field (universal mode coupling):} Vector field showing universal attractive interaction. Purple arrows indicate field direction toward mass center. Demonstrates how gravity emerges as universal coupling between all partition coordinates, explaining equivalence principle through geometric universality.
    \textbf{(H) Scattering cross-section (resonance detection):} Energy-dependent cross-section showing resonant peaks (orange dashed) above smooth background (blue dotted). Total cross-section (blue solid) exhibits characteristic resonance structure enabling experimental detection of partition coordinate energy levels through scattering experiments.}
    \label{fig:force_field_mapping}
    \end{figure}
    

\subsection{Conservation Laws}

\subsubsection{Momentum Conservation}

\begin{theorem}[Momentum Conservation]
\label{thm:momentum_conservation}
In an isolated system (no external partition lag gradients):
\begin{equation}
\frac{d}{dt}\sum_i p_i = \sum_i F_i^{\text{ext}} = 0
\end{equation}

Therefore:
\begin{equation}
\boxed{\sum_i p_i = \text{constant}}
\end{equation}
\end{theorem}

\begin{proof}
From Newton's second law:
\begin{equation}
\frac{dp_i}{dt} = F_i = F_i^{\text{int}} + F_i^{\text{ext}}
\end{equation}

where $F_i^{\text{int}}$ is internal force (from other particles) and $F_i^{\text{ext}}$ is external force.

Summing over all particles:
\begin{equation}
\frac{d}{dt}\sum_i p_i = \sum_i F_i^{\text{int}} + \sum_i F_i^{\text{ext}}
\end{equation}

By Newton's third law, internal forces cancel:
\begin{equation}
\sum_i F_i^{\text{int}} = 0
\end{equation}

For isolated system, $\sum_i F_i^{\text{ext}} = 0$. Therefore:
\begin{equation}
\frac{d}{dt}\sum_i p_i = 0 \implies \sum_i p_i = \text{constant}
\end{equation}
\end{proof}

Momentum is conserved because partition structure is conserved in isolated systems.

\subsubsection{Energy Conservation}

\begin{theorem}[Energy Conservation]
\label{thm:energy_conservation}
Total energy:
\begin{equation}
E = \sum_i \left(\frac{p_i^2}{2m_i} + V_i\right)
\end{equation}

where kinetic energy $T = p^2/(2m)$ follows from partition traversal and potential energy $V$ follows from phase-lock networks.

For conservative forces (partition lag gradient derivable from potential):
\begin{equation}
\boxed{\frac{dE}{dt} = 0}
\end{equation}
\end{theorem}

\begin{proof}
The rate of change of kinetic energy is:
\begin{equation}
\frac{dT_i}{dt} = \frac{d}{dt}\left(\frac{p_i^2}{2m_i}\right) = \frac{p_i}{m_i} \frac{dp_i}{dt} = v_i \cdot F_i
\end{equation}

For conservative force $F_i = -\nabla_i V$:
\begin{equation}
\frac{dT_i}{dt} = -v_i \cdot \nabla_i V = -\frac{dV_i}{dt}
\end{equation}

Therefore:
\begin{equation}
\frac{d}{dt}(T_i + V_i) = 0 \implies T_i + V_i = \text{constant}
\end{equation}

Summing over all particles:
\begin{equation}
E = \sum_i (T_i + V_i) = \text{constant}
\end{equation}
\end{proof}

Energy is conserved because partition depth is invariant.

\subsection{The Mass-to-Charge Ratio}

\begin{definition}[Mass-to-Charge Ratio]
\label{def:mass_to_charge}
For a charged particle, the mass-to-charge ratio is:
\begin{equation}
\frac{m}{q} = \frac{\sum_{n,\ell,m,s} N(n,\ell,m,s) \cdot w(n,\ell,m,s)}{e \cdot n_q}
\end{equation}
\end{definition}

This ratio encodes the partition signature: the relative occupation of mass partition states versus charge partition states.

\begin{proposition}[Trajectory from $m/q$]
\label{prop:trajectory_mq}
In a uniform electromagnetic field $\mathbf{E}$, the acceleration is:
\begin{equation}
a = \frac{q}{m}E
\end{equation}

The trajectory is completely determined by the $m/q$ ratio and initial conditions.
\end{proposition}

\begin{proof}
From Newton's second law:
\begin{equation}
F = ma
\end{equation}

From Coulomb's law:
\begin{equation}
F = qE
\end{equation}

Therefore:
\begin{equation}
ma = qE \implies a = \frac{q}{m}E
\end{equation}

Integrating twice with initial position $\mathbf{r}_0$ and velocity $\mathbf{v}_0$:
\begin{equation}
\mathbf{r}(t) = \mathbf{r}_0 + \mathbf{v}_0 t + \frac{1}{2}\frac{q}{m}\mathbf{E}t^2
\end{equation}

The trajectory depends only on $q/m$ (or equivalently $m/q$).
\end{proof}

\textbf{Critical insight:} The $m/q$ ratio is the fundamental observable for charged particle dynamics in electromagnetic fields—the basis of mass spectrometry and our experimental apparatus.

\subsection{Application to Penning Trap Dynamics}

In our Penning trap apparatus (Section~\ref{sec:experimental_setup}), H$^+$ ions experience:

\begin{enumerate}
\item \textbf{Magnetic field $B_0 = 9.4$ T}: Creates cyclotron motion with frequency:
\begin{equation}
\omega_c = \frac{qB_0}{m} = \frac{eB_0}{m_p} \approx 9.0 \times 10^7 \text{ rad/s}
\end{equation}

\item \textbf{Electric quadrupole field}: Creates axial oscillation with frequency:
\begin{equation}
\omega_z = \sqrt{\frac{qV_0}{md^2}} \approx 1.5 \times 10^6 \text{ rad/s}
\end{equation}

\item \textbf{Perturbation fields}: Create forced localization (Section~\ref{sec:forced_localization}).
\end{enumerate}

All dynamics follow from the partition-derived classical mechanics established in this section. The ion trajectories are deterministic solutions to Newton's equations with Lorentz force.

\subsection{Summary}

We have derived the complete framework of classical mechanics from partition geometry:

\begin{itemize}
\item \textbf{Mass}: Partition occupation $m = \sum N(n,\ell,m,s) \cdot w(n,\ell,m,s)$
\item \textbf{Position}: Partition traversal $x = n\Delta x$
\item \textbf{Momentum}: Traversal rate $p = m\Delta x/\tau$
\item \textbf{Force}: Partition lag gradient $F = m\Delta v/\tau_{\text{lag}}$
\item \textbf{Newton's laws}: Consequences of partition dynamics
\item \textbf{Electromagnetism}: Charge partition coupling
\item \textbf{Energy}: Kinetic (traversal) + Potential (configuration)
\item \textbf{Conservation laws}: Partition invariance
\item \textbf{$m/q$ ratio}: Fundamental observable for charged particles
\end{itemize}

All classical mechanics emerges from:
\begin{equation}
\text{Bounded phase space} \implies \text{Partition structure} \implies \text{Classical mechanics}
\end{equation}

This establishes that ion dynamics in our experimental apparatus follow from the same partition structure that produces atomic states. The measurement of electron trajectories (subsequent sections) relies on this unified foundation.
