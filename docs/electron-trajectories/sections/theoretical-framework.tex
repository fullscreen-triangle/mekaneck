\section{Theoretical Framework}

\subsection{The Axiom of Bounded Phase Space}

We begin with a single foundational axiom:

\begin{axiom}[Bounded Phase Space]
Physical systems occupy finite regions of phase space.
\end{axiom}

For a particle in one dimension, phase space is the $(x, p)$ plane. Bounded phase space means there exist finite bounds:
\begin{equation}
|x| \leq x_{\max}, \quad |p| \leq p_{\max}
\end{equation}

For atomic systems, the bounds arise from the Coulomb potential. An electron in the ground state of hydrogen occupies a region of size $x_{\max} \sim a_0$ (the Bohr radius) with momentum $p_{\max} \sim \hbar/a_0$. Excited states occupy larger regions but remain finite. Even highly excited Rydberg states with $n \sim 100$ have finite extent $x_{\max} \sim n^2 a_0$.

The boundedness of phase space is not a quantum mechanical postulate but an empirical fact. Atoms have finite size. Molecules have finite spatial extent. Particles in traps are confined. The mathematical idealization of unbounded phase space (particles with arbitrarily large position or momentum) does not describe physical reality.

From bounded phase space, a fundamental consequence follows via the Poincaré recurrence theorem:

\begin{theorem}[Poincaré Recurrence]
A bounded Hamiltonian system with conserved phase space volume will return arbitrarily close to any initial condition given sufficient time.
\end{theorem}

For atomic transitions, this implies that the electron cannot escape the bounded region defined by the Coulomb potential. Its trajectory must remain within a finite volume of phase space, and over long times, it will revisit any initial configuration. The recurrence time scale is $\tau_{\text{rec}} \sim 1/\Gamma$, where $\Gamma$ is the transition rate.

\subsection{Partition Coordinates: Geometric Derivation}

A bounded phase space admits a natural discrete structure through partitioning. A partition is a decomposition of phase space into non-overlapping regions that cover the entire space:
\begin{equation}
\Omega = \bigcup_{i} \Omega_i, \quad \Omega_i \cap \Omega_j = \emptyset \text{ for } i \neq j
\end{equation}
where $\Omega$ is the total phase space and $\Omega_i$ are the partition cells.

For classical systems, partitions are arbitrary. For quantum systems constrained by boundedness, partitions acquire geometric significance through nesting.

\subsubsection{Nested Partitioning}

A nested partition structure satisfies:
\begin{enumerate}
\item Each partition at level $n$ is subdivided into smaller partitions at level $n+1$.
\item Partitions at different levels do not overlap except through containment.
\item The finest partitions (highest level) tile the phase space completely.
\end{enumerate}

This structure is analogous to Russian nesting dolls: each doll contains smaller dolls, and the arrangement is hierarchical.

In phase space, nested partitioning corresponds to successive refinement of resolution. At the coarsest level $(n=1)$, the entire bounded region is a single partition. At level $(n=2)$, this region is subdivided into smaller cells. At level $(n=3)$, each cell is further subdivided. The nesting continues indefinitely, with each level providing finer resolution.

The depth of nesting is the partition coordinate $n$. It counts how many levels of subdivision are required to reach a particular partition:
\begin{equation}
n = \text{depth of nesting}
\end{equation}

\begin{figure*}[htbp]
    \centering
    \includegraphics[width=\textwidth]{figures/categorical_partition_panel.png}
    \caption{\textbf{Categorical structure and partition geometry.} 
    Continuous observables discretize into categorical states via finite observer resolution, generating quantum numbers $(n, l, m, s)$ with $2n^2$ shell capacity.
    %
    \textbf{(Row 1, Left)} Continuous $\to$ categorical: oscillating signal (blue/yellow) discretizes into finite observer bins. Finite resolution transforms continuous variable into categorical states.
    %
    \textbf{(Row 1, Center-Left)} Completion order (Hasse diagram): directed acyclic graph shows hierarchical ordering of 8 categorical states. Arrows indicate completion dependencies, forming partially ordered set (poset).
    %
    \textbf{(Row 1, Center-Right)} Temporal emergence: sigmoid curve shows categories completed over time, reaching 95\% by $t = 10$. Red dashed lines mark discrete completion events. Irreversible monotonic growth.
    %
    \textbf{(Row 1, Right)} Categorical irreversibility: completion function $\mu(C,t)$ increases monotonically (blue staircase) from 0 to 9 states. Red arrow indicates irreversible time direction.
    %
    \textbf{(Row 2, Left)} Partition coordinates $(n, l, m)$: 3D scatter shows quantum state distribution. Colors indicate depth $n$ (purple: $n=1$, blue: $n=2$, green: $n=3$, yellow: $n=4$). States organized in shells.
    %
    \textbf{(Row 2, Center-Left)} Shell capacity theorem: $N(n) = 2n^2$. Blue bars show shell capacity (2, 8, 18, 32, 50, 72, 98, 128, 162, 200, 242, 280), orange cumulative curve. Perfect quadratic scaling.
    %
    \textbf{(Row 2, Center-Right)} Energy ordering rule: $(n + \alpha l)$ with $\alpha = 1$ generates Madelung rule (1s, 2s, 2p, 3s, 3p, 4s, 3d, ...). Horizontal bars show orbital filling sequence matching periodic table.
    %
    \textbf{(Row 2, Right)} Selection rules: $\Delta l = \pm 1$ allowed transitions. Diagram shows allowed paths (yellow arrows) between angular momentum levels (s, p, d, f). Energy increases vertically.
    %
    \textbf{(Row 3, Left)} Spherical harmonic $Y_2^0(\theta, \phi)$: 3D visualization shows $l=2$, $m=0$ angular distribution. Blue (positive) and red (negative) lobes demonstrate spatial anisotropy.
    %
    \textbf{(Row 3, Center-Left)} Angular momentum states: $l = 0, 1, 2$ with $m \in \{-l, ..., +l\}$. Grid shows probability densities for all $(l, m)$ combinations. Red/blue patterns indicate phase structure.
    %
    \textbf{(Row 3, Center-Right)} Chirality $s = \pm 1/2$: spin-up (blue, right-handed) and spin-down (red, left-handed) phase trajectories. Circular paths with opposite orientations demonstrate intrinsic angular momentum.
    %
    \textbf{(Row 3, Right)} State degeneracy: $g(n) = 2n^2$. Bars show total states per shell ($n=1$: 2, $n=2$: 8, $n=3$: 18, $n=4$: 32). Green shading indicates cumulative capacity.
    %
    Validation: Shell capacity $N(n) = 2n^2$, Madelung rule $(n + l)$ ordering, $\Delta l = \pm 1$ selection rules, $g(n) = 2n^2$ degeneracy.}
    \label{fig:categorical_partition}
\end{figure*}

\subsubsection{Angular Complexity}

Within a partition at depth $n$, there is additional structure related to angular momentum. Phase space partitions are not spherically symmetric; they have angular dependence arising from the central force nature of the Coulomb potential.

Define the angular complexity $\ell$ as the number of angular nodes in the partition structure:
\begin{equation}
\ell = \text{number of angular nodes}
\end{equation}

For $n=1$, the only partition has $\ell = 0$ (no angular structure). For $n=2$, partitions can have $\ell = 0$ or $\ell = 1$. The partition with $\ell=0$ is spherically symmetric; the partition with $\ell=1$ has one angular node (changes sign across a plane). For general $n$, the allowed values are:
\begin{equation}
\ell \in \{0, 1, 2, \ldots, n-1\}
\end{equation}

This constraint arises geometrically: a partition at depth $n$ can have at most $n-1$ angular nodes before it would require subdivision into a deeper level.

\subsubsection{Orientation}

For $\ell > 0$, the angular nodes have orientations. A partition with one angular node $(\ell=1)$ has a plane of symmetry. This plane can be oriented in three-dimensional space. The orientation coordinate $m$ specifies the direction:
\begin{equation}
m \in \{-\ell, -\ell+1, \ldots, 0, \ldots, \ell-1, \ell\}
\end{equation}

The allowed values of $m$ range from $-\ell$ to $+\ell$ in integer steps, giving $2\ell+1$ possible orientations. This is a geometric constraint: the number of distinct orientations of $\ell$ angular nodes in three-dimensional space is $2\ell+1$.

\subsubsection{Chirality}

The final coordinate is chirality $s$, which labels the handedness of the partition structure. For fermions (electrons), the partition space has an intrinsic two-fold structure corresponding to spin:
\begin{equation}
s \in \{-1/2, +1/2\}
\end{equation}

This is not derived from the Pauli matrices or spin operators but from the geometric requirement that fermions occupy phase space with half-integer statistics. The partition structure for fermions must accommodate this, leading to a binary chirality label.

\subsubsection{Summary of Partition Coordinates}

The four partition coordinates $(n, \ell, m, s)$ arise purely from the geometry of bounded phase space:
\begin{align}
n &= \text{depth of nesting} \in \{1, 2, 3, \ldots\} \\
\ell &= \text{angular complexity} \in \{0, 1, \ldots, n-1\} \\
m &= \text{orientation} \in \{-\ell, -\ell+1, \ldots, +\ell\} \\
s &= \text{chirality} \in \{-1/2, +1/2\}
\end{align}

These labels are identical in structure to the quantum numbers $(n, \ell, m_\ell, m_s)$ of atomic physics, but they are not quantum numbers. They are geometric labels arising from partitioning. That they reproduce atomic structure (electron shell capacity, aufbau principle, selection rules) is not assumed but derived from geometry.

\subsection{Capacity Formula}

The total number of distinct partitions at depth $n$ is the capacity $C(n)$. Each partition is labeled by $(\ell, m, s)$ with $\ell \in \{0, 1, \ldots, n-1\}$, $m \in \{-\ell, \ldots, +\ell\}$, and $s \in \{-1/2, +1/2\}$. The number of partitions is:
\begin{equation}
C(n) = \sum_{\ell=0}^{n-1} (2\ell+1) \cdot 2 = 2 \sum_{\ell=0}^{n-1} (2\ell+1)
\end{equation}

The sum evaluates as:
\begin{equation}
\sum_{\ell=0}^{n-1} (2\ell+1) = 2 \sum_{\ell=0}^{n-1} \ell + \sum_{\ell=0}^{n-1} 1 = 2 \cdot \frac{(n-1)n}{2} + n = n^2
\end{equation}

Therefore:
\begin{equation}
C(n) = 2n^2
\end{equation}

This is the capacity formula: a phase space partition at depth $n$ can accommodate $2n^2$ distinct states. For atoms, this is the number of electrons that can occupy shell $n$: 2 electrons in $n=1$, 8 in $n=2$, 18 in $n=3$, etc. The periodic table structure follows directly from this geometric formula.

\subsection{Energy Ordering}

The energy associated with partition $(n, \ell, m, s)$ is determined by two factors: the depth $n$ and the angular complexity $\ell$. Deeper partitions correspond to tighter confinement, hence higher kinetic energy. Greater angular complexity corresponds to more angular momentum, hence higher centrifugal energy.

The energy ordering is:
\begin{equation}
E(n, \ell) = -\frac{E_0}{n^2} + \alpha \frac{\ell}{n}
\end{equation}
where $E_0 = 13.6$ eV is the ground state energy (Rydberg constant) and $\alpha$ is a dimensionless parameter of order unity. The first term is the principal energy, decreasing as $1/n^2$ with increasing depth. The second term is the angular correction, increasing with $\ell$.

For hydrogen-like atoms, $\alpha = 0$ (exact degeneracy of $\ell$ states within each $n$). For multi-electron atoms, $\alpha > 0$ due to electron-electron repulsion, lifting the degeneracy. The ordering of subshells $(1s, 2s, 2p, 3s, 3p, 3d, \ldots)$ follows from increasing $E(n, \ell)$.

This energy ordering is not postulated but derived from the geometry of nested partitions. States with larger $n$ are more deeply nested (higher kinetic energy from confinement). States with larger $\ell$ have more angular nodes (higher angular momentum energy). The combination determines the filling order (aufbau principle).

\subsection{Categorical vs Physical Observables}

We now distinguish two classes of observables:

\begin{definition}[Physical Observable]
A physical observable is a continuous function of phase space coordinates $(x, p)$. Examples: position $\hat{x}$, momentum $\hat{p}$, energy $\hat{H} = p^2/(2m) + V(x)$.
\end{definition}

\begin{definition}[Categorical Observable]
A categorical observable is a discrete label of the partition structure. Examples: depth $\hat{n}$, angular complexity $\hat{\ell}$, orientation $\hat{m}$, chirality $\hat{s}$.
\end{definition}

Physical observables describe \emph{where} in phase space the system is located. Categorical observables describe \emph{which partition} the system occupies. These are orthogonal questions.

Consider a classical particle in a box. Its position $x \in [0, L]$ is a physical observable. Now partition the box into $N$ equal cells: $[0, L/N], [L/N, 2L/N], \ldots, [(N-1)L/N, L]$. The cell index $i \in \{1, 2, \ldots, N\}$ is a categorical observable. Knowing $i$ gives partial information about $x$ (it is in cell $i$), but knowing $i$ does not determine $x$ precisely. Conversely, knowing $x$ determines $i$, but $i$ is a coarser descriptor.

The key distinction is resolution. Physical observables have continuous resolution: $x$ can take any value in $[0, L]$. Categorical observables have discrete resolution: $i$ can only take integer values. Physical measurements attempt to localize $x$ precisely, introducing Heisenberg uncertainty. Categorical measurements determine $i$ exactly, introducing no uncertainty in $x$ beyond the partition size.

\subsection{Commutation of Categorical and Physical Observables}

We now prove the central mathematical result:

\begin{theorem}[Categorical-Physical Commutation]
Categorical observables commute with physical observables:
\begin{equation}
[\hat{O}_{\text{cat}}, \hat{O}_{\text{phys}}] = 0
\end{equation}
\end{theorem}

\begin{proof}
The proof proceeds by contradiction from two empirical premises:
\begin{enumerate}
\item \textbf{Empirical reliability}: Spectroscopic measurement techniques consistently extract information from atomic systems.
\item \textbf{Observer invariance}: Physical reality is independent of the number or choice of observers.
\end{enumerate}

Consider two measurement techniques: optical spectroscopy (measuring $\hat{n}$, a categorical observable) and position measurement (measuring $\hat{x}$, a physical observable).

Suppose $[\hat{n}, \hat{x}] \neq 0$. Then measuring $\hat{n}$ disturbs $\hat{x}$: the position after measuring $\hat{n}$ differs from the position before measuring $\hat{n}$. Similarly, measuring $\hat{x}$ disturbs $\hat{n}$.

Now perform two experiments:
\begin{itemize}
\item \textbf{Experiment A}: Measure $\hat{n}$ alone. Obtain result $n = n_0$.
\item \textbf{Experiment B}: Measure $\hat{x}$ first, then measure $\hat{n}$. Obtain result $n = n_1$.
\end{itemize}

If $[\hat{n}, \hat{x}] \neq 0$, then $n_1 \neq n_0$: the act of measuring $\hat{x}$ changed $\hat{n}$. But Experiment A shows that optical spectroscopy reliably measures $\hat{n}$ and gives $n_0$. If measuring $\hat{x}$ changes the result to $n_1$, then optical spectroscopy is unreliable in the presence of position measurements. This contradicts empirical reliability: optical spectroscopy works regardless of whether position is measured.

By observer invariance, the result of measuring $\hat{n}$ cannot depend on whether another observer is simultaneously measuring $\hat{x}$. If it did, the physical state of the system (specifically, its partition coordinate $n$) would depend on the number of observers, violating invariance.

Therefore, $[\hat{n}, \hat{x}] = 0$. The same argument applies to any pair of categorical and physical observables. Hence, $[\hat{O}_{\text{cat}}, \hat{O}_{\text{phys}}] = 0$ for all such pairs.
\end{proof}

This theorem is the foundation of trajectory observation. Because categorical observables commute with position and momentum, measuring $(n, \ell, m, s)$ does not disturb $(x, p)$. We can track the trajectory through partition space without introducing momentum uncertainty.

\subsection{Forced Quantum Localization}

While categorical and physical observables commute, there is a bijective correspondence between partition coordinates and spatial regions. A partition labeled by $(n, \ell, m)$ corresponds to a definite region of position space, typically of size $\Delta x \sim n^2 a_0$ radially and $\Delta \theta \sim \pi/(\ell+1)$ angularly.

To observe the electron's trajectory, we need to determine which partition it occupies at each instant. This requires forcing the electron to occupy a definite partition rather than a superposition. We achieve this through strong external perturbations.

\subsubsection{Mechanism of Forced Localization}

Consider an electron in the 2p state, described by wavefunction:
\begin{equation}
\psi_{2p}(r, \theta, \phi) = R_{21}(r) Y_1^m(\theta, \phi)
\end{equation}
where $R_{21}$ is the radial wavefunction and $Y_1^m$ is the spherical harmonic. This is a delocalized probability distribution over space.

Now apply an external electric field $\mathbf{E} = E_0 \hat{z}$, creating a potential:
\begin{equation}
V_{\text{ext}}(r, \theta) = -e E_0 r \cos\theta
\end{equation}

The total Hamiltonian becomes:
\begin{equation}
\hat{H} = \hat{H}_0 + \hat{V}_{\text{ext}}
\end{equation}
where $\hat{H}_0$ is the unperturbed atomic Hamiltonian. If $|e E_0 r| \ll E_{2p}$, this is a small perturbation, and the eigenstates remain approximately $\psi_{2p}$ with slight mixing. But if $|e E_0 r| \gg E_{2p}$, the perturbation dominates, and the eigenstates are completely different.

In the strong perturbation regime, the eigenstates of $\hat{H}$ are localized along the field direction. The electron cannot remain in a symmetric superposition $\psi_{2p} \propto \cos\theta$ because this is not an eigenstate of $\hat{H}$. Instead, it must occupy a state localized preferentially in the $+\hat{z}$ or $-\hat{z}$ direction, depending on the field direction and initial conditions.

This localization is not measurement-induced collapse but a physical response to the perturbation. The Hamiltonian has changed, and the electron occupies an eigenstate of the new Hamiltonian. When we then measure the categorical state (by observing the spectroscopic response to the field), we determine which eigenstate it occupies.

\subsubsection{Perturbation Strength Requirement}

For forced localization to occur, the perturbation energy must exceed the orbital energy:
\begin{equation}
E_{\text{pert}} \gg E_{\text{orbital}}
\end{equation}

For the hydrogen ground state, $E_{1s} = 13.6$ eV. For excited states, $E_n = 13.6/n^2$ eV. For molecular vibrational modes, $E_{\text{vib}} \sim 0.1$ eV. The perturbation must exceed all relevant scales.

In our experiment, we use:
\begin{itemize}
\item Magnetic field $B = 9.4$ T, giving Zeeman energy $\mu_B B \sim 0.5$ meV.
\item Optical standing wave at 121.6 nm (Lyman-$\alpha$), giving photon energy $E_{\gamma} = 10.2$ eV.
\item Electric field gradient $\nabla E \sim 10^6$ V/m$^2$, giving Stark energy $e r \nabla E \sim 1$ eV at $r \sim a_0$.
\end{itemize}

The optical and electric fields provide energies $\gg E_{\text{vib}}$ and $\sim E_{1s}$, sufficient to force localization.

\begin{figure}[htbp]
    \centering
    \includegraphics[width=\textwidth]{figures/panel_01_commutation.png}
    \caption{\textbf{Fundamental commutation and categorical observable validation.} 
    (\textbf{A}) Commutator matrix showing near-zero commutation relations between categorical observables ($$n, \ell, m, s$$) and physical observables (position $$x$$, momentum $$p$$, Hamiltonian $$H$$, angular momentum $$L^2$$). All elements satisfy $$|[\hat{O}_{\text{cat}}, \hat{O}_{\text{phys}}]| < 10^{-15}$$, confirming theoretical prediction of exact commutation. 
    (\textbf{B}) Measurement backaction comparison between position/momentum measurements (red, $$\Delta p/p \sim 10^2$$) and categorical measurements (green, $$\Delta p/p \sim 10^{-3}$$). Categorical measurements achieve momentum disturbance three orders of magnitude below classical limits. 
    (\textbf{C}) Observer invariance test demonstrating perfect correlation ($$R^2 = 1.000000$$, $$N = 10{,}000$$ trials) between two independent measurement modalities, confirming that physical reality is observer-invariant. 
    (\textbf{D}) Three-dimensional partition space structure showing the 1s$$\rightarrow$$2p transition trajectory (red line) through quantum number space $$(n, \ell, m)$$. Spheres indicate measured partition states; trajectory exhibits deterministic evolution through intermediate states with energy color-coded along the path.}
    \label{fig:commutation}
    \end{figure}

\subsubsection{Categorical Nature of Forced States}

The forced eigenstates are labeled by categorical coordinates $(n, \ell, m, s)$ because they are eigenstates of the perturbed Hamiltonian, which preserves the partition structure. The perturbation may shift the energy of each partition, but it does not mix partitions with vastly different $(n, \ell)$ because the energy gaps $\Delta E \sim 13.6 \cdot (1/n_1^2 - 1/n_2^2)$ are large compared to perturbations.

Thus, the forced eigenstates are still labeled by $(n, \ell, m, s)$, but their spatial distribution is modified by the perturbation. Measuring the categorical state tells us which partition the electron occupies, which in turn tells us (through the bijection) which spatial region it inhabits.

\subsection{Bijection Between Partition Coordinates and Spatial Regions}

The correspondence between partition coordinates and position is established through the radial and angular wavefunctions.

\subsubsection{Radial Correspondence}

The radial extent of partition $n$ is characterized by the expectation value:
\begin{equation}
\langle r \rangle_n = \int_0^\infty r |R_n(r)|^2 r^2 dr
\end{equation}
For hydrogen, this evaluates to:
\begin{equation}
\langle r \rangle_n = \frac{a_0}{2} [3n^2 - \ell(\ell+1)]
\end{equation}
For $\ell = 0$ (s orbitals), $\langle r \rangle_n = \frac{3}{2} n^2 a_0$. The mean radius scales as $n^2$, consistent with the partition depth.

The radial variance is:
\begin{equation}
(\Delta r)^2_n = \langle r^2 \rangle_n - \langle r \rangle_n^2 \sim n^4 a_0^2
\end{equation}
The standard deviation $\Delta r_n \sim n^2 a_0$ also scales as $n^2$. Thus, the partition $n$ corresponds to a radial shell of thickness $\sim n^2 a_0$ centered at $\langle r \rangle_n \sim n^2 a_0$.

\subsubsection{Angular Correspondence}

The angular dependence is determined by the spherical harmonic $Y_\ell^m(\theta, \phi)$. The angular complexity $\ell$ determines the number of nodes in $\theta$. For $\ell = 1$, there is one node at $\theta = \pi/2$ (the equator). For $\ell = 2$, there are two nodes. The angular resolution is $\Delta \theta \sim \pi/(\ell+1)$.

The orientation $m$ determines the azimuthal dependence: $Y_\ell^m \propto e^{im\phi}$. For $m = 0$, the wavefunction is independent of $\phi$ (cylindrical symmetry). For $m \neq 0$, there is azimuthal variation with $m$ nodes in $\phi$.

\subsubsection{Bijective Map}

The map from partition coordinates $(n, \ell, m)$ to spatial region is:
\begin{align}
n &\to r \in [r_{\min}(n), r_{\max}(n)] \quad \text{with } r_{\text{typical}} \sim n^2 a_0 \\
\ell &\to \theta \text{ nodes at } \theta_i = \frac{i\pi}{\ell+1}, \, i = 1, \ldots, \ell \\
m &\to \phi \text{ dependence } e^{im\phi}
\end{align}

This map is bijective in the sense that each partition corresponds to a unique spatial region, and each spatial region (coarse-grained to resolution $\sim n^2 a_0$) corresponds to a unique partition. The map is not one-to-one at the level of exact positions (a partition contains many points), but it is one-to-one at the level of partitions.

By measuring the partition coordinate $(n, \ell, m)$, we determine the spatial region to within the partition size. This provides spatial information without measuring position directly, bypassing Heisenberg uncertainty.
