\section{Ternary Representation and S-Entropy Space}

\subsection{Base-3 Encoding of Partition Coordinates}

The ternary trisection algorithm naturally leads to a base-3 (ternary) representation of spatial coordinates. This section formalizes the mathematical structure.

\subsubsection{Ternary Digits (Trits)}

A ternary digit, or trit, takes values $\{0, 1, 2\}$. A sequence of $k$ trits encodes an integer in base 3:
\begin{equation}
N = \sum_{i=0}^{k-1} t_i \cdot 3^i
\end{equation}
where $t_i \in \{0, 1, 2\}$ is the $i$-th trit.

For example, the decimal number $42$ in ternary is:
\begin{equation}
42_{10} = 1120_3 = 1 \cdot 3^3 + 1 \cdot 3^2 + 2 \cdot 3^1 + 0 \cdot 3^0
\end{equation}

\subsubsection{Spatial Coordinate Encoding}

In the ternary trisection algorithm, each trisection step produces a trit $t_k \in \{0, 1, 2\}$ indicating which third of the current region contains the particle:
\begin{align}
t_k = 0 &\Rightarrow \text{particle in left third} \\
t_k = 1 &\Rightarrow \text{particle in middle third} \\
t_k = 2 &\Rightarrow \text{particle in right third}
\end{align}

After $k$ steps, we have a trit string $(t_{k-1}, t_{k-2}, \ldots, t_1, t_0)$ that encodes the particle's position to resolution $L/3^k$, where $L$ is the initial search length.

The position is:
\begin{equation}
x = \sum_{i=0}^{k-1} t_i \cdot \frac{L}{3^{i+1}} = L \sum_{i=0}^{k-1} \frac{t_i}{3^{i+1}}
\end{equation}

This is a ternary fraction: $x = L \cdot (0.t_{k-1} t_{k-2} \cdots t_1 t_0)_3$.

\subsubsection{Three-Dimensional Extension}

For three-dimensional space, each axis is independently encoded in ternary:
\begin{align}
x &= L_x \sum_{i=0}^{k-1} \frac{t_{x,i}}{3^{i+1}} \\
y &= L_y \sum_{i=0}^{k-1} \frac{t_{y,i}}{3^{i+1}} \\
z &= L_z \sum_{i=0}^{k-1} \frac{t_{z,i}}{3^{i+1}}
\end{align}

The complete position requires $3k$ trits: $k$ per dimension.

\subsection{S-Entropy Space}

The ternary representation naturally maps to a three-dimensional coordinate space called S-entropy space, denoted $\mathcal{S} = [0, 1]^3$.

\subsubsection{Definition of S-Entropy Coordinates}

The S-entropy coordinates $(S_k, S_t, S_e)$ are defined as:
\begin{align}
S_k &= \text{knowledge entropy} = \frac{H_k}{H_{\max}} \\
S_t &= \text{temporal entropy} = \frac{H_t}{H_{\max}} \\
S_e &= \text{evolution entropy} = \frac{H_e}{H_{\max}}
\end{align}
where $H_k, H_t, H_e$ are Shannon entropies associated with knowledge, time, and evolution, and $H_{\max} = \log_3 N$ is the maximum entropy (for $N$ possible states in base 3).

Each coordinate $S_i \in [0, 1]$ represents a normalized entropy, with $S_i = 0$ corresponding to complete knowledge (zero entropy) and $S_i = 1$ corresponding to complete ignorance (maximal entropy).

\subsubsection{Bijection Between Ternary Trits and S-Coordinates}

There is a bijective map between trit strings and points in S-entropy space. A trit string $(t_0, t_1, \ldots, t_{k-1})$ with $t_i \in \{0, 1, 2\}$ maps to:
\begin{equation}
S = \sum_{i=0}^{k-1} \frac{t_i}{3^{i+1}} = (0.t_0 t_1 t_2 \cdots)_3
\end{equation}

This is a ternary fraction in $[0, 1]$. Each trit $t_i$ refines the position in S-space by a factor of 3.

For three S-coordinates, we have three independent trit strings:
\begin{align}
S_k &= (0.t_{k,0} t_{k,1} t_{k,2} \cdots)_3 \\
S_t &= (0.t_{t,0} t_{t,1} t_{t,2} \cdots)_3 \\
S_e &= (0.t_{e,0} t_{e,1} t_{e,2} \cdots)_3
\end{align}

Each point $(S_k, S_t, S_e) \in [0,1]^3$ corresponds to an infinite trit string (or finite string for rational coordinates).

\subsection{Hierarchical Ternary Encoding: Multi-Level Structure}

The ternary encoding possesses a natural hierarchical structure that maps directly to molecular and atomic degrees of freedom. This multi-level organization simplifies experimental encoding and provides a unified framework for representing oscillatory, categorical, and partition aspects of quantum systems.

\subsubsection{Three Levels of Ternary Structure}

The ternary encoding operates on three distinct but interconnected levels:

\textbf{Level 1: Temporal Partitioning}

An oscillating system with period $T$ naturally divides into three temporal phases:
\begin{align}
t_1 &\in [0, T/3] \quad \text{(first third of period)} \\
t_2 &\in [T/3, 2T/3] \quad \text{(second third)} \\
t_3 &\in [2T/3, T] \quad \text{(third third)}
\end{align}

These temporal partitions correspond to distinct categorical states of the oscillator. For a vibrating molecule, these represent different phases of the vibrational cycle.

\textbf{Level 2: Spatial Partitioning}

Each temporal phase corresponds to a spatial configuration. For a one-dimensional oscillator:
\begin{align}
t_1 &\leftrightarrow p_1 \quad \text{(position in first third)} \\
t_2 &\leftrightarrow p_2 \quad \text{(position in second third)} \\
t_3 &\leftrightarrow p_3 \quad \text{(position in third third)}
\end{align}

The bijection $t_i \leftrightarrow p_i$ establishes that temporal and spatial partitions are equivalent descriptions of the same underlying categorical structure.

\textbf{Level 3: Partition Decomposition}

The number 3 itself admits three distinct partitions:
\begin{align}
3 &= 3 \quad \text{(single partition)} \\
3 &= 2 + 1 \quad \text{(binary decomposition)} \\
3 &= 1 + 1 + 1 \quad \text{(ternary decomposition)}
\end{align}

This partition structure corresponds to different ways of organizing the three states, relevant for composite systems and hierarchical measurements.


\subsubsection{Molecular Degrees of Freedom as Ternary Digits}

For molecular systems, each degree of freedom naturally encodes a trit value through its three-state structure:

\textbf{Electronic States (Ground, Absorption, Emission):}

The electronic state forms a natural trit:
\begin{align}
\text{Ground state} &\to \text{trit} = 0 \\
\text{Absorption state} &\to \text{trit} = 1 \\
\text{Emission state} &\to \text{trit} = 2
\end{align}

This encoding captures the three fundamental electronic configurations accessible during spectroscopic transitions.

\textbf{Vibrational Modes:}

A molecular vibration with period $T_{\text{vib}}$ divides into three phases:
\begin{align}
\text{Compression} &\to \text{trit} = 0 \\
\text{Equilibrium} &\to \text{trit} = 1 \\
\text{Extension} &\to \text{trit} = 2
\end{align}

\textbf{Rotational States:}

Molecular rotation about an axis admits three orientations relative to a measurement frame:
\begin{align}
\text{Clockwise} &\to \text{trit} = 0 \\
\text{Stationary} &\to \text{trit} = 1 \\
\text{Counterclockwise} &\to \text{trit} = 2
\end{align}

\textbf{Spin States:}

For spin-1 systems, the three spin projections map directly:
\begin{align}
m_s = -1 &\to \text{trit} = 0 \\
m_s = 0 &\to \text{trit} = 1 \\
m_s = +1 &\to \text{trit} = 2
\end{align}

\begin{figure}[htbp]
    \centering
    \includegraphics[width=\textwidth]{figures/hydrogen_bond_dynamics_analysis.png}
    \caption{\textbf{Hydrogen bond dynamics reveal geometric dependence, network connectivity, and quantum tunneling effects.}
    \textbf{(A)} H-bond energy landscape shows geometric dependence on O$\cdots$O distance (2.0--4.0 \AA) and O--H$\cdots$O angle (0--175$^\circ$). Energy (colorbar: 0--800{,}000 eV, blue to red) is minimized at optimal geometry (red star): distance $d_{\text{opt}} = 2.8$ \AA, angle $\theta_{\text{opt}} = 180^\circ$ (linear configuration). Energy increases steeply for $d < 2.5$ \AA (steric repulsion) and $d > 3.5$ \AA (weak interaction). Angular dependence shows preference for linear bonds ($\theta \approx 180^\circ$) with energy penalty for bent configurations. The landscape defines allowed regions for H-bond formation and guides proton transfer dynamics.
    \textbf{(B)} Water cluster snapshot shows H-bond network in 3D space $(x, y, z)$ with coordinates in nm. Purple spheres represent water molecules (50 nodes) connected by H-bonds. The network exhibits characteristic tetrahedral coordination with average degree $\langle k \rangle = 0.08$ (sparse network). Spatial distribution spans $\sim 2 \times 2 \times 2$ nm$^3$ volume. The snapshot captures instantaneous network topology at $t = 0$, providing input for connectivity analysis (panel H).
    \textbf{(C)} H-bond dynamics show formation and breaking over 10 ps trajectory. Blue bars: instantaneous number of H-bonds, fluctuating between 0 and 8. Red solid line: 50-point moving average, oscillating around mean value 2.2 (black dashed line). The dynamics show rapid fluctuations on sub-ps timescale superimposed on slower oscillations with period $\sim 2$ ps. This multi-timescale behavior reflects the hierarchical nature of H-bond networks, with individual bonds breaking/forming rapidly while the overall network structure evolves more slowly.
    \textbf{(D)} H-bond lifetime distribution shows exponential decay. Blue bars: observed lifetimes (histogram). Red curve: exponential fit $P(t) = \lambda e^{-\lambda t}$ with decay constant $\lambda = (0.01 \text{ ps})^{-1} = 100$ ps$^{-1}$. Most H-bonds have lifetimes $< 0.01$ ps, with tail extending to $\sim 0.1$ ps. Mean lifetime $\langle \tau \rangle = 1/\lambda = 0.01$ ps confirms rapid H-bond dynamics. The exponential distribution is characteristic of thermally activated processes with single energy barrier.
    \textbf{(E)} H-bond distance distribution shows peak at optimal distance. Red bars: probability density versus O$\cdots$O distance (2.6--3.4 \AA). Red dashed line marks optimal distance 2.80 \AA. Distribution is approximately Gaussian with mean $\langle d \rangle = 2.9$ \AA and standard deviation $\sigma_d \approx 0.2$ \AA. The peak position agrees with energy landscape minimum (panel A), confirming geometric optimization of H-bond network.
    \textbf{(F)} H-bond angle distribution shows preference for linear bonds. Green bars: probability density versus O--H$\cdots$O angle (150--180$^\circ$). Red dashed line marks optimal angle 180$^\circ$. Distribution peaks at $\theta \approx 175^\circ$ with width $\sigma_\theta \approx 10^\circ$. The near-linear preference reflects sp$^3$ hybridization of water oxygen and maximizes orbital overlap for H-bonding.
    \textbf{(G)} H-bond energy distribution shows mean energy $-498.208$ eV (red dashed line). Orange bars: probability density versus H-bond energy ($-1400$ to 0 eV). Distribution is broad with peak at $\sim -600$ eV and tail extending to $-200$ eV. The negative energies confirm stabilizing nature of H-bonds. Energy spread $\sim 400$ eV reflects geometric and environmental variations in the network.
    \textbf{(H)} H-bond network graph shows connectivity analysis. Purple circles: 50 water molecules (nodes). Lines: H-bonds (edges, 2 total). Network statistics: average degree 0.08, maximum degree 1. The sparse connectivity ($\langle k \rangle \ll 1$) indicates that most molecules are isolated or singly bonded at this snapshot, reflecting the transient nature of H-bond networks. Spatial arrangement shows clustering with isolated molecules at periphery.
    \textbf{(I)} Proton transfer potential shows quantum tunneling through 0.50 eV barrier. Orange curve: double-well potential with donor well (left) and acceptor well (right) separated by barrier at $x = 0$. Gray dotted line: barrier height 0.50 eV. Tunneling rate: $1.41 \times 10^{12}$ Hz (1.41 THz). Proton lifetime in donor well: 0.71 ps. The high tunneling rate enables rapid proton transfer on sub-ps timescale, contributing to the fast H-bond dynamics observed in panel C. Quantum tunneling is essential for proton mobility in H-bond networks and underlies the Grotthuss mechanism for proton conduction in water.}
    \label{fig:hydrogen_bond_dynamics}
    \end{figure}

\subsubsection{Hierarchical Digit Position Encoding}

The complete molecular state is encoded as a multi-digit ternary number, with each digit position corresponding to a specific degree of freedom:

\begin{equation}
\text{Molecular State} = [\text{Elec}][\text{Vib}][\text{Rot}][\text{Spin}][\cdots]
\end{equation}

where each bracket contains a trit value $\{0, 1, 2\}$.

\textbf{Example Encoding:}

For a hydrogen molecule in a specific state:
\begin{itemize}
\item Electronic: Ground state $\to$ trit = 0
\item Vibrational: Compression phase $\to$ trit = 0
\item Rotational: Clockwise $\to$ trit = 0
\item Spin: $m_s = +1$ $\to$ trit = 2
\end{itemize}

Complete state: $[0][0][0][2] = 0002_3$

\subsubsection{Mapping to Partition Coordinates}

Each trit position maps to a partition coordinate $(n, \ell, m, s)$:

\begin{table}[H]
\centering
\caption{Hierarchical Trit-to-Partition Mapping}
\begin{tabular}{ccccc}
\toprule
Digit Position & Degree of Freedom & Trit Value & Partition Coord & Physical Meaning \\
\midrule
1 & Electronic & $\{0,1,2\}$ & $n$ & Principal quantum number \\
2 & Vibrational & $\{0,1,2\}$ & $\ell$ & Angular momentum \\
3 & Rotational & $\{0,1,2\}$ & $m$ & Magnetic quantum number \\
4 & Spin & $\{0,1,2\}$ & $s$ & Spin projection \\
\bottomrule
\end{tabular}
\end{table}

The trit value at each position encodes the mode, phase, or energy level of that degree of freedom.

\subsubsection{S-Entropy Coupling Structure}

Each trit position also contributes to the three S-entropy coordinates $(S_k, S_t, S_e)$:

\begin{align}
S_k &= \text{Knowledge entropy} \quad \leftarrow \text{Electronic trit} \\
S_t &= \text{Temporal entropy} \quad \leftarrow \text{Vibrational trit} \\
S_e &= \text{Evolution entropy} \quad \leftarrow \text{Rotational trit}
\end{align}

The coupling is:
\begin{equation}
S_i = \sum_{j=0}^{k-1} \frac{t_{i,j}}{3^{j+1}}
\end{equation}
where $t_{i,j}$ is the trit value for S-coordinate $i$ at hierarchy level $j$.

\subsubsection{Experimental Simplification}

This hierarchical structure simplifies experimental encoding because:

\begin{enumerate}
\item \textbf{Natural Measurement Basis:} Each spectroscopic modality naturally measures one degree of freedom:
\begin{itemize}
\item Optical spectroscopy $\to$ Electronic trit
\item Raman spectroscopy $\to$ Vibrational trit
\item Microwave spectroscopy $\to$ Rotational trit
\item Magnetic resonance $\to$ Spin trit
\end{itemize}

\item \textbf{Independent Encoding:} Each modality provides one trit independently, enabling parallel measurement without cross-talk.

\item \textbf{Hierarchical Resolution:} Adding more trit positions (higher-order modes) increases resolution without changing the fundamental structure.

\item \textbf{Direct State Identification:} The complete trit string directly identifies the molecular state without requiring complex decoding algorithms.
\end{enumerate}

\subsubsection{Validation Through Hierarchical Consistency}

The hierarchical structure enables validation through consistency checks across levels:

\textbf{Temporal-Spatial Consistency:}

If temporal partition is $t_1$, spatial partition must be $p_1$:
\begin{equation}
t_i = j \Rightarrow p_i = j \quad \forall i, j \in \{1,2,3\}
\end{equation}

\textbf{Mode-Phase-Energy Consistency:}

The trit value must be consistent across mode, phase, and energy interpretations:
\begin{equation}
\text{mode}(t_i) = \text{phase}(t_i) = \text{energy}(t_i)
\end{equation}

\textbf{Partition Coordinate Consistency:}

The partition coordinates $(n,\ell,m,s)$ derived from trit positions must satisfy selection rules:
\begin{align}
\Delta n &\in \mathbb{Z} \\
\Delta \ell &= \pm 1 \\
\Delta m &\in \{0, \pm 1\} \\
\Delta s &= 0
\end{align}

These consistency checks provide additional validation of the measurement and ensure that the hierarchical structure is maintained throughout the trajectory.

\subsubsection{Example: Hydrogen 1s$\to$2p Transition}

For the hydrogen 1s$\to$2p transition, the hierarchical encoding is:

\textbf{Initial State (1s):}
\begin{itemize}
\item Electronic: $n=1$ $\to$ trit = 0 (ground level)
\item Angular: $\ell=0$ $\to$ trit = 0 (no angular momentum)
\item Magnetic: $m=0$ $\to$ trit = 1 (zero projection)
\item Spin: $s=+1/2$ $\to$ trit = 2 (spin up)
\end{itemize}
Trit string: $[0][0][1][2] = 0012_3$

\textbf{Final State (2p):}
\begin{itemize}
\item Electronic: $n=2$ $\to$ trit = 1 (first excited level)
\item Angular: $\ell=1$ $\to$ trit = 1 (one unit angular momentum)
\item Magnetic: $m=0$ $\to$ trit = 1 (zero projection)
\item Spin: $s=+1/2$ $\to$ trit = 2 (spin up)
\end{itemize}
Trit string: $[1][1][1][2] = 1112_3$

The transition changes only the first two trits (electronic and angular), consistent with the selection rules $\Delta n = 1$, $\Delta \ell = 1$, $\Delta m = 0$, $\Delta s = 0$.

\subsubsection{Trit-Coordinate Correspondence Theorem}

\begin{theorem}[Trit-Coordinate Correspondence]
Every sequence of $k$ trits $(t_0, t_1, \ldots, t_{k-1})$ with $t_i \in \{0, 1, 2\}$ corresponds to a unique point in $[0, 1]$ via the map:
\begin{equation}
S = \sum_{i=0}^{k-1} \frac{t_i}{3^{i+1}}
\end{equation}
Conversely, every rational number in $[0, 1]$ with denominator $3^k$ corresponds to a unique trit sequence of length $k$.
\end{theorem}

\begin{proof}
The map $S: \{0,1,2\}^k \to [0,1]$ is:
\begin{equation}
S(t_0, \ldots, t_{k-1}) = \sum_{i=0}^{k-1} \frac{t_i}{3^{i+1}}
\end{equation}

This is injective because distinct trit sequences yield distinct sums (base-3 representation is unique). The range is the set of rational numbers with denominator $3^k$:
\begin{equation}
\text{Range}(S) = \left\{ \frac{n}{3^k} \mid n = 0, 1, \ldots, 3^k - 1 \right\}
\end{equation}

There are $3^k$ such numbers, matching the number of trit sequences of length $k$. Hence, the map is bijective onto its range.

For infinite trit strings (limits as $k \to \infty$), the map extends to all real numbers in $[0, 1]$ via the Cantor set construction.
\end{proof}

\begin{figure}[htbp]
    \centering
    \includegraphics[width=\textwidth]{figures/panel_07_hydrogen_transition.png}
    \caption{\textbf{Complete trajectory reconstruction for hydrogen 1s$$\rightarrow$$2p transition.} 
    (\textbf{A}) Energy diagram showing non-instantaneous transition. Horizontal black lines indicate energy levels (1s at $$-13.6$$ eV, 2s/2p at $$-3.4$$ eV, 3s at $$-1.5$$ eV). Red trajectory line shows continuous evolution from 1s to 2p over $$\tau \sim 10$$ ns, with blue circles marking temporal snapshots at $$t = 0, 0.25\tau, 0.5\tau, 0.75\tau, 1.0\tau$$. Orange boxes indicate transient intermediate states. Trajectory exhibits temporary excursion through higher energy states before settling into 2p. 
    (\textbf{B}) Radial probability density evolution $$|\psi(r,t)|^2$$ as a function of radius and time. Color map shows probability density (blue = 0, yellow = 2.25). Initial 1s state localized at $$r \sim 1 a_0$$ (cyan dashed line). Final 2p state localized at $$r \sim 4 a_0$$ (yellow dashed line). Intermediate times show continuous radial expansion with characteristic 2p node formation. 
    (\textbf{C}) Angular momentum quantum number evolution. Blue curve shows $$\ell(t)$$ increasing from 0 to 2 (approaching final value $$\ell = 1$$ for 2p). Green curve shows $$m(t)$$ remaining constant at 0. Red curve shows $$n(t)$$ evolution from 1 to 2. Gray shaded region indicates quantum jump regime; beige box marks $$\ell$$ transition. Selection rule $$\Delta \ell = \pm 1$$ emerges as geometric constraint on trajectory. 
    (\textbf{D}) Three-dimensional spatial trajectory in Cartesian coordinates (units of $$a_0$$). Blue sphere indicates initial 1s position; red square indicates final 2p position. Purple/orange/magenta curves show trajectory path through intermediate positions. Semi-transparent disks represent probability density cross-sections at key time points. Trajectory exhibits helical structure characteristic of angular momentum change.}
    \label{fig:trajectory}
    \end{figure}

\subsection{Continuous Emergence: From Discrete Trits to Continuous Trajectories}

The trajectory reconstruction involves converting discrete trit strings (from measurements) to continuous spatial paths. This process is formalized by the continuous emergence theorem.

\subsubsection{Discrete Trajectory}

At each measurement step $i = 1, 2, \ldots, N$, we obtain a trit triplet $(t_{x,i}, t_{y,i}, t_{z,i})$ indicating the electron's partition. The discrete trajectory is:
\begin{equation}
\mathcal{T}_{\text{discrete}} = \{(t_{x,i}, t_{y,i}, t_{z,i})\}_{i=1}^N
\end{equation}

\subsubsection{Continuous Trajectory}

We construct a continuous trajectory by mapping each trit string to a position in $[0, 1]^3$ and interpolating:
\begin{equation}
\mathbf{S}(t) = \left( S_k(t), S_t(t), S_e(t) \right)
\end{equation}
where each component is a continuous function of time $t$.

The map from discrete to continuous is:
\begin{equation}
S_\alpha(t_i) = \sum_{j=0}^{i-1} \frac{t_{\alpha,j}}{3^{j+1}}
\end{equation}
for $\alpha \in \{k, t, e\}$ (corresponding to $x, y, z$).

Between measurement times, we interpolate linearly or with splines:
\begin{equation}
S_\alpha(t) = S_\alpha(t_i) + \frac{t - t_i}{t_{i+1} - t_i} \left( S_\alpha(t_{i+1}) - S_\alpha(t_i) \right) \quad \text{for } t \in [t_i, t_{i+1}]
\end{equation}

\subsubsection{Continuous Emergence Theorem}

\begin{theorem}[Continuous Emergence]
As the number of measurement steps $N \to \infty$ and the temporal resolution $\delta t \to 0$, the discrete trajectory converges to a continuous trajectory in S-entropy space:
\begin{equation}
\lim_{N \to \infty} \mathcal{T}_{\text{discrete}} = \mathbf{S}(t) \quad \text{in the metric topology of } C([0, \tau_{\text{transition}}], [0,1]^3)
\end{equation}
where $C([0, \tau], [0,1]^3)$ is the space of continuous functions from $[0, \tau]$ to $[0,1]^3$.
\end{theorem}

\begin{proof}
Each discrete point $\mathbf{S}(t_i)$ is defined by a finite trit string of length $i$. As $i$ increases, the trit string grows, refining the position in S-space by a factor of 3 per step. The error after $i$ steps is:
\begin{equation}
|\mathbf{S}(t) - \mathbf{S}(t_i)| \leq \frac{1}{3^i}
\end{equation}

This is a geometric sequence with ratio $1/3 < 1$, so it converges to zero as $i \to \infty$. Hence, the sequence $\{\mathbf{S}(t_i)\}$ is Cauchy in the metric space $[0,1]^3$ and converges to a unique limit $\mathbf{S}(t)$.

The interpolation between discrete points ensures continuity: for any $\epsilon > 0$, there exists $\delta > 0$ such that $|t - t'| < \delta$ implies $|\mathbf{S}(t) - \mathbf{S}(t')| < \epsilon$. This is the definition of a continuous function.
\end{proof}

This theorem justifies treating the discrete measurement sequence as a continuous trajectory in the limit of infinite temporal resolution.

\subsection{Trajectory Encoding: Position and Path Unification}

A profound property of the ternary representation is that position and trajectory (path) are encoded in the same trit string. This unification simplifies trajectory reconstruction.

\subsubsection{Position Encoding}

The position at time $t$ is encoded as a trit string $(t_0, t_1, \ldots, t_k)$ of length $k = \log_3(L/\Delta x)$, where $\Delta x$ is the spatial resolution. This string specifies the partition containing the particle.

\subsubsection{Path Encoding}

The trajectory from time $0$ to $t$ is encoded as the sequence of trit strings:
\begin{equation}
\text{Path} = \{(t_0^{(i)}, t_1^{(i)}, \ldots, t_k^{(i)})\}_{i=1}^{N(t)}
\end{equation}
where $N(t)$ is the number of measurement steps up to time $t$.

\subsubsection{Unification}

The key insight is that the trit string at time $t$ encodes not only the position at $t$ but also the cumulative effect of all previous positions. This is because the trit string is constructed sequentially: each new trit refines the previous string.

Formally, the trit string $(t_0, t_1, \ldots, t_k)$ encodes:
\begin{itemize}
\item \textbf{Position}: The partition containing the particle is $[S, S + 3^{-k}]$, where $S = \sum_{i=0}^{k-1} t_i/3^{i+1}$.
\item \textbf{Path}: The sequence of partitions visited is implicit in the nested structure of the trit string. Each prefix $(t_0, \ldots, t_j)$ for $j < k$ encodes the partition at an earlier time (coarser resolution).
\end{itemize}

Thus, the complete trit string contains both position and path information.

\begin{figure}[htbp]
    \centering
    \includegraphics[width=\textwidth]{figures/panel_ttr_d3.png}
    \caption{\textbf{Ternary Trajectory Recorder (TTR): $3^k$ Hierarchy Validation.} 
    \textbf{(Top Left)} Trajectories in $3^k$ space for single molecule. Purple lines: trajectory path through three-dimensional S-entropy coordinates $(S_k, S_t, S_e)$. Green sphere: starting configuration. Red sphere: ending configuration. Trajectory explores bounded region [0.30, 0.70]$^3$, demonstrating confined dynamics in categorical phase space. Multiple trajectories shown to illustrate ensemble behavior.
    \textbf{(Top Center)} Trit sequence encodes trajectory as colored bar code. Horizontal axis: step number (0-50). Vertical axis: trit value (0, 1, 2). Blue bars: trit 0 (oscillatory perspective, refine $S_k$). Green bars: trit 1 (categorical perspective, refine $S_t$). Red bars: trit 2 (partition perspective, refine $S_e$). Balanced color distribution indicates equal usage of all three perspectives.
    \textbf{(Top Right)} Perspective balance quantifies trit distribution. Three bars show probability of each perspective: blue (oscillatory, 0.33), green (categorical, 0.32), red (partition, 0.33). Black dashed line: uniform distribution (1/3 $\approx$ 0.333). All three perspectives balanced to within 1\%, validating triple equivalence. Vertical axis: probability (0.00-0.35).
    \textbf{(Middle Left)} Mean squared displacement (MSD) distribution. Three-dimensional surface shows MSD versus depth and steps. Color gradient from purple (low MSD, $\sim$0.010) to yellow (high MSD, $\sim$0.030). Two traces overlaid: orange (radius of gyration), yellow (trajectory length/10). Surface demonstrates diffusive exploration of phase space.
    \textbf{(Middle Center)} Trajectory statistics distribution. Histogram shows count versus trit value (0.2-1.4). Peak at value $\sim$1.2 with count $\sim$8. Distribution skewed toward higher values, indicating preferential occupation of certain categorical regions. Vertical axis: count (0-8).
    \textbf{(Bottom Right)} Transition matrix shows perspective-switching probabilities. Heat map displays transition probability from one perspective (rows: Osc, Cat, Part) to another (columns: Osc, Cat, Part). }
    \label{fig:ttr_validation}
    \end{figure}

\subsubsection{Efficient Representation}

This unification enables efficient representation. Instead of storing a separate trajectory as a list of positions $\{(x_i, y_i, z_i)\}_{i=1}^N$, we store a single trit string $(t_0, t_1, \ldots, t_N)$. The trajectory is implicit in the nested structure of the string.

The storage requirement is $O(N)$ trits, which is logarithmically smaller than storing $N$ floating-point positions (each requiring $\sim 64$ bits).

\subsection{Refinement Along S-Entropy Axes}

The three S-entropy coordinates $(S_k, S_t, S_e)$ correspond to three orthogonal modes of refinement:

\subsubsection{Knowledge Entropy $S_k$}

Refinement along the $S_k$ axis reduces knowledge entropy: we gain information about the system's state. Each trit $t_{k,i}$ narrows the range of possible states by a factor of 3. After $k$ steps:
\begin{equation}
S_k = \frac{\log_3 N_k}{\log_3 N_{\max}} = \frac{k}{\log_3 N_{\max}}
\end{equation}
where $N_k = 3^k$ is the number of possible states after $k$ refinements.

As $k \to \log_3 N_{\max}$, $S_k \to 1$, corresponding to complete knowledge (unique state identification).

\subsubsection{Temporal Entropy $S_t$}

Refinement along the $S_t$ axis reduces temporal uncertainty: we gain information about when the system occupies each state. Each trit $t_{t,i}$ narrows the time window by a factor of 3. After $k$ steps:
\begin{equation}
S_t = \frac{\log_3 N_t}{\log_3 N_{\max}} = \frac{k}{\log_3 N_{\max}}
\end{equation}
where $N_t = 3^k$ is the number of temporal bins after $k$ refinements.

As $k \to \log_3 N_{\max}$, $S_t \to 1$, corresponding to precise timestamping.

\subsubsection{Evolution Entropy $S_e$}

Refinement along the $S_e$ axis reduces evolutionary uncertainty: we gain information about how the system evolves between states. Each trit $t_{e,i}$ narrows the range of possible trajectories by a factor of 3. After $k$ steps:
\begin{equation}
S_e = \frac{\log_3 N_e}{\log_3 N_{\max}} = \frac{k}{\log_3 N_{\max}}
\end{equation}
where $N_e = 3^k$ is the number of possible evolutionary paths after $k$ refinements.

As $k \to \log_3 N_{\max}$, $S_e \to 1$, corresponding to complete determination of the trajectory.

\subsubsection{Orthogonality of Refinement Axes}

The three refinement axes are orthogonal: refining $S_k$ (gaining knowledge about state) does not affect $S_t$ (temporal information) or $S_e$ (evolutionary information). This orthogonality is a consequence of the commutativity of the categorical observables (Theorem 2).

Mathematically:
\begin{equation}
\frac{\partial S_k}{\partial t_{t,i}} = 0, \quad \frac{\partial S_k}{\partial t_{e,i}} = 0
\end{equation}
and similarly for $S_t$ and $S_e$. The three coordinates are independent.

\subsection{Cantor Set Structure and Fractal Dimension}

The S-entropy space has a natural fractal structure related to the Cantor set.

\subsubsection{Ternary Cantor Set}

The standard Cantor set is constructed by iteratively removing the middle third of each interval:
\begin{enumerate}
\item Start with $[0, 1]$.
\item Remove $(1/3, 2/3)$, leaving $[0, 1/3] \cup [2/3, 1]$.
\item Remove the middle third of each remaining interval, leaving four intervals.
\item Repeat infinitely.
\end{enumerate}

The limiting set $\mathcal{C}$ is the Cantor set, with Hausdorff dimension:
\begin{equation}
\dim_H(\mathcal{C}) = \frac{\log 2}{\log 3} \approx 0.631
\end{equation}

\subsubsection{Ternary Representation and the Cantor Set}

Numbers in the Cantor set are precisely those with ternary expansions containing only digits 0 and 2 (no 1s):
\begin{equation}
\mathcal{C} = \left\{ \sum_{i=1}^\infty \frac{t_i}{3^i} \mid t_i \in \{0, 2\} \right\}
\end{equation}

Our trit strings allow $t_i \in \{0, 1, 2\}$, so the S-entropy space contains the Cantor set as a subset but also includes points with $t_i = 1$ (middle-third points).

\subsubsection{Fractal Dimension of Trajectory}

The electron trajectory through S-entropy space has fractal dimension $d_f$ determined by the scaling of visited points. If the trajectory visits $N(r)$ distinct cells of size $r$, then:
\begin{equation}
N(r) \sim r^{-d_f}
\end{equation}

For a smooth curve in 3D, $d_f = 1$ (the curve is 1-dimensional). For a space-filling curve, $d_f = 3$ (it fills the entire volume). For the electron trajectory, we measure $d_f \approx 1.2$, indicating slightly "rough" or fractal behavior due to quantum fluctuations.

\subsection{Computational Efficiency of Ternary Representation}

The ternary representation provides computational advantages for trajectory processing.

\subsubsection{Storage Efficiency}

A trit stores $\log_2 3 \approx 1.58$ bits of information. A sequence of $N$ trits stores $1.58 N$ bits. This is more efficient than binary for representing base-3 partitioning: binary requires $\log_2 3^N = N \log_2 3 \approx 1.58 N$ bits.

Thus, ternary is the natural (most efficient) representation for ternary partitioning.

\subsubsection{Search Efficiency}

The ternary trisection algorithm requires $O(\log_3 N)$ steps to search a space of size $N$. This is $\log_2 3 \approx 1.58$ times faster than binary search ($O(\log_2 N)$ steps).

For $N = 10^{15}$ (the number of distinguishable categorical states), ternary search requires:
\begin{equation}
\log_3(10^{15}) \approx 31.5 \text{ steps}
\end{equation}
versus binary search requiring:
\begin{equation}
\log_2(10^{15}) \approx 49.8 \text{ steps}
\end{equation}

This is a 37\% reduction in the number of measurements.

\subsubsection{Parallelization}

The three spatial dimensions are encoded independently in ternary, enabling parallel processing. Each dimension's trit string can be computed simultaneously, reducing wall-clock time by a factor of 3 (with three parallel processors).

\subsection{Mapping Between S-Entropy Space and Physical Space}

The final step is mapping the trajectory in S-entropy space $\mathbf{S}(t) = (S_k(t), S_t(t), S_e(t))$ to physical space $\mathbf{r}(t) = (x(t), y(t), z(t))$.


\begin{figure}[htbp]
    \centering
    \includegraphics[width=\textwidth]{figures/panel_ternary_computation_1.png}
    \caption{Ternary Representation for Gas Dynamics: S-Entropy Compression. 
    \textbf{Top left:} Full phase space (200 molecules) showing 3D molecular positions and velocities compressed from 18-dimensional space into categorical coordinates. Each point represents one molecule with complete phase space information encoded in ternary addresses.
    \textbf{Top center:} S-Entropy compression demonstration showing dimensional reduction from 18 dimensions (x, y, z, v_x, v_y, v_z for each molecule) to 3 S-entropy coordinates: S_k (knowledge), S_t (temporal), S_e (evolutionary). Each molecule maps to unique point in categorical space.
    \textbf{Top right:} Ternary addresses (3$^k$ hierarchy) showing base-3 encoding where each trit position corresponds to depth in categorical tree. Color coding: 0 = Oscillatory (blue), 1 = Categorical (red), 2 = Partition (yellow). Maximum depth = 10 trits provides 3$^{10}$ = 59,049 unique addresses.
    \textbf{Bottom left:} Sliding window spectrometer tracking S_k (knowledge, yellow), S_t (time, cyan), S_e (evolution, red) entropy components across 30 time windows. The oscillatory behavior demonstrates dynamic categorical transitions in real-time molecular evolution.
    \textbf{Bottom center:} 3$^k$ ternary address tree showing hierarchical structure where each node branches into 3 sub-categories. The tree depth corresponds to measurement precision, with deeper levels providing finer categorical resolution.
    \textbf{Bottom right - Key insight:} \textbf{Oscillator = Processor}: Each molecular oscillator functions as a computational processor where gas dynamics solving is equivalent to running ternary programs. Memory addresses correspond to trajectories in S-space, establishing fundamental equivalence between thermodynamic evolution and categorical computation.
    \textbf{Validation: PASS} - Complete dimensional compression achieved: 18D $\rightarrow$ 3D with perfect information preservation through ternary encoding.}
    \label{fig:ternary_compression_success}
    \end{figure}

\subsubsection{Bijection via Partition Coordinates}

The S-entropy coordinates correspond to partition coordinates:
\begin{align}
S_k &\leftrightarrow n \quad (\text{depth}) \\
S_t &\leftrightarrow \tau \quad (\text{time}) \\
S_e &\leftrightarrow (\ell, m) \quad (\text{angular structure})
\end{align}

Each partition coordinate maps to physical space via the radial and angular wavefunctions (Section 2):
\begin{align}
n &\to r \sim n^2 a_0 \\
\ell, m &\to (\theta, \phi) \quad (\text{angular position})
\end{align}

Combining these:
\begin{align}
\mathbf{r}(t) &= r(n(t)) \, \hat{\mathbf{r}}(\theta(t), \phi(t)) \\
&= \frac{3n(t)^2 - \ell(t)(\ell(t)+1)}{2} a_0 \, \hat{\mathbf{r}}(\theta(t), \phi(t))
\end{align}

\subsubsection{Inverse Map}

Given a trajectory in physical space $\mathbf{r}(t)$, we can compute the corresponding S-entropy trajectory:
\begin{align}
n(t) &= \left\lceil \sqrt{r(t)/a_0} \right\rceil \quad (\text{nearest integer}) \\
\ell(t) &\approx \sqrt{n(t)^2 - 2r(t)/a_0} \quad (\text{from energy matching}) \\
\theta(t), \phi(t) &\to m(t) \quad (\text{from angular position})
\end{align}

This completes the bijection between S-entropy and physical space.
