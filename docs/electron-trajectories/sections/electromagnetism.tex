\section{Electromagnetism from Categorical Current Flow}
\label{sec:electromagnetism}

Our experimental apparatus relies on electromagnetic fields to confine, manipulate, and measure ions. To establish that these fields arise from the same partition structure as atomic states and classical mechanics, we derive electromagnetism from categorical current flow and S-entropy transformations.

\subsection{Dimensional Reduction for Conductors}

\subsubsection{Phase-Lock Networks}

In a conductor, conduction electrons are not localized to specific atoms—they are delocalized across the entire conductor. But this delocalisation does not mean they move independently. Each electron is phase-locked to its neighbours through Coulomb interactions and Pauli exclusion.

\begin{definition}[Phase-Lock Network]
\label{def:phase_lock_network}
A \emph{phase-lock network} is a system of particles with strong mutual coupling, where the coupling time $\tau_c$ is much shorter than the scattering time $\tau_s$:
\begin{equation}
\tau_c \ll \tau_s
\end{equation}
\end{definition}

For conduction electrons:
\begin{align}
\tau_c &\sim 10^{-15} \text{ s} \quad \text{(phase-lock coupling time)} \\
\tau_s &\sim 10^{-14} \text{ s} \quad \text{(scattering time)}
\end{align}

The strong coupling creates a categorical network. When one electron shifts position, it immediately affects all neighbouring electrons through phase-lock coupling. The network responds collectively, not individually.

\subsubsection{Current as Categorical State Propagation}

\begin{theorem}[Newton's Cradle Model]
\label{thm:newtons_cradle}
Current propagates through electron displacement chains—analogous to Newton's cradle—rather than through individual electron drift.
\end{theorem}

\begin{proof}
The drift velocity of electrons in a typical conductor is:
\begin{equation}
v_d \sim 10^{-4} \text{ m/s}
\end{equation}

Yet signals propagate at speeds approaching the speed of light:
\begin{equation}
v_{\text{signal}} \sim 10^8 \text{ m/s}
\end{equation}

This apparent paradox resolves when we recognize that current is the propagation of categorical states through the electron network, not the physical motion of individual electrons. Like Newton's cradle, momentum transfers through successive collisions without individual ball displacement.
\end{proof}

\subsubsection{Dimensional Reduction}

\begin{theorem}[Conductor Dimensional Reduction]
\label{thm:conductor_reduction}
A conductor of length $L$ and cross-sectional area $A$ reduces to:
\begin{equation}
\text{3D Conductor} = \text{0D Cross-Section} \times \text{1D S-Transformation}
\end{equation}
where:
\begin{itemize}
\item The \textbf{0D cross-section} is characterized by the number of parallel conduction paths $N_\parallel = A/a_0^2$, where $a_0$ is the lattice spacing.
\item The \textbf{1D S-transformation} describes categorical state propagation along the conductor length.
\end{itemize}
\end{theorem}

\begin{proof}
Phase-locking imposes the constraint that all electrons in a cross-section must maintain categorical coherence—they cannot occupy independent categorical states. This reduces the cross-sectional degrees of freedom from $N_\parallel$ (number of electrons) to $1$ (the collective cross-sectional state).

The remaining degree of freedom is the propagation of categorical states along the conductor length, described by the S-transformation operator acting on the longitudinal S-coordinate.
\end{proof}

This reduction explains why macroscopic conductors obey simple one-dimensional circuit equations despite their three-dimensional geometry.

\begin{figure}[htbp]
    \centering
    \includegraphics[width=\textwidth]{figures/panel_dimensional_reduction.pdf}
    \caption{\textbf{Dimensional Reduction---Wire as Cross-Section $\times$ S-Transform.}
    (A) 3D wire: cylindrical conductor with infinite degrees of freedom (position of each electron in 3D space).
    (B) 0D cross-section: all radial positions are equivalent for current flow---only the radius $r$ matters, reducing to a point parameter.
    (C) 1D S-transformation along length: S-potential (voltage) varies linearly along wire, with S-coordinates tracking state evolution.
    (D) Complete reduction formula: $\text{Wire} = \int_0^R 2\pi r \, dr \times \mathcal{S}$, giving resistance $R = \rho L/A = \rho L/(\pi r^2)$ from 0D (area) times 1D (length/conductivity).}
    \label{fig:dimensional_reduction}
    \end{figure}

\subsection{Ohm's Law from S-Transformations}

\subsubsection{Partition Lag and Scattering}

\begin{definition}[Scattering Partition Lag]
\label{def:scattering_lag}
The \emph{scattering partition lag} $\tau_s$ is the time delay introduced by electron-lattice scattering events:
\begin{equation}
\tau_s = \frac{m}{ne^2\rho}
\end{equation}
where $m$ is electron mass, $n$ is electron density, $e$ is elementary charge, and $\rho$ is resistivity.
\end{definition}

Each scattering event introduces a time delay in the propagation of categorical states. The accumulation of these delays over many scattering events produces macroscopic resistance.

\subsubsection{Resistivity Formula}

\begin{theorem}[Microscopic Resistivity]
\label{thm:resistivity}
Resistivity arises from scattering partition lag and electron-lattice coupling:
\begin{equation}
\boxed{\rho = \sum_{i,j} \frac{\tau_{s,ij} g_{ij}}{ne^2}}
\end{equation}
where $\tau_{s,ij}$ is the scattering partition lag for interaction pair $(i,j)$ and $g_{ij}$ is the coupling strength.
\end{theorem}

\begin{proof}
The S-transformation rate along the conductor is limited by scattering events. Each scattering introduces lag $\tau_s$ and couples with strength $g$. The resistivity (resistance per unit length per unit area) is:
\begin{equation}
\rho = \frac{1}{\sigma} = \frac{1}{ne^2\mu}
\end{equation}

where mobility $\mu = e\tau_s/m$. Substituting:
\begin{equation}
\rho = \frac{m}{ne^2\tau_s}
\end{equation}

For multiple scattering mechanisms with coupling strengths $g_{ij}$:
\begin{equation}
\rho = \sum_{i,j} \frac{\tau_{s,ij} g_{ij}}{ne^2}
\end{equation}
\end{proof}

\subsubsection{Ohm's Law}

\begin{theorem}[Ohm's Law]
\label{thm:ohms_law}
In the continuum limit of discrete S-transformations:
\begin{equation}
\boxed{V = IR}
\end{equation}
where voltage $V$ is the S-potential difference, current $I$ is the S-transformation rate, and resistance $R = \rho L/A$.
\end{theorem}

\begin{proof}
The S-potential difference along conductor length $L$ is:
\begin{equation}
V = \int_0^L \mathbf{E} \cdot d\mathbf{l}
\end{equation}

The current (S-transformation rate per unit area) is:
\begin{equation}
I = \int_A \mathbf{J} \cdot d\mathbf{A}
\end{equation}

From the resistivity relation $\mathbf{J} = \sigma\mathbf{E} = \mathbf{E}/\rho$:
\begin{equation}
I = \frac{A}{\rho} E
\end{equation}

For uniform field over length $L$:
\begin{equation}
V = EL
\end{equation}

Therefore:
\begin{equation}
V = \frac{\rho L}{A} I = RI
\end{equation}
\end{proof}

This is Ohm's law, derived from categorical state propagation rather than postulated as an empirical relation.

\subsection{Kirchhoff's Laws}

\subsubsection{Current Law}

\begin{theorem}[Kirchhoff's Current Law]
\label{thm:kcl}
At any circuit junction:
\begin{equation}
\boxed{\sum_k I_k = 0}
\end{equation}
\end{theorem}

\begin{proof}
Categorical states are conserved. At a junction, categorical states arriving must equal categorical states departing:
\begin{equation}
\sum_{k \in \text{in}} \frac{dM_k}{dt} = \sum_{k \in \text{out}} \frac{dM_k}{dt}
\end{equation}

Since current $I_k = e \cdot dM_k/dt$ (charge per categorical state times categorical rate):
\begin{equation}
\sum_{k \in \text{in}} I_k = \sum_{k \in \text{out}} I_k
\end{equation}

Defining inward currents as positive and outward as negative:
\begin{equation}
\sum_k I_k = 0
\end{equation}
\end{proof}

Categorical states cannot be created or destroyed at junctions; they can only be redirected along different paths.

\subsubsection{Voltage Law}

\begin{theorem}[Kirchhoff's Voltage Law]
\label{thm:kvl}
Around any closed loop:
\begin{equation}
\boxed{\sum_k V_k = 0}
\end{equation}
\end{theorem}

\begin{proof}
The S-potential is single-valued: traversing any closed path must return to the initial S-potential value. Otherwise, the categorical state structure would be inconsistent.

The voltage $V_k$ across element $k$ is the S-potential difference:
\begin{equation}
V_k = \Phi(x_k^{\text{end}}) - \Phi(x_k^{\text{start}})
\end{equation}

Summing around a closed loop:
\begin{equation}
\sum_k V_k = \sum_k [\Phi(x_k^{\text{end}}) - \Phi(x_k^{\text{start}})] = 0
\end{equation}

because the sum telescopes and the loop closes.
\end{proof}

The S-potential must return to its initial value after traversing any closed path, ensuring consistency of the categorical state structure.

\begin{figure}[htbp]
    \centering
    \includegraphics[width=\textwidth]{figures/panel_ohm_kirchhoff.pdf}
    \caption{\textbf{Ohm's Law and Kirchhoff's Laws from Categorical Dynamics.}
    (A) Ohm's Law $V = IR$: linear relationship emerges from S-dynamics with $R = \tau_s \cdot g \cdot L/A$ where $\tau_s$ is scattering partition lag and $g$ is electron-lattice coupling.
    (B) Resistivity from scattering time: materials with longer scattering time $\tau_s$ (fewer apertures) have lower resistivity $\rho \propto 1/\tau_s$.
    (C) Kirchhoff's Current Law: $\sum I_{in} = \sum I_{out}$ at any node expresses conservation of categorical states---states cannot be created or destroyed at junctions.
    (D) Kirchhoff's Voltage Law: $\sum V_{loop} = 0$ around any closed loop expresses single-valuedness of S-potential---returning to the same point must yield the same categorical state.}
    \label{fig:ohm_kirchhoff}
    \end{figure}

\subsection{Maxwell's Equations}

\subsubsection{Extension to Time-Varying Fields}

Ohm's law and Kirchhoff's laws are quasi-static approximations valid when time derivatives are negligible. Extending to time-varying fields yields Maxwell's equations.

\begin{theorem}[Gauss's Law]
\label{thm:gauss}
\begin{equation}
\boxed{\nabla \cdot \mathbf{E} = \frac{\rho}{\epsilon_0}}
\end{equation}
\end{theorem}

\begin{proof}
Charge $q$ is a partition depth in charge space (Definition~\ref{def:charge}). The electric field $\mathbf{E}$ is the gradient of the S-potential created by charge partition structure.

Applying the divergence theorem to the S-potential:
\begin{equation}
\oint_S \mathbf{E} \cdot d\mathbf{A} = \frac{Q_{\text{enc}}}{\epsilon_0}
\end{equation}

In differential form:
\begin{equation}
\nabla \cdot \mathbf{E} = \frac{\rho}{\epsilon_0}
\end{equation}

where $\rho = dQ/dV$ is charge density.
\end{proof}

\begin{theorem}[No Magnetic Monopoles]
\label{thm:no_monopoles}
\begin{equation}
\boxed{\nabla \cdot \mathbf{B} = 0}
\end{equation}
\end{theorem}

\begin{proof}
Magnetic field $\mathbf{B}$ arises from current (moving charge). Current is categorical state propagation, which has no sources or sinks—it is a flow. Therefore, magnetic field lines form closed loops with no beginning or end:
\begin{equation}
\nabla \cdot \mathbf{B} = 0
\end{equation}
\end{proof}

\begin{theorem}[Faraday's Law]
\label{thm:faraday}
\begin{equation}
\boxed{\nabla \times \mathbf{E} = -\frac{\partial \mathbf{B}}{\partial t}}
\end{equation}
\end{theorem}

\begin{proof}
A changing magnetic field $\partial\mathbf{B}/\partial t$ creates a time-varying S-curl—a rotation in the S-coordinate structure. This S-curl manifests as an electric field curl $\nabla \times \mathbf{E}$.

The negative sign arises from Lenz's law: the induced field opposes the change that created it, ensuring energy conservation.

Integrating around a closed loop:
\begin{equation}
\oint_C \mathbf{E} \cdot d\mathbf{l} = -\frac{d}{dt}\int_S \mathbf{B} \cdot d\mathbf{A}
\end{equation}

In differential form:
\begin{equation}
\nabla \times \mathbf{E} = -\frac{\partial \mathbf{B}}{\partial t}
\end{equation}
\end{proof}

\begin{theorem}[Ampère-Maxwell Law]
\label{thm:ampere_maxwell}
\begin{equation}
\boxed{\nabla \times \mathbf{B} = \mu_0\mathbf{J} + \mu_0\epsilon_0\frac{\partial \mathbf{E}}{\partial t}}
\end{equation}
\end{theorem}

\begin{proof}
Current $\mathbf{J}$ creates magnetic field through the Biot-Savart law (categorical state propagation creates S-curl).

The displacement current term $\epsilon_0\partial\mathbf{E}/\partial t$ represents the rate of S-transformation in time-varying electric fields. A changing electric field is equivalent to a current for the purpose of generating magnetic fields.

Combining:
\begin{equation}
\nabla \times \mathbf{B} = \mu_0\mathbf{J} + \mu_0\epsilon_0\frac{\partial \mathbf{E}}{\partial t}
\end{equation}
\end{proof}



\subsection{The Speed of Light}

\begin{theorem}[Speed of Light from Vacuum Structure]
\label{thm:speed_of_light}
The speed of light emerges from the partition-coupling structure of the electromagnetic vacuum:
\begin{equation}
\boxed{c = \frac{1}{\sqrt{\mu_0 \epsilon_0}}}
\end{equation}
\end{theorem}

\begin{proof}
From Maxwell's equations, electromagnetic waves satisfy:
\begin{equation}
\nabla^2 \mathbf{E} = \mu_0\epsilon_0 \frac{\partial^2 \mathbf{E}}{\partial t^2}
\end{equation}

This is a wave equation with propagation speed:
\begin{equation}
c = \frac{1}{\sqrt{\mu_0\epsilon_0}}
\end{equation}

The vacuum permeability $\mu_0$ represents the electromagnetic partition lag—the inertia of electromagnetic fields. The vacuum permittivity $\epsilon_0$ represents the vacuum field coupling—the flexibility of electromagnetic fields.

The speed of light is determined by the fundamental partition-coupling structure of space:
\begin{equation}
c = \frac{1}{\sqrt{\tau_p^{(\text{EM})} \cdot g^{(\text{EM})}}}
\end{equation}

where $\tau_p^{(\text{EM})} \equiv \mu_0$ and $g^{(\text{EM})} \equiv \epsilon_0$.
\end{proof}

\textbf{Physical interpretation:} The speed of light is not arbitrary. It is the maximum rate at which categorical states can propagate through the vacuum, determined by the fundamental electromagnetic partition-coupling structure of space itself.

\begin{figure}[htbp]
    \centering
    \includegraphics[width=\textwidth]{figures/panel_maxwell_equations.pdf}
    \caption{\textbf{Maxwell's Equations from Categorical S-Dynamics.}
    (A) Gauss's Law: electric field $\mathbf{E} = -\nabla \Phi_S$ as negative gradient of S-potential. Field lines radiate from charges (sources of S-potential).
    (B) Amp\`ere's Law: magnetic field $\mathbf{B} = \nabla \times \mathbf{A}_S$ as curl of S-vector potential. Field lines form closed loops around current (S-flow).
    (C) Coupled E-B oscillation: electromagnetic wave consists of perpendicular E and B fields oscillating 90° out of phase, propagating through S-space.
    (D) Speed of light from S-dynamics: wave equation $\nabla^2 \mathbf{E} = \mu_0 \varepsilon_0 \partial^2\mathbf{E}/\partial t^2$ gives $c = 1/\sqrt{\mu_0\varepsilon_0} = 299{,}792{,}458$ m/s as the S-transformation rate in vacuum.}
    \label{fig:maxwell_equations}
    \end{figure}


\subsection{Application to Penning Trap Fields}

In our Penning trap apparatus (Section~\ref{sec:experimental_setup}), electromagnetic fields arise from:

\subsubsection{Magnetic Field}

The superconducting magnet creates a uniform axial field:
\begin{equation}
\mathbf{B} = B_0 \hat{z} \quad \text{with} \quad B_0 = 9.4 \text{ T}
\end{equation}

This field arises from superconducting currents (zero-resistance categorical state propagation with $\tau_s \to 0$).

\subsubsection{Electric Quadrupole Field}

The ring and endcap electrodes create a quadrupole potential:
\begin{equation}
\Phi(r, z) = \frac{V_0}{2d^2}(z^2 - r^2/2)
\end{equation}

The electric field is:
\begin{equation}
\mathbf{E} = -\nabla\Phi = \frac{V_0}{d^2}(r\hat{r} - 2z\hat{z})
\end{equation}

\subsubsection{Perturbation Fields}

Perturbation fields (Section~\ref{sec:forced_localization}) create position-dependent S-potentials that force electrons into specific categorical states. These fields are derived from the same electromagnetic framework established here.

\subsection{Summary}

We have derived electromagnetism from categorical current flow:

\begin{itemize}
\item \textbf{Current}: Categorical state propagation, not electron drift
\item \textbf{Resistivity}: $\rho = \sum \tau_{s,ij} g_{ij}/(ne^2)$ from partition lag
\item \textbf{Ohm's Law}: $V = IR$ from S-transformation continuum limit
\item \textbf{Kirchhoff's Laws}: From categorical conservation and S-potential single-valuedness
\item \textbf{Maxwell's Equations}: From S-curl dynamics and time-varying fields
\item \textbf{Speed of light}: $c = 1/\sqrt{\mu_0\epsilon_0}$ from vacuum partition-coupling
\end{itemize}

All electromagnetism emerges from:
\begin{equation}
\text{Bounded phase space} \implies \text{Categorical flow} \implies \text{Electromagnetism}
\end{equation}

This establishes that electromagnetic fields in our experimental apparatus arise from the same partition structure that produces atomic states, classical mechanics, and thermodynamics. The complete framework—atomic structure, classical mechanics, thermodynamics, electromagnetism—rests on the single axiom of bounded phase space.

The ions in our Penning trap respond to these electromagnetic fields according to the Lorentz force (Section~\ref{sec:classical_mechanics}), enabling the forced localization and categorical measurement techniques described in subsequent sections.
