\section{Categorical Measurement}

\subsection{Virtual Instruments as Coupling Geometries}

A measurement is not a physical interaction but a categorical relationship established through coupling geometry. This section formalizes this concept.

\begin{definition}[Coupling Geometry]
A coupling geometry $\mathcal{G}$ is a specification of:
\begin{enumerate}
\item Frequency or field modality (optical, vibrational, magnetic, etc.)
\item Spatial mode structure (standing wave pattern, field gradient, etc.)
\item Selection rules (which transitions are allowed)
\item Temporal protocol (continuous, pulsed, etc.)
\end{enumerate}
\end{definition}

The coupling geometry defines which aspects of the system are accessible to measurement. Different geometries access different categorical observables.

\begin{definition}[Virtual Instrument]
A virtual instrument $\mathcal{I}$ is a coupling geometry $\mathcal{G}$ instantiated during measurement. The instrument exists only during the coupling; before and after, there is no instrument, only the system.
\end{definition}

This definition captures the ontology discussed in the introduction: instruments are not physical devices but relationships. A spectrometer sitting on a bench, disconnected from any sample, is not an instrument in this sense. It becomes an instrument only when coupled to a system, establishing a categorical observable through the geometry of the coupling.

\subsection{The Five Modalities}

We employ five distinct coupling geometries, each defining a different categorical observable:

\subsubsection{Optical Absorption: Measuring $n$}

The optical modality couples electromagnetic radiation at frequency $\omega$ to electronic transitions. The coupling is:
\begin{equation}
\hat{H}_{\text{opt}} = -\mathbf{d} \cdot \mathbf{E}(\omega)
\end{equation}
where $\mathbf{d} = -e\mathbf{r}$ is the dipole operator and $\mathbf{E}(\omega)$ is the electric field at frequency $\omega$.

Transitions occur when $\hbar \omega = E_f - E_i$, where $E_i$ and $E_f$ are initial and final state energies. For hydrogen, $E_n = -13.6/n^2$ eV, so:
\begin{equation}
\omega = \frac{13.6 \text{ eV}}{\hbar} \left( \frac{1}{n_i^2} - \frac{1}{n_f^2} \right)
\end{equation}

By scanning $\omega$ and observing absorption, we determine $(n_i, n_f)$. If the initial state $n_i$ is known (e.g., ground state $n_i = 1$), then measuring $\omega$ directly gives $n_f$. The optical modality thus measures the partition depth $n$.

The selection rule for optical transitions is $\Delta \ell = \pm 1$ (electric dipole selection rule). This is a geometric constraint: the dipole operator $\mathbf{r}$ couples only partitions differing by one unit of angular complexity.

\begin{figure}[htbp]
    \centering
    \includegraphics[width=\textwidth]{figures/panel_uvvis_complexity_coordinate.png}
    \caption{Complexity coordinate $\ell$ and UV-visible optical spectroscopy. \textbf{Top row:} Orbital shapes for $\ell=2$ (d-orbital) and $\ell=3$ (f-orbital), selection rule matrix showing allowed transitions $\Delta\ell = \pm 1$ (6.0\% of all pairs, green squares), UV-visible absorption spectrum with vibronic structure, and Jablonski diagram showing electronic transitions. \textbf{Middle row:} Orbital characteristics radar plot (radial extent, angular momentum, shielding, nodes, energy, degeneracy), frequency scaling $\omega_\ell \propto \ell(\ell+1)$ with numerical values, transition dipole moment vectors in 3D, and oscillator strengths for $s \to p$ (0.876), $p \to d$ (0.122), $d \to f$ (0.637) transitions. \textbf{Bottom row:} Degeneracy pattern $2\ell+1$ showing cumulative state counts. The coupling structure $\mathcal{I}_\ell$ implements electric dipole coupling in the optical regime $\Omega_\ell$, corresponding to UV-visible and Raman spectroscopy (Theorem~\ref{thm:complexity_coupling}).}
    \label{fig:complexity_uvvis}
    \end{figure}

\subsubsection{Raman Scattering: Measuring $\ell$}

The Raman modality couples to vibrational modes through inelastic scattering. Incident light at frequency $\omega_0$ is scattered to frequency $\omega = \omega_0 \pm \omega_{\text{vib}}$, where $\omega_{\text{vib}}$ is the vibrational frequency. The coupling is:
\begin{equation}
\hat{H}_{\text{Ram}} = -\frac{\partial \alpha}{\partial Q} Q \, |\mathbf{E}(\omega_0)|^2
\end{equation}
where $\alpha$ is the polarizability, $Q$ is the vibrational coordinate, and $|\mathbf{E}|^2$ is the incident intensity.

Vibrational frequencies are related to angular momentum quantum number by:
\begin{equation}
\omega_{\text{vib}} \propto \sqrt{\ell(\ell+1)}
\end{equation}
because angular momentum introduces centrifugal barriers that modify the effective potential. By measuring $\omega_{\text{vib}}$, we determine $\ell$.

The selection rule for Raman transitions is $\Delta \ell = 0, \pm 2$ (for isotropic scattering), though polarization-dependent Raman can also access $\Delta \ell = \pm 1$. This measures the angular complexity coordinate.

\subsubsection{Magnetic Resonance: Measuring $m$}

The magnetic modality applies a static field $\mathbf{B} = B_0 \hat{z}$ and a rotating field $\mathbf{B}_1(t) = B_1 (\cos\omega t \, \hat{x} + \sin\omega t \, \hat{y})$. The coupling is:
\begin{equation}
\hat{H}_{\text{mag}} = -\boldsymbol{\mu} \cdot \mathbf{B}
\end{equation}
where $\boldsymbol{\mu} = -\mu_B (\mathbf{L} + 2\mathbf{S})/\hbar$ is the magnetic moment, $\mathbf{L}$ is orbital angular momentum, and $\mathbf{S}$ is spin angular momentum.

The static field splits energy levels by orientation:
\begin{equation}
E_m = -\mu_B m B_0
\end{equation}
where $m$ is the magnetic quantum number (orientation coordinate). Transitions occur at:
\begin{equation}
\hbar \omega = \mu_B B_0 \Delta m
\end{equation}

By measuring the resonance frequency $\omega$, we determine $\Delta m$, and if the initial $m$ is known, we determine $m$ directly. The selection rule is $\Delta m = \pm 1$ (magnetic dipole). This measures the orientation coordinate.

\begin{figure}[htbp]
    \centering
    \includegraphics[width=\textwidth]{figures/panel_nmr_chirality_coordinate.png}
    \caption{Chirality coordinate $s$ and nuclear magnetic resonance (NMR) spectroscopy. \textbf{Top row:} Bloch sphere representation of spin states $|\uparrow\rangle$ and $|\downarrow\rangle$, Zeeman energy splitting $\Delta E = \gamma \hbar B$ linear in magnetic field, Boltzmann spin population distribution at various temperatures (100--500 K), and $^1$H NMR spectrum showing chemical shift peaks for different molecular environments. \textbf{Middle row:} NMR relaxation curves for longitudinal ($T_1 = 1.0$ s, blue) and transverse ($T_2 = 0.5$ s, red) magnetization, free induction decay (FID) signal with exponential envelope, spin echo pulse sequence (90°--180°--acquisition), and tissue-dependent NMR properties radar plot (water, fat, brain) showing $T_1$, $T_2$, $T_2^*$, chemical shift, and J-coupling variations. \textbf{Bottom row:} 2D COSY correlation map showing through-bond connectivity, J-coupling multiplet patterns (singlet, doublet, triplet, quartet), Larmor frequency $\omega = \gamma B$ for different nuclei ($^1$H, $^{13}$C, $^{19}$F, $^{31}$P), and two-spin energy level diagram. The coupling structure $\mathcal{I}_s$ implements radio-frequency magnetic resonance at the Larmor frequency in regime $\Omega_s$, corresponding to NMR and ESR spectroscopy (Theorem~\ref{thm:chirality_resonance}).}
    \label{fig:chirality_nmr}
    \end{figure}

\subsubsection{Circular Dichroism: Measuring $s$}

The circular dichroism (CD) modality couples left- and right-circularly polarized light differently to chiral systems. The coupling is:
\begin{equation}
\hat{H}_{\text{CD}} = -\mathbf{d} \cdot \mathbf{E}_L - \mathbf{d} \cdot \mathbf{E}_R
\end{equation}
where $\mathbf{E}_L$ and $\mathbf{E}_R$ are left- and right-circular fields. For chiral systems (those with $s = \pm 1/2$), the absorption differs:
\begin{equation}
\Delta A = A_L - A_R \propto s
\end{equation}

By measuring $\Delta A$, we determine the chirality $s$. This modality is sensitive to the handedness of the partition structure, which for electrons corresponds to spin projection.

\subsubsection{Drift Field Mass Spectrometry: Measuring $\tau$}

The drift modality applies a time-varying electric field that accelerates ions along a drift tube. The time-of-flight (TOF) is:
\begin{equation}
\tau = \sqrt{\frac{2mL}{eV}}
\end{equation}
where $m$ is the ion mass, $L$ is the drift length, $e$ is the charge, and $V$ is the accelerating voltage.

For a given ion (fixed $m, e, V, L$), the TOF $\tau$ is constant. However, when combined with collision-induced dissociation (CID), the ion can fragment into pieces with different $m$, and the TOF spectrum encodes the mass distribution. The temporal evolution coordinate $\tau$ labels which time point in the trajectory we are measuring.

In the context of electron trajectory observation, the drift modality measures the evolution time: at which point during the transition are we observing the system. By synchronizing the drift measurement with optical/Raman/magnetic/CD measurements, we timestamp each categorical snapshot.

\begin{figure}[htbp]
    \centering
    \includegraphics[width=\textwidth]{figures/panel_unified_spectroscopy.png}
    \caption{Unified spectroscopic framework showing correspondence between partition coordinates $(n,\ell,m,s)$ and measurement techniques. \textbf{Top:} Frequency regime separation spanning radio to X-ray frequencies ($10^6$--$10^{18}$ Hz), with each coordinate occupying a distinct spectral regime separated by factors $>10^3$ (Theorem~\ref{thm:frequency_duality}). \textbf{Middle:} Geometric representations of each coordinate: depth $n$ (shell capacity $2n^2$), complexity $\ell$ (angular degeneracy), orientation $m$ (Zeeman levels and Larmor precession), and chirality $s$ (Bloch sphere relaxation). \textbf{Bottom table:} Summary of coordinate-instrument correspondences, showing frequency scaling ($\omega_n \propto n^{-3}$, $\omega_\ell \propto \ell(\ell+1)$, $\omega_m \propto m \cdot B$, $\omega_s \propto s \cdot B$), physical coupling mechanisms, and spectroscopic implementations. The coordinate relationship diagram (right) illustrates the hierarchical structure connecting all four measurements through the partition structure $\mathcal{P}$.}
    \label{fig:unified_spectroscopy}
    \end{figure}
\subsection{Orthogonality of Modalities}

We now prove that the five modalities measure orthogonal categorical observables, following from empirical reliability and observer invariance.

\begin{theorem}[Modality Orthogonality]
The categorical observables measured by optical, Raman, magnetic, CD, and drift modalities commute pairwise:
\begin{equation}
[\hat{O}_i, \hat{O}_j] = 0 \quad \text{for all } i \neq j
\end{equation}
where $\hat{O}_1 = \hat{n}$, $\hat{O}_2 = \hat{\ell}$, $\hat{O}_3 = \hat{m}$, $\hat{O}_4 = \hat{s}$, $\hat{O}_5 = \hat{\tau}$.
\end{theorem}

\begin{proof}
We prove by demonstrating empirical reliability and invoking invariance.

\textbf{Step 1: Empirical reliability.}

Each modality has been used independently for decades with consistent results:
\begin{itemize}
\item Optical spectroscopy (absorption/emission) has measured electronic transitions since Balmer (1885), with reproducible line series.
\item Raman spectroscopy has identified molecular vibrations since Raman (1928), with reproducible peak positions.
\item Magnetic resonance (NMR/EPR) has mapped spin states since Bloch/Purcell (1946), with reproducible spectra.
\item Circular dichroism has distinguished enantiomers since Cotton (1896), with reproducible chirality signatures.
\item Mass spectrometry has determined molecular compositions since Thomson (1897), with reproducible mass-to-charge ratios.
\end{itemize}

The reliability of these techniques is not in question. They are the foundation of analytical chemistry, materials science, and structural biology. If any technique were unreliable, it would not be used.

\textbf{Step 2: Observer invariance.}

Physical reality is independent of how many observers are present. If Observer 1 measures optical absorption and obtains $n = 2$, and Observer 2 independently measures Raman scattering and obtains $\ell = 1$, then Observer 3 using both techniques simultaneously must obtain $(n, \ell) = (2, 1)$.

Suppose, for contradiction, that $[\hat{n}, \hat{\ell}] \neq 0$. Then measuring $\hat{n}$ disturbs $\hat{\ell}$. Observer 3, who measures $\hat{n}$ first, would find $\hat{\ell} \neq 1$ when measuring Raman after optical, contradicting Observer 2's result. But Observer 2 used Raman alone and obtained $\ell = 1$ reliably. This contradicts the reliability of Raman spectroscopy.

Alternatively, if reality is observer-dependent, then the number of observers would change the physical state. But this violates the principle that physical laws are objective. Therefore, $[\hat{n}, \hat{\ell}] = 0$.

\textbf{Step 3: Generalization.}

The same argument applies to any pair of modalities. Since all five techniques work reliably when used alone, and since reality is observer-invariant, all five must measure commuting observables. Therefore:
\begin{equation}
[\hat{O}_i, \hat{O}_j] = 0 \quad \text{for all } i, j \in \{1, 2, 3, 4, 5\}
\end{equation}
\end{proof}

This theorem is the foundation of multi-modal measurement. Because the modalities are orthogonal, we can apply all five simultaneously without mutual interference. Each extracts independent information, over-constraining the system and enabling unique state identification.

\begin{figure}[htbp]
    \centering
    \includegraphics[width=\textwidth]{figures/panel_06_multi_modal.png}
    \caption{\textbf{Multi-modal consistency and redundancy validation.} 
    (\textbf{A}) Cross-modal correlation matrix showing pairwise correlation coefficients $$r$$ between all five measurement modalities (optical, Raman, MRI, circular dichroism, mass spectrometry). All off-diagonal elements satisfy $$r > 0.94$$, with most $$r > 0.95$$, demonstrating high inter-modal consistency. Perfect diagonal ($$r = 1.000$$) confirms self-consistency. Color scale from red ($$r = 0$$) to green ($$r = 1$$). 
    (\textbf{B}) Measurement accuracy as a function of number of modalities used simultaneously. Blue line with circles shows mean accuracy increasing from 50\% (single modality, random guess baseline) to 97\% (all five modalities). Blue shaded region indicates 95\% confidence interval. Gray circles show individual trial results. Redundancy enables error correction: accuracy improves logarithmically with modality count. 
    (\textbf{C}) Measurement timing synchronization across all five modalities over 10 μs observation window. Each row represents one modality; vertical colored bars indicate measurement events (optical: pink, Raman: orange, MRI: green, dichroism: cyan, mass spec: blue). Red vertical lines show atomic clock timing references. Yellow box annotation indicates timing jitter $$< 100$$ ns, ensuring sub-nanosecond synchronization across all channels. 
    (\textbf{D}) Three-dimensional consistency space showing measured quantum numbers $$(n, \ell, m)$$ from $$>10^4$$ simultaneous multi-modal measurements. Point cloud (colored by modality combination) clusters tightly around true value (yellow star) at $$(n, \ell, m) = (2, 1, 0)$$. Scatter width $$\sigma < 0.05$$ in all dimensions demonstrates consistency. Legend indicates single modalities (optical, Raman, MRI), dual combination (optical+Raman), and all five modalities.}
    \label{fig:multimodal}
    \end{figure}

\subsection{Multi-Modal Constraint Satisfaction}

With five orthogonal modalities, we obtain five independent measurements at each time instant:
\begin{equation}
(n, \ell, m, s, \tau) \quad \text{measured simultaneously}
\end{equation}

Each coordinate provides partial information about the system's state:
\begin{itemize}
\item $n$ narrows the radial region to $r \sim n^2 a_0$.
\item $\ell$ narrows the angular region to $\Delta \theta \sim \pi/(\ell+1)$.
\item $m$ narrows the azimuthal region to $\phi$ sectors determined by $e^{im\phi}$.
\item $s$ determines the spin state (binary choice).
\item $\tau$ timestamps the measurement.
\end{itemize}

Together, these five coordinates uniquely identify the partition of phase space the electron occupies. The partition corresponds bijectively to a spatial region, so we know the electron's approximate position without measuring it directly.

\subsubsection{Information Gain per Modality}

Each modality reduces the uncertainty in the system's state by a factor corresponding to the number of possible outcomes:
\begin{align}
\text{Optical: } \quad &N_n \sim n_{\max} \sim 100 \text{ (excited states up to Rydberg)} \\
\text{Raman: } \quad &N_\ell \sim n \sim 10 \text{ (angular complexity up to } \ell \sim 10) \\
\text{Magnetic: } \quad &N_m \sim 2\ell+1 \sim 21 \text{ (orientations for } \ell \sim 10) \\
\text{CD: } \quad &N_s = 2 \text{ (binary chirality)} \\
\text{Drift: } \quad &N_\tau \sim 10^9 \text{ (temporal bins in transition duration)}
\end{align}

The total number of distinguishable states is:
\begin{equation}
N_{\text{total}} = N_n \times N_\ell \times N_m \times N_s \times N_\tau \sim 10^{15}
\end{equation}

This vastly exceeds the number of partitions in atomic phase space ($\sim 10^3$ for typical atoms), ensuring over-constraint: the five modalities uniquely determine the state.

\subsubsection{Redundancy and Error Correction}

The over-constraint provides redundancy, enabling error detection and correction. If one modality gives an inconsistent result (e.g., $m > \ell$, which is geometrically forbidden), we can identify and correct the error using the other modalities.

The redundancy also improves signal-to-noise ratio. Independent measurements of orthogonal observables can be combined statistically to reduce uncertainty. If each modality has measurement uncertainty $\sigma_i$, the combined uncertainty is:
\begin{equation}
\sigma_{\text{combined}} = \left( \sum_{i=1}^5 \sigma_i^{-2} \right)^{-1/2} < \min(\sigma_i)
\end{equation}

This is the multi-modal advantage: using multiple orthogonal techniques improves precision beyond any single technique.

\subsection{Measurement Ontology: Instruments as Relationships}

The conceptual foundation of categorical measurement is that instruments are not physical devices but relationships between observer and system. This section formalizes this ontology.

\begin{definition}[Measurement Relationship]
A measurement is a map $\mathcal{M}: \mathcal{S} \to \mathcal{O}$ from the state space $\mathcal{S}$ of the system to the outcome space $\mathcal{O}$ of the observer. The map is defined by the coupling geometry $\mathcal{G}$.
\end{definition}

The key point is that $\mathcal{M}$ does not exist independently of the coupling. Before coupling, there is no map, no measurement, no instrument. The instrument is the map, and the map is instantiated by establishing the coupling geometry.

\subsubsection{The Fishing Analogy}

A fish in a lake does not have a property "catchability" until a hook is present. The hook defines catchability through its geometry:
\begin{itemize}
\item Hook size determines which fish can bite (too small $\to$ large fish ignore; too large $\to$ small fish cannot bite).
\item Bait type determines which fish are attracted (species-specific preferences).
\item Depth determines which fish are accessible (surface vs deep-water species).
\end{itemize}

Different hooks define different categorical observables of the fish population:
\begin{itemize}
\item Small hook with worm bait at surface $\to$ measures "small surface fish."
\item Large hook with squid bait at depth $\to$ measures "large deep fish."
\end{itemize}

The fish population is the same, but different hooks access different subsets. The hook does not change the fish; it defines which fish count as "catchable" under that coupling geometry.

Similarly, different spectroscopic techniques define different categorical observables of atomic systems:
\begin{itemize}
\item Optical at 121.6 nm $\to$ measures $n$ (depth of nesting).
\item Raman in mid-IR $\to$ measures $\ell$ (angular complexity).
\end{itemize}

The atom is the same, but different techniques access different partition coordinates. The technique does not change the atom; it defines which aspect of the partition structure is measured.

\subsubsection{Instantaneous Coupling}

Because the instrument is a relationship, not a physical object requiring construction or placement, it exists instantaneously upon activation. There is no travel time for the instrument to "reach" the system. The moment we activate the coupling geometry (turn on the laser, apply the magnetic field, etc.), the categorical observable is defined.

This explains faster-than-light "measurement" (more precisely, instantaneous observable definition). Consider measuring Jupiter's atmospheric composition from Earth using spectroscopy. The light from Jupiter takes 40 minutes to reach Earth. But the moment we point the telescope (establish the coupling geometry), we have defined the categorical observable: "What spectral lines does Jupiter emit?" The answer to this question exists now; we simply wait 40 minutes for the signal to arrive to read the answer.

The distinction is subtle but critical. The observable (the question we are asking) is defined instantaneously by the coupling geometry. The outcome (the answer to the question) propagates at light speed. But the definition is mathematical, not physical, and hence not limited by relativity.

\subsubsection{No Physical Backaction}

Because measurement is a relationship, not an interaction, there is no physical backaction. Establishing a categorical observable does not send particles, fields, or forces to the system. It defines a basis for observation, which is mathematical.

When we measure the categorical state (read the outcome), the system responds by revealing which partition it occupies. This response may involve emission or absorption of photons (for optical modality) or precession of magnetic moment (for magnetic modality). But the response is not caused by the measurement; it is the system's natural behavior under the coupling geometry.

The critical point is that categorical measurement does not perturb the system beyond forcing it into an eigenstate of the coupling Hamiltonian. And since the categorical observable commutes with physical observables, this forcing does not disturb position or momentum.

\begin{figure}[htbp]
    \centering
    \includegraphics[width=\textwidth]{figures/panel_09_measurement_ontology.png}
    \caption{\textbf{Measurement ontology: coupling geometry as categorical relationship.} 
    (\textbf{A}) Measurement time versus coupling strength for all five modalities. Colored circles indicate measured values: optical (red), Raman (green), MRI (blue), dichroism (purple), mass spectrometry (orange). Black dashed line shows theoretical scaling $$T \propto g^{-2}$$ (perturbation theory). Cyan box marks categorical limit: as coupling strength $$g \rightarrow 0$$, measurement time $$T \rightarrow 0$$ (instantaneous observable definition). Vertical cyan dashed line indicates zero-coupling asymptote. 
    (\textbf{B}) Information transfer mechanism schematic illustrating measurement as relationship rather than interaction. Blue oval (left) represents ion/system; green oval (right) represents detector/instrument. Black rectangle (center) represents coupling geometry that defines the categorical observable. Blue arrow: no energy transfer from system to instrument. Green arrow: categorical state revealed through geometric relationship. Brown box annotation emphasizes: "Information extracted without physical disturbance." 
    (\textbf{C}) Backaction versus precision phase diagram. Red line marks Heisenberg limit $$\Delta x \cdot \Delta p \geq \hbar/2$$. Red circles show physical measurements (position/momentum), falling on Heisenberg boundary. Green circles show categorical measurements, falling $$\sim 10^3$$ below Heisenberg limit in forbidden region (pink shaded). Beige region (bottom) marks categorical regime where $$\Delta p < \hbar/(2\Delta x)$$ is achievable because measurement does not involve complementary observables. Green shaded region labeled "Forbidden (Heisenberg)" indicates classically inaccessible parameter space. 
    (\textbf{D}) Three-dimensional coupling geometry visualization. Central blue/green sphere represents ion with $$n=1$$ (blue inner) and $$n=2$$ (green outer) spatial regions. Red lines radiating outward show optical coupling geometry (dipole radiation pattern). Blue lines radiating vertically show magnetic coupling geometry (axial field lines). Yellow shaded disk represents spatial mode structure. Coordinate axes in field coordinates (arbitrary units). Geometry defines which categorical observable is measured without physically perturbing the system.}
    \label{fig:ontology}
    \end{figure}

\subsection{Selection of Modalities: Bijection to Partition Coordinates}

The five modalities are not arbitrary choices but mathematically necessary. The partition coordinate space $(n, \ell, m, s, \tau)$ is five-dimensional, so five independent measurements are required for unique identification.

\begin{theorem}[Modality Completeness]
The five modalities (optical, Raman, magnetic, CD, drift) provide a complete basis for partition coordinate space: any state can be uniquely identified by the outcomes $(n, \ell, m, s, \tau)$.
\end{theorem}

\begin{proof}
The partition coordinate space is:
\begin{equation}
\mathcal{P} = \{(n, \ell, m, s, \tau) \mid n \in \mathbb{Z}^+, \, \ell \in \{0, \ldots, n-1\}, \, m \in \{-\ell, \ldots, +\ell\}, \, s \in \{\pm 1/2\}, \, \tau \in \mathbb{R}^+ \}
\end{equation}

This is a five-dimensional discrete space (plus one continuous dimension for time). Each coordinate is independent:
\begin{itemize}
\item $n$ determines the depth but not $\ell$ (multiple $\ell$ values for each $n$).
\item $\ell$ determines angular complexity but not $m$ (multiple $m$ values for each $\ell$).
\item $m$ determines orientation but not $s$ (two $s$ values for each $m$).
\item $s$ determines chirality but not $\tau$ (all times accessible for each $s$).
\item $\tau$ determines when but not the spatial coordinates $(n, \ell, m, s)$.
\end{itemize}

Therefore, to uniquely specify a state, we need to measure all five coordinates. Any subset would leave ambiguity. For example, measuring only $(n, \ell)$ leaves $2(2\ell+1)$ possible states (all $m$ and $s$ values), corresponding to an entire subshell.

The five modalities provide exactly these five measurements. Hence, they are complete.
\end{proof}

This completeness is why we require five modalities in the quintupartite observatory. Fewer modalities would under-determine the state. More modalities would be redundant (providing no additional information, since the partition coordinate space is five-dimensional).

The bijection between modalities and partition coordinates is:
\begin{align}
\text{Optical} &\leftrightarrow n \\
\text{Raman} &\leftrightarrow \ell \\
\text{Magnetic} &\leftrightarrow m \\
\text{CD} &\leftrightarrow s \\
\text{Drift} &\leftrightarrow \tau
\end{align}

This bijection is geometrically determined, not conventional. Each modality couples to the partition structure in a specific way that makes it sensitive to one coordinate.
