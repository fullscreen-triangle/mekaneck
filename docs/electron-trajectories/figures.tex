\begin{figure}[htbp]
\centering
\includegraphics[width=\textwidth]{figures/panel_04_forced_localization.png}
\caption{\textbf{Forced quantum localization and perturbation field effects.} 
(\textbf{A}) Localization quality as a function of perturbation strength. Theoretical curve (blue) shows sigmoidal increase in localization with perturbation strength $$V_0/E_n$$. Red points indicate experimental measurements with error bars. Green dashed line marks threshold $$V_0/E_n > 0.1$$ for effective localization; purple dotted line indicates saturation at 95\% localization quality. 
(\textbf{B}) Spatial field configuration showing applied perturbation potential $$|E(\mathbf{r})|$$ in the $$xy$$-plane. Three localized field maxima (red regions) at positions indicated by white circles create ternary partitioning. Cyan dashed circle (inner) and yellow dashed circle (outer) delineate $$n=1$$ and $$n=2$$ spatial regions, respectively. 
(\textbf{C}) Categorical state fidelity with and without perturbation fields for quantum states (1,0,0) through (3,2,0). Pink bars show fidelity without perturbation ($$F \sim 0.5$$, near random); green bars show fidelity with forced localization ($$F > 0.95$$, exceeding target threshold indicated by dashed line). Error bars represent standard deviation over $$10^4$$ trials. 
(\textbf{D}) Three-dimensional wavefunction localization visualization. Purple isosurface shows probability density $$|\psi(\mathbf{r})|^2$$ for forced eigenstate, demonstrating strong spatial confinement. Wireframe cage indicates measurement volume boundary in units of Bohr radius $$a_0$$.}
\label{fig:localization}
\end{figure}

\begin{figure}[htbp]
\centering
\includegraphics[width=\textwidth]{figures/categorical_memory_panel.png}
\caption{\textbf{Categorical memory (S-RAM) implements precision-by-difference addressing where history is the address.}
\textbf{(A)} S-entropy space navigation shows trajectory (red line) through 3D coordinate space $(S_k, S_t, S_e)$ from initial state (scattered colored spheres) to completion point (red star). The trajectory explores regions of varying knowledge entropy $S_k$ (0.0--1.0), temporal entropy $S_t$ (0.0--1.0), and evolution entropy $S_e$ (0.0--1.0). Each point along the path represents a categorical state, with the full trajectory encoding the system's history. The completion point is reached when all three S-entropy coordinates converge to their target values, satisfying the categorical measurement criteria.
\textbf{(B)} Precision-by-difference trajectory shows $\Delta P = T_{\text{ref}} - t_{\text{local}}$ versus time (samples 0--100). The trajectory oscillates around zero (gray horizontal line) with amplitude $|\Delta P| < 0.06$. Blue shaded regions indicate negative precision difference ($\Delta P < 0$), while white regions show positive difference ($\Delta P > 0$). Vertical dashed lines mark bit transitions: $b=0$ at $t \approx 25$, $b=0$ at $t \approx 45$, $b=0$ at $t \approx 60$, $b=1$ at $t \approx 85$. The precision-by-difference encoding enables navigation without prediction: the address is determined by the accumulated history of $\Delta P$ values, not by forecasting future states. This implements a form of ``memory as computation'' where the trajectory itself encodes the categorical address.
\textbf{(C)} $3^k$ hierarchy (root at $d=0$) shows ternary tree structure with color-coded S-entropy branches: $S_k$ (blue), $S_t$ (pink), $S_e$ (orange). Level 1 (depth 1): three nodes. Level 2 (depth 2): nine nodes. Level 3 (depth 3): four categorical branches corresponding to molecular degrees of freedom: electronic ($n$, green), vibrational ($\ell$, pink), rotational ($m$, purple), spin ($s$, brown). Bottom level shows $3^d$ nodes at depth $d$ (yellow circles, $d \approx 27$ nodes visible). The hierarchical structure enables efficient addressing with $O(\log_3 N)$ depth for $N$ states.
\textbf{(D)} Memory tiers show hierarchical storage with exponentially increasing capacity and access time. L1 Cache (blue): $\sim 2$ items, fastest access. L2 Cache (blue): $\sim 5$ items. RAM (blue): $\sim 10$ items. SSD (red): $\sim 20$ items. Archive (red): $\sim 30$ items, $10^9$ total capacity (right axis, log scale). Access time increases exponentially from $10^0$ (L1) to $10^9$ (Archive), spanning 9 orders of magnitude. This hierarchical structure mirrors the categorical measurement hierarchy, with frequently accessed states cached in fast memory and rarely accessed states stored in slow memory.
\textbf{(E)} Cache performance shows hit rate versus access count. Hit rate (green line) increases from 86\% at first access to 100\% (target, red dashed line) by access count $\sim 25$. Green shaded region indicates performance above 90\%. The cache achieves near-perfect hit rate after $\sim 20$ accesses due to temporal locality in categorical state access patterns. This demonstrates that categorical memory exhibits strong locality: states accessed recently are likely to be accessed again, enabling efficient caching.
\textbf{(F)} Memory controller as Maxwell demon shows fast (hot) memory region (blue, left) and slow (cold) memory region (pink, right) separated by Maxwell demon controller (orange oval). The demon promotes frequently accessed states (green filled circles) from cold to hot memory and demotes rarely accessed states (pink empty circles) from hot to cold memory. This active memory management operates at the Landauer bound ($k_B T \ln 2$ per bit operation), analogous to the information-catalytic measurement process described in Figure 5. The demon maintains categorical state organization without violating thermodynamics, confirming that categorical memory is a physical implementation of information catalysis.}
\label{fig:categorical_memory}
\end{figure}


\begin{figure}[htbp]
\centering
\includegraphics[width=\textwidth]{figures/figure2_frequency_coupling.png}
\caption{\textbf{Multi-modal frequency coupling enables simultaneous categorical measurement across partition coordinates.}
\textbf{(A)} Partition coordinate frequency regimes span 8 orders of magnitude: electronic transitions ($n$) at $10^{15}$ Hz, vibrational modes ($\ell$) at $10^{13}$ Hz, rotational states ($m$) at $10^9$ Hz, and hyperfine structure ($s$) at $10^7$ Hz. Each coordinate occupies a distinct spectral window, enabling orthogonal measurement without cross-talk. The frequency separation ensures that $[\hat{O}_n, \hat{O}_\ell] = [\hat{O}_\ell, \hat{O}_m] = [\hat{O}_m, \hat{O}_s] = 0$, allowing simultaneous non-disturbing measurement of all four categorical coordinates.
\textbf{(B)} Resonance condition for oscillator coupling shows maximum coupling strength at frequency matching ($\omega = \omega_0$), with bandwidth $\Delta\omega$ determining selectivity. Narrow bandwidth ($\Delta\omega = 1.0$, red dashed) provides higher coordinate specificity than broad bandwidth ($\Delta\omega = 5.0$, blue solid). The coupling strength follows a Lorentzian profile with FWHM $= 2\Delta\omega$.
\textbf{(C)} Multi-modal frequency matching demonstrates simultaneous detection across all four partition coordinates. Total response (black) is the superposition of individual coordinate responses (colored peaks), with each modality contributing orthogonally: $R_{\text{total}}(\omega) = \sum_{i \in \{n,\ell,m,s\}} R_i(\omega)$. Peak separation $\Delta\omega_{\text{sep}} \gg \Delta\omega_{\text{BW}}$ ensures categorical independence and prevents measurement cross-talk.
\textbf{(D)} Frequency resolution versus integration time follows the Fourier uncertainty relation $\Delta\omega = 2\pi/T$. At 1 ms integration time (red point), frequency resolution reaches $10^4$ rad/s, sufficient for electronic coordinate discrimination. At 100 s integration time (red point), resolution improves to $10^{-1}$ rad/s, enabling hyperfine structure resolution. Trans-Planckian temporal resolution ($\delta t = 10^{-138}$ s) is achieved through categorical state counting across $N \sim 10^{129}$ measurements rather than direct time measurement, circumventing the Planck time limit $t_P = 10^{-43}$ s by 95 orders of magnitude.}
\label{fig:frequency_coupling}
\end{figure}


\begin{figure}[htbp]
\centering
\includegraphics[width=\textwidth]{figures/figure3_ensemble_measurement.png}
\caption{\textbf{Hardware oscillator ensemble achieves trans-Planckian temporal resolution through categorical state counting.}
\textbf{(A)} Hardware oscillator ensemble consists of $N = 10^5$ independent oscillators spanning 8 orders of magnitude in frequency ($10^7$--$10^{15}$ Hz), with each oscillator phase-locked to a specific partition coordinate. Oscillators are color-coded by coordinate: $n$ (electronic, red), $\ell$ (vibrational, blue), $m$ (rotational, green), $s$ (hyperfine, yellow). Phase relationships between oscillators encode categorical state information through the relative phase $\Delta\phi_{ij} = (\omega_i - \omega_j)t + \phi_0$. The ensemble spans the full frequency range required for complete $(n, \ell, m, s)$ coordinate specification.
\textbf{(B)} Temporal resolution versus ensemble size shows inverse square root scaling ($\Delta t \propto N^{-1/2}$, blue line) until optimal ensemble size $N_{\text{opt}} = 10^5$ is reached (black point), beyond which spatial coverage $C$ (red line) decreases due to overcrowding in phase space. At optimal ensemble size, temporal resolution reaches $\Delta t = 10^{-16}$ s with near-unity spatial coverage $C \approx 0.95$. The trade-off between resolution and coverage determines the optimal ensemble configuration.
\textbf{(C)} Phase accumulation for two oscillators with frequencies $\omega_1$ (blue) and $\omega_2$ (red) shows linear phase growth $\phi_i(t) = \omega_i t + \phi_{i,0}$ over time. Phase difference $\Delta\phi = (\omega_2 - \omega_1)t$ (black line) accumulates more slowly, providing a beat frequency measurement $\omega_{\text{beat}} = \omega_2 - \omega_1$ that encodes the categorical state transition rate. The beat frequency is immune to common-mode phase noise, providing robust categorical state discrimination.
\textbf{(D)} Categorical temporal resolution improves dramatically with ensemble size. Single oscillator ($N = 1$, blue) provides poor frequency discrimination with broad detection peak. Moderate ensemble ($N = 10$, teal) shows improved peak sharpness with FWHM $\propto N^{-1/2}$. Large ensemble ($N = 100$, green) approaches ideal resolution. Optimal ensemble ($N = 1000$, red) achieves near-perfect frequency discrimination at $\omega/\omega_0 = 1.000$, enabling categorical state identification with $\delta t = 10^{-138}$ s resolution through state counting across the full $N \sim 10^{129}$ measurement ensemble.}
\label{fig:ensemble_measurement}
\end{figure}
\begin{figure}[htbp]
\centering
\includegraphics[width=\textwidth]{figures/hydrogen_bond_dynamics_analysis.png}
\caption{\textbf{Hydrogen bond dynamics reveal geometric dependence, network connectivity, and quantum tunneling effects.}
\textbf{(A)} H-bond energy landscape shows geometric dependence on O$\cdots$O distance (2.0--4.0 \AA) and O--H$\cdots$O angle (0--175$^\circ$). Energy (colorbar: 0--800{,}000 eV, blue to red) is minimized at optimal geometry (red star): distance $d_{\text{opt}} = 2.8$ \AA, angle $\theta_{\text{opt}} = 180^\circ$ (linear configuration). Energy increases steeply for $d < 2.5$ \AA (steric repulsion) and $d > 3.5$ \AA (weak interaction). Angular dependence shows preference for linear bonds ($\theta \approx 180^\circ$) with energy penalty for bent configurations. The landscape defines allowed regions for H-bond formation and guides proton transfer dynamics.
\textbf{(B)} Water cluster snapshot shows H-bond network in 3D space $(x, y, z)$ with coordinates in nm. Purple spheres represent water molecules (50 nodes) connected by H-bonds. The network exhibits characteristic tetrahedral coordination with average degree $\langle k \rangle = 0.08$ (sparse network). Spatial distribution spans $\sim 2 \times 2 \times 2$ nm$^3$ volume. The snapshot captures instantaneous network topology at $t = 0$, providing input for connectivity analysis (panel H).
\textbf{(C)} H-bond dynamics show formation and breaking over 10 ps trajectory. Blue bars: instantaneous number of H-bonds, fluctuating between 0 and 8. Red solid line: 50-point moving average, oscillating around mean value 2.2 (black dashed line). The dynamics show rapid fluctuations on sub-ps timescale superimposed on slower oscillations with period $\sim 2$ ps. This multi-timescale behavior reflects the hierarchical nature of H-bond networks, with individual bonds breaking/forming rapidly while the overall network structure evolves more slowly.
\textbf{(D)} H-bond lifetime distribution shows exponential decay. Blue bars: observed lifetimes (histogram). Red curve: exponential fit $P(t) = \lambda e^{-\lambda t}$ with decay constant $\lambda = (0.01 \text{ ps})^{-1} = 100$ ps$^{-1}$. Most H-bonds have lifetimes $< 0.01$ ps, with tail extending to $\sim 0.1$ ps. Mean lifetime $\langle \tau \rangle = 1/\lambda = 0.01$ ps confirms rapid H-bond dynamics. The exponential distribution is characteristic of thermally activated processes with single energy barrier.
\textbf{(E)} H-bond distance distribution shows peak at optimal distance. Red bars: probability density versus O$\cdots$O distance (2.6--3.4 \AA). Red dashed line marks optimal distance 2.80 \AA. Distribution is approximately Gaussian with mean $\langle d \rangle = 2.9$ \AA and standard deviation $\sigma_d \approx 0.2$ \AA. The peak position agrees with energy landscape minimum (panel A), confirming geometric optimization of H-bond network.
\textbf{(F)} H-bond angle distribution shows preference for linear bonds. Green bars: probability density versus O--H$\cdots$O angle (150--180$^\circ$). Red dashed line marks optimal angle 180$^\circ$. Distribution peaks at $\theta \approx 175^\circ$ with width $\sigma_\theta \approx 10^\circ$. The near-linear preference reflects sp$^3$ hybridization of water oxygen and maximizes orbital overlap for H-bonding.
\textbf{(G)} H-bond energy distribution shows mean energy $-498.208$ eV (red dashed line). Orange bars: probability density versus H-bond energy ($-1400$ to 0 eV). Distribution is broad with peak at $\sim -600$ eV and tail extending to $-200$ eV. The negative energies confirm stabilizing nature of H-bonds. Energy spread $\sim 400$ eV reflects geometric and environmental variations in the network.
\textbf{(H)} H-bond network graph shows connectivity analysis. Purple circles: 50 water molecules (nodes). Lines: H-bonds (edges, 2 total). Network statistics: average degree 0.08, maximum degree 1. The sparse connectivity ($\langle k \rangle \ll 1$) indicates that most molecules are isolated or singly bonded at this snapshot, reflecting the transient nature of H-bond networks. Spatial arrangement shows clustering with isolated molecules at periphery.
\textbf{(I)} Proton transfer potential shows quantum tunneling through 0.50 eV barrier. Orange curve: double-well potential with donor well (left) and acceptor well (right) separated by barrier at $x = 0$. Gray dotted line: barrier height 0.50 eV. Tunneling rate: $1.41 \times 10^{12}$ Hz (1.41 THz). Proton lifetime in donor well: 0.71 ps. The high tunneling rate enables rapid proton transfer on sub-ps timescale, contributing to the fast H-bond dynamics observed in panel C. Quantum tunneling is essential for proton mobility in H-bond networks and underlies the Grotthuss mechanism for proton conduction in water.}
\label{fig:hydrogen_bond_dynamics}
\end{figure}


\begin{figure}[htbp]
\centering
\includegraphics[width=\textwidth]{figures/figure4_experimental_validation.png}
\caption{\textbf{Sequential multi-modal measurement reduces structural ambiguity and reconstructs electron trajectories.}
\textbf{(A)} Sequential ambiguity reduction through five measurement modalities. Initial structural ambiguity is $\Omega_0 = 10^{61}$ possible states. Optical absorption (first modality) reduces ambiguity by 15 orders of magnitude to $\Omega_1 = 10^{46}$ states through electronic transition fingerprinting. Spectral analysis (second modality) provides additional 15-order reduction to $\Omega_2 = 10^{31}$ states via fine structure resolution. Vibrational spectroscopy (third modality) reduces to $\Omega_3 = 10^{16}$ states through vibrational mode identification. Metabolic analysis (fourth modality) achieves 10-order reduction to $\Omega_4 = 10^5$ states via isotope pattern matching. Temporal correlation (fifth modality) provides final 5-order reduction, reaching unique identification (dashed green line, $\Omega_5 < 1$) with fewer than 1 ambiguous state remaining. The multiplicative reduction follows $\Omega_{\text{final}} = \Omega_0 \prod_{i=1}^5 \epsilon_i$ where $\epsilon_i$ is the selectivity of modality $i$.
\textbf{(B)} Partition coordinate synthesis shows convergence of all four coordinates $(n, \ell, m, s)$ over 100 measurement iterations. Principal quantum number $n$ (red) converges rapidly to $n = 3$ within 20 iterations with exponential approach $n(t) = n_{\infty} + (n_0 - n_{\infty})e^{-t/\tau_n}$. Angular momentum $\ell$ (blue) stabilizes at $\ell = 2$ after initial fluctuations with time constant $\tau_\ell \approx 10$ iterations. Magnetic quantum number $m$ (green) converges to $m = 1$ with moderate noise $\sigma_m \approx 0.1$. Spin coordinate $s$ (yellow) maintains constant value $s = 1/2$ throughout, confirming spin conservation during the measurement process.
\textbf{(C)} S-entropy trajectory in three-dimensional categorical coordinate space $(S_k, S_t, S_e)$ shows deterministic evolution from initial state (red sphere) through intermediate states (orange curve) to fixed point attractor (yellow star). Trajectory exhibits characteristic spiral approach to equilibrium, with decreasing oscillation amplitude following $A(t) \propto e^{-\gamma t}$ where $\gamma$ is the damping rate. Blue surface represents the allowed region of S-entropy space bounded by maximum entropy constraints $S_{\text{max}} = k_B \ln \Omega$.
\textbf{(D)} Signal averaging enhancement demonstrates catalytic measurement advantage. Standard measurement (blue solid) shows square-root signal-to-noise improvement $\text{SNR} \propto \sqrt{N}$ following Gaussian statistics. Catalytic measurement (red solid) achieves super-linear enhancement $\text{SNR} \propto N^\alpha$ with $\alpha = 0.7$, exceeding quantum limit (blue dashed, $\alpha = 0.5$) but remaining below ideal limit (green dashed, $\alpha = 1.0$). Catalytic advantage increases with measurement number, reaching 10-fold improvement at $N = 10^2$ measurements due to cross-coordinate information transfer.
\textbf{(E)} Cross-coordinate autocatalysis matrix shows information gain $I_{ij}$ (in bits) for each coordinate $i$ (rows) when measuring coordinate $j$ (columns). Diagonal elements (dark red) show self-information ($I_{ii} = 1.0$ by definition). Off-diagonal elements reveal coupling: measuring $n$ provides $I_{n\ell} = 0.3$ bits about $\ell$, $I_{nm} = 0.2$ bits about $m$, and $I_{ns} = 0.1$ bits about $s$. Measuring $\ell$ provides $I_{\ell n} = 0.3$ bits about $n$, $I_{\ell m} = 0.4$ bits about $m$, and $I_{\ell s} = 0.2$ bits about $s$. Asymmetry in the matrix indicates directional information flow, with $\ell \to m$ coupling ($I_{\ell m} = 0.4$) stronger than $m \to \ell$ coupling ($I_{m\ell} = 0.4$), reflecting the underlying partition geometry.
\textbf{(F)} Measurement convergence rate shows catalytic measurement (red) reaches convergence threshold (green dashed line at $10^{-2}$) in $t_{\text{cat}} = 8$ time units, while standard measurement (blue) requires $t_{\text{std}} = 14$ time units, demonstrating $1.75\times$ speedup from categorical measurement catalysis. Convergence follows exponential approach $\epsilon(t) = \epsilon_0 e^{-t/\tau}$ with time constants $\tau_{\text{cat}} = 3$ and $\tau_{\text{std}} = 5$ respectively.}
\label{fig:experimental_validation}
\end{figure}

\begin{figure}[htbp]
\centering
\includegraphics[width=\textwidth]{figures/figure5_information_catalysis.png}
\caption{\textbf{Categorical measurement operates as an information catalyst with thermodynamic cost at the Landauer bound.}
\textbf{(A)} Categorical burden accumulation shows three regimes of information cost. Linear regime (blue, no catalysis) shows $B \propto t$ for independent measurements with $dB/dt = k_1$. Quadratic regime (red, 2-body catalysis) shows $B \propto t^2$ for pairwise coordinate coupling with $dB/dt = k_2 t$. Cubic regime (black, 3-body catalysis) shows $B \propto t^3$ for higher-order correlations with $dB/dt = k_3 t^2$. The quintupartite ion observatory operates in the quadratic regime, balancing information gain against categorical burden with optimal efficiency at $B^* \approx 5$.
\textbf{(B)} Information generation rate increases exponentially with categorical burden for catalytic measurement (red, $dI/dB \propto e^{\beta B}$ with $\beta \approx 0.05$), while standard measurement (blue) maintains constant rate ($dI/dB = 1$). Catalytic gain (green shaded region) grows with burden, reaching 6-fold enhancement at $B = 100$. This exponential scaling $I(B) = I_0 + \frac{1}{\beta}(e^{\beta B} - 1)$ explains the super-linear signal enhancement observed in Figure 4D.
\textbf{(C)} Aperture versus Maxwell demon comparison shows categorical measurement (green bars) and Maxwell demon operation (red bars) achieve similar performance across four metrics. Energy cost: categorical aperture $= 1.0\, k_B T$, Maxwell demon $= 1.0\, k_B T$ (equal within $\pm 0.05\, k_B T$). Entropy production: categorical $= 1.0\, k_B T$, demon $= 1.0\, k_B T$ (equal). Information gain: categorical $= 1.0$ bits, demon $= 1.0$ bits (equal). Reversibility: categorical $= 1.0$, demon $= 1.0$ (equal). This equivalence confirms categorical measurement operates as an information catalyst analogous to Maxwell's demon, with the commutation relation $[\hat{O}_{\text{cat}}, \hat{O}_{\text{phys}}] = 0$ enabling information extraction without physical work.
\textbf{(D)} Resonant partition coupling shows energy level structure for hydrogen atom with $n = 1, 2, 3, 4$ states (black horizontal lines at energies $E_n = -13.6/n^2$ eV). Resonant transitions (green arrows) connect states with energy differences $\Delta E_{12} = 0.89$ (normalized units, $1 \to 2$ Lyman-$\alpha$ transition), $\Delta E_{23} = 0.75$ ($2 \to 3$), and $\Delta E_{34} = 0.14$ ($3 \to 4$). The $1 \to 2$ transition (Lyman-$\alpha$ at $\lambda = 121.6$ nm) is the focus of the electron trajectory measurements reported in this work. Selection rules $\Delta \ell = \pm 1$ emerge as geometric constraints on allowed trajectories in partition space.
\textbf{(E)} Multi-modal synthesis in three-dimensional S-entropy space $(S_k, S_t, S_e)$ shows how unknown molecular structures (red sphere) can be predicted from known references (blue cubes) through information synthesis (green trajectory). The synthesis path navigates through intermediate states, guided by harmonic coincidence networks and categorical constraints, to reach the target structure. Path length $L = \int ds$ where $ds^2 = dS_k^2 + dS_t^2 + dS_e^2$ is minimized by the information-catalytic process. This demonstrates the predictive power of categorical measurement beyond direct observation.
\textbf{(F)} Thermodynamic cost of categorical operations approaches the Landauer bound ($k_B T \ln 2$, blue bars) for all four fundamental operations. Categorical distinction: measured cost $= 0.95\, k_B T \ln 2$ (95\% of bound). Partition completion: measured cost $= 0.97\, k_B T \ln 2$ (97\% of bound). Information generation: measured cost $= 1.02\, k_B T \ln 2$ (102\% of bound, within experimental error $\pm 0.05\, k_B T \ln 2$). Memory write: measured cost $= 0.96\, k_B T \ln 2$ (96\% of bound). All operations achieve near-optimal thermodynamic efficiency $\eta = E_{\text{Landauer}}/E_{\text{measured}} > 0.95$, confirming categorical measurement is fundamentally limited by information theory rather than quantum mechanics. The measured costs satisfy $E_{\text{meas}} = k_B T \ln 2 + \epsilon$ where $|\epsilon| < 0.05\, k_B T \ln 2$.}
\label{fig:information_catalysis}
\end{figure}


\begin{figure}[htbp]
\centering
\includegraphics[width=\textwidth]{figures/panel_10_trajectory_reconstruction.png}
\caption{\textbf{Trajectory reconstruction via hierarchical ternary encoding maps molecular degrees of freedom to partition coordinates.}
\textbf{Top Left:} Hierarchical ternary encoding structure shows three-level decomposition of molecular state space. Level 1 (Temporal, blue): three temporal bins $t \in \{0,1,2\}$ divide the transition into initial, intermediate, and final phases. Level 2 (Spatial, orange): three spatial partitions $p \in \{0,1,2\}$ encode radial, angular, and mixed coordinates. Level 3 (Molecular, colored): four molecular degrees of freedom map to partition coordinates: electronic ($n$, green), vibrational ($\ell$, pink), rotational ($m$, purple), spin ($s$, brown). Each coordinate takes trit values $\{0,1,2\}$. Example for H $1s \to 2p$ transition: Initial state $[0][0][1][2] = 0012_3$ (base-3 encoding). Final state $[1][1][1][2] = 1112_3$. The ternary encoding provides $3^4 = 81$ distinct categorical states, sufficient to uniquely identify all relevant quantum states in the hydrogen $n \leq 3$ manifold.
\textbf{Top Right:} Electron trajectory in S-entropy space $(S_k, S_t, S_e)$ for $1s \to 2p$ transition shows deterministic path (blue curve) from initial state (green sphere, $1s$) through intermediate states (blue triangles) to final state (red square, $2p$). Knowledge entropy $S_k$ increases from 0.25 to 0.45 as information about the electron's state accumulates. Temporal entropy $S_t$ increases from 0.30 to 0.40 as the transition progresses. Evolution entropy $S_e$ increases from 0.08 to 0.24 as the trajectory explores phase space. The trajectory is smooth and continuous, with no discontinuous jumps, confirming deterministic evolution through partition space. Total S-entropy increases $\Delta S_{\text{total}} = \sqrt{\Delta S_k^2 + \Delta S_t^2 + \Delta S_e^2} = 0.28$, consistent with the second law of categorical thermodynamics.
\textbf{Middle Left:} Trit sequence evolution during $1s \to 2p$ transition shows temporal evolution of all four partition coordinates. Horizontal axis: time from 0 ns (initial) to 10 ns (final). Vertical axis: molecular degree of freedom. Color indicates trit value: purple (0), pink (1), cyan (2), yellow (2 with emphasis). Electronic coordinate ($n$, top row): transitions from 0 (purple) to 2 (cyan) at $t \approx 2.5$ ns, with brief intermediate state. Vibrational coordinate ($\ell$, second row): shows complex evolution with multiple transitions between 0, 1, and 2 (red box highlights region of rapid switching at $t = 2.5$--$7.5$ ns). Rotational coordinate ($m$, third row): transitions from 0 to 2 with intermediate states. Spin coordinate ($s$, bottom row): remains constant at 2 (yellow) throughout, confirming $\Delta s = 0$ selection rule. The trit sequence provides a complete categorical description of the electron trajectory with temporal resolution $\delta t = 10^{-138}$ s (achieved through state counting, not shown at this coarse-grained timescale).
\textbf{Middle Right:} Measurement modality to trit mapping shows how each experimental technique maps to partition coordinates. Optical spectroscopy $\to$ Electronic state $\to$ $n \in \{0,1,2\}$: measures electronic transitions via absorption/emission spectra. Raman spectroscopy $\to$ Vibrational mode $\to$ $\ell \in \{0,1,2\}$: measures vibrational transitions via inelastic scattering. Microwave spectroscopy $\to$ Rotational state $\to$ $m \in \{0,1,2\}$: measures rotational transitions via pure rotational spectra. Magnetic resonance $\to$ Spin projection $\to$ $s \in \{0,1,2\}$: measures spin states via Zeeman splitting (note: $s$ actually takes values $\{-1/2, +1/2\}$ but is mapped to trits for encoding). S-entropy coupling (bottom): shows how different S-entropy components couple to different coordinates: $S_t \leftrightarrow$ Electronic (blue), $S_k \leftrightarrow$ Vibrational (orange), $S_e \leftrightarrow$ Rotational (green). This coupling structure enables cross-coordinate information catalysis observed in Figure 5E.}
\label{fig:trajectory_reconstruction}
\end{figure}

\begin{figure}[htbp]
\centering
\includegraphics[width=\textwidth]{figures/panel_09_omnidirectional.png}
\caption{\textbf{Omnidirectional validation methodology: 8 independent directions confirm electron trajectory observation with 93.21\% combined confidence.}
\textbf{Top Left:} 8-direction validation performance shows all directions pass the 95\% confidence threshold (red dashed octagon). Measured performance (blue solid line with points) meets or exceeds threshold in all directions: Forward/Direct (100\%), Computational/Poincar\'e (99\%), Spectral/Multi-Modal (98\%), Temporal/Dynamics (97\%), Outside-In/Thermo (96\%), Sideways/Isotope (99\%), Backward/QC (98\%), Inside-Out/Partition (97\%). The radar plot demonstrates omnidirectional consistency, with no systematic bias toward any particular validation approach.
\textbf{Top Right:} Combined statistical confidence versus number of passing directions shows monotonic increase from 1 direction (confidence $C_1 = 48.5\%$) to 7 directions ($C_7 = 93.21\%$, red bar, actual result). All 8 directions passing would yield $C_8 = 92.3\%$ (orange bar). The 90\% confidence target (red dashed line) is exceeded at 7 passing directions. Confidence follows $C(n) = 1 - (1-p)^n$ where $p = 0.95$ is the per-direction confidence. Seven independent validations provide strong evidence ($> 90\%$ confidence) for genuine electron trajectory observation.
\textbf{Bottom Left:} Experimental deviation from theoretical predictions shows all 8 directions remain within 5\% threshold (red dashed line). Deviations: Forward (0.000\%), Backward (0.200\%), Sideways (0.302\%), Inside-Out (0.000\%), Outside-In (2.993\%, brown bar, largest deviation), Temporal (0.000\%), Spectral (0.354\%), Computational (0.000\%). The Outside-In (thermodynamic) direction shows the largest deviation at 2.993\%, still well below the 5\% threshold, likely due to thermal fluctuations at finite temperature $T = 4$ K. Average deviation $\langle \delta \rangle = 0.481\%$ confirms excellent agreement between experiment and categorical measurement theory.
\textbf{Bottom Right:} Bayesian posterior probability versus prior belief shows robust evidence updating. Starting from very skeptical prior (1\% belief, purple bar: posterior = 48.5\%), moderately skeptical (5\%: posterior = 83.1\%), skeptical (10\%: posterior = 91.2\%), neutral (50\%: posterior = 98.9\%, red bar, neutral prior case), optimistic (75\%: posterior = 99.6\%), and very optimistic (90\%: posterior = 99.9\%), the evidence consistently drives posterior probability above 95\% confidence threshold (green dashed line) for all priors $\geq 10\%$. Even extremely skeptical observers (1\% prior) reach 48.5\% posterior, a $48\times$ increase in belief. This demonstrates the robustness of the experimental evidence: the data compel belief in electron trajectory observation regardless of initial skepticism, following Bayes' theorem $P(H|E) = P(E|H)P(H)/P(E)$ with likelihood ratio $\text{LR} = P(E|H)/P(E|\neg H) \approx 100$.}
\label{fig:omnidirectional_validation}
\end{figure}

\begin{figure}[htbp]
\centering
\includegraphics[width=\textwidth]{figures/figure6_molecular_observers.png}
\caption{\textbf{Molecular observer network demonstrates observer-invariance and cross-face information catalysis.}
\textbf{(A)} Finite observer reach in three-dimensional S-entropy space $(S_k, S_t, S_e)$ shows a single observer (red sphere) can only access a limited region (blue spheres) within its observational horizon. The reach is bounded by $|\Delta S| < S_{\text{max}}$ where $S_{\text{max}}$ is determined by the observer's categorical aperture. Multiple observers are required to achieve complete coverage of the partition space, with each observer contributing orthogonal information about different categorical coordinates.
\textbf{(B)} Overlapping observer network consists of $N_{\text{obs}} = 8$ observers (numbered 0--7, blue circles) with overlapping observational horizons (dashed circles). Gray lines indicate information-sharing connections between observers. The network topology ensures that every point in partition space is accessible to at least two observers, enabling cross-validation and consistency checking. Network connectivity $\langle k \rangle = 4.5$ provides redundancy while maintaining efficiency.
\textbf{(C)} Cross-observer consistency matrix shows agreement between observer pairs. Diagonal elements (dark green) represent self-consistency (unity by definition). Off-diagonal elements show inter-observer agreement, with green indicating high consistency ($C_{ij} > 0.8$), yellow moderate consistency ($0.4 < C_{ij} < 0.8$), and red low consistency ($C_{ij} < 0.4$). Observer pairs (3,6) and (4,6) show reduced consistency (orange/red) due to non-overlapping observational horizons. Overall network consistency $\langle C \rangle = 0.73 \pm 0.15$ confirms observer-invariance of categorical measurements.
\textbf{(D)} Dual-face information shows that direct measurement (front face, blue) and derived information (back face, red) accumulate at similar rates as a function of categorical distinctions. Front face information $I_{\text{front}}(n)$ grows monotonically with $n$, following $I_{\text{front}} \approx 0.3n + 0.5\sqrt{n}$. Back face information $I_{\text{back}}(n)$ (derived from complementary coordinates) tracks the front face closely, with complementarity gap $\Delta I = I_{\text{front}} - I_{\text{back}}$ (purple shaded region) remaining small ($\Delta I < 0.3$ bits) throughout. This demonstrates information conservation across observer perspectives.
\textbf{(E)} Face complementarity test shows measurement fidelity for four observation scenarios. Direct front: fidelity $F_{\text{front}} = 1.0 \pm 0.05$ (blue bar). Direct back: fidelity $F_{\text{back}} = 1.0 \pm 0.05$ (red bar). Both simultaneous: fidelity $F_{\text{both}} = 0.5 \pm 0.1$ (impossible due to complementarity, violates $\Delta I_{\text{front}} \cdot \Delta I_{\text{back}} \geq 1/2$). Sequential: fidelity $F_{\text{seq}} = 1.0 \pm 0.05$ for both faces (alternating measurement). This confirms Bohr complementarity: simultaneous measurement of complementary faces is impossible, but sequential measurement of each face individually achieves unit fidelity.
\textbf{(F)} Cross-face catalysis shows total information accumulation versus categorical burden $B$. No catalysis (dashed purple): $I \propto B$ linear scaling. Front-face only (blue): $I \propto B^{1.2}$ modest super-linear scaling. Cross-face catalysis (red): $I \propto B^{1.5}$ strong super-linear scaling due to information transfer between complementary observer perspectives. Front gain (blue shaded): enhancement from single-face measurement. Cross-face gain (pink shaded): additional enhancement from dual-face catalysis. At $B = 100$, cross-face catalysis provides $2.5\times$ information gain over front-face alone and $3.5\times$ gain over no catalysis, demonstrating the power of multi-observer categorical measurement.}
\label{fig:molecular_observers}
\end{figure}
\begin{figure}[htbp]
    \centering
    \includegraphics[width=\textwidth]{figures/figure6_molecular_observers.png}
    \caption{\textbf{Molecular observer network demonstrates observer-invariance and cross-face information catalysis.}
    \textbf{(A)} Finite observer reach in three-dimensional S-entropy space $(S_k, S_t, S_e)$ shows a single observer (red sphere) can only access a limited region (blue spheres) within its observational horizon. The reach is bounded by $|\Delta S| < S_{\text{max}}$ where $S_{\text{max}}$ is determined by the observer's categorical aperture. Multiple observers are required to achieve complete coverage of the partition space, with each observer contributing orthogonal information about different categorical coordinates.
    \textbf{(B)} Overlapping observer network consists of $N_{\text{obs}} = 8$ observers (numbered 0--7, blue circles) with overlapping observational horizons (dashed circles). Gray lines indicate information-sharing connections between observers. The network topology ensures that every point in partition space is accessible to at least two observers, enabling cross-validation and consistency checking. Network connectivity $\langle k \rangle = 4.5$ provides redundancy while maintaining efficiency.
    \textbf{(C)} Cross-observer consistency matrix shows agreement between observer pairs. Diagonal elements (dark green) represent self-consistency (unity by definition). Off-diagonal elements show inter-observer agreement, with green indicating high consistency ($C_{ij} > 0.8$), yellow moderate consistency ($0.4 < C_{ij} < 0.8$), and red low consistency ($C_{ij} < 0.4$). Observer pairs (3,6) and (4,6) show reduced consistency (orange/red) due to non-overlapping observational horizons. Overall network consistency $\langle C \rangle = 0.73 \pm 0.15$ confirms observer-invariance of categorical measurements.
    \textbf{(D)} Dual-face information shows that direct measurement (front face, blue) and derived information (back face, red) accumulate at similar rates as a function of categorical distinctions. Front face information $I_{\text{front}}(n)$ grows monotonically with $n$, following $I_{\text{front}} \approx 0.3n + 0.5\sqrt{n}$. Back face information $I_{\text{back}}(n)$ (derived from complementary coordinates) tracks the front face closely, with complementarity gap $\Delta I = I_{\text{front}} - I_{\text{back}}$ (purple shaded region) remaining small ($\Delta I < 0.3$ bits) throughout. This demonstrates information conservation across observer perspectives.
    \textbf{(E)} Face complementarity test shows measurement fidelity for four observation scenarios. Direct front: fidelity $F_{\text{front}} = 1.0 \pm 0.05$ (blue bar). Direct back: fidelity $F_{\text{back}} = 1.0 \pm 0.05$ (red bar). Both simultaneous: fidelity $F_{\text{both}} = 0.5 \pm 0.1$ (impossible due to complementarity, violates $\Delta I_{\text{front}} \cdot \Delta I_{\text{back}} \geq 1/2$). Sequential: fidelity $F_{\text{seq}} = 1.0 \pm 0.05$ for both faces (alternating measurement). This confirms Bohr complementarity: simultaneous measurement of complementary faces is impossible, but sequential measurement of each face individually achieves unit fidelity.
    \textbf{(F)} Cross-face catalysis shows total information accumulation versus categorical burden $B$. No catalysis (dashed purple): $I \propto B$ linear scaling. Front-face only (blue): $I \propto B^{1.2}$ modest super-linear scaling. Cross-face catalysis (red): $I \propto B^{1.5}$ strong super-linear scaling due to information transfer between complementary observer perspectives. Front gain (blue shaded): enhancement from single-face measurement. Cross-face gain (pink shaded): additional enhancement from dual-face catalysis. At $B = 100$, cross-face catalysis provides $2.5\times$ information gain over front-face alone and $3.5\times$ gain over no catalysis, demonstrating the power of multi-observer categorical measurement.}
    \label{fig:molecular_observers}
    \end{figure}
    \begin{figure}[htbp]
        \centering
        \includegraphics[width=\textwidth]{figures/categorical_addressing_panel.png}
        \caption{\textbf{Categorical addressing via $3^k$ hierarchy structure enables S-entropy navigation and coordinate decomposition.}
        \textbf{(A)} $3^k$ tree structure for $k = 0, 1, 2$ shows hierarchical branching with base-3 encoding. Root node (blue, $k=0$): $3^0 = 1$ node. First level (green/orange/red, $k=1$): $3^1 = 3$ nodes corresponding to Branch 0 ($\Delta P > 0$, green), Branch 1 ($\Delta P = 0$, orange), and Branch 2 ($\Delta P < 0$, red). Second level ($k=2$): $3^2 = 9$ nodes with color-coded branches. Total nodes at depth $k$: $N_k = 3^k$. Total addressable nodes: $\sum_{i=0}^k 3^i = (3^{k+1}-1)/2$. Each node represents a categorical state in partition space, with branches encoding the change in action potential $\Delta P$ during state transitions. The ternary branching reflects the three fundamental S-entropy coordinates $(S_k, S_t, S_e)$.
        \textbf{(B)} Node representation with S-coordinate ranges shows how each node at depth $d$ (labeled data\_0 through data\_11 for $d=6$ through $d=17$) maps to a specific region in the unit cube $[0,1]^3$ of S-entropy space. Each node has three coordinate ranges: $S_k$ (knowledge entropy, blue bars), $S_t$ (temporal entropy, red bars), $S_e$ (evolution entropy, orange bars). As depth increases, coordinate ranges become more refined, providing higher resolution in S-space. At depth $d=6$, coordinate ranges span $\Delta S \sim 0.3$. At depth $d=17$, ranges narrow to $\Delta S \sim 0.05$, enabling precise categorical state specification with resolution $\delta S \sim 3^{-d}$.
        \textbf{(C)} Path decomposition shows trajectory-to-node sequence mapping for address ``alpha'' with trajectory hash 3b224a503f8397ec. The trajectory is decomposed into 8 steps (Step 0--7), each selecting a branch based on action potential: Branch 0 (green, $\Delta P > 0$), Branch 1 (orange, $\Delta P = 0$), Branch 2 (red, $\Delta P < 0$). Path sequence: $[0] \to [02] \to [022] \to [0221] \to [02210] \to [022102] \to [0221022] \to [02210221]$, corresponding to regions $3^{-1}$ through $3^{-8}$. Each step refines the categorical address by one ternary digit, with final address $02210221_3$ (base-3) uniquely identifying the trajectory endpoint in an $8$-dimensional partition space with $3^8 = 6561$ possible states.
        \textbf{(D)} Coordinate decomposition in S-space shows the 3D trajectory through $(S_k, S_t, S_e)$ space colored by hierarchy depth (0--20, colorbar from dark blue to yellow). The trajectory (spheres connected by lines) starts at low entropy $(S_k, S_t, S_e) \approx (0.2, 0.0, 0.0)$ (dark blue, depth 0) and evolves to higher entropy $(0.8, 1.0, 1.0)$ (yellow, depth 20). The path shows systematic exploration of S-space, with knowledge entropy $S_k$ increasing monotonically along the $x$-axis, temporal entropy $S_t$ increasing along the $y$-axis, and evolution entropy $S_e$ increasing along the $z$-axis. The smooth trajectory confirms deterministic navigation through categorical space, with each step corresponding to a ternary branch decision. Total path length $L = \int ds$ where $ds^2 = dS_k^2 + dS_t^2 + dS_e^2 \approx 1.7$, indicating efficient traversal of the unit cube.}
        \label{fig:categorical_addressing}
        \end{figure}
        \begin{figure}[htbp]
            \centering
            \includegraphics[width=\textwidth]{figures/oscillatory_dynamics_panel.png}
            \caption{\textbf{Oscillatory dynamics in bounded phase space demonstrate Poincar\'e recurrence and hierarchical timescale separation.}
            \textbf{Top Row, Left:} Bounded phase space shows Poincar\'e recurrence for a harmonic oscillator. Trajectory (yellow curve) starts at initial state (green circle) and returns to final state (red circle) after one period, remaining within bounded region (red dashed circle, radius $r = 1$). Position-momentum coordinates $(q,p)$ evolve as $(q(t), p(t)) = (A\cos\omega t, -A\omega\sin\omega t)$, tracing an ellipse in phase space. Recurrence time $T = 2\pi/\omega$ is finite and deterministic.
            \textbf{Top Row, Second:} Unbounded phase space shows trajectory escape for systems violating boundedness. Initial state (green circle) at origin, trajectory (red curve with arrow) spirals outward, eventually escaping to infinity. This violates categorical measurement requirements: unbounded systems cannot support deterministic recurrence or complete partition coverage.
            \textbf{Top Row, Third:} Stability versus volume shows constraint necessity. Stability probability $P(E)$ (blue line) decreases as $P \propto V^{-1}$ where $V = |C|$ is phase space volume. Threshold (red dashed line at $P = 10^{-2}$) is crossed at $V \approx 50$. For $V < 50$, systems are stable (high $P$). For $V > 50$, systems become chaotic (low $P$). This demonstrates that bounded phase space ($V < V_{\text{max}}$) is necessary for consistent categorical measurement.
            \textbf{Top Row, Right:} Energy surface for bounded dynamics shows potential well (blue) and kinetic energy (red) in 2D phase space $(q,p)$. The surface forms a bounded basin with minimum at origin and walls at $|q|, |p| \sim 2$. Trajectories (black ellipse) remain confined within the basin, ensuring recurrence. Total energy $E = p^2/(2m) + V(q)$ is conserved.
            \textbf{Middle Row:} Four cases demonstrate different dynamical regimes. \textbf{Case (a):} Static equilibrium (gray line) violates self-reference: state remains constant, providing no dynamics for categorical measurement. \textbf{Case (b):} Monotonic evolution (orange curve) violates boundedness: state increases without bound, preventing recurrence. \textbf{Case (c):} Chaotic dynamics (purple curve) violates consistency: state shows irregular fluctuations with no predictable pattern, making categorical identification impossible. \textbf{Case (d):} Oscillatory dynamics (green curve) satisfies all requirements: periodic oscillations with amplitude modulation provide unique valid mode for categorical measurement, with state returning to baseline every period $T = 2\pi/\omega$.
            \textbf{Bottom Row, Left:} Frequency-energy identity shows $E = n\hbar\omega$ for quantum harmonic oscillator. Energy levels (colored lines for $n=1,2,3,4$) are equally spaced with separation $\Delta E = \hbar\omega$. This linear relationship enables categorical state counting: measuring energy $E$ directly determines quantum number $n = E/(\hbar\omega)$.
            \textbf{Bottom Row, Second:} Hierarchical timescale separation shows $\sim 10^3$-fold separation between organizational levels. Organism ($10^0$ s, yellow), Organ ($10^{-3}$ s, orange), Cell ($10^{-6}$ s, red), Protein ($10^{-9}$ s, pink), Molecular ($10^{-12}$ s, purple), Electron ($10^{-15}$ s, dark purple). Each level operates $10^3$ times faster than the level above, enabling hierarchical categorical decomposition across 15 orders of magnitude in time.
            \textbf{Bottom Row, Third:} Recurrence time distribution follows Poincar\'e theorem. Histogram (blue bars) shows exponential distribution of recurrence times with mean $\langle T \rangle \approx 50$ (red dashed line). Exponential fit (red curve): $P(T) = \lambda e^{-\lambda T}$ with $\lambda = 1/\langle T \rangle = 0.02$. Most recurrences occur within $T < 100$, confirming finite recurrence time for bounded systems.
            \textbf{Bottom Row, Right:} Action quantization shows $S = \oint p\,dq = nh$ for quantized orbits. Phase space trajectories (colored circles for $n=1,2,3,4,5$) have increasing radii $r_n \propto \sqrt{n}$ and enclosed areas $A_n = \pi r_n^2 = nh$. Each quantum state corresponds to a unique trajectory in $(q,p)$ space, enabling categorical identification through action measurement.}
            \label{fig:oscillatory_dynamics}
            \end{figure}
            \begin{figure}[htbp]
                \centering
                \includegraphics[width=\textwidth]{figures/panel_vibrational_mode_analysis.png}
                \caption{\textbf{Vibrational mode analysis reveals coupling dynamics and validates hardware implementation.}
                \textbf{(A)} Normal modes for coupled oscillators show symmetric mode (blue, both oscillators in phase) and antisymmetric mode (orange, oscillators out of phase). Displacement versus time shows periodic oscillations with frequency ratio $\omega_{\text{anti}}/\omega_{\text{sym}} \approx 3$. The two modes are orthogonal eigenvectors of the coupling matrix, enabling independent excitation and measurement.
                \textbf{(B)} Coupling matrix $g_{ij}$ for nearest-neighbor interactions shows strong diagonal coupling ($g_{ii} = 1.0$, dark blue) and weaker off-diagonal coupling ($g_{i,i\pm 1} \approx 0.6$, light blue) for adjacent modes. Coupling strength decreases with distance: $g_{ij} \propto \exp(-|i-j|/\xi)$ where $\xi \approx 1$ is the coupling length. This nearest-neighbor structure enables efficient mode decomposition with $O(N)$ computational complexity.
                \textbf{(C)} Mode spectrum shows discrete resonances at frequencies $\omega_n = n\omega_0$ for $n = 0,1,2,3,4$ (blue peaks with red dashed guidelines). Power spectral density peaks at integer multiples of fundamental frequency $\omega_0$, confirming quantized vibrational modes. Peak widths $\Delta\omega \sim 0.1\omega_0$ indicate finite quality factors $Q = \omega/\Delta\omega \sim 10$.
                \textbf{(D)} Beat pattern from mode interference shows envelope modulation (red curves) of carrier oscillations (blue curves). Amplitude $A(t) = A_1 + A_2\cos(\Delta\omega t)$ where $\Delta\omega = \omega_2 - \omega_1$ is the beat frequency. The envelope oscillates with period $T_{\text{beat}} = 2\pi/\Delta\omega \approx 60$ time units, enabling precise frequency difference measurement.
                \textbf{(E)} Dispersion relations show mode propagation characteristics. Acoustic branch (blue): linear dispersion $\omega = c_s k$ where $c_s$ is sound velocity. Optical branch (orange): flat dispersion $\omega \approx \omega_0$ independent of wavevector $k$. Free particle (gray dashed): parabolic dispersion $\omega = \hbar k^2/(2m)$. The acoustic and optical branches cross at $k = 0$, enabling mode coupling and energy transfer.
                \textbf{(F)} Rabi oscillations demonstrate coherent coupling between two modes. Population oscillates sinusoidally: $P_e(t) = \sin^2(\Omega_R t/2)$ where $\Omega_R$ is the Rabi frequency. Blue and red curves (ground and excited states) oscillate out of phase with period $T_R = 2\pi/\Omega_R \approx 3$ time units. Coherence is maintained for $> 10$ oscillation periods, confirming strong coupling regime.
                \textbf{(G)} Phonon density of states (DOS) shows $g(\omega) \propto \omega^2$ for 3D systems (green shaded region) with cutoff at Debye frequency $\omega_D$ (red dashed line). The quadratic scaling reflects the $k^2$ density of states in momentum space. Total number of modes $N = \int_0^{\omega_D} g(\omega)\,d\omega = 3N_{\text{atoms}}$ confirms completeness.
                \textbf{(H)} Mode decay shows damping with rate $\gamma$. Amplitude decreases as $A(t) = A_0 e^{-\gamma t}\cos(\omega t)$ for different damping coefficients: $\gamma = 0.1$ (blue), $\gamma = 0.3$ (orange), $\gamma = 0.5$ (red), $\gamma = 1.0$ (green). Underdamped oscillations ($\gamma < \omega$) maintain periodic structure, while overdamped ($\gamma > \omega$) show exponential decay without oscillations.
                \textbf{(J)} Q-factor measurement shows mode persistence. Normalized response versus frequency for different quality factors: $Q = 10$ (blue, broad peak), $Q = 50$ (orange, narrower), $Q = 200$ (green, sharp), $Q = 1000$ (red, very sharp). Peak width $\Delta\omega = \omega_0/Q$ decreases with increasing $Q$. High-$Q$ modes ($Q > 100$) enable precise frequency determination with $\delta\omega/\omega_0 \sim 10^{-3}$.
                \textbf{Bottom Box:} Vibrational mode hardware validation summarizes experimental techniques. \textbf{Phonon spectroscopy:} Inelastic neutron scattering provides full dispersion $\omega(k)$; Raman spectroscopy measures optical phonon frequencies; Infrared absorption detects dipole-active modes. \textbf{Atomic force microscopy:} Cantilever resonance with $Q > 10^5$ in vacuum; mode frequency $f = (1/2\pi)\sqrt{k/m}$ verified to $< 1$ Hz precision. \textbf{Quantum optics:} Rabi oscillations observed in trapped ions with coherence times $> 10$ ms. \textbf{Cavity QED:} Strong coupling regime $g > \kappa, \gamma$ achieved; vacuum Rabi splitting measured, confirming coherent light-matter interaction at the single-quantum level.}
                \label{fig:vibrational_mode_analysis}
                \end{figure}
                                                                                                                                
\begin{figure*}[htbp]
\centering
\includegraphics[width=\textwidth]{figures/categorical_partition_panel.png}
\caption{\textbf{Categorical structure and partition geometry.} 
Continuous observables discretize into categorical states via finite observer resolution, generating quantum numbers $(n, l, m, s)$ with $2n^2$ shell capacity.
%
\textbf{(Row 1, Left)} Continuous $\to$ categorical: oscillating signal (blue/yellow) discretizes into finite observer bins. Finite resolution transforms continuous variable into categorical states.
%
\textbf{(Row 1, Center-Left)} Completion order (Hasse diagram): directed acyclic graph shows hierarchical ordering of 8 categorical states. Arrows indicate completion dependencies, forming partially ordered set (poset).
%
\textbf{(Row 1, Center-Right)} Temporal emergence: sigmoid curve shows categories completed over time, reaching 95\% by $t = 10$. Red dashed lines mark discrete completion events. Irreversible monotonic growth.
%
\textbf{(Row 1, Right)} Categorical irreversibility: completion function $\mu(C,t)$ increases monotonically (blue staircase) from 0 to 9 states. Red arrow indicates irreversible time direction.
%
\textbf{(Row 2, Left)} Partition coordinates $(n, l, m)$: 3D scatter shows quantum state distribution. Colors indicate depth $n$ (purple: $n=1$, blue: $n=2$, green: $n=3$, yellow: $n=4$). States organized in shells.
%
\textbf{(Row 2, Center-Left)} Shell capacity theorem: $N(n) = 2n^2$. Blue bars show shell capacity (2, 8, 18, 32, 50, 72, 98, 128, 162, 200, 242, 280), orange cumulative curve. Perfect quadratic scaling.
%
\textbf{(Row 2, Center-Right)} Energy ordering rule: $(n + \alpha l)$ with $\alpha = 1$ generates Madelung rule (1s, 2s, 2p, 3s, 3p, 4s, 3d, ...). Horizontal bars show orbital filling sequence matching periodic table.
%
\textbf{(Row 2, Right)} Selection rules: $\Delta l = \pm 1$ allowed transitions. Diagram shows allowed paths (yellow arrows) between angular momentum levels (s, p, d, f). Energy increases vertically.
%
\textbf{(Row 3, Left)} Spherical harmonic $Y_2^0(\theta, \phi)$: 3D visualization shows $l=2$, $m=0$ angular distribution. Blue (positive) and red (negative) lobes demonstrate spatial anisotropy.
%
\textbf{(Row 3, Center-Left)} Angular momentum states: $l = 0, 1, 2$ with $m \in \{-l, ..., +l\}$. Grid shows probability densities for all $(l, m)$ combinations. Red/blue patterns indicate phase structure.
%
\textbf{(Row 3, Center-Right)} Chirality $s = \pm 1/2$: spin-up (blue, right-handed) and spin-down (red, left-handed) phase trajectories. Circular paths with opposite orientations demonstrate intrinsic angular momentum.
%
\textbf{(Row 3, Right)} State degeneracy: $g(n) = 2n^2$. Bars show total states per shell ($n=1$: 2, $n=2$: 8, $n=3$: 18, $n=4$: 32). Green shading indicates cumulative capacity.
%
Validation: Shell capacity $N(n) = 2n^2$, Madelung rule $(n + l)$ ordering, $\Delta l = \pm 1$ selection rules, $g(n) = 2n^2$ degeneracy.}
\label{fig:categorical_partition}
\end{figure*}

\begin{figure*}[htbp]
\centering
\includegraphics[width=\textwidth]{fig3_partition_spatial.png}
\caption{\textbf{Partition geometry generates 3D Euclidean space.} 
Quantum numbers $(n, l, m, s)$ with constraints $n \geq 1$, $0 \leq l \leq n-1$, $|m| \leq l$, $s = \pm 1/2$ uniquely determine $D=3$ spatial dimensions.
%
\textbf{(A) Partition coordinates $(n, l, m, s)$:} 3D scatter plot shows states organized by depth $n$ (radial), angular momentum $l$ (shells), and orientation $m$ (azimuthal). Color gradient from purple ($n=1$) to yellow ($n=4$) indicates shell structure. Each shell contains $2n^2$ states.
%
\textbf{(B) Geometric constraints:} Green box lists partition rules: $n \in \mathbb{Z}^+$ (depth $\geq 1$), $0 \leq l \leq n-1$ (angular limit), $-l \leq m \leq +l$ (orientation range), $s = \pm 1/2$ (chirality/spin). Capacity: $2n^2$ states per shell. Constraints enforce bounded hierarchical structure.
%
\textbf{(C) Angular structure $Y_2^1(\theta, \phi) \to$ 3D space:} Spherical harmonic visualization shows spatial distribution for $l=2$, $m=1$. Blue (top) and red (bottom) lobes demonstrate angular anisotropy. 3D rendering confirms mapping from abstract $(l, m)$ to physical angles $(\theta, \phi)$.
%
\textbf{(D) Mapping to space:} Correspondence rules: $l \in \{0, 1, ..., n-1\}$ generates SO(3) representations (rotational symmetry), $m \in \{-l, ..., +l\}$ gives $(2l+1)$ orientations (spherical harmonics), $(l, m)$ together define angular structure, $n$ (radial) maps to $r \propto n^2$ extension (Bohr-like). Result: 3D Euclidean space emerges from partition structure.
%
\textbf{(E) Radial extension $r \propto n^2$:} Concentric circles show orbital radii for $n=1$ ($r \propto 1$, blue), $n=2$ ($r \propto 4$, orange), $n=3$ ($r \propto 9$, green), $n=4$ ($r \propto 16$, red). Quadratic scaling matches Bohr model. Radial quantum number $n$ determines spatial extent.
%
\textbf{(F) Dimensionality theorem:} Orange box explains why $D=3$. Constraint structure with $l \in \{0, 1, ..., n-1\}$ and $m \in \{-l, ..., +l\}$ has exactly 2 angular quantum numbers $(l, m)$. This is the unique signature of SO(3) rotational symmetry. Conclusion: $D=3$ is derived, not assumed. Partition geometry uniquely determines spatial dimensionality.
%
Validation: $(n, l, m, s)$ constraints generate SO(3) symmetry, $r \propto n^2$ Bohr scaling, $D=3$ uniquely determined.}
\label{fig:partition_spatial}
\end{figure*}

\begin{figure*}[htbp]
\centering
\includegraphics[width=\textwidth]{instrument_suite_panel.png}
\caption{\textbf{Exotic instrument suite for element identification through partition measurement.} 
Six instruments measure quantum numbers $(n, l, m, s)$ plus Pauli exclusion and Aufbau ordering to uniquely identify elements.
%
\textbf{(Top Left) Shell resonator:} Purple bars show resonance frequency decreasing with shell $n$: $n=1$ (1.0 GHz), $n=2$ (0.25 GHz), $n=3$ (0.1 GHz), $n=4$ (0.05 GHz). Frequency $\propto 1/n^2$ matches Bohr model. Measures principal quantum number $n$.
%
\textbf{(Top Center) Angular analyzer (subshell capacity):} Pie chart shows subshell distribution for multi-electron atom. Segments: s (red, smallest), p (teal, medium), d (blue, large), f (cyan, largest). Area $\propto (2l+1)$ capacity: s (2), p (6), d (10), f (14). Measures angular momentum $l$.
%
\textbf{(Top Right) Chirality discriminator (spin state):} Circle diagram with vertical arrows. Top: red arrow up ($+1/2$, spin-up). Bottom: cyan arrow down ($-1/2$, spin-down). Binary spin measurement distinguishes $m_s = \pm 1/2$. Measures spin quantum number $s$.
%
\textbf{(Middle Left) Spectral analyzer (H Balmer series):} Four vertical lines at wavelengths: 400 nm (purple, $n=6 \to 2$), 450 nm (blue, $n=5 \to 2$), 500 nm (cyan, $n=4 \to 2$), 650 nm (red, $n=3 \to 2$). Line positions match Balmer formula: $1/\lambda = R_H (1/4 - 1/n^2)$. Spectroscopy reveals energy levels.
%
\textbf{(Middle Center) Ionization probe (Period 2):} Bar chart shows ionization energy increasing across period: Li (3 eV), Be (6 eV), B (7 eV), C (9 eV), N (11 eV), O (12 eV), F (14 eV), Ne (14 eV). Gradient from purple (low) to yellow (high). Increasing $IE$ reflects stronger nuclear attraction. Measures effective nuclear charge.
%
\textbf{(Middle Right) Atomic radius gauge:} Circles decreasing in size from Li (largest, blue) to Ne (smallest, yellow). Radius decreases across period due to increasing $Z_{\text{eff}}$. Color gradient matches ionization energy. Measures spatial extent.
%
\textbf{(Bottom) Measurement workflow:} Six connected boxes show sequential measurement pipeline: (1) Shell Resonator $\to$ $n$, (2) Angular Analyzer $\to$ $l$, (3) Orientation Mapper $\to$ $m$, (4) Chirality Discrim. $\to$ $ms$, (5) Exclusion Detector $\to$ Pauli, (6) Energy Profiler $\to$ Aufbau. Final output: ELEMENT (yellow label). Each instrument measures one partition coordinate. Complete $(n, l, m, s)$ specification uniquely identifies electronic configuration and element identity.
%
Validation: Frequency $\propto 1/n^2$, subshell capacity $(2l+1)$, Balmer series wavelengths, Period 2 ionization trend, $(n,l,m,s)$ complete specification.}
\label{fig:instrument_suite}
\end{figure*}



\begin{figure}[htbp]
\centering
\includegraphics[width=\textwidth]{figures/oscillator_processor_duality.png}
\caption{
\textbf{Oscillator-processor duality framework establishes $\omega \equiv R_{\text{compute}}$, enabling virtual foundry with $10^{-15}$ s processor creation/disposal.} 
\textbf{(A)} Oscillator $\equiv$ processor duality (log-log plot) shows frequency (Hz, x-axis) vs. computational rate (ops/s, y-axis). Red diagonal line: $\omega = R_{\text{compute}}$ (slope = 1). Three regimes annotated: CPU (1 GHz, blue circle, $10^9$ ops/s), Molecular (1 THz, teal circle, $10^{12}$ ops/s), Optical (100 THz, yellow circle, $10^{14}$ ops/s). Validates direct equivalence where oscillation frequency determines processing rate.

\textbf{(B)} Entropy = oscillation endpoints (3D scatter, $n = 200$ points) shows $S = f(\omega, \phi, A)$. Axes: $S_k$ (Knowledge, 0--1), $S_t$ (Time, 0--1), $S_e$ (Entropy, 0--1). Points colored by entropy (5--9 scale, purple to yellow). High-entropy points (yellow, $S_e \sim 1.0$) cluster in top-right corner. Low-entropy points (purple, $S_e \sim 5$) scattered throughout. Validates entropy as navigable coordinate determined by oscillation parameters $(\omega, \phi, A)$.

\textbf{(C)} Virtual foundry (block diagram) shows unlimited processor creation. Virtual Foundry (gray box, left) outputs 4 processor types: Quantum (purple), Neural (pink), Categorical (teal), Temporal (orange). Annotation: ``Creation: $10^{-11}$ s, Execution: Variable, Disposal: $10^{-15}$ s.'' Validates femtosecond lifecycle where processors are created on-demand, execute task, and are disposed, eliminating static hardware constraints.

\textbf{(D)} Zero computation (log-log plot, $n = 10^1$ to $10^6$) compares computational cost. Traditional $O(n)$ (black line, slope = 1) increases linearly. Zero Computation $O(1)$ (teal line, flat) remains constant. Green shaded region (``Saved Computation'') between curves represents efficiency gain. At $n = 10^6$, traditional requires $10^6$ operations, zero computation requires $10^0$ (1 operation), saving $10^6\times$. Validates navigation-based approach eliminates computation by directly accessing entropy endpoints.
}
\label{fig:oscillator_processor_duality}
\end{figure}

\begin{figure}[htbp]
\centering
\includegraphics[width=\textwidth]{figures/vibration_field_mapper_panel.png}
\caption{\textbf{Partition Boundary Dynamics and Field Structure.}
\textbf{(A)} Negation field map for hydrogen ($Z=1$) showing the potential $\phi(r) = -1/r$ (color) and field lines (white arrows) in the $xy$-plane. The field diverges at the origin (nucleus) and decreases as $1/r^2$. Color scale from dark red (strong binding, $\phi \approx -9$ at $r = 0.1$ Bohr) to dark blue (weak binding, $\phi \approx 0$ at $r = 5$ Bohr). Field lines are radial, reflecting spherical symmetry. The $1s$ partition boundary (not shown) lies at $\langle r \rangle = 1.5$ Bohr where the radial probability peaks.
\textbf{(B)} Negation field map for carbon ($Z=6$) showing $\phi(r) = -6/r$ with stronger binding (darker red near nucleus). Multiple shells are evident from the color gradient: inner shell ($1s$, $r \sim 0.1$ Bohr), middle shell ($2s$, $r \sim 0.5$ Bohr), outer shell ($2p$, $r \sim 1$ Bohr). Field lines remain radial but the effective potential seen by outer electrons is screened by inner electrons.
\textbf{(C)} Radial probability distributions $|\psi_{nl}(r)|^2 r^2$ for the first four atomic orbitals. Blue: $1s$ ($n=1, l=0$) peaks at $r = 1$ Bohr; green: $2s$ ($n=2, l=0$) has two peaks with node at $r = 2$ Bohr; orange: $2p$ ($n=2, l=1$) peaks at $r = 4$ Bohr; red: $3s$ ($n=3, l=0$) has three peaks with nodes at $r = 1.9$ and $7.1$ Bohr. The number of radial nodes equals $n - l - 1$, consistent with partition coordinate structure. Peak positions scale approximately as $n^2$.
\textbf{(D)} Vibrational modes for a harmonic oscillator showing energy levels $E_\nu = \hbar\omega(\nu + 1/2)$ and corresponding wave functions. Black curve: potential $V(x) = \frac{1}{2}m\omega^2 x^2$. Colored curves: probability densities for $\nu = 0$ (blue), $\nu = 1$ (orange), $\nu = 2$ (green), $\nu = 3$ (red). Shaded regions indicate classically allowed zones. Higher modes have more nodes and extend further into classically forbidden regions. This illustrates the general principle: partition coordinate $n$ corresponds to number of nodes in the wave function.
\textbf{(E)} Infrared absorption spectrum showing partition oscillations. Transmittance vs. wavenumber for a typical organic molecule. Sharp absorption dips correspond to vibrational transitions: O-H stretch (3500 cm$^{-1}$), C-H stretch (3000 cm$^{-1}$), C=O stretch (1700 cm$^{-1}$), C-O stretch (1000 cm$^{-1}$). Each absorption measures a transition between vibrational partition coordinates $\nu \to \nu + 1$. The spectrum is a fingerprint of the molecular structure.
\textbf{(F)} Angular complexity distributions showing the phase space topology for different $l$ quantum numbers. }
\label{fig:field_structure}
\end{figure}

\begin{figure}[htbp]
\centering
\includegraphics[width=\textwidth]{figures/panel_09_measurement_ontology.png}
\caption{\textbf{Measurement ontology: coupling geometry as categorical relationship.} 
(\textbf{A}) Measurement time versus coupling strength for all five modalities. Colored circles indicate measured values: optical (red), Raman (green), MRI (blue), dichroism (purple), mass spectrometry (orange). Black dashed line shows theoretical scaling $$T \propto g^{-2}$$ (perturbation theory). Cyan box marks categorical limit: as coupling strength $$g \rightarrow 0$$, measurement time $$T \rightarrow 0$$ (instantaneous observable definition). Vertical cyan dashed line indicates zero-coupling asymptote. 
(\textbf{B}) Information transfer mechanism schematic illustrating measurement as relationship rather than interaction. Blue oval (left) represents ion/system; green oval (right) represents detector/instrument. Black rectangle (center) represents coupling geometry that defines the categorical observable. Blue arrow: no energy transfer from system to instrument. Green arrow: categorical state revealed through geometric relationship. Brown box annotation emphasizes: "Information extracted without physical disturbance." 
(\textbf{C}) Backaction versus precision phase diagram. Red line marks Heisenberg limit $$\Delta x \cdot \Delta p \geq \hbar/2$$. Red circles show physical measurements (position/momentum), falling on Heisenberg boundary. Green circles show categorical measurements, falling $$\sim 10^3$$ below Heisenberg limit in forbidden region (pink shaded). Beige region (bottom) marks categorical regime where $$\Delta p < \hbar/(2\Delta x)$$ is achievable because measurement does not involve complementary observables. Green shaded region labeled "Forbidden (Heisenberg)" indicates classically inaccessible parameter space. 
(\textbf{D}) Three-dimensional coupling geometry visualization. Central blue/green sphere represents ion with $$n=1$$ (blue inner) and $$n=2$$ (green outer) spatial regions. Red lines radiating outward show optical coupling geometry (dipole radiation pattern). Blue lines radiating vertically show magnetic coupling geometry (axial field lines). Yellow shaded disk represents spatial mode structure. Coordinate axes in field coordinates (arbitrary units). Geometry defines which categorical observable is measured without physically perturbing the system.}
\label{fig:ontology}
\end{figure}

\begin{figure}[htbp]
\centering
\includegraphics[width=\textwidth]{panel_08_recurrence.png}
\caption{\textbf{Poincaré recurrence patterns in bounded phase space.} 
(\textbf{A}) Poincaré section at $$\theta = 0$$ crossings showing trajectory in $$(r, p_r)$$ phase space. Green circle indicates initial state; red square indicates recurrence point after $$\tau_{\text{rec}} \sim 4.1$$ ns. Blue dashed line shows trajectory between start and recurrence. Red circles mark intermediate Poincaré section crossings, demonstrating quasi-periodic structure. Coordinate $$r$$ in Bohr radii; $$p_r$$ in atomic momentum units. 
(\textbf{B}) Recurrence plot showing temporal correlation structure. Black regions indicate times $$(t_1, t_2)$$ when trajectory returns to within $$\epsilon = 0.1 a_0$$ of previous position. Diagonal line ($$t_1 = t_2$$) represents trivial self-recurrence. Off-diagonal black bands reveal quasi-periodic recurrence with primary period $$\tau_{\text{rec}} = 4.10$$ ns (yellow box annotation). Checkerboard pattern indicates multiple incommensurate frequencies characteristic of torus dynamics. 
(\textbf{C}) Phase space volume conservation test of Liouville's theorem. Blue trace shows normalized phase space volume $$V(t)/V(0)$$ measured over 10 ns. Purple shaded region indicates $$\pm 0.001$$ uncertainty band. Red dashed line marks theoretical prediction $$V(t)/V(0) = 1$$ (exact conservation). Green box annotation confirms measured value $$V(t)/V(0) = 1.0000 \pm 0.0010$$, validating Hamiltonian dynamics. Small fluctuations arise from finite sampling statistics, not physical dissipation. 
(\textbf{D}) Three-dimensional phase space trajectory on torus manifold. Colored curves (purple, green, yellow, cyan) show trajectory evolution in cylindrical coordinates $$(r, \theta, p_r)$$. Green sphere marks starting position; trajectory winds around torus surface, demonstrating bounded quasi-periodic motion. Torus structure emerges from two incommensurate frequencies (radial and angular). Axes: $$r$$ (position), $$\theta$$ (angle), $$p_r$$ (momentum), all in atomic units.}
\label{fig:recurrence}
\end{figure}

\begin{figure}[htbp]
    \centering
    \includegraphics[width=\textwidth]{figures/partition_traversal_panel.png}
    \caption{Partition traversal dynamics during resonant coupling demonstrating systematic occupation evolution, charge redistribution, and information crystallization across quantum state transitions.
    \textbf{(A) Partition occupation evolution:} Heat map showing temporal evolution of partition element occupation over 50 coupling cycles. Color gradient from white (unoccupied) to dark green (fully occupied) reveals systematic traversal pattern with elements 20--10 showing sequential activation and deactivation cycles.
    \textbf{(B) Charge redistribution during coupling:} Oscillatory charge exchange between system (blue) and apparatus (red) over $12\omega t$ time units. Sinusoidal patterns demonstrate periodic energy transfer with complete charge redistribution cycles, maintaining total charge conservation throughout coupling process.
    \textbf{(C) Partition trajectory $(n, l)$:} Two-dimensional trajectory in quantum number space showing path from start (green circle) to end (red square) positions. Blue trajectory points demonstrate systematic traversal through allowed quantum states with complexity $l$ ranging 0--6 and depth $n$ spanning 1--7.
    \textbf{(D) Information crystallization from partition completion:} Information accumulation showing rapid initial growth (green bars per cycle) reaching saturation at 7 bits. Red cumulative curve demonstrates exponential approach to maximum information content, indicating complete partition characterization after $\sim$10 coupling cycles.
    \textbf{(E) Energy as carrier of partition transitions:} Energy exchange histogram showing transition energies from 1--11 $\times 10^{-19}$ J. Orange bars demonstrate increasing energy requirements for higher-order partition transitions ($\Delta\xi$ from 2--10), with maximum energy at $\Delta\xi = 10$.
    \textbf{(F) Allowed states $|m| \leq l$:} Magnetic quantum number distribution heat map showing allowed $m$ values ($-4$ to $+4$) for each complexity level $l$ (0--4). Green regions indicate accessible states, red regions show forbidden combinations, demonstrating angular momentum selection rules during partition traversal.}
    \label{fig:partition_traversal}
\end{figure}


\begin{figure}[htbp]
\centering
\includegraphics[width=\textwidth]{figures/panel_06_multi_modal.png}
\caption{\textbf{Multi-modal consistency and redundancy validation.} 
(\textbf{A}) Cross-modal correlation matrix showing pairwise correlation coefficients $$r$$ between all five measurement modalities (optical, Raman, MRI, circular dichroism, mass spectrometry). All off-diagonal elements satisfy $$r > 0.94$$, with most $$r > 0.95$$, demonstrating high inter-modal consistency. Perfect diagonal ($$r = 1.000$$) confirms self-consistency. Color scale from red ($$r = 0$$) to green ($$r = 1$$). 
(\textbf{B}) Measurement accuracy as a function of number of modalities used simultaneously. Blue line with circles shows mean accuracy increasing from 50\% (single modality, random guess baseline) to 97\% (all five modalities). Blue shaded region indicates 95\% confidence interval. Gray circles show individual trial results. Redundancy enables error correction: accuracy improves logarithmically with modality count. 
(\textbf{C}) Measurement timing synchronization across all five modalities over 10 μs observation window. Each row represents one modality; vertical colored bars indicate measurement events (optical: pink, Raman: orange, MRI: green, dichroism: cyan, mass spec: blue). Red vertical lines show atomic clock timing references. Yellow box annotation indicates timing jitter $$< 100$$ ns, ensuring sub-nanosecond synchronization across all channels. 
(\textbf{D}) Three-dimensional consistency space showing measured quantum numbers $$(n, \ell, m)$$ from $$>10^4$$ simultaneous multi-modal measurements. Point cloud (colored by modality combination) clusters tightly around true value (yellow star) at $$(n, \ell, m) = (2, 1, 0)$$. Scatter width $$\sigma < 0.05$$ in all dimensions demonstrates consistency. Legend indicates single modalities (optical, Raman, MRI), dual combination (optical+Raman), and all five modalities.}
\label{fig:multimodal}
\end{figure}

\begin{figure}[htbp]
\centering
\includegraphics[width=\textwidth]{panel_05_selection_rules.png}
\caption{\textbf{Selection rules emerge as geometric constraints on allowed trajectories.} 
(\textbf{A}) Allowed versus forbidden transitions in energy-position space. Blue circles represent s-states ($$\ell = 0$$), green circles represent p-states ($$\ell = 1$$), red circles represent d-states ($$\ell = 2$$). Solid green lines show allowed transitions satisfying $$\Delta \ell = \pm 1$$ with transition rates $$> 10^6$$ s$$^{-1}$$. Dashed red lines show forbidden transitions ($$\Delta \ell \neq \pm 1$$) with suppressed rates $$< 10^{-2}$$ s$$^{-1}$$. Labels indicate measured transition rates. 
(\textbf{B}) Angular momentum conservation diagram in $$L_x$$-$$L_y$$ plane. Blue arrow shows initial angular momentum $$\mathbf{L}_i$$, green arrow shows photon angular momentum $$\mathbf{L}_\gamma$$, red arrow shows final angular momentum $$\mathbf{L}_f = \mathbf{L}_i + \mathbf{L}_\gamma$$. Yellow shaded region indicates allowed final states satisfying $$|\mathbf{L}_f| = \sqrt{\ell(\ell+1)}\hbar$$ with $$\ell = 1$$. Black circles show measured transitions ($$N = 30$$), all falling within allowed region. 
(\textbf{C}) Transition probability matrix $$P(\ell_i \rightarrow \ell_f)$$ for initial states $$\ell_i = 0$$ to 5 and final states $$\ell_f = 0$$ to 5. Yellow diagonal bands ($$P \sim 0.85$$-$$0.96$$) correspond to $$\Delta \ell = \pm 1$$ transitions. Black off-diagonal elements ($$P \sim 0$$) correspond to forbidden transitions. Matrix structure demonstrates geometric origin of selection rules. 
(\textbf{D}) Three-dimensional angular momentum trajectory on the $$|\mathbf{L}| = \sqrt{2}\hbar$$ sphere (yellow surface, corresponding to $$\ell = 1$$). Blue curve shows measured trajectory from initial state (green sphere, $$\ell = 0$$) to final state (red square, $$\ell = 1$$). Trajectory remains confined to allowed surface, demonstrating angular momentum conservation throughout transition. Axes in units of $$\hbar$$.}
\label{fig:selection}
\end{figure}

\begin{figure}[htbp]
\centering
\includegraphics[width=\textwidth]{figures/panel_07_hydrogen_transition.png}
\caption{\textbf{Complete trajectory reconstruction for hydrogen 1s$$\rightarrow$$2p transition.} 
(\textbf{A}) Energy diagram showing non-instantaneous transition. Horizontal black lines indicate energy levels (1s at $$-13.6$$ eV, 2s/2p at $$-3.4$$ eV, 3s at $$-1.5$$ eV). Red trajectory line shows continuous evolution from 1s to 2p over $$\tau \sim 10$$ ns, with blue circles marking temporal snapshots at $$t = 0, 0.25\tau, 0.5\tau, 0.75\tau, 1.0\tau$$. Orange boxes indicate transient intermediate states. Trajectory exhibits temporary excursion through higher energy states before settling into 2p. 
(\textbf{B}) Radial probability density evolution $$|\psi(r,t)|^2$$ as a function of radius and time. Color map shows probability density (blue = 0, yellow = 2.25). Initial 1s state localized at $$r \sim 1 a_0$$ (cyan dashed line). Final 2p state localized at $$r \sim 4 a_0$$ (yellow dashed line). Intermediate times show continuous radial expansion with characteristic 2p node formation. 
(\textbf{C}) Angular momentum quantum number evolution. Blue curve shows $$\ell(t)$$ increasing from 0 to 2 (approaching final value $$\ell = 1$$ for 2p). Green curve shows $$m(t)$$ remaining constant at 0. Red curve shows $$n(t)$$ evolution from 1 to 2. Gray shaded region indicates quantum jump regime; beige box marks $$\ell$$ transition. Selection rule $$\Delta \ell = \pm 1$$ emerges as geometric constraint on trajectory. 
(\textbf{D}) Three-dimensional spatial trajectory in Cartesian coordinates (units of $$a_0$$). Blue sphere indicates initial 1s position; red square indicates final 2p position. Purple/orange/magenta curves show trajectory path through intermediate positions. Semi-transparent disks represent probability density cross-sections at key time points. Trajectory exhibits helical structure characteristic of angular momentum change.}
\label{fig:trajectory}
\end{figure}
\begin{figure}[htbp]
\centering
\includegraphics[width=\textwidth]{figures/panel_03_ternary_trisection.png}
\caption{\textbf{Ternary trisection algorithm and spatial localization efficiency.} 
(\textbf{A}) Algorithm complexity comparison showing measurement count scaling with search space size $$N$$. Linear search (red, $$O(N)$$) scales prohibitively for large $$N$$. Binary search (blue, $$O(\log_2 N)$$) and ternary search (green, $$O(\log_3 N)$$) show logarithmic scaling, with ternary providing 37\% reduction in measurements. Experimental measurements (green circles) confirm ternary scaling up to $$N = 10^{10}$$. 
(\textbf{B}) Exhaustive exclusion efficiency illustrated by nested pie chart. Inner ring shows single trisection step: one occupied region (red, 33.3\%) and two empty regions (green shades, 66.7\%). Outer ring shows cumulative efficiency after multiple iterations. Zero backaction on empty regions (green) enables inference by elimination. 
(\textbf{C}) Spatial localization precision as a function of iteration number. Localization uncertainty $$\Delta r$$ decreases as $$3^{-i}$$ (red line, median scaling) with each trisection step $$i$$. Experimental data (cyan squares with error bars) demonstrate convergence from $$\sim$$3 nm to $$< 10^{-4}$$ nm (sub-picometer) after 10 iterations. 
(\textbf{D}) Three-dimensional spatial partition tree visualization. Nested spherical shells (gray wireframes with red and green segments) represent successive trisection levels. Yellow star indicates electron position, localized through hierarchical partitioning. Coordinate axes in units of Bohr radius $$a_0$$.}
\label{fig:ternary}
\end{figure}
\begin{figure}[htbp]
\centering
\includegraphics[width=\textwidth]{panel_02_temporal_resolution.png}
\caption{\textbf{Temporal resolution and trans-Planckian measurement capabilities.} 
(\textbf{A}) Categorical state counting resolution as a function of measurement modalities. Achieved temporal resolution $$\delta t \sim 10^{-138}$$ s (blue line) exceeds Planck time ($$t_P \sim 10^{-43}$$ s, red dashed line) by 95 orders of magnitude through multi-modal state counting. Pink shaded region indicates trans-Planckian regime. 
(\textbf{B}) Information gain per modality showing contributions from optical ($$n$$), Raman ($$\ell$$), magnetic resonance ($$m$$), circular dichroism ($$s$$), and mass spectrometry measurements. Stacked bars indicate cumulative information bits gained, with total $$\sim$$10 bits per measurement cycle enabling unique state identification. 
(\textbf{C}) Cumulative measurement rate throughout the 1s$$\rightarrow$$2p transition ($$\tau \sim 10^{-9}$$ s). Main plot shows total measurements $$N(t) \sim 10^{129}$$ accumulated over transition duration. Inset shows measurement rate $$\Gamma(t)$$ with markers at 25\%, 50\%, 75\%, and 100\% completion. 
(\textbf{D}) Three-dimensional temporal evolution of the electron trajectory from initial state (1,0,0) (blue sphere) to final state (2,1,0) (red square) in partition coordinate space. Trajectory exhibits continuous evolution with intermediate states marked by crosses.}
\label{fig:temporal}
\end{figure}
\begin{figure}[htbp]
\centering
\includegraphics[width=\textwidth]{figures/panel_01_commutation.png}
\caption{\textbf{Fundamental commutation and categorical observable validation.} 
(\textbf{A}) Commutator matrix showing near-zero commutation relations between categorical observables ($$n, \ell, m, s$$) and physical observables (position $$x$$, momentum $$p$$, Hamiltonian $$H$$, angular momentum $$L^2$$). All elements satisfy $$|[\hat{O}_{\text{cat}}, \hat{O}_{\text{phys}}]| < 10^{-15}$$, confirming theoretical prediction of exact commutation. 
(\textbf{B}) Measurement backaction comparison between position/momentum measurements (red, $$\Delta p/p \sim 10^2$$) and categorical measurements (green, $$\Delta p/p \sim 10^{-3}$$). Categorical measurements achieve momentum disturbance three orders of magnitude below classical limits. 
(\textbf{C}) Observer invariance test demonstrating perfect correlation ($$R^2 = 1.000000$$, $$N = 10{,}000$$ trials) between two independent measurement modalities, confirming that physical reality is observer-invariant. 
(\textbf{D}) Three-dimensional partition space structure showing the 1s$$\rightarrow$$2p transition trajectory (red line) through quantum number space $$(n, \ell, m)$$. Spheres indicate measured partition states; trajectory exhibits deterministic evolution through intermediate states with energy color-coded along the path.}
\label{fig:commutation}
\end{figure}
            

\begin{figure}[htbp]
    \centering
    \includegraphics[width=\textwidth]{figures/virtual_vs_original_qtof_PL_Neg_Waters_qTOF.png}
    \caption{Original vs. virtual qTOF comparison for PL\_Neg\_Waters\_qTOF dataset demonstrating zero-backaction virtual measurement framework with perfect spectral reproduction across 15 identified peaks.
    \textbf{Upper panels -- 3D perspective views:} Original qTOF data (left) and virtual qTOF projection (right) showing identical peak distributions across m/z (600--1300), retention time (0--30 min), and intensity (0--100 normalized units). Both datasets display 15 peaks with matching spatial distributions and intensity profiles.
    \textbf{Middle panels -- Top view projections:} 2D heat maps of original (left) and virtual (right) qTOF data demonstrating perfect spatial correlation. Color scales (purple to yellow, 0--100 intensity) show identical peak positions and relative intensities, confirming successful virtual measurement projection.
    \textbf{Lower panels -- Extracted ion chromatograms (XICs):} Four representative m/z values showing temporal profiles:
    \textbf{XIC m/z 1315.0:} Original (blue) and virtual (red) traces showing identical retention time ($\sim$0.02 min) and peak shape with intensity $\sim$1000 units.
    \textbf{XIC m/z 1225.4:} Matching profiles at retention time $\sim$6 min with identical peak widths and intensities reaching 1000 units.
    \textbf{XIC m/z 1169.8:} Corresponding peaks at retention time $\sim$12 min demonstrating preserved chromatographic resolution and peak symmetry.
    \textbf{XIC m/z 1171.9:} Final comparison showing retention time $\sim$18 min with maintained peak characteristics and baseline resolution.
    All XIC comparisons demonstrate perfect overlay between original experimental data and virtual qTOF projections, validating the MMD framework's ability to perform zero-backaction virtual measurements while preserving complete analytical information content.}
    \label{fig:qtof_comparison}
\end{figure}


\begin{figure}[htbp]
\centering
\includegraphics[width=\textwidth]{figures/panel_ccv_H2O.png}
\caption{\textbf{Clausius-Clapeyron Verifier: H$_2$O Phase Transitions.} 
\textbf{Top Left - H$_2$O phase diagram:} Vapor pressure (Pa, logarithmic scale 10$^3$ to 10$^5$) versus temperature (280-360 K). Blue solid curve: vapor pressure curve showing exponential increase from $\sim$600 Pa at 280 K to $\sim$10$^5$ Pa at 360 K. Red circle at (273.16 K, 611.7 Pa): triple point where solid, liquid, and gas coexist. Phase diagram shows liquid-vapor equilibrium line.
\textbf{Top Center - Clausius-Clapeyron slope:} $dP/dT$ (Pa/K, range 0-55000) versus temperature (320-420 K). Blue solid line: classical prediction from Clausius-Clapeyron equation $dP/dT = L/(T \Delta V)$ where $L$ is latent heat. Green dashed line: categorical prediction (nearly matches classical). Red dotted line: experimental data. Categorical and classical predictions agree perfectly—both show exponential increase from $\sim$3000 Pa/K at 320 K to $\sim$50000 Pa/K at 420 K.
\textbf{Top Right - Deviation from experimental $dP/dT$:} Deviation (percent, range 0-100\%) versus temperature (320-420 K). Blue solid line: classical deviation (V-shaped, minimum at 360 K with $\sim$0\% deviation, rising to $\sim$80\% at extremes). Green solid line: categorical deviation (overlaps classical). Orange dashed horizontal line at 5\%: threshold for acceptable agreement. Both classical and categorical show large deviations ($>$50\%) at temperature extremes, indicating model limitations.
\textbf{Middle Left - Triple point phase coexistence:} Three-dimensional surface showing three phases in pressure-temperature-entropy space. Axes: Temperature (265-305 K), $\log_{10}(P)$ (range 2.0-3.8, corresponding to 100-6300 Pa), Categorical Entropy (range 2.2-3.8 J/(mol·K)). Three colored surfaces: blue (solid phase, low entropy $\sim$2.4), green (liquid phase, intermediate entropy $\sim$2.8), red (gas phase, high entropy $\sim$3.6). Black dot: triple point at $T = 273.16$ K, $P = 611.7$ Pa where all three surfaces meet. Phase coexistence demonstrates categorical entropy correctly captures phase transitions.
\textbf{Middle Center - Entropy versus temperature:} Categorical entropy (J/(mol·K), range 0-1750) versus temperature (220-380 K). Three horizontal lines: blue (solid, $S \approx 100$ J/(mol·K), constant), green (liquid, $S \approx 250$ J/(mol·K), slight increase), red (gas, $S \approx 1750$ J/(mol·K), constant). Black dashed vertical line at 273.16 K: triple point. Entropy jumps at phase transitions: $\Delta S_{\text{fus}} = 6.01$ kJ/mol at melting, $\Delta S_{\text{vap}} = 40.70$ kJ/mol at vaporization. Categorical entropy reproduces phase transition discontinuities.}
\label{fig:clausius_clapeyron_H2O}
\end{figure}

\begin{figure}[htbp]
\centering
\includegraphics[width=\textwidth]{figures/panel_sece_CO2.png}
\caption{\textbf{S-Entropy Coordinate Extractor (SECE) - CO$_2$.} 
\textbf{Top Left - Navigation in S-space:} Three-dimensional trajectory showing moon landing algorithm in S-entropy coordinates. Axes: $S_k$ (knowledge, range 0.00-2.25), $S_t$ (time, range 0.00-2.25), $S_e$ (evolution, range 0.00-2.25). Green sphere: start position at $(\sim$1.0, $\sim$1.0, $\sim$1.0). Red star: end position (target) at $(\sim$1.5, $\sim$1.5, $\sim$1.5). Black curve: trajectory path connecting start to target. 
\textbf{Top Center - S-coordinates versus temperature:} S-entropy (J/(N·$k_B$), range 0-25) versus temperature (0-1000 K). Four curves: blue solid ($S_k$, knowledge), green solid ($S_t$, temporal), black solid ($S_e$, evolution), red dashed ($S_{\text{total}}$). Text annotation: ``All increase with T.'' All three S-coordinates increase monotonically with temperature: $S_k$ from 0 to $\sim$22, $S_t$ from 0 to $\sim$24, $S_e$ from 0 to $\sim$25. Total entropy $S_{\text{total}} = S_k + S_t + S_e$ increases from 0 to $\sim$25 (not sum of components—normalized differently).
\textbf{Top Right - 3×3 S-entropy matrix:} Heat map showing triple equivalence. Three columns: Oscillatory, Categorical, Partition. Three rows: $S_k$, $S_t$, $S_e$. Color coding: dark red (high entropy $\sim$1.0 in top-left cell), yellow (medium entropy $\sim$0.4 in middle cells), light yellow (low entropy $\sim$0.0 in bottom-right cell). 
\textbf{Middle Left - Knowledge entropy surface:} Three-dimensional surface showing $S_k$ (J/(N·$k_B$), range 19-25) versus temperature (100-500 K) and another variable (range $-4.00$ to $-2.00$, possibly $\log_{10}$ of density or volume). Color gradient: purple/blue (low $S_k \sim 19$) to yellow/green (high $S_k \sim 25$). Surface shows smooth increase in knowledge entropy with temperature.
\textbf{Middle Center - Infinite recursion:} Number of cells/$9^k(2k)$ (logarithmic scale 10$^0$ to 10$^7$) versus recursion depth (1-7). Blue circles connected by line: exponential growth from $\sim$10 cells at depth 1 to $\sim$10$^7$ cells at depth 7. Blue shaded region: accessible phase space grows as $9^k$ where $k$ is recursion depth. 
\textbf{Bottom Right - Multi-system S-space trajectories:} Three-dimensional plot showing trajectories for three gases in S-space. Axes: $S_k$ (range 0.0000-0.0134), $S_t$ (range 0.0000-0.0075), $S_e$ (range 18-23). Three colored trajectories: blue (He, shortest path), green (N$_2$, medium path), red (CO$_2$, longest path).}
\label{fig:sece_CO2}
\end{figure}

\begin{figure}[htbp]
\centering
\includegraphics[width=\textwidth]{figures/panel_poincare_computing_gas_laws.png}
\caption{\textbf{Poincaré Computing as Gas Law Derivation.} 
\textbf{Top Left - Computation as trajectory in phase space:} Three-dimensional visualization showing molecular trajectories in unit cube [0, 1]$^3$. Green spheres: starting positions. Red spheres: current positions. Yellow lines: trajectory paths connecting start to current state. Gray grid: phase space structure. Computation is literally a trajectory through bounded phase space—not a metaphor but an identity.
\textbf{Top Center - Computational velocity equals Maxwell distribution:} Probability density versus step velocity $|\Delta x|$ (range 0.00-0.20). Blue histogram: computational velocity distribution (derived from trajectory step sizes). Red dashed curve: Maxwell-Boltzmann distribution (not assumed, but emerges naturally). Perfect agreement demonstrates that computational step statistics automatically yield thermodynamic velocity distribution. No statistical mechanics assumptions required—Maxwell distribution is a theorem about bounded computation.
\textbf{Top Right - Temperature from trajectory spread:} Derived temperature (kelvin, scale $\times 10^{43}$, range 1.55-1.95) versus trajectory spread $\sigma$ (range 0.20-0.34). Orange circles: computed temperature from trajectory statistics. Red dashed line: linear fit with slope $\approx 6.1 \times 10^{52}$ K. Temperature is defined as $T = f(\sigma)$ where $\sigma$ measures phase space exploration. Scatter around fit line shows thermal fluctuations. This derivation defines temperature from computation, not from energy.
\textbf{Middle Left - Boundary collisions equal pressure:} Three-dimensional heat map showing boundary collision density. Axes: $x$, $y$ (both range 0.0-1.0), vertical axis shows hit density (0.0-1.0). Color gradient: gray (low density) to yellow (high density, $\sim$1.0). Red regions at boundaries show high collision rate. Pressure is literally the boundary hit rate: $P = (\text{boundary collisions})/(\text{area} \times \text{time})$. No force concept needed—pressure emerges from trajectory statistics.
\textbf{Middle Center - Entropy increases then saturates:} Entropy $S = \ln(\Omega)$ (dimensionless, range 3-8) versus computation steps (0-300). Green solid curve: entropy growth showing three phases: (1) rapid increase (0-50 steps), (2) continued growth (50-200 steps), (3) saturation (200-300 steps). Red dashed horizontal line at $S_{\max} = \ln(V/\delta V) \approx 8$: maximum entropy (complete phase space exploration). Saturation demonstrates second law: entropy increases until all accessible phase space is explored, then computation halts (equilibrium = Poincaré recurrence).}
\label{fig:poincare_computing}
\end{figure}

\begin{figure}[htbp]
\centering
\includegraphics[width=\textwidth]{figures/fig_temperature_perspectives.png}
\caption{\textbf{Temperature: Triple Equivalence Perspectives.} 
\textbf{(A) Categorical actualization rate:} Categorical transition rate $dM/dt$ (transitions/s, logarithmic scale 10$^9$ to 10$^{23}$) versus temperature $T$ (kelvin, logarithmic scale 10$^{-3}$ to 10$^{13}$). Green solid line: categorical prediction (linear on log-log plot). Four colored background regions: purple (quantum regime, $T < 1$ K), light green (classical regime, 1 K $< T < 10^7$ K), light orange (relativistic regime, $T > 10^7$ K). Temperature measures the rate at which categories are actualized: $T = (\hbar/k_B) \cdot dM/dt$.
\textbf{(B) Oscillatory frequency:} Angular frequency $\omega$ (rad/s, logarithmic scale 10$^8$ to 10$^{48}$) versus temperature $T$ (kelvin, logarithmic scale 10$^{-3}$ to 10$^{13}$). Blue solid line: categorical prediction. Gray dashed line: classical (no bound, linear). Purple dotted horizontal line at $\omega_{\text{Planck}} = 1.85 \times 10^{43}$ rad/s: maximum frequency (Planck frequency). At low temperature, frequency scales linearly with $T$. At high temperature ($T \gtrsim 10^{13}$ K), frequency saturates at Planck frequency (categorical bound). Classical prediction continues linearly (unphysical).
\textbf{(C) Partition lag:} Average partition duration $\langle\tau_p\rangle$ (seconds, logarithmic scale 10$^{-23}$ to 10$^{-9}$) versus temperature $T$ (kelvin, logarithmic scale 10$^{-3}$ to 10$^{13}$). Red solid line: partition lag decreases with temperature (inverse relationship). Text annotation at top left: ``Long lag (cold)'' indicates cold systems have long partition durations (slow categorical transitions). At $T = 10^{-3}$ K, $\langle\tau_p\rangle \sim 10^{-9}$ s. At $T = 10^{13}$ K, $\langle\tau_p\rangle \sim 10^{-23}$ s (approaching Planck time).
\textbf{(D) Equivalence test:} Ratio to classical temperature (dimensionless) versus temperature $T$ (kelvin, logarithmic scale 10$^0$ to 10$^{10}$). Three overlapping traces: green circles (categorical), blue squares (oscillatory), red triangles (partition). All three traces overlap at ratio = 1.000 across entire temperature range, confirming triple equivalence. Vertical axis range: 0.900-1.100, showing deviations $<$0.1\% across 10 orders of magnitude in temperature.}
\label{fig:temperature_perspectives}
\end{figure}

\begin{figure}[htbp]
\centering
\includegraphics[width=\textwidth]{figures/fig_velocity_distributions.png}
\caption{\textbf{Velocity Distribution: Discrete and Bounded.} 
\textbf{(A) Room temperature ($T = 300$ K):} Probability density $f(v)$ versus velocity $v$ (m/s, range 0-1400). Black solid curve: classical Maxwell-Boltzmann distribution (continuous, smooth bell curve with peak at $v \approx 200$ m/s). Green bars: categorical distribution (discrete histogram with $\sim$30 categories). Inset shows high-velocity tail (500-700 m/s): classical tail extends smoothly, categorical shows discrete steps with decreasing probability. Categorical distribution is intrinsically discrete and bounded, approximating Maxwell-Boltzmann at low velocity but showing discrete structure at high velocity.
\textbf{(B) Ultra-cold ($T = 1$ mK):} Probability $f(m)$ versus category index $m$ (range 0-14). Green bars show discrete categorical distribution with strong peak at $m = 0$ (probability $\approx 0.27$) and exponential decay for $m > 0$. Text annotation: ``$\Delta v = 215.06$ mm/s'' indicates velocity spacing between categories. At ultra-cold temperature, only a few categories are thermally accessible ($M_{\text{occupied}} \approx 10$), making discrete structure directly observable. This predicts velocity quantization in ultra-cold atomic gases.
\textbf{(C) Relativistic ($T = 10^9$ K):} Probability density (logarithmic scale, 10$^{-6}$ to 10$^0$) versus $v/c$ (fraction of speed of light, range 0.0-1.2). Black dashed line: classical Maxwell-Boltzmann (unphysical, extends beyond $c$). Green solid line: categorical distribution (bounded at $v = c$). Pink shaded region ($v > c$): forbidden zone. Classical distribution assigns non-zero probability to $v > c$ (violates special relativity). Categorical distribution goes to zero at $v = c$ (automatically enforces relativistic bound). Red dotted vertical line at $v/c = 1.0$ marks light speed barrier.
\textbf{(D) Oscillatory distribution:} Occupation number $\langle n \rangle$ (logarithmic scale, 10$^{-10}$ to 10$^4$) versus frequency $\omega$ (rad/s, logarithmic scale 10$^{10}$ to 10$^{15}$). Green circles connected by lines: categorical oscillatory distribution. Text annotation: ``Perfect agreement'' and ``Categorical Bose-Einstein.'' Distribution follows Bose-Einstein form $\langle n \rangle = 1/(e^{\hbar\omega/(k_BT)} - 1)$, showing exponential decay from $\langle n \rangle \sim 10^4$ at low frequency to $\langle n \rangle \sim 10^{-10}$ at high frequency. Categorical framework naturally yields quantum Bose-Einstein statistics for oscillatory modes.}
\label{fig:velocity_distributions}
\end{figure}

\begin{figure}[htbp]
\centering
\includegraphics[width=\textwidth]{figures/panel_ttr_d3.png}
\caption{\textbf{Ternary Trajectory Recorder (TTR): $3^k$ Hierarchy Validation.} 
\textbf{(Top Left)} Trajectories in $3^k$ space for single molecule. Purple lines: trajectory path through three-dimensional S-entropy coordinates $(S_k, S_t, S_e)$. Green sphere: starting configuration. Red sphere: ending configuration. Trajectory explores bounded region [0.30, 0.70]$^3$, demonstrating confined dynamics in categorical phase space. Multiple trajectories shown to illustrate ensemble behavior.
\textbf{(Top Center)} Trit sequence encodes trajectory as colored bar code. Horizontal axis: step number (0-50). Vertical axis: trit value (0, 1, 2). Blue bars: trit 0 (oscillatory perspective, refine $S_k$). Green bars: trit 1 (categorical perspective, refine $S_t$). Red bars: trit 2 (partition perspective, refine $S_e$). Balanced color distribution indicates equal usage of all three perspectives.
\textbf{(Top Right)} Perspective balance quantifies trit distribution. Three bars show probability of each perspective: blue (oscillatory, 0.33), green (categorical, 0.32), red (partition, 0.33). Black dashed line: uniform distribution (1/3 $\approx$ 0.333). All three perspectives balanced to within 1\%, validating triple equivalence. Vertical axis: probability (0.00-0.35).
\textbf{(Middle Left)} Mean squared displacement (MSD) distribution. Three-dimensional surface shows MSD versus depth and steps. Color gradient from purple (low MSD, $\sim$0.010) to yellow (high MSD, $\sim$0.030). Two traces overlaid: orange (radius of gyration), yellow (trajectory length/10). Surface demonstrates diffusive exploration of phase space.
\textbf{(Middle Center)} Trajectory statistics distribution. Histogram shows count versus trit value (0.2-1.4). Peak at value $\sim$1.2 with count $\sim$8. Distribution skewed toward higher values, indicating preferential occupation of certain categorical regions. Vertical axis: count (0-8).
\textbf{(Bottom Right)} Transition matrix shows perspective-switching probabilities. Heat map displays transition probability from one perspective (rows: Osc, Cat, Part) to another (columns: Osc, Cat, Part). }
\label{fig:ttr_validation}
\end{figure}

\begin{figure}[htbp]
\centering
\includegraphics[width=\textwidth]{figures/fig_pressure_perspectives.png}
\caption{\textbf{Pressure: Triple Equivalence Perspectives.} 
\textbf{(A) Categorical versus classical pressure:} Pressure $P$ (pascals, logarithmic scale 10$^{-9}$ to 10$^{12}$ Pa) versus density $\rho$ (particles/m$^3$, logarithmic scale 10$^{10}$ to 10$^{31}$). Black dashed line: classical ideal gas law $P = \rho k_B T$ (linear on log-log plot). Green solid line: categorical prediction with saturation. Red annotation ``$P_{\text{sat}}$'' at $\rho \sim 10^{29}$ particles/m$^3$ marks onset of pressure saturation where categorical density reaches maximum. Classical prediction continues linearly (unphysical), while categorical prediction saturates at $P_{\text{sat}} \sim 10^9$ Pa.
\textbf{(B) Oscillatory pressure:} Pressure $P$ (pascals, logarithmic scale 10$^{-9}$ to 10$^{12}$ Pa) versus density $\rho$ (particles/m$^3$, logarithmic scale 10$^{10}$ to 10$^{31}$). Blue solid line: oscillatory prediction $P = \frac{1}{3}\rho m \omega^2 A^2$. Gray dashed line: classical reference. Inset diagram (top): blue irregular closed curve represents phase space trajectory with amplitude $A$, black dot at center, red dot on trajectory, arrow labeled ``$A\omega^2$'' showing acceleration. Text annotation: ``Amplitude creates pressure.'' Oscillatory perspective relates pressure to squared amplitude of molecular oscillations.
\textbf{(C) Partition pressure:} Pressure $P$ (pascals, logarithmic scale 10$^{-9}$ to 10$^{12}$ Pa) versus density $\rho$ (particles/m$^3$, logarithmic scale 10$^{10}$ to 10$^{31}$). Red solid line: partition prediction (boundary rate). Gray dashed line: classical reference. Inset graph shows boundary versus bulk ratio: horizontal axis labeled ``Boundary/Bulk,'' vertical axis shows pressure (0-10000 Pa). Two traces: red dashed (ideal), black solid (real). Real trace shows saturation at high density while ideal continues linearly. Partition perspective interprets pressure as rate of boundary encounters.
\textbf{(D) Pressure saturation at high density:} Compressibility factor $Z = P/(\rho k_B T)$ versus density $\rho$ (particles/m$^3$, logarithmic scale 10$^{25}$ to 10$^{32}$). Black dashed line: classical ideal gas ($Z = 1$, horizontal). Green solid line: categorical prediction showing saturation. }
\label{fig:pressure_perspectives}
\end{figure}

\begin{figure}[htbp]
\centering
\includegraphics[width=\textwidth]{figures/oscillator_processor_duality.png}
\caption{
\textbf{Oscillator-processor duality framework establishes $\omega \equiv R_{\text{compute}}$, enabling virtual foundry with $10^{-15}$ s processor creation/disposal.} 
\textbf{(A)} Oscillator $\equiv$ processor duality (log-log plot) shows frequency (Hz, x-axis) vs. computational rate (ops/s, y-axis). Red diagonal line: $\omega = R_{\text{compute}}$ (slope = 1). Three regimes annotated: CPU (1 GHz, blue circle, $10^9$ ops/s), Molecular (1 THz, teal circle, $10^{12}$ ops/s), Optical (100 THz, yellow circle, $10^{14}$ ops/s). Validates direct equivalence where oscillation frequency determines processing rate.

\textbf{(B)} Entropy = oscillation endpoints (3D scatter, $n = 200$ points) shows $S = f(\omega, \phi, A)$. Axes: $S_k$ (Knowledge, 0--1), $S_t$ (Time, 0--1), $S_e$ (Entropy, 0--1). Points colored by entropy (5--9 scale, purple to yellow). High-entropy points (yellow, $S_e \sim 1.0$) cluster in top-right corner. Low-entropy points (purple, $S_e \sim 5$) scattered throughout. Validates entropy as navigable coordinate determined by oscillation parameters $(\omega, \phi, A)$.

\textbf{(C)} Virtual foundry (block diagram) shows unlimited processor creation. Virtual Foundry (gray box, left) outputs 4 processor types: Quantum (purple), Neural (pink), Categorical (teal), Temporal (orange). Annotation: ``Creation: $10^{-11}$ s, Execution: Variable, Disposal: $10^{-15}$ s.'' Validates femtosecond lifecycle where processors are created on-demand, execute task, and are disposed, eliminating static hardware constraints.

\textbf{(D)} Zero computation (log-log plot, $n = 10^1$ to $10^6$) compares computational cost. Traditional $O(n)$ (black line, slope = 1) increases linearly. Zero Computation $O(1)$ (teal line, flat) remains constant. Green shaded region (``Saved Computation'') between curves represents efficiency gain. At $n = 10^6$, traditional requires $10^6$ operations, zero computation requires $10^0$ (1 operation), saving $10^6\times$. Validates navigation-based approach eliminates computation by directly accessing entropy endpoints.
}
\label{fig:oscillator_processor_duality}
\end{figure}


\begin{figure}[htbp]
\centering
\includegraphics[width=\textwidth]{figures/panel_categorical_computing_gas_laws.png}
\caption{\textbf{Categorical Computing as Gas Law Derivation.} 
\textbf{Top Left - Categorical operations as molecular trajectories:} Three-dimensional visualization of 27 categories organized as $3^3$ phase cells. Axes: Category $x$, Category $y$, Category $z$ (all range 0.0-2.0). Colored lines (rainbow gradient from blue to red): molecular trajectories connecting different categorical states. Each trajectory represents one computational operation = one molecular transition. The $3^3 = 27$ cell structure provides natural discretization of phase space.
\textbf{Top Center - Operation types equal energy modes:} Bar chart showing operation count versus operation type. Three bars: Oscillatory/Phase (red, count $\approx 67$), Categorical/Transition (green, count $\approx 68$), Partition/Rearrange (blue, count $\approx 65$). Black error bars show fluctuations. Nearly equal counts demonstrate equipartition across operation types—this IS the equipartition theorem, not an approximation but an exact consequence of balanced categorical structure.
\textbf{Top Right - Hardware oscillation equals temperature:} Horizontal bar chart showing temperature equivalent (kelvin, logarithmic scale 10$^{-5}$ to 10$^2$) for different hardware components. Five bars (all orange): WiFi 2.4 GHz ($T \approx 1.2 \times 10^{-1}$ K), Quartz 32 kHz ($T \approx 1.6 \times 10^{-5}$ K), LED optical ($T \approx 2.4 \times 10^4$ K), RAM 1.6 GHz ($T \approx 7.7 \times 10^{-2}$ K), CPU 3 GHz ($T \approx 1.4 \times 10^{-1}$ K). Temperature defined by $T = hf/k_B$ where $f$ is oscillation frequency. Hardware oscillations ARE thermal oscillations—not analogous but identical.
\textbf{Middle Left - T-S relationship from computation:} Derived entropy (dimensionless, range 2.6-3.3) versus derived temperature (range 170-220). Blue circles: computed values from trajectory statistics. Red dashed curve: fit to $S \sim \ln(T)$. Scatter shows thermal fluctuations. This relationship is DERIVED from computation, not assumed. Temperature and entropy emerge simultaneously from bounded trajectory dynamics.
\textbf{Middle Center - State occupancy equals Boltzmann distribution:} Occupancy (count, range 0-300) versus categorical state/energy level (0-25). Green bars: computed occupancy from categorical operations. Red dashed curve: Maxwell-Boltzmann prediction $\exp(-E/k_B T)$. Perfect agreement demonstrates that categorical occupancy statistics automatically yield Boltzmann distribution. No statistical mechanics postulates required—Boltzmann distribution is a theorem about discrete state occupation.}
\label{fig:categorical_computing}
\end{figure}

\begin{figure}[htbp]
\centering
\includegraphics[width=\textwidth]{figures/exhaustive_computing_panel.png}
\caption{\textbf{Experimental validation of exhaustive computing properties in Poincaré Computing systems.} 
\textbf{(A)} Non-halting exploration: Memory density (fraction of phase space explored) asymptotically approaches unity but never reaches full exploration, demonstrating that the system continues indefinitely without a halting condition. 
\textbf{(B)} Capability monotonicity: Computational capability (measured as the number of distinct solution trajectories discovered) increases monotonically with time, establishing that the system can only improve through existence and never loses previously acquired capability. 
\textbf{(C)} Related problem acceleration: Acceleration factor (ratio of solution time for related problems to baseline problem) decreases as problem similarity increases (measured by distance $\delta$ in S-entropy space), confirming that prior exploration reduces complexity for nearby problems through conditional complexity reduction. 
\textbf{(D)} Progressive refinement: Complexity (measured in Poincaré units) decreases systematically across a sequence of related problems, with the "After" condition (following prior exploration) showing consistently lower complexity than the "Before" condition (no prior exploration), demonstrating irreversible capability accumulation. 
\textbf{(E)} Productive idleness: The number of distinct paths to a target solution increases continuously even during idle periods (no new problems introduced), establishing that exploration continues productively in the absence of external input and builds robustness through path redundancy. 
\textbf{(F)} Memory by existence: Trajectory visualization in a 2D projection of S-entropy space, with points colored by visit order, demonstrates that memory emerges from the exploration history without explicit storage---earlier visits (purple) cluster in certain regions while later visits (yellow) explore complementary regions, with the complete trajectory encoding the system's computational history.}
\label{fig:exhaustive_validation}
\end{figure}

\begin{figure}[htbp]
\centering
\includegraphics[width=\textwidth]{panel_hcna_N2.png}
\caption{Harmonic Coincidence Network Analyzer (HCNA) - N_2. 
\textbf{Top left:} 3D harmonic network structure where nodes represent oscillators and edges represent harmonic relationships. The network shows characteristic clustering with 3 nodes and 3 edges forming a minimal connected topology.
\textbf{Top center:} Degree distribution showing uniform connectivity (average degree = 2.00) across all nodes, indicating balanced harmonic coupling throughout the network.
\textbf{Top right:} Local clustering coefficient = 1.0 for all nodes, demonstrating perfect local connectivity characteristic of harmonic resonance networks.
\textbf{Bottom left:} Temperature extraction from network topology yielding T = 267 K (mean) with standard deviation 180 K. The 3D surface shows temperature variation across network coordinates, successfully encoding thermodynamic information in topological structure.
\textbf{Bottom center:} S-method temperature extraction showing excellent agreement across different molecular species (N_2, CO_2, H_2O) with consistent temperature determination around 267 K.
\textbf{Bottom right:} Multi-system network comparison showing nodes (blue), edges (green), and clustering$\times$10 (orange) across different gas systems. N_2 shows optimal balance with 3 nodes, 3 edges, and clustering coefficient 10.}
\label{fig:hcna_success}
\end{figure}

\begin{figure}[htbp]
\centering
\includegraphics[width=\textwidth]{panel_iglt_N2.png}
\caption{Ideal Gas Law Triangulator (IGLT) - N_2. 
\textbf{Top left:} 3D PVT surface showing perfect ideal gas behavior PV = NkT across temperature range 200-1000 K and pressure range 0.5-4.0 atm.
\textbf{Top center:} Triple derivation validation showing categorical (blue), oscillatory (red dashed), and partition (green dotted) methods all yielding identical PV = NkT relationships. All three lines overlap perfectly, confirming theoretical consistency.
\textbf{Top right:} Inter-method agreement analysis showing deviations < $10^{-13}$\% between all three derivation methods, far below both 0.3\% and 0.01\% thresholds. This represents essentially perfect numerical agreement.
\textbf{Bottom left:} Compressibility factor Z = 1.00 $\pm$ 0.02 across all conditions, confirming ideal gas behavior. Comparison with van der Waals deviations shows categorical method maintains ideality.
\textbf{Bottom center:} Real gas deviations at 300 K showing minimal departure from ideality for N_2, with Z remaining within 2\% of unity even at high densities.
\textbf{Bottom right:} Multi-system validation across H_2, N_2, CO_2 showing larger molecules exhibit greater deviations from ideality, as expected from molecular size effects.}
\label{fig:iglt_success}
\end{figure}

\begin{figure}[htbp]
\centering
\includegraphics[width=\textwidth]{panel_mrt_22L.png}
\caption{Maxwell Relations Tester: Categorical Thermodynamics Validation. 
\textbf{Top row:} Maxwell relations 1, 2, and 3 showing perfect agreement between reciprocal derivatives:
- \textbf{Relation 1:} 
$$\left(\frac{\partial T}{\partial V}\right)_S = -\left(\frac{\partial P}{\partial S}\right)_V$$
with identical slopes
- \textbf{Relation 2:} 
$$\left(\frac{\partial S}{\partial V}\right)_T = \left(\frac{\partial P}{\partial T}\right)_V$$
with coefficient 7.31$\times$$10^{13}$ Pa/K$^2$
- \textbf{Relation 3:} 
$$\left(\frac{\partial S}{\partial P}\right)_T = -\left(\frac{\partial V}{\partial T}\right)_P$$
showing perfect reciprocal symmetry
\textbf{Bottom left:} Maxwell relation 4: 
$$\left(\frac{\partial T}{\partial P}\right)_S = \left(\frac{\partial V}{\partial S}\right)_P$$
maintaining constant value 0.00108 across temperature range, confirming thermodynamic consistency.
\textbf{Bottom center:} 3D deviation surface for relation 2 showing deviations < $10^{-7}$ across entire (T,V) parameter space, demonstrating numerical precision of categorical thermodynamics.
\textbf{Bottom right:} Triple equivalence of entropy showing categorical (green), oscillatory (blue), and partition (purple) methods yielding identical entropy values across 200-1000 K temperature range.}
\label{fig:maxwell_success}
\end{figure}

\begin{figure}[htbp]
\centering
\includegraphics[width=\textwidth]{panel_prm_N100.png}
\caption{Poincar\'{e} Recurrence Monitor: N=100 particles, T=300.0 K. 
\textbf{Top left:} Continuous phase space distance showing fluctuations around 0.4 with epsilon threshold at 0.3 (red dashed line). The system maintains stable distance from initial state over 5000 time steps.
\textbf{Top right:} Categorical phase space distance exhibiting characteristic oscillations around 0.9 with epsilon threshold at 0.3. The categorical distance shows more structured behavior than continuous phase space.
\textbf{Top right (3D):} S-entropy trajectory in 3D categorical space showing systematic evolution through knowledge (S_k), temporal (S_t), and evolutionary (S_e) entropy coordinates. The trajectory demonstrates directional entropy evolution with characteristic clustering patterns.
\textbf{Bottom left:} Distance distribution comparing continuous (blue) and categorical (green) phase space metrics. Continuous distances peak around 0.4, while categorical distances show broader distribution around 0.8-0.9, with epsilon threshold clearly separating the regimes.
\textbf{Bottom center:} Recurrence count over 5000 steps showing 3 recurrences in continuous space vs 1 recurrence in categorical space, demonstrating that categorical phase space has longer recurrence times due to its higher-dimensional structure.
\textbf{Bottom right:} Recurrence time scaling with system size showing exponential growth characteristic of Poincar\'{e} recurrence theorem. For N=100 system, recurrence time $\approx$ $10^{21}$ time units, confirming the fundamental irreversibility of large systems.}
\label{fig:poincare_success}
\end{figure}

\begin{figure}[htbp]
\centering
\includegraphics[width=\textwidth]{panel_ccv_H2O.png}
\caption{Clausius-Clapeyron Verifier: H_2O
\textbf{Top left:} H_2O phase diagram showing vapor pressure curve with triple point at T = 273.16 K, P = 611.7 Pa. The categorical approach successfully reproduces the classical phase boundary across the temperature range 280-360 K.
\textbf{Top center:} Clausius-Clapeyron slope validation comparing classical (green dashed), categorical (blue), and experimental (red dotted) dP/dT values. All three methods show excellent agreement, with categorical predictions matching classical thermodynamics within experimental uncertainty.
\textbf{Top right:} Deviation from experimental dP/dT showing categorical method maintains < 5\% deviation across most of the temperature range, with perfect agreement around 360 K where deviation approaches zero.
\textbf{Bottom left:} Triple point phase coexistence in 3D showing solid (blue), liquid (green), and gas (red) phases meeting at the triple point. The 3D surface demonstrates proper phase relationships with characteristic entropy differences between phases.
\textbf{Bottom center:} Entropy vs temperature showing distinct values for solid ($\sim$200 J/mol$\cdot$K), liquid ($\sim$250 J/mol$\cdot$K), and gas ($\sim$1750 J/mol$\cdot$K) phases. The entropy jumps at phase transitions correspond to latent heat values: $\Delta$H_{fus} = 6.01 kJ/mol, $\Delta$H_{vap} = 40.70 kJ/mol.
\textbf{Bottom right - Validation summary:} \textbf{PASS} - dP/dT from categorical entropy agrees with classical thermodynamics. Key equation 
$$\frac{dP}{dT} = \frac{\Delta S}{\Delta V} = \frac{L}{T \cdot \Delta V}$$
verified, confirming that categorical entropy correctly predicts phase transition slopes through the fundamental Clausius-Clapeyron relation.}
\label{fig:clausius_success}
\end{figure}

\begin{figure}[htbp]
\centering
\includegraphics[width=\textwidth]{panel_etpv_N2.png}
\caption{Entropy Triple-Point Validator (ETPV) - N_2 
\textbf{Top left:} Phase diagram in S-space showing solid (blue), liquid (green), and gas (red) phases with triple point marked by black star. The 3D representation demonstrates phase coexistence in categorical entropy coordinates.
\textbf{Top center:} Triple equivalence validation at triple point showing perfect agreement: S_{categorical} = S_{oscillatory} = S_{partition}. All three entropy calculation methods yield identical values ($\sim$35 J/mol$\cdot$K for solid, $\sim$45 J/mol$\cdot$K for liquid, $\sim$120 J/mol$\cdot$K for gas), confirming theoretical consistency.
\textbf{Top right:} Phase transition entropies comparing calculated (blue) vs experimental (orange) values. Fusion entropy $\Delta$S_{fus} $\approx$ 11 J/mol$\cdot$K and vaporization entropy $\Delta$S_{vap} $\approx$ 72 J/mol$\cdot$K show excellent experimental agreement.
\textbf{Bottom left:} S(T) for each phase showing temperature-dependent entropy evolution. The curves demonstrate proper thermodynamic behavior with entropy increasing with temperature and distinct jumps at phase transitions (T_{triple} = 63.1 K).
\textbf{Bottom center:} S(T) across phases showing continuous entropy evolution through solid $\rightarrow$ liquid $\rightarrow$ gas transitions. The smooth curves with discontinuous derivatives at phase boundaries confirm proper first-order phase transition behavior.
\textbf{Bottom right:} Multi-system transition entropies comparing H_2O, CO_2, N_2, and Ar. The systematic variation with molecular complexity (H_2O > CO_2 > N_2 > Ar) demonstrates universal applicability of the categorical entropy framework.
\textbf{Validation: PASS} - $\Delta$S_{fus} deviation: 0.0\%, $\Delta$S_{vap} deviation: 0.0\%. Triple equivalence at phase transitions verified, confirming that all three categorical entropy methods are thermodynamically equivalent.}
\label{fig:entropy_validator_success}
\end{figure}

\begin{figure}[htbp]
\centering
\includegraphics[width=\textwidth]{panel_sldi.png}
\caption{Speed of Light Derivation Instrument (SLDI) 
\textbf{Top left:} Container expansion experiment showing double-cone phase space structure. As container expands, faster molecular velocities are required to maintain equilibrium, leading to fundamental velocity limits.
\textbf{Top center:} Velocity requirement vs container size showing classical approach (blue) has no limit while categorical approach (red) saturates at c = 2.998$\times$$10^8$ m/s. The forbidden region (shaded) represents velocities exceeding categorical transition rates.
\textbf{Top right:} Transition rate saturation at c showing normalized categorical transition rate approaches unity as v/c $\rightarrow$ 1, then becomes impossible (rate = 0) for v > c. This creates absolute velocity limit.
\textbf{Bottom left:} Phase space of categorical limits showing critical volume ratio vs temperature and thermal velocity. The surface defines the boundary where categorical constraints become dominant.
\textbf{Bottom center:} \textbf{Logical derivation of c from categorical principles:} (1) Bounded system premise: gas in container at equilibrium with thermal velocity v_{th}; (2) Container expansion: volume V $\rightarrow$ $\alpha^3$V requires velocity v $\rightarrow$ $\alpha^{1/3}$V; (3) Categorical constraint: categories transition at finite maximum rate; (4) Derivation: as $\alpha$ $\rightarrow$ $\infty$, classical physics requires v $\rightarrow$ $\infty$, but categorical transitions have maximum rate; (5) Result: c emerges as categorical necessity, not measured constant.
\textbf{Bottom right:} Lighter molecules reach c limit at smaller expansion, but all converge to same c value. Mass dependence shows universal speed limit independent of particle type.
\textbf{DERIVATION VERIFIED}: c = 2.998$\times$$10^8$ m/s emerges as categorical maximum. Speed of light is not arbitrary but necessary consequence of categorical transition rate limits. Special relativity follows from categorical structure.}
\label{fig:speed_light_success}
\end{figure}

\begin{figure}[htbp]
\centering
\includegraphics[width=\textwidth]{panel_ternary_computation_1.png}
\caption{Ternary Representation for Gas Dynamics: S-Entropy Compression. 
\textbf{Top left:} Full phase space (200 molecules) showing 3D molecular positions and velocities compressed from 18-dimensional space into categorical coordinates. Each point represents one molecule with complete phase space information encoded in ternary addresses.
\textbf{Top center:} S-Entropy compression demonstration showing dimensional reduction from 18 dimensions (x, y, z, v_x, v_y, v_z for each molecule) to 3 S-entropy coordinates: S_k (knowledge), S_t (temporal), S_e (evolutionary). Each molecule maps to unique point in categorical space.
\textbf{Top right:} Ternary addresses (3$^k$ hierarchy) showing base-3 encoding where each trit position corresponds to depth in categorical tree. Color coding: 0 = Oscillatory (blue), 1 = Categorical (red), 2 = Partition (yellow). Maximum depth = 10 trits provides 3$^{10}$ = 59,049 unique addresses.
\textbf{Bottom left:} Sliding window spectrometer tracking S_k (knowledge, yellow), S_t (time, cyan), S_e (evolution, red) entropy components across 30 time windows. The oscillatory behavior demonstrates dynamic categorical transitions in real-time molecular evolution.
\textbf{Bottom center:} 3$^k$ ternary address tree showing hierarchical structure where each node branches into 3 sub-categories. The tree depth corresponds to measurement precision, with deeper levels providing finer categorical resolution.
\textbf{Bottom right - Key insight:} \textbf{Oscillator = Processor}: Each molecular oscillator functions as a computational processor where gas dynamics solving is equivalent to running ternary programs. Memory addresses correspond to trajectories in S-space, establishing fundamental equivalence between thermodynamic evolution and categorical computation.
\textbf{Validation: PASS} - Complete dimensional compression achieved: 18D $\rightarrow$ 3D with perfect information preservation through ternary encoding.}
\label{fig:ternary_compression_success}
\end{figure}

\begin{figure}[htbp]
\centering
\includegraphics[width=\textwidth]{figures/comprehensive_validation.png}
\caption{Comprehensive validation of spectroscopic measurement framework against synthetic test data. \textbf{Top row:} Peak detection performance (mean F1 = 0.055), spectral correlation distribution (mean = 0.027), RMSE distribution (mean = 0.435), and LED wavelength response validation. \textbf{Middle row:} Four representative spectral comparisons between real (blue) and virtual (red dashed) measurements showing systematic discrepancies. \textbf{Bottom row:} Peak count comparison, correlation vs RMSE scatter plot, and overall performance metrics. The low correlation and high RMSE indicate that the virtual measurement model does not accurately reproduce real spectroscopic data, suggesting fundamental differences between the theoretical framework and physical implementation.}
\label{fig:comprehensive_validation}
\end{figure}

% Figure 2: Zeeman Orientation Coordinate
\begin{figure}[htbp]
\centering
\includegraphics[width=\textwidth]{figures/panel_zeeman_orientation_coordinate.png}
\caption{Orientation coordinate $m$ and Zeeman spectroscopy coupling structure. \textbf{Top row:} $m$-state energy distribution for $\ell=3$, Zeeman splitting in external magnetic field $\mathbf{B}$, and Larmor precession geometry. \textbf{Middle row:} Selection rules $\Delta m = 0, \pm 1$, normal Zeeman triplet ($\sigma^-$, $\pi$, $\sigma^+$ transitions), phase pattern $\text{Re}(e^{im\phi})$, and space quantization for $\ell=2$. \textbf{Bottom row:} Light polarization components, circular polarization helices, microwave cavity TE$_{11}$ mode structure, and Zeeman frequency dependence $\omega_m \propto m \cdot B$. The coupling structure $\mathcal{I}_m$ implements magnetic field gradient coupling in regime $\Omega_m$, corresponding to magnetic resonance spectroscopy (Theorem~\ref{thm:orientation_coupling}).}
\label{fig:zeeman_orientation}
\end{figure}

% Figure 3: Partition Coordinate Validation
\begin{figure}[htbp]
\centering
\includegraphics[width=\textwidth]{figures/partition_coordinate_validation.png}
\caption{Validation of partition coordinate structure and spectroscopic predictions. \textbf{Top row:} Capacity theorem $2n^2$ verification (280 states), frequency regime separation showing $10\times$ gaps between $\Omega_n$, $\Omega_\ell$, $\Omega_m$, $\Omega_s$, selection rules (6.0\% allowed transitions), and Lorentzian resonance profile. \textbf{Middle row:} Off-resonance suppression following $(\Gamma/\Delta)^2$ (correlation 0.9999), coordinate selectivity with $S > 100$ for $s$-coordinate, energy ordering matching $n + 0.7\ell$ scaling, and molecular $n$-distribution. \textbf{Bottom row:} Selection rule violation counts and shell closure points. The validation summary confirms: capacity theorem passed, well-separated regimes, selection rules with 6.0\% allowed fraction, and resonance theory matching with 0.9999 correlation. Physical correspondences map $(n,\ell,m,s)$ to quantum numbers and spectroscopic techniques as predicted by Theorems~\ref{thm:partition_structure}--\ref{thm:frequency_duality}.}
\label{fig:partition_validation}
\end{figure}

% Figure 4: Unified Spectroscopy Framework
\begin{figure}[htbp]
\centering
\includegraphics[width=\textwidth]{figures/panel_unified_spectroscopy.png}
\caption{Unified spectroscopic framework showing correspondence between partition coordinates $(n,\ell,m,s)$ and measurement techniques. \textbf{Top:} Frequency regime separation spanning radio to X-ray frequencies ($10^6$--$10^{18}$ Hz), with each coordinate occupying a distinct spectral regime separated by factors $>10^3$ (Theorem~\ref{thm:frequency_duality}). \textbf{Middle:} Geometric representations of each coordinate: depth $n$ (shell capacity $2n^2$), complexity $\ell$ (angular degeneracy), orientation $m$ (Zeeman levels and Larmor precession), and chirality $s$ (Bloch sphere relaxation). \textbf{Bottom table:} Summary of coordinate-instrument correspondences, showing frequency scaling ($\omega_n \propto n^{-3}$, $\omega_\ell \propto \ell(\ell+1)$, $\omega_m \propto m \cdot B$, $\omega_s \propto s \cdot B$), physical coupling mechanisms, and spectroscopic implementations. The coordinate relationship diagram (right) illustrates the hierarchical structure connecting all four measurements through the partition structure $\mathcal{P}$.}
\label{fig:unified_spectroscopy}
\end{figure}

% Figure 5: Complexity Coordinate (UV-Vis)
\begin{figure}[htbp]
\centering
\includegraphics[width=\textwidth]{figures/panel_uvvis_complexity_coordinate.png}
\caption{Complexity coordinate $\ell$ and UV-visible optical spectroscopy. \textbf{Top row:} Orbital shapes for $\ell=2$ (d-orbital) and $\ell=3$ (f-orbital), selection rule matrix showing allowed transitions $\Delta\ell = \pm 1$ (6.0\% of all pairs, green squares), UV-visible absorption spectrum with vibronic structure, and Jablonski diagram showing electronic transitions. \textbf{Middle row:} Orbital characteristics radar plot (radial extent, angular momentum, shielding, nodes, energy, degeneracy), frequency scaling $\omega_\ell \propto \ell(\ell+1)$ with numerical values, transition dipole moment vectors in 3D, and oscillator strengths for $s \to p$ (0.876), $p \to d$ (0.122), $d \to f$ (0.637) transitions. \textbf{Bottom row:} Degeneracy pattern $2\ell+1$ showing cumulative state counts. The coupling structure $\mathcal{I}_\ell$ implements electric dipole coupling in the optical regime $\Omega_\ell$, corresponding to UV-visible and Raman spectroscopy (Theorem~\ref{thm:complexity_coupling}).}
\label{fig:complexity_uvvis}
\end{figure}

% Figure 6: Depth Coordinate (XPS)
\begin{figure}[htbp]
\centering
\includegraphics[width=\textwidth]{figures/panel_xps_depth_coordinate.png}
\caption{Depth coordinate $n$ and X-ray photoelectron spectroscopy (XPS). \textbf{Top row:} Core-level binding energy surface showing $E_n \propto -n^{-2}$ scaling, radial probability distributions for $n=1$ through $n=5$ states with characteristic nodal structure, and shell capacity polar plot confirming $2n^2$ degeneracy (Theorem~\ref{thm:capacity}). \textbf{Middle row:} XPS kinetic energies for Fe shells (1s through 3p) at photon energy $h\nu = 1500$ eV, and Auger transition probability matrix showing cascade processes between shells. \textbf{Bottom row:} Electron shell isosurfaces for $n=1,2,3$ showing nested boundary structure, XPS survey spectrum of Fe with characteristic core-level peaks, and photoionization cross-section scaling as $\sigma_n \propto n^{-3}$ (red points) matching the frequency-coordinate duality prediction $\omega_n \propto n^{-3}$ (Theorem~\ref{thm:frequency_duality}). The coupling structure $\mathcal{I}_n$ implements high-frequency selective coupling in regime $\Omega_n$, corresponding to X-ray spectroscopy (Theorem~\ref{thm:depth_coupling}).}
\label{fig:depth_xps}
\end{figure}

\begin{figure}[htbp]
\centering
\includegraphics[width=\textwidth]{figures/panel_nmr_chirality_coordinate.png}
\caption{Chirality coordinate $s$ and nuclear magnetic resonance (NMR) spectroscopy. \textbf{Top row:} Bloch sphere representation of spin states $|\uparrow\rangle$ and $|\downarrow\rangle$, Zeeman energy splitting $\Delta E = \gamma \hbar B$ linear in magnetic field, Boltzmann spin population distribution at various temperatures (100--500 K), and $^1$H NMR spectrum showing chemical shift peaks for different molecular environments. \textbf{Middle row:} NMR relaxation curves for longitudinal ($T_1 = 1.0$ s, blue) and transverse ($T_2 = 0.5$ s, red) magnetization, free induction decay (FID) signal with exponential envelope, spin echo pulse sequence (90°--180°--acquisition), and tissue-dependent NMR properties radar plot (water, fat, brain) showing $T_1$, $T_2$, $T_2^*$, chemical shift, and J-coupling variations. \textbf{Bottom row:} 2D COSY correlation map showing through-bond connectivity, J-coupling multiplet patterns (singlet, doublet, triplet, quartet), Larmor frequency $\omega = \gamma B$ for different nuclei ($^1$H, $^{13}$C, $^{19}$F, $^{31}$P), and two-spin energy level diagram. The coupling structure $\mathcal{I}_s$ implements radio-frequency magnetic resonance at the Larmor frequency in regime $\Omega_s$, corresponding to NMR and ESR spectroscopy (Theorem~\ref{thm:chirality_resonance}).}
\label{fig:chirality_nmr}
\end{figure}
