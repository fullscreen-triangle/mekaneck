\section{Autocatalytic Charge Redistribution in Closed Systems}
\label{sec:autocatalytic}

We establish that charge redistribution in closed systems (no external reservoir) exhibits autocatalytic dynamics with perpetual oscillation arising from thermodynamic necessity.

\subsection{Charge Conservation Constraint}

In closed hybrid microfluidic circuits, total charge is conserved:
\begin{equation}
Q_{\text{total}} = \int_V \rho(\mathbf{r},t) \, d^3r = \text{const}
\label{eq:charge_conservation_closed}
\end{equation}

\begin{axiom}[No External Reservoir]
\label{ax:no_ground}
The system has no access to external charge reservoirs (ground). All charge redistribution must occur within the bounded volume $V$.
\end{axiom}

This constraint distinguishes closed circuits from open systems where charge can dissipate to ground.

\subsection{Variance Minimization Principle}

\begin{principle}[Charge Variance Minimization]
\label{prin:variance_min}
Systems minimize charge distribution variance:
\begin{equation}
\sigma^2[\rho] = \frac{1}{V} \int_V (\rho(\mathbf{r},t) - \langle \rho \rangle)^2 \, d^3r \to \min
\end{equation}
where $\langle \rho \rangle = Q_{\text{total}} / V$ is the mean charge density.
\end{principle}

Variance minimization drives charge redistribution from high-density to low-density regions through current flux:
\begin{equation}
\mathbf{J} = -\sigma \nabla \phi = -\sigma \nabla \left( \frac{\delta F}{\delta \rho} \right)
\end{equation}
where $\sigma$ is conductivity and $F[\rho]$ is the free energy functional.

\subsection{The Autocatalytic Cycle}

\begin{theorem}[Autocatalytic Charge Redistribution]
\label{thm:autocatalytic}
In closed systems, charge redistribution is autocatalytic: local imbalance triggers redistribution, which creates new imbalance, triggering further redistribution. Formally:
\begin{equation}
\Delta \rho_1(\mathbf{r}_1, t_1) \to \mathbf{J}_{1 \to 2} \to \Delta \rho_2(\mathbf{r}_2, t_2) \to \mathbf{J}_{2 \to 3} \to \cdots
\end{equation}
This cycle is self-sustaining and perpetual in closed systems.
\end{theorem}

\begin{proof}
Consider local charge imbalance $\Delta \rho(\mathbf{r}_1, t_1) = \rho(\mathbf{r}_1, t_1) - \langle \rho \rangle > 0$ at position $\mathbf{r}_1$.

\textbf{Step 1 (Variance increase):} The imbalance increases system variance:
\begin{equation}
\sigma^2[\rho](t_1) > \sigma^2[\rho](t_0)
\end{equation}

\textbf{Step 2 (Thermodynamic drive):} By Principle~\ref{prin:variance_min}, the system responds by minimizing variance through charge redistribution.

\textbf{Step 3 (Compensatory redistribution):} Charge flows from $\mathbf{r}_1$ (high density) to region $\mathbf{r}_2$ (low density):
\begin{equation}
\mathbf{J}_{1 \to 2} = -\sigma \nabla \phi \quad \text{with} \quad \nabla \cdot \mathbf{J} = -\frac{\partial \rho}{\partial t}
\end{equation}

\textbf{Step 4 (New imbalance created):} By charge conservation (Equation~\ref{eq:charge_conservation_closed}), charge leaving $\mathbf{r}_1$ must accumulate at $\mathbf{r}_2$. This creates new imbalance:
\begin{equation}
\Delta \rho(\mathbf{r}_2, t_2) = \rho(\mathbf{r}_2, t_2) - \langle \rho \rangle > 0
\end{equation}

\textbf{Step 5 (Cycle continuation):} The new imbalance at $\mathbf{r}_2$ triggers variance minimization, driving redistribution to $\mathbf{r}_3$, and so on.

\textbf{Perpetuity:} In closed systems (no ground), charge cannot dissipate externally. Every redistribution creates compensatory imbalance elsewhere. The cycle continues indefinitely.
\end{proof}

\subsection{Oscillatory Dynamics}

\begin{corollary}[Perpetual Oscillation]
\label{cor:perpetual_oscillation}
Closed charge-coupled circuits exhibit perpetual oscillation around equilibrium charge distribution. The equilibrium $\nabla \rho = 0$ (uniform distribution) is never reached.
\end{corollary}

\begin{proof}
The equilibrium condition is uniform charge distribution:
\begin{equation}
\rho_{\text{eq}}(\mathbf{r}) = \langle \rho \rangle = \frac{Q_{\text{total}}}{V} \quad \forall \mathbf{r} \in V
\end{equation}

At equilibrium, variance is minimized: $\sigma^2[\rho_{\text{eq}}] = 0$.

However, reaching this equilibrium requires:
\begin{enumerate}
\item Simultaneous charge redistribution across all regions
\item Infinite precision in charge placement
\item Zero thermal fluctuations
\end{enumerate}

In physical systems, thermal fluctuations continuously perturb charge distribution:
\begin{equation}
\rho(\mathbf{r}, t) = \rho_{\text{eq}} + \delta \rho(\mathbf{r}, t)
\end{equation}
where $\delta \rho$ represents fluctuations with $\langle \delta \rho \rangle = 0$ but $\langle (\delta \rho)^2 \rangle > 0$.

These fluctuations trigger variance minimization responses, which create compensatory imbalances (Theorem~\ref{thm:autocatalytic}). The system oscillates around equilibrium without reaching it.

The oscillation is not force-driven (no external forcing) but arises from categorical necessity: charge must occupy some configuration at all times, and with bounded phase space, systems cycle through accessible configurations.
\end{proof}

\subsection{Timescales and Frequencies}

The autocatalytic cycle operates on characteristic timescales determined by circuit parameters.

\begin{definition}[Redistribution Timescale]
\label{def:redistribution_time}
The charge redistribution timescale is:
\begin{equation}
\tau_{\text{redist}} = \frac{L^2}{\sigma / \epsilon}
\end{equation}
where $L$ is the characteristic length scale, $\sigma$ is conductivity, and $\epsilon$ is permittivity.
\end{definition}

This is the RC time constant for charge redistribution across distance $L$.

\begin{definition}[Oscillation Frequency]
\label{def:oscillation_freq}
The characteristic oscillation frequency is:
\begin{equation}
\omega_{\text{osc}} = \frac{1}{\tau_{\text{redist}}} = \frac{\sigma}{\epsilon L^2}
\end{equation}
\end{definition}

For typical hybrid microfluidic circuits with $L \sim 10^{-3}$ m, $\sigma \sim 10^{-2}$ S/m, $\epsilon \sim 10^{-10}$ F/m:
\begin{equation}
\omega_{\text{osc}} \sim \frac{10^{-2}}{10^{-10} \cdot 10^{-6}} = 10^{14} \text{ rad/s} \sim 10^{13} \text{ Hz}
\end{equation}

This corresponds to infrared frequencies, consistent with thermal oscillations.

\subsection{Energy Landscape}

The autocatalytic cycle can be visualized as dynamics on a free energy landscape.

\begin{definition}[Charge Distribution Free Energy]
\label{def:free_energy}
The free energy functional for charge distribution is:
\begin{equation}
F[\rho] = \int_V \left[ f(\rho(\mathbf{r})) + \frac{\epsilon}{2} |\nabla \phi|^2 \right] d^3r
\end{equation}
where $f(\rho)$ is the local free energy density and $\phi$ satisfies $\nabla \cdot (\epsilon \nabla \phi) = -\rho$.
\end{definition}

\begin{theorem}[Free Energy Minimization Drives Redistribution]
\label{thm:free_energy_min}
Charge redistribution follows the gradient flow:
\begin{equation}
\frac{\partial \rho}{\partial t} = \nabla \cdot \left( \sigma \nabla \frac{\delta F}{\delta \rho} \right)
\end{equation}
This is a dissipative dynamics that decreases free energy: $dF/dt \leq 0$.
\end{theorem}

\begin{proof}
The variational derivative of free energy with respect to charge density is:
\begin{equation}
\frac{\delta F}{\delta \rho} = f'(\rho) + \phi
\end{equation}

The charge current is:
\begin{equation}
\mathbf{J} = -\sigma \nabla \left( \frac{\delta F}{\delta \rho} \right)
\end{equation}

By continuity equation:
\begin{equation}
\frac{\partial \rho}{\partial t} = -\nabla \cdot \mathbf{J} = \nabla \cdot \left( \sigma \nabla \frac{\delta F}{\delta \rho} \right)
\end{equation}

The free energy evolution is:
\begin{equation}
\frac{dF}{dt} = \int_V \frac{\delta F}{\delta \rho} \frac{\partial \rho}{\partial t} d^3r = -\int_V \sigma \left| \nabla \frac{\delta F}{\delta \rho} \right|^2 d^3r \leq 0
\end{equation}

Free energy decreases monotonically until a local minimum is reached.
\end{proof}

\begin{remark}[Multiple Minima]
The free energy landscape typically has multiple local minima corresponding to different charge configurations. The autocatalytic cycle causes the system to transition between these minima, never settling at a single equilibrium.
\end{remark}

\subsection{Comparison with Open Systems}

\begin{theorem}[Closed vs. Open System Dynamics]
\label{thm:closed_vs_open}
Closed and open systems exhibit qualitatively different dynamics:

\textbf{Closed system (no ground):}
\begin{itemize}
\item $Q = \text{const}$ (charge conserved)
\item Perpetual oscillation
\item $\sigma^2[\rho] > 0$ (non-zero variance maintained)
\item Equilibrium never reached
\end{itemize}

\textbf{Open system (ground available):}
\begin{itemize}
\item $Q \to Q_{\text{reservoir}}$ (charge dissipates)
\item Dynamics cease
\item $\sigma^2[\rho] \to 0$ (variance vanishes)
\item Equilibrium reached
\end{itemize}
\end{theorem}

\begin{proof}
In open systems with ground access, charge can dissipate to external reservoir:
\begin{equation}
\frac{dQ}{dt} = -\gamma (Q - Q_{\text{reservoir}})
\end{equation}

This enables static equilibrium: charge flows to ground until system reaches reservoir potential. Dynamics cease when $Q = Q_{\text{reservoir}}$.

In closed systems, no such dissipation pathway exists. Charge must redistribute internally, creating perpetual oscillation through the autocatalytic mechanism (Theorem~\ref{thm:autocatalytic}).
\end{proof}

\subsection{Experimental Signatures}

The autocatalytic charge redistribution has observable signatures:

\begin{enumerate}
\item \textbf{Persistent oscillations:} Closed circuits exhibit oscillations that do not decay over time, unlike damped oscillations in open systems.

\item \textbf{Frequency spectrum:} The oscillation frequency $\omega_{\text{osc}} = \sigma / (\epsilon L^2)$ depends on circuit geometry and material properties, providing a diagnostic.

\item \textbf{Variance persistence:} Charge distribution variance $\sigma^2[\rho]$ remains non-zero indefinitely in closed systems, while it decays to zero in open systems.

\item \textbf{Phase coherence:} Oscillations across different regions maintain phase coherence $R > 0.8$ in closed systems, indicating synchronized autocatalytic dynamics.
\end{enumerate}

These signatures distinguish autocatalytic redistribution from externally driven oscillations or transient responses.
