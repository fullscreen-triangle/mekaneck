\section{Multi-Circuit Systems with External Coupling}
\label{sec:multi_circuit}

We analyze charge distribution dynamics in multi-circuit systems where a primary circuit couples to an ensemble of external circuits with distinct architectures.

\subsection{System Architecture}

Consider a multi-circuit system:
\begin{equation}
\mathcal{S} = \mathcal{C}_{\text{primary}} \cup \bigcup_{i=1}^{N} \mathcal{C}_{\text{ext},i}
\end{equation}
where:
\begin{itemize}
\item $\mathcal{C}_{\text{primary}}$ is the primary circuit with architecture type $\alpha$
\item $\{\mathcal{C}_{\text{ext},i}\}_{i=1}^{N}$ are external circuits with architecture type $\beta \neq \alpha$
\end{itemize}

\begin{definition}[Circuit Architecture]
\label{def:architecture}
The \emph{architecture} of a circuit is its internal structural organization: component arrangement, connection topology, and geometric configuration. Two circuits have different architectures if their internal structures differ.
\end{definition}

\subsection{Charge Distribution Continuity}

\begin{theorem}[Continuous Charge Distribution Across Boundaries]
\label{thm:charge_continuity}
The total charge distribution is continuous across circuit boundaries:
\begin{equation}
\rho_{\text{total}}(\mathbf{r},t) = \rho_{\text{primary}}(\mathbf{r},t) + \sum_{i=1}^{N} \rho_{\text{ext},i}(\mathbf{r},t)
\end{equation}
with no discontinuity at interfaces $\partial \mathcal{C}_{\text{primary}} \cap \partial \mathcal{C}_{\text{ext},i}$.
\end{theorem}

\begin{proof}
Charge conservation requires:
\begin{equation}
\nabla \cdot \mathbf{J} = -\frac{\partial \rho}{\partial t}
\end{equation}

At circuit boundaries, current continuity demands:
\begin{equation}
\mathbf{J}_{\text{primary}} \cdot \hat{\mathbf{n}} = \mathbf{J}_{\text{ext},i} \cdot \hat{\mathbf{n}}
\end{equation}
where $\hat{\mathbf{n}}$ is the interface normal.

This continuity condition ensures that charge does not accumulate or deplete at boundaries. The charge distribution $\rho(\mathbf{r},t)$ is continuous across interfaces, though its gradient $\nabla \rho$ may be discontinuous (reflecting different conductivities).

The total charge distribution spans both primary and external circuits as a single continuous field, regardless of architectural differences.
\end{proof}

\subsection{State Equivalence Principle}

\begin{principle}[Functional Emergence from Context]
\label{prin:state_equivalence}
Two circuits with different architectures exhibit equivalent functional behavior if they maintain the same local charge balance:
\begin{equation}
\langle \rho_A(\mathbf{r},t) \rangle_{\text{local}} = \langle \rho_B(\mathbf{r},t) \rangle_{\text{local}} \implies F_A = F_B
\end{equation}
where $F$ denotes functional output (flux, phase, frequency).
\end{principle}

\begin{theorem}[Architecture-Independent Function]
\label{thm:architecture_independent}
Functional behavior emerges from local charge balance (context) rather than from circuit architecture (intrinsic structure).
\end{theorem}

\begin{proof}
Consider two circuits $\mathcal{C}_A$ and $\mathcal{C}_B$ with different architectures (different internal structures) but same local charge balance:
\begin{equation}
\langle \rho_A(\mathbf{r},t) \rangle_{\text{local}} = \langle \rho_B(\mathbf{r},t) \rangle_{\text{local}} = \rho_0
\end{equation}

The functional output (e.g., current flux) depends on local charge gradient:
\begin{equation}
\mathbf{J} = -\sigma \nabla \phi \propto \nabla \rho
\end{equation}

If local charge balance matches ($\rho_A = \rho_B$ locally), then local gradients match ($\nabla \rho_A = \nabla \rho_B$ locally), and thus functional outputs match ($\mathbf{J}_A = \mathbf{J}_B$ locally).

The internal architecture (how the circuit achieves this charge distribution) is irrelevant to the functional output. Only the charge distribution itself (the context) determines function.

This establishes that function emerges from context (charge balance) rather than from intrinsic identity (architecture).
\end{proof}

\subsection{Synchronization Across Architectures}

\begin{theorem}[Cross-Architecture Synchronization]
\label{thm:cross_sync}
External circuits with different architectures synchronize with primary circuit through phase-lock coupling when coupling strength exceeds frequency variance:
\begin{equation}
K_{\text{coupling}} > \sigma(\omega)
\end{equation}
where $K_{\text{coupling}}$ is coupling strength and $\sigma(\omega)$ is frequency variance.
\end{theorem}

\begin{proof}
The phase dynamics of coupled oscillators follow the Kuramoto model:
\begin{equation}
\frac{d\phi_i}{dt} = \omega_i + \frac{K_{\text{coupling}}}{N} \sum_{j=1}^{N} \sin(\phi_j - \phi_i)
\end{equation}
where $\omega_i$ is the natural frequency of oscillator $i$.

Synchronization occurs when coupling dominates frequency variance. The order parameter:
\begin{equation}
R = \frac{1}{N} \left| \sum_{j=1}^{N} e^{i\phi_j} \right|
\end{equation}
approaches unity ($R \to 1$) when:
\begin{equation}
K_{\text{coupling}} > K_c = \frac{2}{\pi} \sigma(\omega)
\end{equation}

For $K_{\text{coupling}} > K_c$, all oscillators (regardless of architecture) synchronize to common frequency and phase, achieving $R > 0.8$.

The architecture differences (different $\omega_i$ values) are overcome by strong coupling, establishing synchronized dynamics across architectural boundaries.
\end{proof}

\subsection{Functional Adoption}

\begin{theorem}[External Circuit Functional Adoption]
\label{thm:functional_adoption}
When external circuit $\mathcal{C}_{\text{ext}}$ couples to primary circuit $\mathcal{C}_{\text{primary}}$, the external circuit adopts the functional behavior of the primary circuit:
\begin{equation}
F_{\text{ext}}(t) \to F_{\text{primary}}(t) \quad \text{as} \quad t \to \tau_{\text{sync}}
\end{equation}
where $\tau_{\text{sync}}$ is the synchronization timescale.
\end{theorem}

\begin{proof}
The synchronization timescale is:
\begin{equation}
\tau_{\text{sync}} = \frac{1}{K_{\text{coupling}} - K_c}
\end{equation}

For $t < \tau_{\text{sync}}$, the external circuit operates at its natural frequency $\omega_{\text{ext}}$ with functional output $F_{\text{ext}}(0)$ determined by its architecture.

For $t > \tau_{\text{sync}}$, phase-lock coupling forces the external circuit to oscillate at the primary circuit frequency $\omega_{\text{primary}}$. By Theorem~\ref{thm:architecture_independent}, matching frequency and phase implies matching functional output:
\begin{equation}
\omega_{\text{ext}}(t) = \omega_{\text{primary}}(t) \implies F_{\text{ext}}(t) = F_{\text{primary}}(t)
\end{equation}

The external circuit has "adopted" the functional behavior of the primary circuit through charge balance matching, despite architectural differences.
\end{proof}

\begin{corollary}[Context Determines Function]
\label{cor:context_function}
Circuit function is not intrinsic property but emerges from charge distribution context (coupling to other circuits).
\end{corollary}

\begin{proof}
Consider external circuit $\mathcal{C}_{\text{ext}}$ in two contexts:

\textbf{Context 1 (isolated):} $\mathcal{C}_{\text{ext}}$ operates independently with natural frequency $\omega_{\text{ext}}$ and functional output $F_{\text{ext}}^{(1)}$.

\textbf{Context 2 (coupled to $\mathcal{C}_{\text{primary}}$):} $\mathcal{C}_{\text{ext}}$ synchronizes to $\omega_{\text{primary}}$ with functional output $F_{\text{ext}}^{(2)} = F_{\text{primary}}$.

The same circuit exhibits different functional behavior in different contexts. Function is context-dependent, not intrinsic.

This establishes that "circuit identity" (defined by architecture) does not determine function. Only charge distribution context (coupling configuration) determines function.
\end{proof}

\subsection{Charge Balance Matching}

\begin{definition}[Local Charge Balance]
\label{def:local_balance}
The \emph{local charge balance} at position $\mathbf{r}$ is:
\begin{equation}
\langle \rho \rangle_{\text{local}}(\mathbf{r}) = \frac{1}{V_{\text{local}}} \int_{V_{\text{local}}(\mathbf{r})} \rho(\mathbf{r}', t) \, d^3r'
\end{equation}
where $V_{\text{local}}(\mathbf{r})$ is a neighborhood around $\mathbf{r}$.
\end{equation}

\begin{theorem}[Synchronization Through Balance Matching]
\label{thm:balance_matching}
External circuits synchronize with primary circuit by matching local charge balance:
\begin{equation}
\langle \rho_{\text{ext},i} \rangle_{\text{local}} \to \langle \rho_{\text{primary}} \rangle_{\text{local}}
\end{equation}
through variance minimization dynamics.
\end{theorem}

\begin{proof}
The coupled system minimizes total variance:
\begin{equation}
\sigma^2_{\text{total}} = \int_V (\rho_{\text{total}}(\mathbf{r}) - \langle \rho_{\text{total}} \rangle)^2 d^3r
\end{equation}

Variance is minimized when local charge distributions match across circuits:
\begin{equation}
\rho_{\text{ext},i}(\mathbf{r}) \approx \rho_{\text{primary}}(\mathbf{r}) \quad \forall \mathbf{r}
\end{equation}

The dynamics drive external circuits toward this configuration through charge redistribution:
\begin{equation}
\frac{\partial \rho_{\text{ext},i}}{\partial t} = \nabla \cdot \left( \sigma_i \nabla \frac{\delta F_{\text{total}}}{\delta \rho_{\text{total}}} \right)
\end{equation}

The timescale for balance matching is:
\begin{equation}
\tau_{\text{balance}} = \frac{L^2}{\sigma_{\text{eff}} / \epsilon_{\text{eff}}}
\end{equation}

For $t > \tau_{\text{balance}}$, local charge balance matches, establishing functional equivalence (Theorem~\ref{thm:architecture_independent}).
\end{proof}

\subsection{Trajectory Alignment in Phase Space}

\begin{definition}[Phase Space Trajectory]
\label{def:phase_trajectory}
The phase space trajectory of circuit $\mathcal{C}$ is:
\begin{equation}
\gamma_{\mathcal{C}}(t) = (\rho_{\mathcal{C}}(t), \mathbf{J}_{\mathcal{C}}(t)) \in \mathbb{R}^{2d}
\end{equation}
where $d$ is the number of spatial degrees of freedom.
\end{definition}

\begin{theorem}[Trajectory Alignment Through Coupling]
\label{thm:trajectory_alignment}
External circuits with different architectures trace parallel trajectories in phase space when coupled to primary circuit:
\begin{equation}
\gamma_{\text{ext},i}(t) \parallel \gamma_{\text{primary}}(t) \quad \forall i
\end{equation}
\end{theorem}

\begin{proof}
The phase space velocity is:
\begin{equation}
\frac{d\gamma}{dt} = \left( \frac{\partial \rho}{\partial t}, \frac{\partial \mathbf{J}}{\partial t} \right)
\end{equation}

For coupled circuits with matched charge balance (Theorem~\ref{thm:balance_matching}):
\begin{equation}
\rho_{\text{ext},i}(t) \approx \rho_{\text{primary}}(t)
\end{equation}

The charge dynamics are:
\begin{equation}
\frac{\partial \rho}{\partial t} = -\nabla \cdot \mathbf{J} \propto \nabla^2 \rho
\end{equation}

If $\rho_{\text{ext},i} \approx \rho_{\text{primary}}$, then $\nabla^2 \rho_{\text{ext},i} \approx \nabla^2 \rho_{\text{primary}}$, and thus:
\begin{equation}
\frac{\partial \rho_{\text{ext},i}}{\partial t} \approx \frac{\partial \rho_{\text{primary}}}{\partial t}
\end{equation}

Similarly for current:
\begin{equation}
\frac{\partial \mathbf{J}_{\text{ext},i}}{\partial t} \approx \frac{\partial \mathbf{J}_{\text{primary}}}{\partial t}
\end{equation}

Therefore:
\begin{equation}
\frac{d\gamma_{\text{ext},i}}{dt} \parallel \frac{d\gamma_{\text{primary}}}{dt}
\end{equation}

The trajectories are parallel in phase space, indicating synchronized dynamics despite architectural differences.
\end{proof}

\subsection{No Intrinsic Identity}

\begin{corollary}[Circuit Identity is Not Intrinsic]
\label{cor:no_intrinsic_identity}
There is no objective criterion for "circuit identity" based on architecture. Only charge distribution continuity is physically meaningful.
\end{corollary}

\begin{proof}
Consider two circuits $\mathcal{C}_A$ and $\mathcal{C}_B$ with different architectures but coupled to the same primary circuit $\mathcal{C}_{\text{primary}}$.

By Theorem~\ref{thm:functional_adoption}, both circuits adopt the functional behavior of $\mathcal{C}_{\text{primary}}$:
\begin{equation}
F_A(t) = F_B(t) = F_{\text{primary}}(t)
\end{equation}

An external observer measuring functional output cannot distinguish $\mathcal{C}_A$ from $\mathcal{C}_B$. The architectural differences are invisible at the functional level.

The only physically meaningful distinction is charge distribution $\rho(\mathbf{r}, t)$, which is continuous across all circuits (Theorem~\ref{thm:charge_continuity}).

"Circuit identity" (architectural type) is an observer-imposed label, not an intrinsic physical property. The physical reality is continuous charge distribution spanning multiple architectural regions.
\end{proof}

\subsection{Experimental Signatures}

Multi-circuit systems with external coupling exhibit:

\begin{enumerate}
\item \textbf{Functional equivalence:} External circuits with different architectures exhibit identical functional outputs when coupled to primary circuit.

\item \textbf{Synchronization:} Phase coherence $R > 0.8$ across all circuits despite architectural differences.

\item \textbf{Charge continuity:} No discontinuities in charge distribution at circuit boundaries.

\item \textbf{Trajectory alignment:} Parallel phase space trajectories across circuits with different architectures.

\item \textbf{Context-dependent function:} Same external circuit exhibits different functional behavior in different coupling contexts.
\end{enumerate}

These signatures distinguish context-dependent functional emergence from intrinsic architectural determination.
