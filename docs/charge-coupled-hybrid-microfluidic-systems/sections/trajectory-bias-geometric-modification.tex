\section{Trajectory Bias Through Geometric Modification}
\label{sec:trajectory_bias}

We establish that previous charge redistribution events modify circuit geometry, creating trajectory biases that influence future charge dynamics without requiring information storage.

\subsection{Geometric Modification from Charge Flow}

\begin{definition}[Geometric Modification]
\label{def:geometric_modification}
\emph{Geometric modification} is the alteration of circuit properties (conductivity, free energy landscape) resulting from previous charge redistribution events:
\begin{equation}
\sigma(\mathbf{r},t) = \sigma_0(\mathbf{r}) + \int_0^t f[\rho(\mathbf{r},t')] \, dt'
\end{equation}
\begin{equation}
F[\rho,t] = F_0[\rho] + \int_0^t g[\rho(\mathbf{r},t')] \, dt'
\end{equation}
where $\sigma_0$ and $F_0$ are initial properties, and $f$, $g$ are modification functionals.
\end{definition}

The integrals represent accumulated modifications from charge flow history. These modifications are geometric (changes to circuit structure), not informational (stored data).

\begin{theorem}[Charge Flow Modifies Circuit Geometry]
\label{thm:flow_modifies_geometry}
Charge redistribution creates persistent geometric modifications that bias future charge dynamics:
\begin{equation}
\mathbf{J}(t) = -\sigma(\mathbf{r},t) \nabla \phi(\mathbf{r},t)
\end{equation}
where $\sigma(\mathbf{r},t)$ depends on charge flow history through Definition~\ref{def:geometric_modification}.
\end{theorem}

\begin{proof}
Consider charge flow at time $t'$:
\begin{equation}
\mathbf{J}(\mathbf{r},t') = -\sigma(\mathbf{r},t') \nabla \phi(\mathbf{r},t')
\end{equation}

This flow dissipates energy:
\begin{equation}
\frac{dE}{dt'} = -\int_V \frac{\mathbf{J}^2}{\sigma} d^3r < 0
\end{equation}

The dissipated energy modifies circuit structure through:
\begin{itemize}
\item Joule heating: $\Delta T \propto \mathbf{J}^2 / \sigma$
\item Electrochemical reactions: $\Delta c \propto \mathbf{J}$
\item Structural rearrangement: $\Delta \sigma \propto \int \mathbf{J}^2 dt'$
\end{itemize}

These modifications accumulate over time:
\begin{equation}
\sigma(\mathbf{r},t) = \sigma_0(\mathbf{r}) + \int_0^t \alpha \frac{\mathbf{J}^2(\mathbf{r},t')}{\sigma(\mathbf{r},t')} dt'
\end{equation}

where $\alpha$ is modification rate constant.

Future charge flow at time $t > t'$ experiences modified conductivity $\sigma(\mathbf{r},t)$, creating trajectory bias toward paths of previous flow.
\end{proof}

\subsection{Trajectory Bias Without Information Storage}

\begin{definition}[Trajectory Bias]
\label{def:trajectory_bias}
A \emph{trajectory bias} is preferential selection of phase space trajectories due to geometric modifications, without explicit information storage:
\begin{equation}
P(\gamma_i | \rho_0) = \frac{e^{-\beta F[\gamma_i]}}{\sum_j e^{-\beta F[\gamma_j]}}
\end{equation}
where $F[\gamma_i]$ is free energy of trajectory $\gamma_i$, modified by previous charge flow history.
\end{definition}

\begin{theorem}[Trajectory Selection Without Storage]
\label{thm:trajectory_no_storage}
Geometric modifications bias trajectory selection without storing information about previous events. The bias is consequence of modified landscape, not retrieved data.
\end{theorem}

\begin{proof}
Consider two scenarios:

\textbf{Scenario 1 (First occurrence):} Charge distribution $\rho_0$ evolves along trajectory $\gamma_1$ with unmodified landscape:
\begin{equation}
\frac{\partial \rho}{\partial t} = \nabla \cdot \left( \sigma_0 \nabla \frac{\delta F_0}{\delta \rho} \right)
\end{equation}

This creates geometric modifications along $\gamma_1$.

\textbf{Scenario 2 (Subsequent occurrence):} Similar charge distribution $\rho_0'$ (with $\|\rho_0' - \rho_0\| < \epsilon$) evolves with modified landscape:
\begin{equation}
\frac{\partial \rho}{\partial t} = \nabla \cdot \left( \sigma(\mathbf{r},t) \nabla \frac{\delta F[\rho,t]}{\delta \rho} \right)
\end{equation}

The modified $\sigma$ and $F$ bias trajectory toward $\gamma_1$ because:
\begin{itemize}
\item Conductivity increased along $\gamma_1$ path: $\sigma(\mathbf{r} \in \gamma_1) > \sigma_0(\mathbf{r})$
\item Free energy reduced along $\gamma_1$ path: $F[\gamma_1,t] < F_0[\gamma_1]$
\end{itemize}

The trajectory preferentially follows $\gamma_1$ not because information about $\gamma_1$ is "stored" and "retrieved," but because the landscape has been geometrically modified to favor that path.

**Analogy**: Water flowing down hillside carves channel. Subsequent water follows same channel not because first water "stored information" about the path, but because channel geometry biases flow direction.

No information storage occurs—only geometric modification that biases future dynamics.
\end{proof}

\subsection{The River Bed Analogy}

\begin{example}[River Flow as Trajectory Bias]
\label{ex:river_flow}
Water flowing in river provides direct analogy to trajectory bias without storage:

\textbf{River dynamics}:
\begin{itemize}
\item Water flows downhill following path of least resistance
\item Flow erodes riverbed, deepening channel along flow path
\item Subsequent water follows same path due to deepened channel
\item River has no "memory" of previous water—only modified geometry
\end{itemize}

\textbf{Circuit dynamics}:
\begin{itemize}
\item Charge flows following gradient of least free energy
\item Flow modifies conductivity, lowering resistance along flow path
\item Subsequent charge follows same path due to modified conductivity
\item Circuit has no "memory" of previous charge—only modified geometry
\end{itemize}

The river doesn't "know" its path or "remember" previous flow. It simply follows the carved geometry. Similarly, circuits don't "store" or "retrieve" information—they follow geometrically modified trajectories.
\end{example}

\subsection{Variance-Dependent Modification Strength}

\begin{theorem}[High Variance Creates Stronger Modifications]
\label{thm:variance_modification}
Charge distributions with higher variance create stronger geometric modifications:
\begin{equation}
\frac{d\sigma}{dt} \propto \sigma^2[\rho] = \int_V (\rho(\mathbf{r}) - \langle \rho \rangle)^2 d^3r
\end{equation}
\end{theorem}

\begin{proof}
The modification rate depends on energy dissipation:
\begin{equation}
\frac{d\sigma}{dt} = \alpha \int_V \frac{\mathbf{J}^2}{\sigma} d^3r
\end{equation}

Current flux is:
\begin{equation}
\mathbf{J} = -\sigma \nabla \phi \propto -\sigma \nabla \rho
\end{equation}

Therefore:
\begin{equation}
\frac{d\sigma}{dt} \propto \int_V \sigma |\nabla \rho|^2 d^3r
\end{equation}

By integration by parts and using charge conservation:
\begin{equation}
\int_V |\nabla \rho|^2 d^3r \propto \int_V (\rho - \langle \rho \rangle)^2 d^3r = \sigma^2[\rho]
\end{equation}

Higher variance → larger gradients → stronger currents → more energy dissipation → stronger geometric modification.
\end{proof}

\begin{corollary}[Near-Equilibrium Events Create Weak Modifications]
\label{cor:near_equilibrium_weak}
Charge distributions near equilibrium (low variance) create minimal geometric modifications:
\begin{equation}
\sigma^2[\rho] \to 0 \quad \Rightarrow \quad \frac{d\sigma}{dt} \to 0
\end{equation}
\end{corollary}

\begin{proof}
By Theorem~\ref{thm:variance_modification}, modification rate is proportional to variance. For near-equilibrium distributions with $\sigma^2[\rho] \approx 0$, modification rate approaches zero.

Near-equilibrium events create negligible trajectory bias because they produce minimal geometric modification.
\end{proof}

\subsection{High-Variance Events and Trajectory Preference}

\begin{theorem}[High-Variance Events Create Strong Trajectory Bias]
\label{thm:high_variance_bias}
Charge distributions with high variance (far from equilibrium) create strong trajectory biases that persist over long timescales:
\begin{equation}
\Delta \sigma \propto \sigma^2[\rho] \cdot \tau_{\text{event}}
\end{equation}
where $\tau_{\text{event}}$ is event duration.
\end{theorem}

\begin{proof}
The accumulated modification from event of duration $\tau_{\text{event}}$ is:
\begin{equation}
\Delta \sigma = \int_0^{\tau_{\text{event}}} \frac{d\sigma}{dt} dt = \int_0^{\tau_{\text{event}}} \alpha \sigma^2[\rho(t)] dt
\end{equation}

For approximately constant variance during event:
\begin{equation}
\Delta \sigma \approx \alpha \sigma^2[\rho] \cdot \tau_{\text{event}}
\end{equation}

High-variance events ($\sigma^2[\rho] \gg k_B T / V$) create modifications $\Delta \sigma \gg \Delta \sigma_{\text{thermal}}$, establishing strong trajectory preferences.

These modifications persist until:
\begin{itemize}
\item Thermal fluctuations erase modifications: $\tau_{\text{persist}} \sim \Delta \sigma / (k_B T)$
\item Competing trajectories create stronger modifications
\item System undergoes structural reorganization
\end{itemize}

For typical circuits, $\tau_{\text{persist}} \gg \tau_{\text{event}}$, so trajectory bias persists long after initiating event.
\end{proof}

\begin{example}[Near-Miss vs. Success]
\label{ex:near_miss}
Consider two charge redistribution events:

\textbf{Event A (Success):} Trajectory reaches equilibrium attractor:
\begin{itemize}
\item Initial variance: $\sigma^2[\rho_0] = \sigma_0^2$
\item Final variance: $\sigma^2[\rho_f] \approx 0$ (equilibrium)
\item Average variance: $\langle \sigma^2 \rangle \approx \sigma_0^2 / 2$
\item Modification: $\Delta \sigma_A \propto \sigma_0^2 \tau / 2$
\end{itemize}

\textbf{Event B (Near-miss):} Trajectory approaches but doesn't reach attractor:
\begin{itemize}
\item Initial variance: $\sigma^2[\rho_0] = \sigma_0^2$
\item Final variance: $\sigma^2[\rho_f] \approx \sigma_0^2 / 2$ (near equilibrium)
\item Average variance: $\langle \sigma^2 \rangle \approx 3\sigma_0^2 / 4$
\item Modification: $\Delta \sigma_B \propto 3\sigma_0^2 \tau / 4$
\end{itemize}

Near-miss creates stronger modification: $\Delta \sigma_B / \Delta \sigma_A = 3/2$.

Near-miss trajectories create stronger biases because they maintain high variance longer, producing more geometric modification. This explains why "near-miss" events create stronger trajectory preferences than "success" events.
\end{example}

\subsection{Trajectory Selection from Phase Space}

\begin{theorem}[Trajectories Pre-Exist in Phase Space]
\label{thm:trajectories_preexist}
All possible trajectories exist in phase space before any charge redistribution event. Geometric modifications bias selection among pre-existing trajectories rather than creating new trajectories.
\end{theorem}

\begin{proof}
The phase space of circuit with $N$ degrees of freedom is:
\begin{equation}
\Gamma = \{(\rho(\mathbf{r}), \mathbf{J}(\mathbf{r})) : \int \rho \, d^3r = Q_{\text{total}}\}
\end{equation}

For bounded phase space (finite energy, finite volume), $\Gamma$ is compact set with finite measure.

A trajectory is curve $\gamma: [0,T] \to \Gamma$ satisfying:
\begin{equation}
\frac{d\gamma}{dt} = \mathcal{F}[\gamma(t)]
\end{equation}
where $\mathcal{F}$ is the dynamics operator.

The set of all possible trajectories is:
\begin{equation}
\mathcal{T} = \{\gamma : [0,T] \to \Gamma \text{ satisfying dynamics}\}
\end{equation}

This set is determined by:
\begin{itemize}
\item Phase space structure $\Gamma$ (fixed by circuit architecture)
\item Dynamics operator $\mathcal{F}$ (fixed by charge conservation and variance minimization)
\item Boundary conditions (fixed by circuit geometry)
\end{itemize}

Geometric modifications change the probability distribution over $\mathcal{T}$:
\begin{equation}
P(\gamma | \text{history}) = \frac{e^{-\beta F[\gamma,\text{history}]}}{\sum_{\gamma' \in \mathcal{T}} e^{-\beta F[\gamma',\text{history}]}}
\end{equation}

but do not change the set $\mathcal{T}$ itself. All trajectories pre-exist in phase space; modifications only bias selection probabilities.
\end{proof}

\subsection{No Information Storage Requirement}

\begin{theorem}[Trajectory Bias Requires No Information Storage]
\label{thm:no_storage_required}
The trajectory bias mechanism operates without storing information about previous events. Only geometric modifications (structural changes) are required.
\end{theorem}

\begin{proof}
Information storage requires:
\begin{enumerate}
\item Encoding: Mapping events to stored representations
\item Retention: Maintaining stored representations over time
\item Retrieval: Accessing stored representations when needed
\item Decoding: Interpreting stored representations to influence behavior
\end{enumerate}

Trajectory bias requires none of these:

\textbf{No encoding}: Previous charge flow directly modifies geometry through energy dissipation. No mapping to representation occurs.

\textbf{No retention mechanism}: Geometric modifications persist through structural stability, not through active maintenance of stored data.

\textbf{No retrieval process}: Current charge flow responds to current geometric state. No access to stored past events occurs.

\textbf{No decoding}: Modified geometry directly influences charge dynamics through conductivity and free energy landscape. No interpretation of stored information occurs.

The mechanism is purely geometric: past events modify structure → modified structure biases future dynamics. No information processing occurs.
\end{proof}

\begin{corollary}[Trajectory Bias Explains Apparent "Memory" Without Storage]
\label{cor:apparent_memory}
Circuits exhibit behavior that appears to "remember" previous events, but this behavior emerges from geometric trajectory bias rather than information storage.
\end{corollary}

\begin{proof}
Consider circuit that previously experienced charge distribution $\rho_1$ along trajectory $\gamma_1$. When similar distribution $\rho_1'$ occurs later, circuit preferentially follows trajectory $\gamma_1'$ similar to $\gamma_1$.

This appears as "memory" of previous event: circuit "remembers" $\rho_1$ and "recalls" appropriate response $\gamma_1$.

But actual mechanism is:
\begin{itemize}
\item Previous trajectory $\gamma_1$ modified geometry along its path
\item Modified geometry biases current trajectory toward $\gamma_1'$
\item No information about $\rho_1$ or $\gamma_1$ is stored or retrieved
\item Only geometric modifications influence current dynamics
\end{itemize}

The "memory" is trajectory preference created by geometric modification, not stored information about past events.
\end{proof}

\subsection{Temporal Consistency Without Predetermined Storage}

\begin{theorem}[Trajectory Bias Maintains Temporal Consistency]
\label{thm:temporal_consistency_bias}
Geometric modifications ensure that similar charge distributions produce similar trajectories across time, maintaining temporal consistency without requiring predetermined information storage.
\end{theorem}

\begin{proof}
For charge distribution $\rho_0$ occurring at time $t_1$, trajectory is:
\begin{equation}
\gamma_1(t) = \int_0^t \mathcal{F}[\rho(t'), \sigma(t_1)] dt'
\end{equation}

For similar distribution $\rho_0'$ (with $\|\rho_0' - \rho_0\| < \epsilon$) occurring at time $t_2 > t_1$, trajectory is:
\begin{equation}
\gamma_2(t) = \int_0^t \mathcal{F}[\rho(t'), \sigma(t_2)] dt'
\end{equation}

If $\gamma_1$ created geometric modifications, then $\sigma(t_2)$ is biased toward $\gamma_1$ path, making:
\begin{equation}
\|\gamma_2 - \gamma_1\| < \|\gamma_2^{\text{unbiased}} - \gamma_1\|
\end{equation}

Similar initial conditions produce similar trajectories across time due to geometric bias, not because information about "correct" response is stored and retrieved.

Temporal consistency emerges from geometric modification, not from information storage.
\end{proof}

\subsection{Experimental Signatures}

Trajectory bias through geometric modification exhibits:

\begin{enumerate}
\item \textbf{Path preference}: Repeated charge redistributions preferentially follow previously traversed trajectories.

\item \textbf{Variance-dependent strength}: High-variance events create stronger trajectory biases than low-variance events.

\item \textbf{Near-miss enhancement}: Events that approach but don't reach equilibrium create stronger biases than events that reach equilibrium.

\item \textbf{Persistence timescales}: Trajectory biases persist on timescales $\tau_{\text{persist}} \sim \Delta \sigma / (k_B T)$, much longer than initiating events.

\item \textbf{No storage requirement}: Trajectory bias operates through geometric modification without requiring information encoding, retention, or retrieval mechanisms.

\item \textbf{Pre-existing trajectories}: All possible trajectories exist in phase space; modifications only bias selection probabilities.
\end{enumerate}

These signatures distinguish trajectory bias (geometric mechanism) from information storage (encoding/retrieval mechanism).
