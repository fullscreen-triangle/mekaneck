\section{Subsystem Replacement and Identity Persistence}
\label{sec:subsystem_replacement}

We analyze charge distribution dynamics under sequential subsystem replacement, establishing that circuit identity is finite information that dissipates through replacement operations.

\subsection{Identity as Information}

\begin{definition}[Identity Information]
\label{def:identity_info}
The \emph{identity information} $I_{\text{id}}$ of a circuit $\mathcal{C}$ is the minimum information required to distinguish $\mathcal{C}$ from all other circuits in a reference class:
\begin{equation}
I_{\text{id}}(\mathcal{C}) = \min_{D} H(D(\mathcal{C}))
\end{equation}
where $D$ ranges over all distinguishing functions and $H$ is Shannon entropy.
\end{definition}

The identity information quantifies "how much information is needed to specify this particular circuit."

\begin{theorem}[Identity Information is Finite]
\label{thm:identity_finite}
For any physical circuit $\mathcal{C}$ with bounded phase space volume $V_{\text{phase}} < \infty$:
\begin{equation}
I_{\text{id}}(\mathcal{C}) < \infty
\end{equation}
\end{theorem}

\begin{proof}
A circuit occupies bounded region of phase space with volume $V_{\text{phase}}$. The circuit state is specified by charge distribution $\rho(\mathbf{r})$ and current distribution $\mathbf{J}(\mathbf{r})$ to precision $\delta$.

The number of distinguishable states is:
\begin{equation}
N_{\text{states}} = \frac{V_{\text{phase}}}{\delta^{2d}}
\end{equation}
where $d$ is the number of spatial degrees of freedom.

The identity information is at most:
\begin{equation}
I_{\text{id}}(\mathcal{C}) \leq \ln N_{\text{states}} = \ln\left(\frac{V_{\text{phase}}}{\delta^{2d}}\right) < \infty
\end{equation}

For finite $V_{\text{phase}}$ and finite precision $\delta > 0$, identity information is finite.
\end{proof}

\begin{definition}[Identity Entropy]
\label{def:identity_entropy}
The \emph{identity entropy} is the thermodynamic entropy associated with distinguishability:
\begin{equation}
S_{\text{id}} = \kB I_{\text{id}}
\end{equation}
This is the minimum entropy that must be dissipated to completely erase circuit identity.
\end{definition}

\subsection{Subsystem Replacement as Partition-Composition}

\begin{definition}[Subsystem Replacement]
\label{def:replacement}
A \emph{subsystem replacement} operation on circuit $\mathcal{C}$ consists of two sequential operations:
\begin{enumerate}
\item \textbf{Partition (removal):} Remove subsystem $\mathcal{C}_{\text{old}}$ from $\mathcal{C}$:
\begin{equation}
\mathcal{C}' = \mathcal{C} \setminus \{\mathcal{C}_{\text{old}}\}
\end{equation}

\item \textbf{Composition (addition):} Add new subsystem $\mathcal{C}_{\text{new}}$ to $\mathcal{C}'$:
\begin{equation}
\mathcal{C}'' = \mathcal{C}' \cup \{\mathcal{C}_{\text{new}}\}
\end{equation}
\end{enumerate}

The complete replacement is:
\begin{equation}
\mathcal{C} \xrightarrow{\text{remove } \mathcal{C}_{\text{old}}} \mathcal{C}' \xrightarrow{\text{add } \mathcal{C}_{\text{new}}} \mathcal{C}''
\end{equation}
\end{definition}

\begin{theorem}[Replacement Generates Entropy]
\label{thm:replacement_entropy}
Each subsystem replacement generates entropy:
\begin{equation}
\Delta S_{\text{replacement}} = S_{\text{partition}} + S_{\text{composition}} > 0
\end{equation}
where $S_{\text{partition}}$ is entropy from removal and $S_{\text{composition}}$ is entropy from addition.
\end{theorem}

\begin{proof}
\textbf{Partition entropy:} Removing subsystem $\mathcal{C}_{\text{old}}$ creates undetermined residue—information about the exact state of connections between $\mathcal{C}_{\text{old}}$ and the rest of $\mathcal{C}$ is lost:
\begin{equation}
S_{\text{partition}} = \kB \ln\left(\frac{W_{\text{before}}}{W_{\text{after}}}\right) + S_{\text{residue}}^{(\text{removal})}
\end{equation}
where $W$ is the number of accessible configurations.

\textbf{Composition entropy:} Adding subsystem $\mathcal{C}_{\text{new}}$ generates entropy because $\mathcal{C}_{\text{new}} \neq \mathcal{C}_{\text{old}}$—the new subsystem has different charge distribution, requiring system reconfiguration:
\begin{equation}
S_{\text{composition}} = S_{\text{residue}}^{(\text{addition})}
\end{equation}

The total entropy is:
\begin{equation}
\Delta S_{\text{replacement}} = S_{\text{partition}} + S_{\text{composition}} > 0
\end{equation}

The inequality is strict because at least one operation (partition or composition) generates positive entropy.
\end{proof}

\subsection{Cumulative Identity Loss}

\begin{theorem}[Cumulative Entropy from Sequential Replacements]
\label{thm:cumulative}
After $n$ sequential replacements, cumulative entropy is:
\begin{equation}
S_{\text{cumulative}}(n) = \sum_{i=1}^{n} \Delta S_i
\end{equation}
If all replacements are statistically similar:
\begin{equation}
S_{\text{cumulative}}(n) = n \cdot \langle \Delta S \rangle
\end{equation}
where $\langle \Delta S \rangle$ is average entropy per replacement.
\end{theorem}

\begin{proof}
Each replacement $i$ generates entropy $\Delta S_i > 0$ (Theorem~\ref{thm:replacement_entropy}). These contributions are additive because each replacement is independent thermodynamic process.

The cumulative entropy after $n$ replacements is:
\begin{equation}
S_{\text{cumulative}}(n) = \sum_{i=1}^{n} \Delta S_i
\end{equation}

For statistically similar replacements (same subsystem type, same procedure):
\begin{equation}
\Delta S_i \approx \langle \Delta S \rangle = \frac{1}{n} \sum_{i=1}^{n} \Delta S_i
\end{equation}

Therefore:
\begin{equation}
S_{\text{cumulative}}(n) \approx n \cdot \langle \Delta S \rangle
\end{equation}

Cumulative entropy grows linearly with number of replacements.
\end{proof}

\begin{theorem}[Identity Dissipation Threshold]
\label{thm:threshold}
Original identity is thermodynamically dissipated when cumulative replacement entropy exceeds identity entropy:
\begin{equation}
S_{\text{cumulative}}(n^*) \geq S_{\text{id}}(\mathcal{C})
\end{equation}
The threshold number of replacements is:
\begin{equation}
n^* = \frac{S_{\text{id}}(\mathcal{C})}{\langle \Delta S \rangle} = \frac{I_{\text{id}}(\mathcal{C})}{\langle \Delta I \rangle}
\end{equation}
where $\langle \Delta I \rangle = \langle \Delta S \rangle / \kB$ is average information loss per replacement.
\end{theorem}

\begin{proof}
Identity information $I_{\text{id}}(\mathcal{C})$ is total information required to distinguish original circuit. Each replacement dissipates information:
\begin{equation}
\Delta I = \frac{\Delta S_{\text{replacement}}}{\kB}
\end{equation}

After $n$ replacements, cumulative information loss is:
\begin{equation}
I_{\text{lost}}(n) = \sum_{i=1}^{n} \Delta I_i = \frac{S_{\text{cumulative}}(n)}{\kB}
\end{equation}

Remaining identity information is:
\begin{equation}
I_{\text{remaining}}(n) = I_{\text{id}}(\mathcal{C}) - I_{\text{lost}}(n)
\end{equation}

When cumulative loss equals total identity:
\begin{equation}
I_{\text{lost}}(n^*) = I_{\text{id}}(\mathcal{C})
\end{equation}
the circuit no longer contains sufficient information to be identified as the original. Identity has been thermodynamically dissipated.

Solving for $n^*$:
\begin{equation}
n^* \cdot \langle \Delta I \rangle = I_{\text{id}}(\mathcal{C}) \quad \Rightarrow \quad n^* = \frac{I_{\text{id}}(\mathcal{C})}{\langle \Delta I \rangle}
\end{equation}
\end{proof}

\subsection{Fractional Identity Decay}

\begin{corollary}[Fractional Identity Remaining]
\label{cor:fractional_identity}
The fractional identity remaining after $n$ replacements is:
\begin{equation}
f_{\text{id}}(n) = 1 - \frac{n}{n^*} = 1 - \frac{S_{\text{cumulative}}(n)}{S_{\text{id}}}
\end{equation}
for $n \leq n^*$. For $n > n^*$, identity is completely dissipated: $f_{\text{id}}(n) = 0$.
\end{corollary}

\begin{proof}
Remaining identity information is:
\begin{equation}
I_{\text{remaining}}(n) = I_{\text{id}}(\mathcal{C}) - n \cdot \langle \Delta I \rangle
\end{equation}

Fractional identity is:
\begin{equation}
f_{\text{id}}(n) = \frac{I_{\text{remaining}}(n)}{I_{\text{id}}(\mathcal{C})} = 1 - \frac{n \cdot \langle \Delta I \rangle}{I_{\text{id}}(\mathcal{C})} = 1 - \frac{n}{n^*}
\end{equation}

This is linear decay from $f_{\text{id}}(0) = 1$ (full original identity) to $f_{\text{id}}(n^*) = 0$ (no original identity).

For $n > n^*$, cumulative information loss exceeds total identity information, but identity cannot be negative. Therefore $f_{\text{id}}(n) = 0$ for all $n > n^*$.
\end{proof}

\subsection{Charge Distribution Adoption}

\begin{theorem}[Replacement Subsystem Adopts Host Charge Distribution]
\label{thm:charge_adoption}
When subsystem $\mathcal{C}_{\text{new}}$ replaces $\mathcal{C}_{\text{old}}$ in circuit $\mathcal{C}$, the replacement subsystem adopts the host circuit charge distribution:
\begin{equation}
\rho_{\mathcal{C}_{\text{new}}}(\mathbf{r},t) \to \rho_{\mathcal{C}}(\mathbf{r},t) \Big|_{\text{region of } \mathcal{C}_{\text{old}}}
\end{equation}
as $t \to \tau_{\text{adopt}}$, where $\tau_{\text{adopt}}$ is the adoption timescale.
\end{theorem}

\begin{proof}
After replacement, the new subsystem $\mathcal{C}_{\text{new}}$ is coupled to the host circuit $\mathcal{C}' = \mathcal{C} \setminus \{\mathcal{C}_{\text{old}}\}$.

The coupled system minimizes total variance (Section~\ref{sec:multi_circuit}):
\begin{equation}
\sigma^2_{\text{total}} = \int_{\mathcal{C}' \cup \mathcal{C}_{\text{new}}} (\rho - \langle \rho \rangle)^2 d^3r
\end{equation}

Variance minimization drives charge redistribution:
\begin{equation}
\frac{\partial \rho_{\mathcal{C}_{\text{new}}}}{\partial t} = \nabla \cdot \left( \sigma \nabla \frac{\delta F}{\delta \rho} \right)
\end{equation}

The equilibrium charge distribution in $\mathcal{C}_{\text{new}}$ matches the host circuit distribution:
\begin{equation}
\rho_{\mathcal{C}_{\text{new}}}^{\text{eq}} = \rho_{\mathcal{C}}^{\text{eq}} \Big|_{\text{region of } \mathcal{C}_{\text{old}}}
\end{equation}

The timescale for reaching equilibrium is:
\begin{equation}
\tau_{\text{adopt}} = \frac{L_{\text{new}}^2}{\sigma_{\text{eff}} / \epsilon_{\text{eff}}}
\end{equation}

For $t > \tau_{\text{adopt}}$, the replacement subsystem has adopted the host charge distribution, losing its original charge configuration.
\end{proof}

\begin{corollary}[Functional Continuity Under Replacement]
\label{cor:functional_continuity}
System-level function remains continuous despite subsystem replacement:
\begin{equation}
F_{\mathcal{C}}(t) \approx F_{\mathcal{C}}(t_0) \quad \forall t > \tau_{\text{adopt}}
\end{equation}
where $t_0$ is time before replacement.
\end{corollary}

\begin{proof}
By Theorem~\ref{thm:charge_adoption}, replacement subsystem adopts host charge distribution. By Theorem~\ref{thm:architecture_independent} (Section~\ref{sec:multi_circuit}), functional output depends on charge distribution, not on subsystem architecture.

If charge distribution is preserved ($\rho_{\mathcal{C}}(t) \approx \rho_{\mathcal{C}}(t_0)$), then functional output is preserved ($F_{\mathcal{C}}(t) \approx F_{\mathcal{C}}(t_0)$), despite subsystem replacement.

System-level function exhibits continuity through replacement operations.
\end{proof}

\subsection{Sequential Replacement Dynamics}

Consider sequential replacement of all $N$ subsystems:
\begin{equation}
\mathcal{C}(t_0) = \{\mathcal{C}_1, \mathcal{C}_2, \ldots, \mathcal{C}_N\} \to \mathcal{C}(t_f) = \{\mathcal{C}_1', \mathcal{C}_2', \ldots, \mathcal{C}_N'\}
\end{equation}

\begin{theorem}[Complete Replacement Dissipates All Original Identity]
\label{thm:complete_replacement}
After replacing all $N$ subsystems, original identity is completely dissipated:
\begin{equation}
f_{\text{id}}(N) = 0
\end{equation}
The circuit $\mathcal{C}(t_f)$ shares no components with original circuit $\mathcal{C}(t_0)$.
\end{theorem}

\begin{proof}
Assume identity is uniformly distributed among subsystems, so each subsystem carries identity fraction $1/N$.

After replacing $k$ subsystems:
\begin{itemize}
\item $N - k$ original subsystems remain, carrying total identity $(N-k)/N$
\item $k$ new subsystems carry zero original identity
\end{itemize}

Fractional identity is:
\begin{equation}
f_{\text{id}}(k) = \frac{N - k}{N}
\end{equation}

For complete replacement ($k = N$):
\begin{equation}
f_{\text{id}}(N) = \frac{N - N}{N} = 0
\end{equation}

All original identity has been dissipated. The circuit after complete replacement is distinct from the original—it shares no components.
\end{proof}

\subsection{Charge Distribution Continuity}

\begin{theorem}[Identity is Not Intrinsic]
\label{thm:identity_not_intrinsic}
Circuit identity based on subsystem composition is observer-dependent labeling, not intrinsic physical property. Only charge distribution continuity $\rho(\mathbf{r},t)$ is physically meaningful.
\end{theorem}

\begin{proof}
Consider sequential replacement of all subsystems. At each step, charge distribution remains continuous (Theorem~\ref{thm:charge_adoption}):
\begin{equation}
\rho_{\mathcal{C}}(t) \text{ continuous for all } t \in [t_0, t_f]
\end{equation}

The question "Is $\mathcal{C}(t_f)$ the same circuit as $\mathcal{C}(t_0)$?" has no objective answer based on subsystem composition:
\begin{itemize}
\item By subsystem criterion: $\mathcal{C}(t_f) \neq \mathcal{C}(t_0)$ (no shared components)
\item By charge distribution criterion: $\mathcal{C}(t_f)$ is continuous evolution of $\mathcal{C}(t_0)$ (charge distribution evolved continuously)
\item By functional criterion: $F_{\mathcal{C}}(t_f) \approx F_{\mathcal{C}}(t_0)$ (functional continuity preserved)
\end{itemize}

Different criteria yield different answers. "Circuit identity" is observer-imposed partition, not intrinsic property.

The physical reality is continuous charge distribution $\rho(\mathbf{r},t)$ evolving over time. Subsystem labels are convenient descriptions, not fundamental ontology.
\end{proof}

\subsection{Experimental Signatures}

Subsystem replacement dynamics exhibit:

\begin{enumerate}
\item \textbf{Linear identity decay:} Fractional identity $f_{\text{id}}(n) = 1 - n/N$ decreases linearly with number of replacements.

\item \textbf{Functional continuity:} System-level function $F_{\mathcal{C}}(t)$ remains approximately constant despite replacements.

\item \textbf{Charge adoption timescale:} Replacement subsystems adopt host charge distribution on timescale $\tau_{\text{adopt}} \sim L^2 / (\sigma/\epsilon)$.

\item \textbf{Entropy accumulation:} Cumulative entropy $S_{\text{cumulative}}(n) = n \langle \Delta S \rangle$ grows linearly with replacements.

\item \textbf{Threshold dissipation:} Complete identity dissipation occurs at $n^* = I_{\text{id}} / \langle \Delta I \rangle$ replacements.
\end{enumerate}
