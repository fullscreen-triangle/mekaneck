\section{Charge Balance as Universal Attractor}
\label{sec:charge_attractor}

We establish that uniform charge distribution (charge balance) functions as universal attractor for all circuit trajectories in closed systems.

\subsection{Equilibrium Charge Distribution}

\begin{definition}[Uniform Charge Distribution]
\label{def:uniform_distribution}
The \emph{uniform charge distribution} is:
\begin{equation}
\rho_{\text{eq}}(\mathbf{r}) = \langle \rho \rangle = \frac{Q_{\text{total}}}{V} \quad \forall \mathbf{r} \in V
\end{equation}
where $Q_{\text{total}}$ is total charge and $V$ is system volume.
\end{definition}

At uniform distribution, charge density is constant throughout the system. There are no gradients:
\begin{equation}
\nabla \rho_{\text{eq}} = 0
\end{equation}

\begin{theorem}[Uniform Distribution Minimizes Variance]
\label{thm:uniform_minimizes}
Among all charge distributions with fixed total charge $Q_{\text{total}}$, the uniform distribution minimizes variance:
\begin{equation}
\sigma^2[\rho_{\text{eq}}] = 0 < \sigma^2[\rho] \quad \forall \rho \neq \rho_{\text{eq}}
\end{equation}
\end{theorem}

\begin{proof}
The charge distribution variance is:
\begin{equation}
\sigma^2[\rho] = \frac{1}{V} \int_V (\rho(\mathbf{r}) - \langle \rho \rangle)^2 d^3r
\end{equation}

For uniform distribution $\rho_{\text{eq}}(\mathbf{r}) = \langle \rho \rangle$:
\begin{equation}
\sigma^2[\rho_{\text{eq}}] = \frac{1}{V} \int_V (\langle \rho \rangle - \langle \rho \rangle)^2 d^3r = 0
\end{equation}

For any non-uniform distribution, there exist regions where $\rho(\mathbf{r}) \neq \langle \rho \rangle$, giving:
\begin{equation}
\sigma^2[\rho] = \frac{1}{V} \int_V (\rho(\mathbf{r}) - \langle \rho \rangle)^2 d^3r > 0
\end{equation}

Therefore $\sigma^2[\rho_{\text{eq}}] < \sigma^2[\rho]$ for all $\rho \neq \rho_{\text{eq}}$.

Uniform distribution is the unique global minimum of variance functional.
\end{proof}

\subsection{Attractor Dynamics}

\begin{definition}[Attractor]
\label{def:attractor}
A state $\rho^*$ is an \emph{attractor} if trajectories starting from nearby initial conditions approach $\rho^*$ as $t \to \infty$:
\begin{equation}
\|\rho(t) - \rho^*\| \to 0 \quad \text{as} \quad t \to \infty
\end{equation}
for all initial conditions in basin of attraction.
\end{definition}

\begin{theorem}[Uniform Distribution is Universal Attractor]
\label{thm:universal_attractor}
In closed charge-coupled circuits, uniform charge distribution $\rho_{\text{eq}}$ is universal attractor: all trajectories approach $\rho_{\text{eq}}$ asymptotically.
\end{theorem}

\begin{proof}
The charge dynamics are governed by variance minimization (Section~\ref{sec:autocatalytic}):
\begin{equation}
\frac{\partial \rho}{\partial t} = \nabla \cdot \left( \sigma \nabla \frac{\delta F}{\delta \rho} \right)
\end{equation}

The free energy functional is:
\begin{equation}
F[\rho] = \int_V \left[ f(\rho) + \frac{\epsilon}{2} |\nabla \phi|^2 \right] d^3r
\end{equation}

Free energy decreases monotonically (Theorem~\ref{thm:free_energy_min}):
\begin{equation}
\frac{dF}{dt} = -\int_V \sigma \left| \nabla \frac{\delta F}{\delta \rho} \right|^2 d^3r \leq 0
\end{equation}

The global minimum of $F[\rho]$ subject to charge conservation $\int \rho d^3r = Q_{\text{total}}$ is the uniform distribution $\rho_{\text{eq}}$.

Therefore, all trajectories evolve toward $\rho_{\text{eq}}$ as $t \to \infty$:
\begin{equation}
\rho(t) \to \rho_{\text{eq}} \quad \text{as} \quad t \to \infty
\end{equation}

Uniform distribution is universal attractor for all initial conditions.
\end{proof}

\subsection{Asymptotic Approach in Closed Systems}

\begin{theorem}[Asymptotic Approach Without Reaching]
\label{thm:asymptotic}
In closed systems, trajectories approach uniform distribution asymptotically but never reach it:
\begin{equation}
\|\rho(t) - \rho_{\text{eq}}\| \to 0 \quad \text{but} \quad \rho(t) \neq \rho_{\text{eq}} \quad \forall t < \infty
\end{equation}
\end{theorem}

\begin{proof}
The approach to equilibrium is governed by:
\begin{equation}
\frac{d}{dt} \|\rho(t) - \rho_{\text{eq}}\|^2 = -2 \int_V \sigma \left| \nabla \frac{\delta F}{\delta \rho} \right|^2 d^3r < 0
\end{equation}

The distance to equilibrium decreases monotonically. However, reaching exact equilibrium $\rho(t) = \rho_{\text{eq}}$ requires:
\begin{enumerate}
\item Zero gradient: $\nabla \rho = 0$ everywhere
\item Zero thermal fluctuations: $\delta \rho_{\text{thermal}} = 0$
\item Infinite precision: $\epsilon \to 0$
\end{enumerate}

In physical systems, thermal fluctuations continuously perturb charge distribution:
\begin{equation}
\rho(\mathbf{r}, t) = \rho_{\text{eq}} + \delta \rho_{\text{thermal}}(\mathbf{r}, t)
\end{equation}
where $\langle \delta \rho_{\text{thermal}} \rangle = 0$ but $\langle (\delta \rho_{\text{thermal}})^2 \rangle = k_B T / V > 0$.

These fluctuations prevent exact equilibrium. The system oscillates around $\rho_{\text{eq}}$ with amplitude:
\begin{equation}
\langle (\rho - \rho_{\text{eq}})^2 \rangle \sim k_B T / V
\end{equation}

The distance to equilibrium approaches thermal floor:
\begin{equation}
\|\rho(t) - \rho_{\text{eq}}\| \to \sqrt{k_B T / V} \quad \text{as} \quad t \to \infty
\end{equation}

but never reaches zero. Equilibrium is approached asymptotically but not achieved.
\end{proof}

\subsection{Perpetual Oscillation Around Attractor}

\begin{corollary}[Oscillation Around Equilibrium]
\label{cor:oscillation_equilibrium}
Closed systems exhibit perpetual oscillation around uniform distribution with amplitude determined by thermal energy:
\begin{equation}
\rho(\mathbf{r}, t) = \rho_{\text{eq}} + A(\mathbf{r}) \cos(\omega t + \phi(\mathbf{r}))
\end{equation}
where $A(\mathbf{r}) \sim \sqrt{k_B T / V}$ is thermal amplitude.
\end{corollary}

\begin{proof}
Thermal fluctuations create charge imbalances with characteristic energy $k_B T$. These imbalances drive autocatalytic redistribution (Section~\ref{sec:autocatalytic}) on timescale $\tau_{\text{redist}} = L^2 / (\sigma/\epsilon)$.

The oscillation frequency is:
\begin{equation}
\omega = \frac{1}{\tau_{\text{redist}}} = \frac{\sigma}{\epsilon L^2}
\end{equation}

The amplitude is determined by thermal energy:
\begin{equation}
\frac{1}{2} \epsilon A^2 V \sim k_B T \quad \Rightarrow \quad A \sim \sqrt{\frac{k_B T}{\epsilon V}}
\end{equation}

The system oscillates perpetually around $\rho_{\text{eq}}$ with thermal amplitude and characteristic frequency $\omega$.
\end{proof}

\subsection{Basin of Attraction}

\begin{theorem}[Global Basin of Attraction]
\label{thm:global_basin}
The uniform distribution $\rho_{\text{eq}}$ has global basin of attraction: all physically realizable initial conditions evolve toward $\rho_{\text{eq}}$.
\end{theorem}

\begin{proof}
The basin of attraction is the set of initial conditions that evolve toward attractor. For uniform distribution:
\begin{equation}
\mathcal{B}(\rho_{\text{eq}}) = \left\{ \rho_0 : \lim_{t \to \infty} \rho(t; \rho_0) = \rho_{\text{eq}} \right\}
\end{equation}

The free energy $F[\rho]$ is convex functional with unique global minimum at $\rho_{\text{eq}}$. Gradient flow dynamics:
\begin{equation}
\frac{\partial \rho}{\partial t} = \nabla \cdot \left( \sigma \nabla \frac{\delta F}{\delta \rho} \right)
\end{equation}
decrease free energy monotonically from any initial condition.

Therefore, all initial conditions with finite free energy $F[\rho_0] < \infty$ evolve toward global minimum $\rho_{\text{eq}}$.

The basin of attraction is:
\begin{equation}
\mathcal{B}(\rho_{\text{eq}}) = \left\{ \rho : F[\rho] < \infty, \int \rho d^3r = Q_{\text{total}} \right\}
\end{equation}

This includes all physically realizable charge distributions (finite energy, conserved charge). The basin is global.
\end{proof}

\subsection{Lyapunov Function}

\begin{theorem}[Free Energy as Lyapunov Function]
\label{thm:lyapunov}
The free energy functional $F[\rho]$ serves as Lyapunov function for charge dynamics, proving stability of uniform distribution.
\end{theorem}

\begin{proof}
A Lyapunov function $L[\rho]$ satisfies:
\begin{enumerate}
\item $L[\rho] \geq L[\rho_{\text{eq}}]$ for all $\rho$ (minimum at equilibrium)
\item $\frac{dL}{dt} \leq 0$ along trajectories (monotonic decrease)
\item $\frac{dL}{dt} = 0$ only at $\rho = \rho_{\text{eq}}$ (strict decrease away from equilibrium)
\end{enumerate}

The free energy satisfies all three conditions:

\textbf{Condition 1:} $F[\rho_{\text{eq}}]$ is global minimum (Theorem~\ref{thm:uniform_minimizes}), so $F[\rho] \geq F[\rho_{\text{eq}}]$ for all $\rho$.

\textbf{Condition 2:} Free energy decreases along trajectories (Theorem~\ref{thm:free_energy_min}):
\begin{equation}
\frac{dF}{dt} = -\int_V \sigma \left| \nabla \frac{\delta F}{\delta \rho} \right|^2 d^3r \leq 0
\end{equation}

\textbf{Condition 3:} $\frac{dF}{dt} = 0$ requires $\nabla (\delta F / \delta \rho) = 0$ everywhere, which occurs only at equilibrium $\rho = \rho_{\text{eq}}$.

Therefore, $F[\rho]$ is Lyapunov function, proving that $\rho_{\text{eq}}$ is stable attractor.
\end{proof}

\subsection{Convergence Rate}

\begin{theorem}[Exponential Convergence to Attractor]
\label{thm:exponential_convergence}
The distance to equilibrium decays exponentially:
\begin{equation}
\|\rho(t) - \rho_{\text{eq}}\| = \|\rho_0 - \rho_{\text{eq}}\| e^{-\lambda t}
\end{equation}
where $\lambda = \sigma / (\epsilon L^2)$ is the convergence rate.
\end{theorem}

\begin{proof}
Linearize dynamics around equilibrium:
\begin{equation}
\rho(\mathbf{r}, t) = \rho_{\text{eq}} + \delta \rho(\mathbf{r}, t)
\end{equation}

The perturbation evolution is:
\begin{equation}
\frac{\partial \delta \rho}{\partial t} = \frac{\sigma}{\epsilon} \nabla^2 \delta \rho
\end{equation}

This is diffusion equation with solution:
\begin{equation}
\delta \rho(\mathbf{r}, t) = \sum_k A_k e^{-\lambda_k t} \psi_k(\mathbf{r})
\end{equation}
where $\psi_k$ are eigenfunctions of Laplacian and:
\begin{equation}
\lambda_k = \frac{\sigma}{\epsilon} k^2
\end{equation}

The slowest decay mode has $k \sim 1/L$, giving:
\begin{equation}
\lambda_{\min} = \frac{\sigma}{\epsilon L^2}
\end{equation}

The distance to equilibrium decays as:
\begin{equation}
\|\rho(t) - \rho_{\text{eq}}\| \sim e^{-\lambda_{\min} t}
\end{equation}

Convergence is exponential with rate $\lambda_{\min} = \sigma / (\epsilon L^2)$.
\end{proof}

\subsection{Universal Dynamics}

\begin{theorem}[All Circuits Evolve Toward Charge Balance]
\label{thm:universal_evolution}
Regardless of initial configuration, circuit architecture, or coupling topology, all closed charge-coupled circuits evolve toward uniform charge distribution as universal attractor.
\end{theorem}

\begin{proof}
The evolution toward uniform distribution follows from three universal principles:

\textbf{Principle 1 (Charge conservation):} $Q_{\text{total}} = \text{const}$ in closed systems.

\textbf{Principle 2 (Variance minimization):} Systems minimize $\sigma^2[\rho]$ through thermodynamic necessity.

\textbf{Principle 3 (Free energy minimization):} Dynamics follow gradient flow $\partial \rho / \partial t = -\nabla (\delta F / \delta \rho)$.

These three principles are independent of:
\begin{itemize}
\item Circuit architecture (component arrangement)
\item Initial charge distribution $\rho_0(\mathbf{r})$
\item Coupling topology (how subsystems connect)
\item System size or geometry
\end{itemize}

All closed circuits satisfy these principles, therefore all evolve toward uniform distribution $\rho_{\text{eq}}$.

Charge balance is universal attractor for entire class of closed charge-coupled circuits.
\end{proof}

\subsection{Comparison with Open Systems}

\begin{theorem}[Attractor Reachability Distinguishes Regimes]
\label{thm:attractor_reachability}
The key distinction between closed and open systems is attractor reachability:

\textbf{Closed systems:} Attractor approached asymptotically but never reached.

\textbf{Open systems:} Attractor reached and maintained.
\end{theorem}

\begin{proof}
In closed systems, thermal fluctuations prevent exact equilibrium (Theorem~\ref{thm:asymptotic}). The system oscillates around attractor with thermal amplitude.

In open systems, dissipation to reservoir overcomes thermal fluctuations. The system reaches exact equilibrium:
\begin{equation}
\rho(t) = \rho_{\text{reservoir}} \quad \text{for} \quad t > \tau_{\text{diss}}
\end{equation}

The distinction is:
\begin{itemize}
\item Closed: $\|\rho(t) - \rho_{\text{eq}}\| \to \sqrt{k_B T / V} > 0$ (asymptotic approach)
\item Open: $\|\rho(t) - \rho_{\text{reservoir}}\| \to 0$ (exact equilibrium)
\end{itemize}

Attractor reachability distinguishes the two regimes.
\end{proof}

\subsection{Experimental Signatures}

Charge balance as universal attractor exhibits:

\begin{enumerate}
\item \textbf{Monotonic variance decrease:} $\sigma^2[\rho](t)$ decreases monotonically toward thermal floor $k_B T / V$.

\item \textbf{Exponential convergence:} Distance to equilibrium decays as $e^{-\lambda t}$ with rate $\lambda = \sigma / (\epsilon L^2)$.

\item \textbf{Oscillation around equilibrium:} Perpetual oscillation with amplitude $\sim \sqrt{k_B T / V}$ and frequency $\omega = \sigma / (\epsilon L^2)$.

\item \textbf{Universal evolution:} All initial conditions converge to same attractor regardless of architecture or coupling.

\item \textbf{Thermal floor:} Minimum achievable variance $\sigma^2_{\min} = k_B T / V$ set by thermal fluctuations.
\end{enumerate}

These signatures confirm charge balance as universal attractor for closed charge-coupled circuits.
