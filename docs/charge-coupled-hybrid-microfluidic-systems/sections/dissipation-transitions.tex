\section{Dissipation Transitions in Open Systems}
\label{sec:dissipation}

We analyze the transition from closed system dynamics (perpetual oscillation) to open system dynamics (dissipation to equilibrium) when external charge reservoirs become accessible.

\subsection{Closed vs. Open System Regimes}

\begin{definition}[Closed System]
\label{def:closed_system}
A \emph{closed system} has no access to external charge reservoirs. Total charge is conserved:
\begin{equation}
Q_{\text{total}} = \int_V \rho(\mathbf{r},t) \, d^3r = \text{const}
\end{equation}
\end{definition}

\begin{definition}[Open System]
\label{def:open_system}
An \emph{open system} has access to external charge reservoir (ground) at potential $\phi_{\text{reservoir}}$. Charge can flow to/from reservoir:
\begin{equation}
\frac{dQ}{dt} = -\gamma (Q - Q_{\text{reservoir}})
\end{equation}
where $\gamma$ is the dissipation rate.
\end{definition}

\subsection{Dissipation Dynamics}

\begin{theorem}[Exponential Charge Dissipation]
\label{thm:exponential_dissipation}
When external reservoir becomes accessible, charge dissipates exponentially:
\begin{equation}
Q(t) = Q_{\text{reservoir}} + (Q_0 - Q_{\text{reservoir}}) e^{-\gamma t}
\end{equation}
where $Q_0$ is initial charge and $\tau_{\text{diss}} = \gamma^{-1}$ is dissipation timescale.
\end{theorem}

\begin{proof}
The charge evolution equation is:
\begin{equation}
\frac{dQ}{dt} = -\gamma (Q - Q_{\text{reservoir}})
\end{equation}

This is first-order linear ODE with solution:
\begin{equation}
Q(t) - Q_{\text{reservoir}} = (Q_0 - Q_{\text{reservoir}}) e^{-\gamma t}
\end{equation}

Rearranging:
\begin{equation}
Q(t) = Q_{\text{reservoir}} + (Q_0 - Q_{\text{reservoir}}) e^{-\gamma t}
\end{equation}

For $t \to \infty$:
\begin{equation}
Q(t) \to Q_{\text{reservoir}}
\end{equation}

Charge dissipates exponentially to reservoir value on timescale $\tau_{\text{diss}} = 1/\gamma$.
\end{proof}

\subsection{Dissipation Cascade}

\begin{theorem}[Dissipation Triggers Hierarchical Collapse]
\label{thm:dissipation_cascade}
Charge dissipation triggers cascade of system degradation:
\begin{enumerate}
\item Phase coherence collapse: $R(t) \to 0$
\item Hierarchical depth reduction: $D(t) \to 0$
\item Dynamics cessation: $\|\mathbf{J}(t)\| \to 0$
\end{enumerate}
\end{theorem}

\begin{proof}
\textbf{Step 1 (Phase coherence collapse):}

Phase coherence is maintained by coupling energy:
\begin{equation}
E_{\text{coupling}} \sim Q^2 / C
\end{equation}
where $C$ is capacitance.

As charge dissipates ($Q \to Q_{\text{reservoir}}$), coupling energy decreases. When $E_{\text{coupling}} < k_B T$ (thermal energy), phase coherence collapses:
\begin{equation}
R(t) = R_0 e^{-t/\tau_{\text{phase}}} \to 0
\end{equation}

The phase coherence timescale is:
\begin{equation}
\tau_{\text{phase}} = \frac{C k_B T}{2 \gamma Q_0^2}
\end{equation}

\textbf{Step 2 (Hierarchical depth reduction):}

Hierarchical depth depends on multi-scale flux:
\begin{equation}
D = \frac{1}{n} \sum_{i=1}^{n} \mathbb{1}[F_i > F_{\text{threshold}}]
\end{equation}

Flux at scale $i$ is:
\begin{equation}
F_i \sim Q \cdot \omega_i
\end{equation}

As charge dissipates, flux decreases at all scales. Scales fall below threshold sequentially, starting from finest scales:
\begin{equation}
D(t) = D_0 e^{-t/\tau_{\text{depth}}} \to 0
\end{equation}

The depth reduction timescale is:
\begin{equation}
\tau_{\text{depth}} = \frac{1}{\gamma} \ln\left(\frac{Q_0}{Q_{\text{threshold}}}\right)
\end{equation}

\textbf{Step 3 (Dynamics cessation):}

Current flux is:
\begin{equation}
\mathbf{J} = -\sigma \nabla \phi \propto Q
\end{equation}

As charge dissipates to reservoir value, potential gradients vanish:
\begin{equation}
\nabla \phi \to 0 \quad \Rightarrow \quad \mathbf{J} \to 0
\end{equation}

Dynamics cease when $\|\mathbf{J}\| < J_{\text{threshold}}$.

The cascade proceeds sequentially: charge dissipation → phase collapse → depth reduction → dynamics cessation.
\end{proof}

\subsection{Critical Thresholds}

\begin{definition}[Critical Charge]
\label{def:critical_charge}
The \emph{critical charge} $Q_{\text{critical}}$ is the minimum charge required to sustain dynamics:
\begin{equation}
Q_{\text{critical}} = \sqrt{\frac{C k_B T}{K_{\text{coupling}}}}
\end{equation}
where $K_{\text{coupling}}$ is coupling strength.
\end{definition}

\begin{theorem}[Irreversibility Threshold]
\label{thm:irreversibility}
Once charge dissipates below critical threshold $Q < Q_{\text{critical}}$, dynamics cannot resume. The transition is irreversible.
\end{theorem}

\begin{proof}
For $Q < Q_{\text{critical}}$, coupling energy is insufficient to maintain phase coherence:
\begin{equation}
E_{\text{coupling}} = \frac{Q^2}{2C} < \frac{Q_{\text{critical}}^2}{2C} = \frac{k_B T}{2}
\end{equation}

Thermal fluctuations dominate coupling, preventing synchronization. Phase coherence remains $R < 0.1$ (decoherent).

Without phase coherence, hierarchical depth cannot be sustained:
\begin{equation}
D < D_{\text{critical}} \approx 0.4
\end{equation}

Without hierarchical depth, autocatalytic redistribution (Section~\ref{sec:autocatalytic}) cannot operate. Dynamics remain ceased.

To resume dynamics requires:
\begin{enumerate}
\item Recharge system: $Q \to Q_0 > Q_{\text{critical}}$
\item Re-establish phase coherence: $R \to R_0 > 0.8$
\item Rebuild hierarchy: $D \to D_0 > 0.6$
\end{enumerate}

But in open systems with ground access, any recharge immediately dissipates back to reservoir:
\begin{equation}
Q(t) \to Q_{\text{reservoir}} < Q_{\text{critical}}
\end{equation}

The system is trapped in dissipated state. Transition is irreversible.
\end{proof}

\subsection{Equilibrium with Reservoir}

\begin{definition}[Thermodynamic Equilibrium]
\label{def:thermo_equilibrium}
A system is in \emph{thermodynamic equilibrium} with reservoir when:
\begin{enumerate}
\item Charge balance: $Q = Q_{\text{reservoir}}$
\item Potential balance: $\phi = \phi_{\text{reservoir}}$
\item Zero flux: $\mathbf{J} = 0$
\item Maximum entropy: $S = S_{\max}$
\end{enumerate}
\end{definition}

\begin{theorem}[Equilibrium is Stable Fixed Point]
\label{thm:equilibrium_stable}
Thermodynamic equilibrium with reservoir is stable fixed point. Small perturbations decay exponentially.
\end{theorem}

\begin{proof}
Consider small perturbation from equilibrium:
\begin{equation}
Q(t) = Q_{\text{reservoir}} + \delta Q(t)
\end{equation}

The evolution of perturbation is:
\begin{equation}
\frac{d(\delta Q)}{dt} = -\gamma \delta Q
\end{equation}

Solution:
\begin{equation}
\delta Q(t) = \delta Q_0 e^{-\gamma t} \to 0
\end{equation}

Perturbations decay exponentially with timescale $\tau_{\text{diss}} = 1/\gamma$. Equilibrium is stable.

Any attempt to drive system away from equilibrium (by adding charge, creating potential gradients) is counteracted by dissipation to reservoir. System returns to equilibrium.
\end{proof}

\subsection{Comparison of Closed and Open Dynamics}

\begin{theorem}[Qualitative Distinction Between Regimes]
\label{thm:regime_distinction}
Closed and open systems exhibit qualitatively different dynamics:

\textbf{Closed system (no reservoir):}
\begin{itemize}
\item $Q = \text{const}$ (charge conserved)
\item Perpetual oscillation
\item $R > 0.8$ (phase coherence maintained)
\item $D \approx 1$ (full hierarchy maintained)
\item $\sigma^2[\rho] > 0$ (non-zero variance)
\item Equilibrium never reached
\end{itemize}

\textbf{Open system (reservoir accessible):}
\begin{itemize}
\item $Q \to Q_{\text{reservoir}}$ (charge dissipates)
\item Dynamics cease
\item $R \to 0$ (phase coherence collapses)
\item $D \to 0$ (hierarchy collapses)
\item $\sigma^2[\rho] \to 0$ (variance vanishes)
\item Equilibrium reached and stable
\end{itemize}
\end{theorem}

\begin{proof}
The distinction arises from charge conservation constraint.

In closed systems, charge cannot dissipate externally. Variance minimization drives internal redistribution, creating autocatalytic cycle (Section~\ref{sec:autocatalytic}). The system oscillates perpetually around equilibrium without reaching it.

In open systems, charge dissipates to reservoir. Variance minimization is satisfied by dissipation rather than redistribution. The autocatalytic cycle breaks down. The system reaches static equilibrium with reservoir.

The transition from closed to open regime is discontinuous: opening access to reservoir qualitatively changes system behavior from oscillatory to dissipative.
\end{proof}

\subsection{Timescale Hierarchy}

The dissipation cascade operates on multiple timescales:

\begin{equation}
\tau_{\text{diss}} < \tau_{\text{phase}} < \tau_{\text{depth}}
\end{equation}

\begin{enumerate}
\item \textbf{Charge dissipation:} $\tau_{\text{diss}} = 1/\gamma \sim 10^{-9}$ s (nanosecond)

\item \textbf{Phase coherence collapse:} $\tau_{\text{phase}} = C k_B T / (2 \gamma Q_0^2) \sim 10^{-8}$ s (tens of nanoseconds)

\item \textbf{Hierarchical depth reduction:} $\tau_{\text{depth}} = \gamma^{-1} \ln(Q_0 / Q_{\text{threshold}}) \sim 10^{-7}$ s (hundreds of nanoseconds)
\end{enumerate}

The cascade proceeds sequentially on these timescales.

\subsection{Irreversibility and Entropy Production}

\begin{theorem}[Dissipation Produces Entropy]
\label{thm:dissipation_entropy}
Dissipation to reservoir generates entropy:
\begin{equation}
\Delta S_{\text{dissipation}} = \int_0^{\infty} \frac{\gamma (Q(t) - Q_{\text{reservoir}})^2}{T} dt
\end{equation}
This entropy is irreversibly dissipated to environment.
\end{theorem}

\begin{proof}
The entropy production rate is:
\begin{equation}
\frac{dS}{dt} = \frac{1}{T} \frac{dQ}{dt} (\phi - \phi_{\text{reservoir}})
\end{equation}

Using $dQ/dt = -\gamma (Q - Q_{\text{reservoir}})$ and $\phi - \phi_{\text{reservoir}} \propto (Q - Q_{\text{reservoir}})$:
\begin{equation}
\frac{dS}{dt} = \frac{\gamma (Q - Q_{\text{reservoir}})^2}{T}
\end{equation}

Total entropy produced:
\begin{equation}
\Delta S_{\text{dissipation}} = \int_0^{\infty} \frac{\gamma (Q - Q_{\text{reservoir}})^2}{T} dt
\end{equation}

Using $Q(t) = Q_{\text{reservoir}} + (Q_0 - Q_{\text{reservoir}}) e^{-\gamma t}$:
\begin{equation}
\Delta S_{\text{dissipation}} = \int_0^{\infty} \frac{\gamma (Q_0 - Q_{\text{reservoir}})^2 e^{-2\gamma t}}{T} dt = \frac{(Q_0 - Q_{\text{reservoir}})^2}{2T}
\end{equation}

This entropy is dissipated to environment and cannot be recovered. The dissipation is irreversible.
\end{proof}

\subsection{Experimental Signatures}

Dissipation transitions exhibit:

\begin{enumerate}
\item \textbf{Exponential charge decay:} $Q(t) = Q_{\text{reservoir}} + (Q_0 - Q_{\text{reservoir}}) e^{-\gamma t}$

\item \textbf{Phase coherence collapse:} $R(t)$ drops from $> 0.8$ to $< 0.1$ on timescale $\tau_{\text{phase}}$

\item \textbf{Hierarchical depth reduction:} $D(t)$ decreases from $\approx 1$ to $< 0.4$ on timescale $\tau_{\text{depth}}$

\item \textbf{Dynamics cessation:} Current flux $\|\mathbf{J}(t)\|$ drops below detection threshold

\item \textbf{Irreversibility:} Systems with $Q < Q_{\text{critical}}$ cannot resume dynamics without external recharge

\item \textbf{Entropy production:} Measurable heat dissipation $\Delta S_{\text{dissipation}} \cdot T$ during transition
\end{enumerate}

These signatures distinguish dissipation transitions from transient perturbations or temporary decoherence.
