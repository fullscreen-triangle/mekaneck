\section{Circuit Equations of State for Five Regimes}
\label{sec:circuit_regimes}

We derive equations of state for five distinct operational regimes of hybrid microfluidic circuits. All equations reduce to the universal form $PV = N\kB T \cdot \mathcal{S}(V,N,\{n_i,\ell_i,m_i,s_i\})$ where $\mathcal{S}$ is a temperature-independent structural factor encoding partition geometry.

\subsection{Regime 1: Coherent Flow Circuits}

Coherent flow circuits exhibit high phase coherence $R > 0.8$ with synchronized oscillatory dynamics across all hierarchical scales.

\subsubsection{Physical Characteristics}

\textbf{Phase coherence}: $R = N^{-1}|\sum_{j=1}^N e^{i\phi_j}| > 0.8$

\textbf{Phase variance}: $\sigma^2(\phi) < 0.1$ rad$^2$

\textbf{Hierarchical depth}: $D = 1.0$ (all scales active)

\textbf{Coupling strength}: $K_{\text{coupling}} > \sigma(\omega)$ (exceeds frequency variance)

\textbf{Timescale separation}: Clear 10$\times$ separation between hierarchical levels

\subsubsection{Partition Function Derivation}

For coherent flow, oscillators phase-lock into collective modes. The partition function is:
\begin{equation}
Z_{\text{coherent}} = \sum_{\{n,\ell,m,s\}} g(n,\ell,m,s) \exp\left(-\frac{E(n,\ell,m,s)}{\kB T}\right) \times \left(1 + \frac{R^2}{1-R^2}\right)
\end{equation}

The coherence enhancement factor $(1 + R^2/(1-R^2))$ accounts for reduced entropy due to phase-locking. As $R \to 1$ (perfect coherence), this factor diverges, reflecting the singular nature of complete synchronization.

\subsubsection{Equation of State}

From the partition function:
\begin{equation}
P = \kB T \frac{\partial \ln Z_{\text{coherent}}}{\partial V}
\end{equation}

For ideal gas baseline with coherence correction:
\begin{equation}
PV = N\kB T \cdot \left(1 + \frac{R^2}{1-R^2}\right)
\label{eq:eos_coherent}
\end{equation}

\textbf{Structural factor}:
\begin{equation}
\mathcal{S}_{\text{coherent}}(R) = 1 + \frac{R^2}{1-R^2}
\end{equation}

This is temperature-independent, depending only on phase coherence $R$.

\subsubsection{Limiting Behavior}

\textbf{Low coherence} ($R \to 0$): $\mathcal{S}_{\text{coherent}} \to 1$ (ideal gas)

\textbf{High coherence} ($R \to 1$): $\mathcal{S}_{\text{coherent}} \to \infty$ (phase transition)

\textbf{Typical coherent flow} ($R = 0.85$): $\mathcal{S}_{\text{coherent}} = 1 + 0.72/0.28 \approx 3.6$

\subsubsection{Internal Energy}

\begin{equation}
U = -\frac{\partial \ln Z_{\text{coherent}}}{\partial \beta} = \frac{3}{2}N\kB T \cdot \left(1 + \frac{2R^2}{(1-R^2)^2}\right)
\end{equation}

where $\beta = 1/(\kB T)$. The coherence term increases internal energy due to collective mode excitations.

\subsubsection{Entropy}

\begin{equation}
S_{\text{coherent}} = \kB \ln Z_{\text{coherent}} + \frac{U}{T} = N\kB \left[\frac{3}{2}\ln T + \ln\left(1 + \frac{R^2}{1-R^2}\right)\right] + \text{const}
\end{equation}

Coherence reduces entropy through the logarithmic term.

\subsubsection{Chemical Potential}

\begin{equation}
\mu_{\text{coherent}} = -\kB T \ln\left(\frac{Z_{\text{coherent}}}{N}\right) = -\kB T \ln\left(\frac{V}{N\lambda_T^3}\right) - \kB T \ln\left(1 + \frac{R^2}{1-R^2}\right)
\end{equation}

where $\lambda_T = h/\sqrt{2\pi m \kB T}$ is thermal de Broglie wavelength.

\subsubsection{Experimental Signatures}

\textbf{(1) Pressure enhancement}: $P/P_{\text{ideal}} = 1 + R^2/(1-R^2) \approx 3.6$ for $R = 0.85$

\textbf{(2) Specific heat}: $C_V = \partial U/\partial T$ shows anomaly near $R = 1$

\textbf{(3) Compressibility}: $\kappa_T = -V^{-1}(\partial V/\partial P)_T$ reduced by factor $\mathcal{S}_{\text{coherent}}$

\textbf{(4) Sound speed}: $c_s = \sqrt{(\partial P/\partial \rho)_S}$ enhanced by coherence

\subsection{Regime 2: Turbulent Flow Circuits}

Turbulent circuits exhibit low phase coherence $R < 0.3$ with chaotic dynamics and large phase variance.

\subsubsection{Physical Characteristics}

\textbf{Phase coherence}: $R < 0.3$

\textbf{Phase variance}: $\sigma^2(\phi) > 2.0$ rad$^2$

\textbf{Hierarchical depth}: $D < 0.4$ (cascade failure at intermediate scales)

\textbf{Coupling strength}: $K_{\text{coupling}} < \sigma(\omega)$ (insufficient for phase-locking)

\textbf{Lyapunov exponent}: $\lambda > 0$ (positive, indicating chaos)

\subsubsection{Partition Function Derivation}

For turbulent flow, phase variance reduces accessible states. The partition function is:
\begin{equation}
Z_{\text{turbulent}} = \sum_{\{n,\ell,m,s\}} g(n,\ell,m,s) \exp\left(-\frac{E(n,\ell,m,s)}{\kB T}\right) \times \exp\left(-\frac{\sigma^2(\phi)}{2\pi^2}\right)
\end{equation}

The variance suppression factor $\exp(-\sigma^2(\phi)/(2\pi^2))$ accounts for entropy reduction due to chaotic fluctuations preventing access to ordered states.

\subsubsection{Equation of State}

\begin{equation}
PV = N\kB T \cdot \left(1 - \frac{\sigma^2(\phi)}{2\pi^2}\right)
\label{eq:eos_turbulent}
\end{equation}

\textbf{Structural factor}:
\begin{equation}
\mathcal{S}_{\text{turbulent}}(\sigma^2) = 1 - \frac{\sigma^2(\phi)}{2\pi^2}
\end{equation}

This is temperature-independent, depending only on phase variance.

\subsubsection{Limiting Behavior}

\textbf{Low variance} ($\sigma^2 \to 0$): $\mathcal{S}_{\text{turbulent}} \to 1$ (ideal gas)

\textbf{Maximum variance} ($\sigma^2 \to 2\pi^2$): $\mathcal{S}_{\text{turbulent}} \to 0$ (complete disorder)

\textbf{Typical turbulent flow} ($\sigma^2 = 2.3$ rad$^2$): $\mathcal{S}_{\text{turbulent}} = 1 - 2.3/19.7 \approx 0.88$

\subsubsection{Internal Energy}

\begin{equation}
U = \frac{3}{2}N\kB T \cdot \left(1 + \frac{\sigma^2(\phi)}{\pi^2}\right)
\end{equation}

Variance increases internal energy through chaotic fluctuations.

\subsubsection{Entropy}

\begin{equation}
S_{\text{turbulent}} = N\kB \left[\frac{3}{2}\ln T - \frac{\sigma^2(\phi)}{2\pi^2}\right] + \text{const}
\end{equation}

Paradoxically, turbulence reduces thermodynamic entropy by restricting accessible phase space.

\subsubsection{Hierarchical Depth Collapse}

Turbulent circuits exhibit cascade failure. Hierarchical depth:
\begin{equation}
D = \frac{1}{n}\sum_{i=1}^n \mathbb{1}[F_i > F_{\text{threshold}}]
\end{equation}

For turbulent flow with $\sigma^2 > 2.0$:
\begin{equation}
D \approx 0.35 \pm 0.05
\end{equation}

Only the first 2 levels remain active; levels 3-5 fail due to insufficient coupling.

\subsubsection{Experimental Signatures}

\textbf{(1) Pressure reduction}: $P/P_{\text{ideal}} \approx 0.88$ for typical turbulence

\textbf{(2) Broad spectral lines}: Frequency spectrum shows continuous distribution

\textbf{(3) Intermittency}: Temporal dynamics exhibit bursting behavior

\textbf{(4) Mixing enhancement}: Diffusion coefficient increases by factor $\sim 10^2$

\subsection{Regime 3: Hierarchical Cascade Circuits}

Multi-scale circuits with information compression across hierarchical levels through flux cascades.

\subsubsection{Physical Characteristics}

\textbf{Hierarchical structure}: $n$ distinct temporal scales with 10$\times$ separation

\textbf{Flux ratios}: $F_i^{\text{out}}/F_i^{\text{in}} < 1$ at each level

\textbf{Information compression}: $I = \sum_i \alpha_i \log_2(F_i^{\text{in}}/F_i^{\text{out}})$ bits

\textbf{Depth}: $D \in [0.4, 1.0]$ depending on cascade integrity

\textbf{Coupling hierarchy}: $K_i$ varies across scales

\subsubsection{Partition Function Derivation}

For hierarchical cascades, each level contributes multiplicatively:
\begin{equation}
Z_{\text{cascade}} = \prod_{i=1}^n Z_i = \prod_{i=1}^n \sum_{\{n_i,\ell_i,m_i,s_i\}} g_i \exp\left(-\frac{E_i}{\kB T}\right) \times \left(1 + \frac{F_i^{\text{out}}}{F_i^{\text{in}}}\right)
\end{equation}

The flux ratio factor accounts for state space reduction at each level.

\subsubsection{Equation of State}

\begin{equation}
PV = N\kB T \cdot \prod_{i=1}^n \left(1 + \frac{F_i^{\text{out}}}{F_i^{\text{in}}}\right)
\label{eq:eos_cascade}
\end{equation}

\textbf{Structural factor}:
\begin{equation}
\mathcal{S}_{\text{cascade}}(\{F_i\}) = \prod_{i=1}^n \left(1 + \frac{F_i^{\text{out}}}{F_i^{\text{in}}}\right)
\end{equation}

This is temperature-independent, depending only on flux ratios.

\subsubsection{Information-Thermodynamic Connection}

Information compression at level $i$:
\begin{equation}
I_i = \alpha_i \log_2\left(\frac{F_i^{\text{in}}}{F_i^{\text{out}}}\right)
\end{equation}

Total information:
\begin{equation}
I_{\text{total}} = \sum_{i=1}^n I_i = \sum_{i=1}^n \alpha_i \log_2\left(\frac{F_i^{\text{in}}}{F_i^{\text{out}}}\right)
\end{equation}

For healthy cascade with $n=5$ levels and typical flux ratios:
\begin{equation}
I_{\text{total}} \approx 7-9 \text{ bits}
\end{equation}

\subsubsection{Cascade Failure Criterion}

Cascade fails when flux at level $i$ drops below threshold:
\begin{equation}
F_i < F_{\text{threshold}} = 0.1 \times F_i^{\text{baseline}}
\end{equation}

This causes all downstream levels ($j > i$) to fail, reducing hierarchical depth:
\begin{equation}
D_{\text{failed}} = \frac{i-1}{n}
\end{equation}

\subsubsection{Internal Energy}

\begin{equation}
U = \frac{3}{2}N\kB T \cdot \sum_{i=1}^n \left(1 + \frac{F_i^{\text{out}}}{F_i^{\text{in}}}\right)
\end{equation}

Each level contributes additively to internal energy.

\subsubsection{Free Energy as Trajectory Functional}

Helmholtz free energy:
\begin{equation}
F[\gamma] = \int_{\gamma} \left(U(\Scoord) - TS(\Scoord)\right) d\ell - \sum_{i=1}^n \kB T \ln\left(1 + \frac{F_i^{\text{out}}}{F_i^{\text{in}}}\right)
\end{equation}

Minimization yields equilibrium flux ratios.

\subsubsection{Experimental Signatures}

\textbf{(1) Pressure scaling}: $P \propto \prod_i (1 + F_i^{\text{out}}/F_i^{\text{in}})$

\textbf{(2) Multi-scale coherence}: Phase coherence $R_i$ measured at each scale

\textbf{(3) Information capacity}: Measured through entropy production rates

\textbf{(4) Cascade integrity}: Depth $D$ measured through flux tracing

\subsection{Regime 4: Aperture-Dominated Circuits}

Circuits where geometric confinement through molecular apertures dominates dynamics.

\subsubsection{Physical Characteristics}

\textbf{Aperture density}: $\rho_{\mathcal{A}} = N_{\mathcal{A}}/V$ (apertures per volume)

\textbf{Partition depth}: $n$ determines aperture capacity $C(n) = 2n^2$

\textbf{Variance selection}: $\sigma^2(\phi|\mathcal{A}) < \sigma^2_{\text{threshold}}$

\textbf{Catalytic reduction}: Factor $\sim 10^{38}$ from $10^{44}$ to $10^6$ states

\textbf{Composition}: Non-commutative aperture algebra

\subsubsection{Partition Function Derivation}

For aperture-dominated circuits, accessible states limited by partition capacity:
\begin{equation}
Z_{\text{aperture}} = \sum_{n=1}^{n_{\max}} C(n) \exp\left(-\frac{E(n)}{\kB T}\right) = \sum_{n=1}^{n_{\max}} 2n^2 \exp\left(-\frac{E(n)}{\kB T}\right)
\end{equation}

The capacity $C(n) = 2n^2$ appears explicitly as degeneracy factor.

\subsubsection{Equation of State}

\begin{equation}
PV = N\kB T \cdot \frac{C(n)}{C_{\max}} = N\kB T \cdot \frac{2n^2}{2n_{\max}^2}
\label{eq:eos_aperture}
\end{equation}

\textbf{Structural factor}:
\begin{equation}
\mathcal{S}_{\text{aperture}}(n) = \frac{n^2}{n_{\max}^2}
\end{equation}

This is temperature-independent, depending only on partition depth.

\subsubsection{Aperture Composition Effects}

When apertures compose $\mathcal{A}_1 \otimes \mathcal{A}_2$, effective capacity:
\begin{equation}
C_{\text{eff}} = C_1 \times C_2 \times (1 + K_{12}\cos(\Delta\phi_{12}))
\end{equation}

where $K_{12}$ is coupling strength and $\Delta\phi_{12}$ is phase difference.

Constructive interference ($\Delta\phi_{12} = 0$): $C_{\text{eff}} = C_1 C_2 (1 + K_{12})$

Destructive interference ($\Delta\phi_{12} = \pi$): $C_{\text{eff}} = C_1 C_2 (1 - K_{12})$

\subsubsection{Internal Energy}

\begin{equation}
U = \frac{3}{2}N\kB T \cdot \left(1 + \frac{2n}{n_{\max}}\right)
\end{equation}

Partition depth increases energy through aperture confinement.

\subsubsection{Entropy}

\begin{equation}
S_{\text{aperture}} = N\kB \left[\frac{3}{2}\ln T + 2\ln n - 2\ln n_{\max}\right] + \text{const}
\end{equation}

Entropy increases with partition depth (more accessible states).

\subsubsection{Chemical Potential}

\begin{equation}
\mu_{\text{aperture}} = -\kB T \ln\left(\frac{V}{N\lambda_T^3}\right) - 2\kB T \ln\left(\frac{n}{n_{\max}}\right)
\end{equation}

Apertures reduce chemical potential by increasing accessible states.

\subsubsection{Experimental Signatures}

\textbf{(1) Pressure scaling}: $P \propto n^2$ (quadratic in partition depth)

\textbf{(2) Capacity sequence}: $C(n) = 2, 8, 18, 32, 50, \ldots$ observable in spectroscopy

\textbf{(3) Variance reduction}: $\sigma^2(\phi)$ drops by factor $\sim 10^2$ in apertures

\textbf{(4) Catalytic efficiency}: Measured through state space reduction

\subsection{Regime 5: Phase-Locked Network Circuits}

Circuits exhibiting Kuramoto synchronization with coupling-dependent coherence.

\subsubsection{Physical Characteristics}

\textbf{Network topology}: Graph $\mathcal{G} = (\mathcal{V}, \mathcal{E})$ with $N$ nodes

\textbf{Coupling matrix}: $K_{ij}$ for edges $(i,j) \in \mathcal{E}$

\textbf{Frequency distribution}: $g(\omega)$ with variance $\sigma(\omega)$

\textbf{Order parameter}: $R = N^{-1}|\sum_j e^{i\phi_j}|$

\textbf{Critical coupling}: $K_c = 2/(\pi g(0))$ for synchronization transition

\subsubsection{Partition Function Derivation}

For phase-locked networks, coupling strength determines accessible states:
\begin{equation}
Z_{\text{sync}} = \sum_{\{n,\ell,m,s\}} g(n,\ell,m,s) \exp\left(-\frac{E(n,\ell,m,s)}{\kB T}\right) \times \left(1 + \frac{K_{\text{coupling}}}{\sigma(\omega)}\right)
\end{equation}

The coupling enhancement factor $(1 + K_{\text{coupling}}/\sigma(\omega))$ accounts for synchronization-induced state reduction.

\subsubsection{Equation of State}

\begin{equation}
PV = N\kB T \cdot \left(1 + \frac{K_{\text{coupling}}}{\sigma(\omega)}\right)
\label{eq:eos_sync}
\end{equation}

\textbf{Structural factor}:
\begin{equation}
\mathcal{S}_{\text{sync}}(K) = 1 + \frac{K_{\text{coupling}}}{\sigma(\omega)}
\end{equation}

This is temperature-independent, depending only on coupling-to-variance ratio.

\subsubsection{Synchronization Transition}

Phase transition occurs at critical coupling:
\begin{equation}
K_c = \frac{2}{\pi g(0)}
\end{equation}

For $K < K_c$: Incoherent state with $R \approx 0$

For $K > K_c$: Partially synchronized with $R = \sqrt{1 - K_c/K}$

For $K \gg K_c$: Fully synchronized with $R \to 1$

\subsubsection{Order Parameter Evolution}

Near critical point:
\begin{equation}
R \sim (K - K_c)^{\beta}
\end{equation}

with critical exponent $\beta = 1/2$ (mean-field universality class).

\subsubsection{Internal Energy}

\begin{equation}
U = \frac{3}{2}N\kB T \cdot \left(1 + \frac{2K_{\text{coupling}}}{\sigma(\omega)}\right)
\end{equation}

Coupling increases energy through collective mode excitations.

\subsubsection{Entropy}

\begin{equation}
S_{\text{sync}} = N\kB \left[\frac{3}{2}\ln T + \ln\left(1 + \frac{K_{\text{coupling}}}{\sigma(\omega)}\right)\right] + \text{const}
\end{equation}

Synchronization increases entropy through enhanced phase space accessibility.

\subsubsection{Network Topology Effects}

\textbf{Complete graph}: All-to-all coupling, $K_{\text{eff}} = K$

\textbf{Ring lattice}: Nearest-neighbor coupling, $K_{\text{eff}} = K/2$

\textbf{Small-world}: Shortcuts enhance synchronization, $K_{\text{eff}} = K(1 + p)$

\textbf{Scale-free}: Hub nodes dominate, $K_{\text{eff}} = K\langle k^2\rangle/\langle k\rangle$

where $\langle k\rangle$ is mean degree and $\langle k^2\rangle$ is second moment.

\subsubsection{Experimental Signatures}

\textbf{(1) Pressure enhancement}: $P/P_{\text{ideal}} = 1 + K/\sigma(\omega)$

\textbf{(2) Critical behavior}: Power-law scaling near $K_c$

\textbf{(3) Hysteresis}: First-order transition for certain network topologies

\textbf{(4) Chimera states}: Coexisting synchronized and desynchronized regions

\subsection{Universal Form and Temperature Scaling}

All five regimes reduce to the universal form:
\begin{equation}
PV = N\kB T \cdot \mathcal{S}(V,N,\{n_i,\ell_i,m_i,s_i\})
\end{equation}

where the structural factor $\mathcal{S}$ is:

\begin{center}
\begin{tabular}{ll}
\toprule
\textbf{Regime} & \textbf{Structural Factor} \\
\midrule
Coherent flow & $\mathcal{S} = 1 + R^2/(1-R^2)$ \\
Turbulent flow & $\mathcal{S} = 1 - \sigma^2(\phi)/(2\pi^2)$ \\
Hierarchical cascade & $\mathcal{S} = \prod_i (1 + F_i^{\text{out}}/F_i^{\text{in}})$ \\
Aperture-dominated & $\mathcal{S} = n^2/n_{\max}^2$ \\
Phase-locked network & $\mathcal{S} = 1 + K_{\text{coupling}}/\sigma(\omega)$ \\
\bottomrule
\end{tabular}
\end{center}

\textbf{Key observation}: All structural factors are temperature-independent, confirming that temperature functions as a universal scaling factor rather than a structural parameter.

\subsection{Thermodynamic Consistency}

All five equations satisfy:

\textbf{(1) Maxwell relations}:
\begin{equation}
\left(\frac{\partial S}{\partial V}\right)_T = \left(\frac{\partial P}{\partial T}\right)_V
\end{equation}

\textbf{(2) Stability criteria}:
\begin{equation}
\left(\frac{\partial P}{\partial V}\right)_T < 0, \quad C_V > 0, \quad \kappa_T > 0
\end{equation}

\textbf{(3) Third law}: $S \to 0$ as $T \to 0$

\textbf{(4) Extensivity}: $S(N,V,T) = N s(v,T)$ where $v = V/N$

These consistency checks confirm the equations are thermodynamically valid.
