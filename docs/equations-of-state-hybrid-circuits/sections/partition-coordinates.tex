\section{Partition Coordinates from Finite Observational Resolution}
\label{sec:partition_coordinates}

We derive a coordinate system for labeling distinguishable states in bounded hybrid microfluidic circuits. The derivation proceeds from geometric constraints on nested spherical partitions without invoking quantum mechanical postulates.

\subsection{Axiomatic Foundation}

\begin{axiom}[Bounded Phase Space]
\label{axiom:bounded_circuit}
Every hybrid microfluidic circuit observable for finite time $t_{\text{obs}}$ occupies a bounded region of phase space. There exist finite constants $L$, $E_{\max}$, and $T$ such that:
\begin{enumerate}[nosep]
\item \textbf{Spatial boundedness}: All position coordinates satisfy $|q_i| \leq L$ where $L < \infty$
\item \textbf{Energetic boundedness}: Total energy satisfies $E \leq E_{\max} < \infty$
\item \textbf{Temporal boundedness}: Any distinguishable process completes within $T < \infty$
\end{enumerate}
\end{axiom}

\begin{axiom}[Finite Observational Resolution]
\label{axiom:resolution_circuit}
Any observation distinguishes among finite alternatives. For observable $Q$ and measurement $\mathcal{M}$, there exists finite set $\{q_1, \ldots, q_n\}$ with $n < \infty$.

Equivalently, phase space $\mathcal{M} \subset \mathbb{R}^{2d}$ partitions into finite cells:
\begin{equation}
\mathcal{M} = \bigcup_{k=1}^{n} C_k
\end{equation}
where cells are mutually exclusive, exhaustive, and finite in number.
\end{axiom}

\begin{remark}
With finite resolution $(\Delta q > 0, \Delta p > 0)$ and bounded phase space, distinguishable states number $n = \Omega/(\Delta q \cdot \Delta p) < \infty$ where $\Omega$ is phase space volume.
\end{remark}

\subsection{Radial Partition Depth}

\begin{definition}[Principal Partition Coordinate]
For circuit with spatial extent $L$ and radial resolution $\Delta r$, the \textit{radial partition depth} is:
\begin{equation}
n = \frac{L}{\Delta r}
\end{equation}
\end{definition}

\begin{proposition}[Shell Volume Scaling]
Shell $n$ at radius $r \in [(n-1)\Delta r, n\Delta r]$ has volume:
\begin{equation}
V_n = 4\pi(\Delta r)^3(3n^2 - 3n + 1) \approx 4\pi n^2 (\Delta r)^3
\end{equation}
for large $n$.
\end{proposition}

\begin{proof}
Volume of sphere with radius $r$ is $V(r) = (4\pi/3)r^3$. Shell $n$ volume is:
\begin{align}
V_n &= V(n\Delta r) - V((n-1)\Delta r) \\
&= \frac{4\pi}{3}\left[(n\Delta r)^3 - ((n-1)\Delta r)^3\right] \\
&= \frac{4\pi}{3}(\Delta r)^3\left[n^3 - (n-1)^3\right] \\
&= \frac{4\pi}{3}(\Delta r)^3(3n^2 - 3n + 1)
\end{align}

For $n \gg 1$, dominant term is $3n^2$:
\begin{equation}
V_n \approx 4\pi n^2 (\Delta r)^3
\end{equation}
\end{proof}

\begin{corollary}[Quadratic Scaling]
Shell volume scales quadratically with partition depth: $V_n \propto n^2$.
\end{corollary}

\subsection{Angular Complexity Coordinate}

\begin{definition}[Angular Complexity]
Within shell $n$, angular momentum states define the \textit{angular complexity coordinate} $\ell$, satisfying:
\begin{equation}
\ell \in \{0, 1, \ldots, n-1\}
\end{equation}
\end{definition}

\begin{theorem}[Angular Constraint]
\label{thm:angular_constraint}
Angular complexity cannot exceed radial depth: $\ell < n$.
\end{theorem}

\begin{proof}
Maximum angular momentum at radius $r_n = n\Delta r$ with momentum $p_{\max}$ is:
\begin{equation}
L_{\max} = r_n \cdot p_{\max} = n\Delta r \cdot p_{\max}
\end{equation}

Quantized angular momentum is $L = \hbar \ell$. Therefore:
\begin{equation}
\hbar \ell \leq n\Delta r \cdot p_{\max}
\end{equation}

Solving for $\ell$:
\begin{equation}
\ell \leq \frac{n\Delta r \cdot p_{\max}}{\hbar}
\end{equation}

For consistency with geometric constraints, we require $\ell < n$, yielding:
\begin{equation}
\Delta r \cdot p_{\max} < \hbar
\end{equation}

This is satisfied when resolution $\Delta r$ and momentum $p_{\max}$ respect the uncertainty principle.
\end{proof}

\subsection{Orientation Coordinate}

\begin{definition}[Orientation]
The angular momentum projection onto chosen axis defines the \textit{orientation coordinate} $m$:
\begin{equation}
m \in \{-\ell, -\ell+1, \ldots, \ell-1, \ell\}
\end{equation}
\end{definition}

\begin{proposition}[Orientation Count]
For angular complexity $\ell$, there are $2\ell + 1$ distinguishable orientations.
\end{proposition}

\begin{proof}
Values of $m$ range from $-\ell$ to $+\ell$ in integer steps. Total count:
\begin{equation}
N_m = \ell - (-\ell) + 1 = 2\ell + 1
\end{equation}
\end{proof}

\subsection{Chirality Coordinate}

\begin{definition}[Chirality]
Intrinsic angular momentum (spin) defines the \textit{chirality coordinate} $s$:
\begin{equation}
s \in \begin{cases}
\{-1/2, +1/2\} & \text{fermions} \\
\{0, \pm 1, \pm 2, \ldots\} & \text{bosons}
\end{cases}
\end{equation}
\end{definition}

For hybrid microfluidic circuits with molecular constituents (fermions), $s \in \{-1/2, +1/2\}$.

\subsection{Complete Partition Coordinate System}

\begin{definition}[Partition Coordinates]
\label{def:partition_coords_circuit}
The partition coordinates $(n, \ell, m, s)$ characterize discrete states in bounded phase space:
\begin{itemize}[nosep]
\item $n \in \{1, 2, 3, \ldots\}$: radial partition depth
\item $\ell \in \{0, 1, \ldots, n-1\}$: angular complexity
\item $m \in \{-\ell, -\ell+1, \ldots, \ell\}$: orientation
\item $s \in \{-1/2, +1/2\}$: chirality (for fermions)
\end{itemize}
\end{definition}

\subsection{Capacity Theorem}

\begin{theorem}[Partition Capacity]
\label{thm:capacity_circuit}
The number of distinguishable states at partition depth $n$ is:
\begin{equation}
C(n) = 2n^2
\end{equation}
\end{theorem}

\begin{proof}
For fixed $n$, angular complexity ranges $\ell \in \{0, 1, \ldots, n-1\}$. For each $\ell$, orientation ranges $m \in \{-\ell, \ldots, +\ell\}$ (total $2\ell + 1$ values). Chirality has 2 values.

Total states:
\begin{align}
C(n) &= \sum_{\ell=0}^{n-1} (2\ell + 1) \times 2 \\
&= 2 \sum_{\ell=0}^{n-1} (2\ell + 1) \\
&= 2 \left[2\sum_{\ell=0}^{n-1} \ell + \sum_{\ell=0}^{n-1} 1\right] \\
&= 2 \left[2 \cdot \frac{(n-1)n}{2} + n\right] \\
&= 2[n(n-1) + n] \\
&= 2n^2
\end{align}
\end{proof}

\begin{corollary}[Capacity Sequence]
The capacity sequence is:
\begin{equation}
C(1) = 2, \quad C(2) = 8, \quad C(3) = 18, \quad C(4) = 32, \quad C(5) = 50, \quad \ldots
\end{equation}
\end{corollary}

\subsection{Energy Levels}

\begin{proposition}[Energy-Coordinate Relation]
Energy at partition depth $n$ scales as:
\begin{equation}
E_n = E_0 \frac{n^2}{n_{\max}^2}
\end{equation}
where $E_0$ is ground state energy and $n_{\max}$ is maximum partition depth.
\end{proposition}

\begin{proof}
For particle in spherical box with radius $L$, energy eigenvalues scale as $E_n \propto n^2$ (from radial Schrödinger equation). Normalizing to maximum energy $E_{\max}$ at $n = n_{\max}$:
\begin{equation}
E_n = E_{\max} \frac{n^2}{n_{\max}^2}
\end{equation}

Ground state energy is $E_0 = E_{\max}/n_{\max}^2$, yielding:
\begin{equation}
E_n = E_0 n^2
\end{equation}
\end{proof}

\subsection{Partition Density of States}

\begin{definition}[Density of States]
The density of states at partition depth $n$ is:
\begin{equation}
\rho(n) = \frac{dC(n)}{dn} = \frac{d(2n^2)}{dn} = 4n
\end{equation}
\end{definition}

\begin{proposition}[Linear Density Growth]
Density of states grows linearly with partition depth.
\end{proposition}

\subsection{Maximum Partition Depth}

\begin{proposition}[Maximum Depth]
For circuit with spatial extent $L$ and minimum resolution $\Delta r_{\min}$, maximum partition depth is:
\begin{equation}
n_{\max} = \frac{L}{\Delta r_{\min}}
\end{equation}
\end{proposition}

\begin{proof}
Partition depth $n = L/\Delta r$. Minimum resolution $\Delta r_{\min}$ (e.g., from uncertainty principle) yields maximum depth:
\begin{equation}
n_{\max} = \frac{L}{\Delta r_{\min}}
\end{equation}
\end{proof}

\begin{corollary}[Total State Count]
Total distinguishable states in circuit:
\begin{equation}
N_{\text{total}} = \sum_{n=1}^{n_{\max}} C(n) = \sum_{n=1}^{n_{\max}} 2n^2 = 2 \sum_{n=1}^{n_{\max}} n^2 = 2 \cdot \frac{n_{\max}(n_{\max}+1)(2n_{\max}+1)}{6} \approx \frac{2n_{\max}^3}{3}
\end{equation}
\end{corollary}

\subsection{Partition Coordinate Independence}

\begin{theorem}[Coordinate Independence]
\label{thm:coordinate_independence}
Partition coordinates $(n, \ell, m, s)$ are independent of coordinate system choice.
\end{theorem}

\begin{proof}
The coordinates arise from geometric constraints (spherical symmetry, angular momentum quantization, spin) which are invariant under coordinate transformations. Specifically:
\begin{itemize}[nosep]
\item $n$ counts radial shells (rotation invariant)
\item $\ell$ quantifies angular momentum magnitude (scalar, rotation invariant)
\item $m$ is projection onto arbitrary axis (changes under rotation, but set $\{m\}$ is invariant)
\item $s$ is intrinsic spin (Lorentz invariant)
\end{itemize}
\end{proof}

\subsection{Experimental Determination}

\textbf{(1) Mass spectrometry}: Fragment patterns reveal partition coordinates through mass-to-charge ratios.

\textbf{(2) Spectroscopy}: Electronic transitions measure $n$ (energy levels), rotational transitions measure $\ell$ (angular momentum).

\textbf{(3) Magnetic resonance}: Zeeman splitting measures $m$ (orientation), spin-spin coupling measures $s$ (chirality).

\textbf{(4) Capacity verification}: Count states at each $n$, verify $C(n) = 2n^2$.

\textbf{(5) Density of states}: Measure $\rho(n)$, verify linear growth $\rho(n) = 4n$.

This partition coordinate framework establishes that hybrid microfluidic circuits admit discrete state labeling $(n, \ell, m, s)$ arising from geometric constraints in bounded phase space, with capacity $C(n) = 2n^2$ following from nested spherical partition structure.
