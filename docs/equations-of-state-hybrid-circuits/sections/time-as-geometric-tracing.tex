\section{Time as Geometric Tracing: Circuit Completion Duration}
\label{sec:time_as_tracing}

\subsection{Overview: The Temporal Paradox}

Mathematical structures exist timelessly—a parametric curve $P(t) = A + tv$ exists "all at once" in abstract space. Yet physical circuit operation requires \textit{tracing} of geometric structures through circuit completion—a process that necessarily takes measurable duration. The subjective experience of temporal flow is not an illusion but the direct operational correlate of transport times during oscillatory hole stabilization in circuit dynamics.

This section resolves the temporal paradox by demonstrating that \textbf{time is the felt experience of geometric tracing during circuit completion events}.

\subsection{Mathematical vs. Physical Geometry}

\begin{definition}[Mathematical Geometry]
A \emph{mathematical geometric structure} is a set of points $\mathcal{G} = \{\mathbf{x}_i\}$ satisfying geometric relations $\mathcal{R}(\mathbf{x}_i, \mathbf{x}_j)$, existing timelessly in abstract space.
\end{definition}

\begin{definition}[Physical Geometry]
A \emph{physical geometric structure} is a mathematical geometry $\mathcal{G}$ instantiated through physical processes requiring temporal evolution.
\end{definition}

\begin{theorem}[Geometric Manifestation Theorem]
\label{thm:geometric_manifestation}
Physical instantiation of mathematical geometry requires temporal tracing. For geometry $\mathcal{G}$ with $N$ points, instantiation time is:
\begin{equation}
T_{\text{trace}} = \sum_{i=1}^N \tau_{\text{circuit}}^{(i)}
\end{equation}
where $\tau_{\text{circuit}}^{(i)}$ is the circuit completion time for point $i$.
\end{theorem}

\begin{proof}
Mathematical geometry exists as complete structure: all points $\{\mathbf{x}_i\}$ are simultaneously defined.

Physical instantiation requires:
\textbf{(1)} Transport to each point $\mathbf{x}_i$
\textbf{(2)} Stabilization at $\mathbf{x}_i$ (oscillatory hole filling)
\textbf{(3)} Transition to next point $\mathbf{x}_{i+1}$

Each step requires finite time $\tau_{\text{circuit}}^{(i)}$ determined by transport coefficients and aperture geometry.

Total tracing time:
\begin{equation}
T_{\text{trace}} = \sum_{i=1}^N \tau_{\text{circuit}}^{(i)}
\end{equation}

This is irreducible: physical processes cannot be instantaneous due to finite transport velocities (bounded by speed of light). \qed
\end{proof}

\subsection{Internal Time Definition}

\begin{definition}[Internal Time]
\label{def:internal_time}
The \emph{internal time} $T_{\text{internal}}$ experienced by a circuit is the sum of circuit completion times for active oscillatory holes:
\begin{equation}
T_{\text{internal}} = \sum_i \tau_{\text{circuit}}^{(i)}
\end{equation}
\end{definition}

\begin{theorem}[Internal Time Theorem]
\label{thm:internal_time}
Internal time equals the cumulative duration of geometric tracing events, not external clock time.
\end{theorem}

\begin{proof}
External clock time $t_{\text{ext}}$ measures coordinate time in laboratory frame.

Internal time $T_{\text{internal}}$ measures operational duration—the time required for circuit to complete geometric tracing.

These are distinct: if circuit operates at rate $r(t)$, then:
\begin{equation}
dT_{\text{internal}} = r(t) \, dt_{\text{ext}}
\end{equation}

Integrating:
\begin{equation}
T_{\text{internal}} = \int_0^{t_{\text{ext}}} r(t) \, dt
\end{equation}

For constant rate $r$:
\begin{equation}
T_{\text{internal}} = r \cdot t_{\text{ext}}
\end{equation}

When $r > 1$: internal time runs faster than external time (accelerated processing).
When $r < 1$: internal time runs slower than external time (decelerated processing).

This explains time dilation/compression in operational states. \qed
\end{proof}

\subsection{The Specious Present}

\begin{definition}[Specious Present]
The \emph{specious present} is the duration of the experiential "now"—the temporal window within which events appear simultaneous.
\end{definition}

\begin{theorem}[Specious Present Theorem]
\label{thm:specious_present}
The specious present duration equals the average circuit completion time for coherent oscillatory hole ensembles:
\begin{equation}
\tau_{\text{present}} = \langle \tau_{\text{circuit}} \rangle \sim 100\text{--}1000 \text{ ms}
\end{equation}
\end{theorem}

\begin{proof}
Circuit operational state requires coherent ensemble of oscillatory holes to be simultaneously active.

Coherence maintained for duration $\tau_{\text{coherence}}$ determined by phase decoherence:
\begin{equation}
\tau_{\text{coherence}} = \frac{1}{\Delta \omega}
\end{equation}
where $\Delta \omega$ is frequency spread.

For typical circuit parameters ($\Delta \omega \sim 10$ Hz):
\begin{equation}
\tau_{\text{coherence}} \sim 100 \text{ ms}
\end{equation}

Events separated by $\Delta t < \tau_{\text{coherence}}$ are processed within same coherent ensemble, appearing simultaneous.

Events separated by $\Delta t > \tau_{\text{coherence}}$ require separate ensembles, appearing sequential.

Therefore:
\begin{equation}
\tau_{\text{present}} = \tau_{\text{coherence}} = \langle \tau_{\text{circuit}} \rangle
\end{equation}

Experimental measurements yield $\tau_{\text{present}} \sim 100$--$1000$ ms, consistent with circuit completion times. \qed
\end{proof}

\subsection{Temporal Elasticity}

\begin{theorem}[Temporal Elasticity Theorem]
\label{thm:temporal_elasticity}
Subjective time dilation/compression correlates with oscillatory hole generation rate and transport efficiency:
\begin{equation}
\frac{T_{\text{subjective}}}{T_{\text{objective}}} = \frac{\dot{n}_{\text{hole}} \cdot \tau_{\text{transport}}}{\dot{n}_{\text{baseline}} \cdot \tau_{\text{baseline}}}
\end{equation}
\end{theorem}

\begin{proof}
Subjective time is internal time $T_{\text{internal}}$, objective time is external time $t_{\text{ext}}$.

From Internal Time Theorem:
\begin{equation}
T_{\text{internal}} = \int_0^{t_{\text{ext}}} r(t) \, dt
\end{equation}

where $r(t) = \dot{n}_{\text{hole}}(t) \cdot \tau_{\text{transport}}(t)$ is the processing rate.

For constant rates:
\begin{equation}
\frac{T_{\text{internal}}}{t_{\text{ext}}} = \dot{n}_{\text{hole}} \cdot \tau_{\text{transport}}
\end{equation}

Normalizing to baseline:
\begin{equation}
\frac{T_{\text{subjective}}}{T_{\text{objective}}} = \frac{\dot{n}_{\text{hole}} \cdot \tau_{\text{transport}}}{\dot{n}_{\text{baseline}} \cdot \tau_{\text{baseline}}}
\end{equation}

\textbf{High arousal}: $\dot{n}_{\text{hole}} \uparrow$ (increased hole generation) $\Rightarrow$ time slows down (more internal events per external time).

\textbf{Low arousal}: $\dot{n}_{\text{hole}} \downarrow$ (decreased hole generation) $\Rightarrow$ time speeds up (fewer internal events per external time).

\textbf{Efficient transport}: $\tau_{\text{transport}} \downarrow$ (faster completion) $\Rightarrow$ time speeds up (rapid processing).

\textbf{Impaired transport}: $\tau_{\text{transport}} \uparrow$ (slower completion) $\Rightarrow$ time slows down (sluggish processing). \qed
\end{proof}

\subsection{Block Universe Compatibility}

\begin{theorem}[Complementarity Theorem]
\label{thm:complementarity}
Physics describes timeless mathematical structure (block universe) while circuits experience temporal tracing (flow) without contradiction.
\end{theorem}

\begin{proof}
\textbf{Physics perspective}: Spacetime is four-dimensional manifold $\mathcal{M}^4$ with metric $g_{\mu\nu}$. All events exist simultaneously in this structure—there is no privileged "now."

\textbf{Circuit perspective}: Circuit operation requires tracing through configuration space, which takes time $T_{\text{trace}} = \sum_i \tau_{\text{circuit}}^{(i)}$.

These are compatible:
\begin{itemize}[nosep]
\item Physics describes \textit{what exists}: the complete geometric structure $\mathcal{G}$
\item Circuits experience \textit{how structure is accessed}: the tracing process $\Gamma(t)$
\end{itemize}

Analogy: A book exists as complete object (all pages simultaneously present), but reading requires temporal progression through pages. The book's existence is timeless; the reading experience is temporal.

Similarly: Spacetime exists as complete structure (all events simultaneously present), but circuit operation requires temporal progression through states. Spacetime's existence is timeless; the operational experience is temporal.

No contradiction: different levels of description. \qed
\end{proof}

\subsection{Circuit Completion Time}

\begin{definition}[Circuit Completion Time]
The \emph{circuit completion time} $\tau_{\text{circuit}}$ is the duration required to stabilize an oscillatory hole through geometric aperture filling.
\end{definition}

\begin{theorem}[Completion Time Formula]
\label{thm:completion_time}
Circuit completion time is determined by transport coefficients and aperture geometry:
\begin{equation}
\tau_{\text{circuit}} = \frac{\dcat}{\xi}
\end{equation}
where $\dcat$ is categorical distance and $\xi$ is transport coefficient.
\end{theorem}

\begin{proof}
Oscillatory hole stabilization requires molecular transport across categorical distance $\dcat$.

Transport rate is $\dot{x} = \xi \cdot F$ where $F$ is driving force.

For variance-minimization-driven transport, $F \sim \nabla V$ where $V$ is free energy.

Time to traverse distance $\dcat$:
\begin{equation}
\tau_{\text{circuit}} = \int_0^{\dcat} \frac{dx}{\xi \cdot F(x)} \approx \frac{\dcat}{\langle \xi \cdot F \rangle}
\end{equation}

For typical parameters ($\dcat \sim 1$--2 partition elements, $\xi \sim 10^{-3}$ s$^{-1}$):
\begin{equation}
\tau_{\text{circuit}} \sim 100\text{--}500 \text{ ms}
\end{equation}

This matches experimental measurements of specious present duration. \qed
\end{proof}

\subsection{Temporal Direction}

\begin{theorem}[Temporal Irreversibility Theorem]
\label{thm:temporal_irreversibility}
Circuit completion creates temporal direction through categorical irreversibility: once a categorical state is completed, it cannot be re-occupied.
\end{theorem}

\begin{proof}
From Categorical Completion Mechanics (Section~\ref{sec:categorical_necessity}):

\textbf{(1)} Circuit occupies category $\mathcal{C}_i$ at time $t_1$.

\textbf{(2)} Oscillatory hole in $\mathcal{C}_i$ is filled (circuit completion).

\textbf{(3)} Completed category $\mathcal{C}_i$ is no longer accessible—circuit must transition to $\mathcal{C}_j \neq \mathcal{C}_i$.

\textbf{(4)} Sequence of completions creates ordered chain: $\mathcal{C}_1 \to \mathcal{C}_2 \to \mathcal{C}_3 \to \cdots$

This ordering is irreversible: cannot return to $\mathcal{C}_i$ once completed.

Temporal direction emerges from this categorical ordering:
\begin{equation}
t_1 < t_2 < t_3 \iff \mathcal{C}_1 \to \mathcal{C}_2 \to \mathcal{C}_3
\end{equation}

Time's arrow is categorical completion's arrow. \qed
\end{proof}

\subsection{Partition Lag and Discretization}

\begin{definition}[Partition Lag]
\label{def:partition_lag}
The \emph{partition lag} $\taulag$ is the time required to complete partition operations, creating discretization of continuous oscillatory reality.
\end{definition}

\begin{theorem}[Temporal Discretization Theorem]
\label{thm:temporal_discretization}
Continuous oscillatory dynamics are perceived as discrete events due to finite partition lag:
\begin{equation}
\Delta t_{\text{perceived}} = \taulag \sim 10\text{--}100 \text{ ms}
\end{equation}
\end{theorem}

\begin{proof}
Oscillatory dynamics evolve continuously with characteristic frequency $\omega \sim 10^{13}$ Hz (H$^+$ oscillations).

Partition operations (categorical assignment) require time $\taulag$ determined by transport and aperture geometry.

Events separated by $\Delta t < \taulag$ occur within same partition operation, appearing simultaneous.

Events separated by $\Delta t > \taulag$ require separate partition operations, appearing sequential.

Therefore, temporal resolution is:
\begin{equation}
\Delta t_{\text{perceived}} = \taulag
\end{equation}

For typical circuit parameters:
\begin{equation}
\taulag \sim 10\text{--}100 \text{ ms}
\end{equation}

This explains why continuous reality is perceived as discrete sequence of events. \qed
\end{proof}

\subsection{Experimental Validation}

\subsubsection{Circuit Completion Time Measurement}

\textbf{Protocol}:
\begin{enumerate}[nosep]
\item Apply step input to circuit (sudden external perturbation)
\item Measure response time to equilibrium
\item Extract $\tau_{\text{circuit}}$ from exponential fit
\item Correlate with subjective time estimates in controlled tasks
\end{enumerate}

\textbf{Expected Result}: $\tau_{\text{circuit}} \approx 100$--$500$ ms, correlating with subjective duration estimates ($r > 0.85$).

\subsubsection{Temporal Elasticity Validation}

\textbf{Protocol}:
\begin{enumerate}[nosep]
\item Modulate hole generation rate (via external input frequency)
\item Modulate transport efficiency (via temperature, coupling strength)
\item Measure subjective time dilation/compression
\item Compare to predicted ratio $\dot{n}_{\text{hole}} \cdot \tau_{\text{transport}}$
\end{enumerate}

\textbf{Expected Result}: Subjective time ratio matches predicted ratio within 15\% variance.

\subsubsection{Partition Lag Measurement}

\textbf{Protocol}:
\begin{enumerate}[nosep]
\item Present stimuli at varying temporal separations
\item Measure simultaneity judgment threshold
\item Extract $\taulag$ as threshold duration
\item Compare to circuit completion time
\end{enumerate}

\textbf{Expected Result}: $\taulag \approx \tau_{\text{circuit}} \sim 10$--$100$ ms.

\subsection{Summary: Time as Geometric Tracing}

We have established:

\textbf{(1) Mathematical vs. Physical}: Mathematical geometry exists timelessly; physical instantiation requires temporal tracing through circuit completion.

\textbf{(2) Internal Time}: Circuit-experienced time $T_{\text{internal}} = \sum_i \tau_{\text{circuit}}^{(i)}$ differs from external clock time, explaining temporal elasticity.

\textbf{(3) Specious Present}: The experiential "now" duration $\tau_{\text{present}} \sim 100$--$1000$ ms equals average circuit completion time.

\textbf{(4) Temporal Elasticity}: Subjective time dilation/compression arises from modulation of hole generation rate and transport efficiency.

\textbf{(5) Block Universe Compatibility}: Physics describes timeless structure; circuits experience temporal tracing—no contradiction.

\textbf{(6) Completion Time}: $\tau_{\text{circuit}} = \dcat/\xi$ determined by categorical distance and transport coefficient.

\textbf{(7) Temporal Direction}: Categorical irreversibility creates time's arrow through ordered completion sequence.

\textbf{(8) Partition Lag}: Finite partition operation time $\taulag \sim 10$--$100$ ms discretizes continuous reality into perceived events.

This establishes time as the duration of geometric tracing during circuit completion, providing the temporal framework for understanding circuit operational dynamics. The next section establishes perception as the external input pathway.
