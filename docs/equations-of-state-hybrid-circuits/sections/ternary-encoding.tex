\section{Ternary Encoding and Continuous Emergence}
\label{sec:ternary_encoding}

The three-dimensional S-entropy space $\Sspace = [0,1]^3$ admits natural encoding through ternary representation, providing a discrete-to-continuous bridge.

\subsection{Ternary Representation Basics}

A $k$-trit ternary string is a sequence $(t_1, t_2, \ldots, t_k)$ where each trit $t_i \in \{0, 1, 2\}$. The set of all $k$-trit strings has cardinality $3^k$.

\textbf{Geometric interpretation}: Each trit specifies refinement along one of three orthogonal axes in $[0,1]^3$:
\begin{align}
t_i = 0 &\leftrightarrow \text{refine along } \Sk \text{ axis} \\
t_i = 1 &\leftrightarrow \text{refine along } \St \text{ axis} \\
t_i = 2 &\leftrightarrow \text{refine along } \Se \text{ axis}
\end{align}

\begin{proof}[Proof of Theorem~\ref{thm:ternary}]
We construct an explicit bijection $\phi: \{0,1,2\}^k \to \mathcal{C}_k$ where $\mathcal{C}_k$ is the set of cells in the $3^k$ partition of $[0,1]^3$.

\textbf{Base case} ($k=1$): Three trits $\{0, 1, 2\}$ map to three cells obtained by dividing $[0,1]^3$ along one axis:
\begin{align}
\phi(0) &= [0, 1] \times [0, 1/3] \times [0, 1] \\
\phi(1) &= [0, 1] \times [1/3, 2/3] \times [0, 1] \\
\phi(2) &= [0, 1] \times [2/3, 1] \times [0, 1]
\end{align}

Wait, this is incorrect. Let me fix the base case. For $k=1$, we should have $3^1 = 3$ cells total, but we need to partition the full 3D space. Let me reconsider...

Actually, for $k=1$, each trit indicates which third of the space along its respective axis. So:
\begin{align}
\phi(0) &= [0, 1/3] \times [0, 1] \times [0, 1] \quad \text{(refine } \Sk \text{)} \\
\phi(1) &= [0, 1] \times [0, 1/3] \times [0, 1] \quad \text{(refine } \St \text{)} \\
\phi(2) &= [0, 1] \times [0, 1] \times [0, 1/3] \quad \text{(refine } \Se \text{)}
\end{align}

No, this still doesn't partition the space correctly. Let me reconsider the theorem statement.

The correct interpretation: A $k$-trit string specifies a sequence of $k$ refinements. Each refinement subdivides the current cell into 3 parts along one axis. After $k$ refinements, we have $3^k$ cells.

\textbf{Recursive construction}:
\begin{itemize}
\item Start with cell $C_0 = [0,1]^3$
\item Trit $t_1$ specifies axis: subdivide into 3 parts along that axis
\item Trit $t_2$ specifies next axis: subdivide each of the 3 cells into 3 parts
\item Continue for $k$ trits, yielding $3^k$ cells
\end{itemize}

The mapping is bijective by construction: distinct trit sequences produce distinct refinement sequences, and every cell in the $3^k$ partition corresponds to exactly one refinement sequence.
\end{proof}

\subsection{Continuous Emergence}

\begin{corollary}[Continuous Limit]
As $k \to \infty$, the discrete $3^k$ cell structure converges to the continuous space $[0,1]^3$:
\begin{equation}
\lim_{k \to \infty} \text{Cell}(t_1, t_2, \ldots, t_k) = \Scoord \in [0,1]^3
\end{equation}
where $\Scoord$ is the unique point in $[0,1]^3$ corresponding to the infinite trit sequence.
\end{corollary}

\begin{proof}
Each trit refines position by factor of 3. After $k$ refinements, position is determined to within $\pm 1/(2 \cdot 3^k)$ along each axis. As $k \to \infty$, this uncertainty vanishes:
\begin{equation}
\lim_{k \to \infty} \frac{1}{3^k} = 0
\end{equation}
Therefore, the infinite trit sequence specifies a unique point in $[0,1]^3$.
\end{proof}

\subsection{Ternary Arithmetic}

Ternary strings support arithmetic operations:

\textbf{Addition}: Component-wise with carry:
\begin{equation}
(t_1, t_2, \ldots, t_k) + (t_1', t_2', \ldots, t_k') = (s_1, s_2, \ldots, s_k, c)
\end{equation}
where $s_i = (t_i + t_i' + c_{i-1}) \mod 3$ and $c_i = \lfloor(t_i + t_i' + c_{i-1})/3\rfloor$ is the carry.

\textbf{Multiplication}: Distributive over addition with ternary multiplication table.

\textbf{Comparison}: Lexicographic ordering.

These operations enable computational algorithms operating directly on ternary-encoded S-entropy coordinates.

\subsection{Information Density}

Ternary encoding achieves information density:
\begin{equation}
\rho_{\text{ternary}} = \frac{\log_2 3^k}{k} = \log_2 3 \approx 1.585 \text{ bits/trit}
\end{equation}

This exceeds binary encoding ($1$ bit/bit) but is less than optimal for three-dimensional space. However, the natural correspondence between trits and axes makes ternary encoding more efficient for S-entropy space operations.

\subsection{Hierarchical Structure}

Ternary encoding naturally represents hierarchical structure:

\textbf{Level 0}: $3^0 = 1$ cell (entire space)

\textbf{Level 1}: $3^1 = 3$ cells (first refinement)

\textbf{Level 2}: $3^2 = 9$ cells (second refinement)

\textbf{Level $k$}: $3^k$ cells ($k$-th refinement)

Each level provides finer resolution while maintaining hierarchical relationships. This structure is ideal for multi-scale circuit analysis where different phenomena occur at different resolutions.

\subsection{Computational Applications}

Ternary encoding enables:

\textbf{(1) Efficient state representation}: $k$ trits encode $3^k$ states, requiring $O(k)$ storage vs. $O(3^k)$ for explicit enumeration.

\textbf{(2) Hierarchical search}: Coarse-to-fine search through ternary tree with depth $k$ and branching factor 3.

\textbf{(3) Adaptive resolution}: Refine only regions of interest by extending trit sequences locally.

\textbf{(4) Parallel computation}: Independent trit positions can be processed in parallel.

\textbf{(5) Error correction}: Redundant encoding through multiple trit sequences converging to same point.
