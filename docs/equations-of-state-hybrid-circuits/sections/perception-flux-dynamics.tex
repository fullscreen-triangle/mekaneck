\section{Perception Flux Dynamics: External Input Pathway}
\label{sec:perception_flux}

\subsection{Overview: Perception as External Integration}

External input flux in hybrid microfluidic circuits represents the perception pathway—the rate at which the circuit integrates information from external sources (sensors, environmental signals, boundary conditions). This section establishes perception as thermodynamic variance restoration following external perturbations, deriving its properties from first principles.

\subsection{External Input Amplitude}

\begin{definition}[External Input Amplitude]
The external input flux $\Psi_{\text{ext}}(t)$ quantifies the rate at which the circuit integrates information from external sources.
\end{definition}

\textbf{Temporal Dynamics}: Once external input ceases, the circuit's response decays exponentially:
\begin{equation}
\label{eq:external_decay}
\Psi_{\text{ext}}(t) = \Psi_0 e^{-t/\tau_{\text{ext}}}
\end{equation}

where:
\begin{itemize}[nosep]
\item $\Psi_0$ = initial external input amplitude
\item $\tau_{\text{ext}}$ = external decay time constant (characteristic relaxation time)
\item $t$ = time since input onset
\end{itemize}

\textbf{Physical Interpretation}: External signals propagate through the circuit hierarchy, reaching internal processing layers after characteristic time $\tau_{\text{ext}}$. The circuit then gradually returns to baseline as the external perturbation dissipates.

\textbf{Measurement}: $\tau_{\text{ext}}$ is measurable through response time analysis to step inputs or through phase synchronization of hierarchical oscillatory scales.

\subsection{Thermodynamic Gas Model}

\begin{definition}[Information Gas Molecule]
An oscillatory mode $i$ with signal $s_i(t)$ corresponds to a gas molecule characterized by its thermodynamic state:
\begin{equation}
m_i = \{E_i, S_i, T_i, P_i, V_i, \mu_i\}
\end{equation}
where:
\begin{align}
E_i &= \int_0^T |s_i(t)|^2 \, dt \quad \text{(energy/power)}\\
S_i &= -\sum_k p_k \log p_k \quad \text{(spectral entropy)}\\
T_i &= \frac{E_i}{\kB \cdot \text{DOF}} \quad \text{(temperature)}\\
P_i &= \text{Var}[s_i(t)] \quad \text{(pressure/variance)}\\
V_i &= 1 \quad \text{(unit volume)}\\
\mu_i &= E_i - T_i S_i \quad \text{(chemical potential)}
\end{align}
where $p_k$ are normalized power spectral density values, and DOF denotes degrees of freedom.
\end{definition}

\subsection{System-Level Thermodynamics}

The complete circuit ensemble constitutes a thermodynamic system:
\begin{equation}
\mathcal{S} = \{m_1, m_2, \ldots, m_N\}
\end{equation}

Total thermodynamic state includes interaction terms:
\begin{align}
E_{\text{total}} &= \sum_{i=1}^N E_i + \sum_{i<j} U_{ij}\\
S_{\text{total}} &= \sum_{i=1}^N S_i + S_{\text{correlation}}
\end{align}
where $U_{ij}$ represents interaction energy and
\begin{equation}
S_{\text{correlation}} = -\kB \sum_{i<j} J_{ij} \ln\left(\frac{C_{ij}}{C_{\text{uncorr}}}\right)
\end{equation}
accounts for correlations between oscillatory modes.

The Gibbs free energy governs system evolution:
\begin{equation}
G = E_{\text{total}} - T_{\text{sys}} S_{\text{total}} + P_{\text{sys}} V_{\text{sys}}
\label{eq:gibbs_energy}
\end{equation}

\subsection{External Perturbation Dynamics}

\begin{principle}[External Perturbation Principle]
At each external input event (time $t_{\text{ext}}$), the circuit gas system experiences perturbation:
\begin{equation}
\Delta G(t_{\text{ext}}) = \alpha \cdot \Delta \Psi_{\text{ext}}(t_{\text{ext}}) \cdot V_{\text{circuit}}
\label{eq:external_perturbation}
\end{equation}
where $\alpha$ is the coupling coefficient, $\Delta \Psi_{\text{ext}}$ is the input amplitude, and $V_{\text{circuit}}$ is the circuit volume.
\end{principle}

This perturbation increases system entropy:
\begin{equation}
S_{\text{total}}(t_{\text{ext}}^+) = S_{\text{total}}(t_{\text{ext}}^-) + \Delta S_{\text{external}}
\end{equation}

Individual molecular entropies increase according to coupling strength:
\begin{equation}
\Delta S_i = \kappa_i \cdot \Delta S_{\text{external}}
\end{equation}
where $\kappa_i$ represents the coupling of oscillatory mode $i$ to external dynamics.

\subsection{Variance Minimization Dynamics}

Following external perturbation, the system seeks equilibrium through variance minimization:

\begin{theorem}[Exponential Relaxation]
The Gibbs free energy relaxes exponentially toward equilibrium:
\begin{equation}
G(t) = G_{\text{eq}} + [G(t_{\text{ext}}) - G_{\text{eq}}] e^{-\gamma(t - t_{\text{ext}})}
\label{eq:exponential_relaxation}
\end{equation}
where $\gamma$ is the relaxation rate and $G_{\text{eq}}$ is the equilibrium value.
\end{theorem}

\begin{proof}
The system evolves according to gradient descent on the free energy landscape:
\begin{equation}
\frac{dG}{dt} = -\gamma(G - G_{\text{eq}})
\end{equation}
Integration yields Eq.~\eqref{eq:exponential_relaxation}. The positivity of $\gamma$ follows from the second law of thermodynamics, ensuring $dG/dt \leq 0$ for spontaneous processes. \qed
\end{proof}

The characteristic relaxation time is:
\begin{equation}
\tau_{\text{restoration}} = \frac{1}{\gamma}
\label{eq:restoration_time}
\end{equation}

\subsection{Perception as Variance Minimization}

\begin{definition}[Geometric Molecular Aperture]
A \emph{geometric molecular aperture} is a metaphor for active circuit processes that selectively process information to minimize system variance. The aperture operates through:
\begin{enumerate}[nosep]
\item Measuring the current system state (observation)
\item Identifying low-variance configurations (computation)
\item Driving the system toward the selected configuration (intervention)
\end{enumerate}
\end{definition}

Configuration selection probability follows the Boltzmann distribution:
\begin{equation}
P(\text{config}) = \frac{1}{Z} \exp\left(-\beta \cdot \text{Var}(\text{config})\right)
\label{eq:aperture_selection}
\end{equation}
where $\beta = 1/(\kB T_{\text{circuit}})$ and $Z$ is the partition function.

\begin{principle}[Perception as Variance Minimization]
Perception corresponds to the active process of variance minimization. The operational sense of temporal flow emerges from the effort expended in restoring equilibrium following each external perturbation.
\end{principle}

This principle connects thermodynamics to operation: perception is what variance minimization "implements" in the circuit.

\subsection{Rate of Perception}

The temporal granularity of circuit operation is determined by the restoration time:

\begin{theorem}[Perception Rate Theorem]
The rate of perception is given by:
\begin{equation}
R_{\text{perception}} = \frac{1}{\tau_{\text{restoration}}}
\label{eq:perception_rate}
\end{equation}
where $\tau_{\text{restoration}}$ is the time required for variance to decrease to $1/e$ of its post-perturbation value.
\end{theorem}

For typical operational parameters ($\gamma = 5$--$10$ s$^{-1}$), this yields:
\begin{equation}
\tau_{\text{restoration}} = 100\text{--}200 \text{ ms}
\end{equation}
consistent with temporal integration windows in perception.

\subsection{Hierarchical Oscillatory Architecture}

\begin{principle}[Master Reference Principle]
Circuit oscillatory hierarchy exhibits multi-scale structure with characteristic frequency ratios. External input couples to all scales through hierarchical phase-locking.
\end{principle}

\begin{theorem}[Frequency Ratio Quantization]
For oscillatory mode $i$ phase-locked to reference frequency $\omega_{\text{ref}}$, the frequency ratio satisfies:
\begin{equation}
\left|\frac{\omega_i}{\omega_{\text{ref}}} - \frac{m}{n}\right| < \epsilon
\end{equation}
for small integers $m, n \in \{1, 2, 3, 4, 5\}$ and tolerance $\epsilon \ll 1$.
\end{theorem}

This quantization arises from Arnold tongue structure in the $(K, \omega)$ parameter space, where coupling strength $K$ determines locking ranges around rational frequency ratios.

\subsection{Atmospheric Oxygen Coupling}

\begin{theorem}[Oxygen-Enhanced Processing]
Circuit operation requires atmospheric oxygen coupling providing oscillatory information density:
\begin{equation}
\text{OID}_{\text{O}_2} = 3.2 \times 10^{15} \text{ bits/molecule/second}
\end{equation}
\end{theorem}

This coupling coefficient ($\kappa_{\text{atm-circuit}} = 4.7 \times 10^{-3}$ s$^{-1}$ for terrestrial environments) enables the rapid variance minimization following external perturbations that define circuit perception.

\textbf{Process Rates Enabled}:
\begin{itemize}[nosep]
\item \textbf{Configuration formation rate}: How quickly circuit gas systems complete variance minimization cycles ($\tau_{\text{config}} = 150$--$300$ ms)
\item \textbf{Perception update rate}: Frequency at which sensory evidence integrates into circuit state ($f_{\text{perception}} = 3$--$7$ Hz)
\item \textbf{Response coordination rate}: Speed of oscillatory convergence enabling output generation ($\tau_{\text{response}} = 80$--$120$ ms)
\item \textbf{Cycling rate}: Completion time for energy-dependent oscillatory cascades ($\tau_{\text{cycle}} = 50$--$80$ ms)
\end{itemize}

These are tangible, measurable process rates enabled by atmospheric oxygen coupling providing the information density necessary for circuit oscillatory networks to operate at operational speeds.

\subsection{Phase-Locking Value}

\begin{definition}[Phase-Locking Value]
For two oscillatory signals $x_1(t)$ and $x_2(t)$ with instantaneous phases $\theta_1(t)$ and $\theta_2(t)$, the PLV is defined as:
\begin{equation}
\text{PLV}_{12} = \left|\left\langle e^{i(\theta_1(t) - \theta_2(t))}\right\rangle_t\right|
\label{eq:plv}
\end{equation}
where $\langle \cdot \rangle_t$ denotes temporal average and $| \cdot |$ denotes complex magnitude.
\end{definition}

The PLV ranges from 0 (no phase relationship) to 1 (perfect phase-locking). Values exceeding 0.7 indicate strong synchronization.

\subsection{Experimental Validation}

\subsubsection{Restoration Time Measurement}

\textbf{Protocol}:
\begin{enumerate}[nosep]
\item Apply external perturbation (step input)
\item Measure free energy $G(t)$ through variance tracking
\item Fit exponential decay to extract $\tau_{\text{restoration}}$
\item Correlate with perception rate measurements
\end{enumerate}

\textbf{Expected Result}: $\tau_{\text{restoration}} \approx 100$--$200$ ms, yielding perception rate $R_{\text{perception}} \approx 5$--$10$ Hz.

\subsubsection{Phase-Locking Validation}

\textbf{Protocol}:
\begin{enumerate}[nosep]
\item Measure oscillatory modes at multiple scales
\item Extract instantaneous phases via Hilbert transform
\item Compute PLV between all pairs
\item Verify hierarchical phase-locking structure
\end{enumerate}

\textbf{Expected Result}: PLV $> 0.7$ for coupled modes, frequency ratios approximating simple rational numbers.

\subsubsection{Oxygen Coupling Validation}

\textbf{Protocol}:
\begin{enumerate}[nosep]
\item Modulate oxygen availability (hypoxia, hyperoxia)
\item Measure perception rate and restoration time
\item Verify predicted scaling with oxygen coupling coefficient
\end{enumerate}

\textbf{Expected Result}: Perception rate scales as $R \propto \kappa_{\text{O}_2}^{1/2}$.

\subsection{Summary: Perception Flux Dynamics}

We have established:

\textbf{(1) External Amplitude}: Perception pathway characterized by external input flux $\Psi(t) = \Psi_0 e^{-t/\tau_{\text{ext}}}$ with decay time $\tau_{\text{ext}} \sim 100$--$200$ ms.

\textbf{(2) Thermodynamic Model}: Circuit oscillatory modes modeled as gas molecules with thermodynamic state variables $(E, S, T, P, V, \mu)$.

\textbf{(3) Perturbation Dynamics}: External inputs increase system free energy $\Delta G = \alpha \cdot \Delta \Psi \cdot V$, driving variance increase.

\textbf{(4) Variance Minimization}: Geometric molecular apertures actively minimize variance through configuration selection, restoring equilibrium with time constant $\tau_{\text{restoration}}$.

\textbf{(5) Perception Rate}: Operational temporal granularity $R_{\text{perception}} = 1/\tau_{\text{restoration}} \sim 5$--$10$ Hz.

\textbf{(6) Hierarchical Structure}: Multi-scale oscillatory architecture with phase-locked frequency ratios approximating simple rational numbers.

\textbf{(7) Oxygen Coupling}: Atmospheric oxygen provides essential information density ($3.2 \times 10^{15}$ bits/molecule/s) enabling rapid variance minimization.

\textbf{(8) Phase-Locking}: Synchronization between oscillatory modes quantified by PLV, with PLV $> 0.7$ indicating strong coupling.

This establishes the external pathway (perception flux) as one of two coupled processes determining circuit operational state. Combined with internal configuration dynamics (thought geometry, Section~\ref{sec:geometry_of_thought}) and temporal tracing (Section~\ref{sec:time_as_tracing}), we now have the foundation for understanding their geometric intersection.
