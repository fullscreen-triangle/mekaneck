%==============================================================================
\section{Contradictions in Temporal Catalysis}
\label{sec:temporal}
%==============================================================================

The temporal interpretation of catalysis, while computationally tractable and historically productive, generates three fundamental logical contradictions when examined rigorously. These contradictions are not merely empirical anomalies that might be resolved through refinement of the temporal framework, but rather represent deep conceptual incoherence in the notion that catalysis operates through temporal acceleration. The present section formalizes these contradictions as theorems, demonstrates their logical structure through formal proofs, and shows that no resolution is possible within the temporal framework. Each contradiction independently refutes temporal catalysis; collectively, they establish that catalytic action must be understood through an alternative conceptual framework.

\subsection{The Instantaneous Concentration Paradox}
\label{sec:instantaneous_paradox}

The first contradiction concerns the relationship between substrate concentration and reaction velocity. The temporal interpretation predicts that reaction velocity should increase without bound as substrate concentration increases, yet experimental observation demonstrates finite saturation velocity. This contradiction reveals that catalysis cannot operate through temporal acceleration.

\begin{theorem}[Instantaneous Concentration Paradox]
\label{thm:instantaneous}
If catalysts operated by temporal acceleration with acceleration factor dependent on substrate availability, then reaction velocity would be unbounded at high substrate concentration. Michaelis-Menten kinetics demonstrates a finite saturation velocity independent of substrate concentration. Therefore, catalysts do not operate by temporal acceleration.
\end{theorem}

\begin{proof}
Assume catalysts operate by temporal acceleration characterised by an acceleration factor $\alpha([S])$ that depends on substrate concentration $[S]$. Under this assumption, the catalyzed reaction velocity $v$ relates to the uncatalyzed velocity $v_0$ through:
\begin{equation}
v = \alpha([S]) \cdot v_0
\label{eq:temporal_acceleration}
\end{equation}

The temporal interpretation posits that a higher substrate concentration provides more opportunities for catalytic encounters per unit time, thereby increasing the effective temporal compression. This implies that the acceleration factor increases monotonically with substrate concentration. In the limit of infinite substrate availability, the acceleration factor should diverge:
\begin{equation}
\lim_{[S] \to \infty} \alpha([S]) = \infty
\label{eq:alpha_divergence}
\end{equation}

Combining Equations~\ref{eq:temporal_acceleration} and \ref{eq:alpha_divergence}, the temporal interpretation predicts unbounded reaction velocity at high substrate concentration:
\begin{equation}
\lim_{[S] \to \infty} v = \lim_{[S] \to \infty} \left[\alpha([S]) \cdot v_0\right] = v_0 \cdot \lim_{[S] \to \infty} \alpha([S]) = \infty
\label{eq:velocity_divergence}
\end{equation}

However, experimental measurements of enzyme kinetics, systematically characterized by \citet{michaelis1913} and subsequently verified across thousands of enzyme systems \citep{cornish-bowden2012}, demonstrate saturation behavior described by the Michaelis-Menten equation:
\begin{equation}
v = \frac{V_{\max}[S]}{K_M + [S]}
\label{eq:michaelis_menten_full}
\end{equation}
where $V_{\max}$ is the maximum velocity and $K_M$ is the Michaelis constant characterizing the substrate concentration at which $v = V_{\max}/2$.

Taking the limit of Equation~\ref{eq:michaelis_menten_full} as substrate concentration approaches infinity:
\begin{equation}
\lim_{[S] \to \infty} v = \lim_{[S] \to \infty} \frac{V_{\max}[S]}{K_M + [S]} = \lim_{[S] \to \infty} \frac{V_{\max}}{K_M/[S] + 1} = V_{\max} < \infty
\label{eq:vmax_finite}
\end{equation}

The maximum velocity $V_{\max}$ is finite and independent of substrate concentration, determined solely by the total enzyme concentration $[E]_{\text{total}}$ and the turnover number $k_{\text{cat}}$:
\begin{equation}
V_{\max} = k_{\text{cat}}[E]_{\text{total}}
\label{eq:vmax_definition}
\end{equation}

The finite saturation value given by Equation~\ref{eq:vmax_finite} directly contradicts the unbounded velocity predicted by Equation~\ref{eq:velocity_divergence}. The temporal acceleration hypothesis is therefore logically inconsistent with experimental observation.
\end{proof}

The standard resolution within temporal theory invokes the concept of enzyme saturation: at high substrate concentration, all enzyme active sites are occupied, and further increases in substrate concentration cannot increase velocity because no additional binding sites are available. However, this resolution merely relocates the contradiction rather than resolving it. If catalysis operates through temporal acceleration, why does the catalytic step itself (the chemical transformation of bound substrate to product) have a finite rate $k_{\text{cat}}$ that cannot be further accelerated? The temporal framework provides no answer beyond asserting that this rate is "intrinsic" to the enzyme, which is a restatement of the observation rather than an explanation.

Furthermore, the saturation argument fails to address the conceptual incoherence. If temporal acceleration is the mechanism of catalysis, and substrate binding removes the limitation of substrate availability, then the bound substrate should undergo transformation at an accelerated rate that increases with the degree of catalytic optimization. Highly evolved enzymes should exhibit higher $k_{\text{cat}}$ values approaching infinity, yet no such trend is observed. Enzymes catalysing simple reactions (e.g., carbonic anhydrase with $k_{\text{cat}} \approx 10^6~\text{s}^{-1}$) and enzymes catalysing complex reactions (e.g., Rubisco with $k_{\text{cat}} \approx 3~\text{s}^{-1}$) both exhibit finite turnover numbers with no apparent correlation to evolutionary age or selective pressure.

\begin{figure*}[htbp]
\centering
\includegraphics[width=0.95\textwidth]{figures/arg1_temporal_triviality.png}
\caption{\textbf{Temporal Triviality—Any Configuration Occurs Naturally Through Thermal Fluctuations.}
\textbf{(A)} Boltzmann probability landscape showing all configurations are thermally accessible. The probability distribution $P(\text{config}) = \exp(-E/k_BT)/Z$ ensures every spatial arrangement, including ``sorted'' states, occurs naturally through fluctuations.
\textbf{(B)} Poincaré recurrence times as a function of sorting degree. Higher sorting corresponds to exponentially longer recurrence times $\tau_{\text{rec}} \sim \exp(N\Delta S)$, but all states eventually recur. The horizontal dashed line indicates laboratory timescales; even highly sorted states recur within observable time for small systems.
\textbf{(C)} Configuration space flow field showing all trajectories converge to equilibrium. The flow follows $\dot{\mathbf{q}} = -\nabla_{\mathbf{q}} F(\mathbf{q})$ where $F$ is the free energy. Red squares mark ``sorted'' configurations; yellow circles mark equilibrium. All paths lead to the central attractor, demonstrating that sorted states are unstable fixed points.
\textbf{(D)} Entropy evolution over time showing fluctuations enable access to all states. The solid black line shows total entropy $S(t) = -k_B \sum_i p_i \ln p_i$ increasing monotonically toward equilibrium (horizontal dashed line). The dotted red line marks the entropy of the ``sorted'' state. Yellow triangles indicate moments when the system spontaneously visits sorted configurations through thermal fluctuations, demonstrating temporal triviality: the demon's purported action is redundant.}
\label{fig:temporal_triviality}
\end{figure*}

\begin{remark}[Categorical Interpretation of Saturation]
\label{rem:categorical_vmax}
The categorical framework provides a natural interpretation of saturation without invoking temporal acceleration. The maximum velocity is determined by the categorical distance $d_{\mathcal{C}}$ traversed per catalytic cycle and the time $\tau_{\text{step}}$ required for each categorical transition:
\begin{equation}
V_{\max} = \frac{[E]_{\text{total}}}{\tau_{\text{cat}}} = \frac{[E]_{\text{total}}}{d_{\mathcal{C}} \cdot \tau_{\text{step}}}
\label{eq:vmax_categorical}
\end{equation}
where $\tau_{\text{cat}} = d_{\mathcal{C}} \cdot \tau_{\text{step}}$ is the total time per catalytic cycle. Saturation occurs because the categorical distance $d_{\mathcal{C}}$ is fixed by the topology of the reaction pathway and cannot be reduced by increasing substrate concentration. The enzyme provides a pathway through categorical space with a specific number of topological transitions, and traversing this pathway requires a minimum time determined by the intrinsic timescale $\tau_{\text{step}}$ of molecular reconfiguration (typically $10^{-13}$ to $10^{-9}$~s for bond rotations, proton transfers, and conformational changes). Substrate concentration affects the frequency of pathway entry but not the pathway length or the speed of traversal.
\end{remark}

\subsection{The Reversible Reaction Paradox}
\label{sec:reversible_paradox}

The second contradiction concerns reversible reactions. Catalysts accelerate both forward and reverse reactions without altering thermodynamic equilibrium constants, a principle established experimentally by \citet{haldane1930} and verified universally across all known catalytic systems. The temporal interpretation cannot account for this bidirectional acceleration without invoking the logical impossibility of time flowing in opposite directions simultaneously.

\begin{theorem}[Reversible Reaction Paradox]
\label{thm:reversible}
If catalysts operated by temporal acceleration, they could not preserve equilibrium constants in reversible reactions. All catalysts preserve equilibrium constants. Therefore, catalysts do not operate by temporal acceleration.
\end{theorem}

\begin{proof}
Consider a reversible reaction between species A and B:
\begin{equation}
\ce{A <=>[$k_f$][$k_r$] B}
\label{eq:reversible_reaction}
\end{equation}
where $k_f$ and $k_r$ are the forward and reverse rate constants. The equilibrium constant is defined by the ratio of these rate constants:
\begin{equation}
K_{\text{eq}} = \frac{k_f}{k_r} = \frac{[\text{B}]_{\text{eq}}}{[\text{A}]_{\text{eq}}}
\label{eq:keq_definition}
\end{equation}
where $[\text{A}]_{\text{eq}}$ and $[\text{B}]_{\text{eq}}$ are equilibrium concentrations.

The temporal acceleration hypothesis must account for how catalysts affect $k_f$ and $k_r$. We examine all logically possible cases.

\textbf{Case 1: Forward acceleration only.}

Suppose the catalyst accelerates only the forward reaction by a factor $\alpha > 1$:
\begin{equation}
k_f' = \alpha \cdot k_f, \quad k_r' = k_r
\label{eq:forward_only}
\end{equation}

The catalyzed equilibrium constant becomes:
\begin{equation}
K_{\text{eq}}' = \frac{k_f'}{k_r'} = \frac{\alpha k_f}{k_r} = \alpha \cdot K_{\text{eq}} > K_{\text{eq}}
\label{eq:keq_forward_shift}
\end{equation}

This predicts that the equilibrium shifts toward products, increasing the equilibrium concentration of B relative to A. However, experimental measurements demonstrate that catalysts do not alter equilibrium constants \citep{haldane1930, alberty1953}:
\begin{equation}
K_{\text{eq}}^{\text{catalyzed}} = K_{\text{eq}}^{\text{uncatalyzed}}
\label{eq:keq_preservation}
\end{equation}

Case 1 contradicts Equation~\ref{eq:keq_preservation}.

\textbf{Case 2: Reverse acceleration only.}

Suppose the catalyst accelerates only the reverse reaction by a factor $\alpha > 1$:
\begin{equation}
k_f' = k_f, \quad k_r' = \alpha \cdot k_r
\label{eq:reverse_only}
\end{equation}

The catalyzed equilibrium constant becomes:
\begin{equation}
K_{\text{eq}}' = \frac{k_f'}{k_r'} = \frac{k_f}{\alpha k_r} = \frac{K_{\text{eq}}}{\alpha} < K_{\text{eq}}
\label{eq:keq_reverse_shift}
\end{equation}

This predicts that the equilibrium shifts toward reactants, decreasing the equilibrium concentration of B relative to A. This again contradicts Equation~\ref{eq:keq_preservation}.

\textbf{Case 3: Asymmetric bidirectional acceleration.}

Suppose the catalyst accelerates both directions but by different factors $\alpha_f \neq \alpha_r$:
\begin{equation}
k_f' = \alpha_f \cdot k_f, \quad k_r' = \alpha_r \cdot k_r
\label{eq:asymmetric_acceleration}
\end{equation}

The catalyzed equilibrium constant becomes:
\begin{equation}
K_{\text{eq}}' = \frac{k_f'}{k_r'} = \frac{\alpha_f k_f}{\alpha_r k_r} = \frac{\alpha_f}{\alpha_r} \cdot K_{\text{eq}}
\label{eq:keq_asymmetric}
\end{equation}

Preservation of the equilibrium constant requires $\alpha_f = \alpha_r$, reducing this case to Case 4 below.

\textbf{Case 4: Symmetric bidirectional acceleration.}

Suppose the catalyst accelerates both directions by the same factor $\alpha$:
\begin{equation}
k_f' = \alpha \cdot k_f, \quad k_r' = \alpha \cdot k_r
\label{eq:symmetric_acceleration}
\end{equation}

The catalyzed equilibrium constant becomes:
\begin{equation}
K_{\text{eq}}' = \frac{k_f'}{k_r'} = \frac{\alpha k_f}{\alpha k_r} = \frac{k_f}{k_r} = K_{\text{eq}}
\label{eq:keq_preserved}
\end{equation}

This case preserves the equilibrium constant, satisfying Equation~\ref{eq:keq_preservation}. However, it requires that the catalyst accelerate time equally in both the forward direction (A $\to$ B) and the reverse direction (B $\to$ A).

The forward reaction proceeds along the reaction coordinate from reactants toward products, corresponding to time flowing forward. The reverse reaction proceeds along the same reaction coordinate from products toward reactants, corresponding to time flowing backward. Temporal acceleration in both directions simultaneously requires time to flow forward and backward at once, which is a logical impossibility. Time is a one-dimensional parameter with a distinguished direction (the thermodynamic arrow of time); it cannot flow in opposite directions simultaneously.

All four cases lead to contradiction. Cases 1, 2, and 3 contradict experimental observation. Case 4 contradicts logical coherence. The temporal acceleration hypothesis is therefore untenable.
\end{proof}

The temporal framework might attempt to resolve this paradox by arguing that "acceleration" refers not to temporal flow but to the rate of barrier crossing, and that catalysts reduce activation barriers equally for forward and reverse reactions. However, this resolution abandons the temporal interpretation. If catalysis reduces barriers rather than accelerating time, then the mechanism is not temporal acceleration but energy landscape modification. Furthermore, the barrier reduction explanation is circular: it explains why both rates increase by asserting that both barriers decrease, but provides no independent criterion for why both barriers decrease by the same factor. The categorical framework provides such a criterion, as formalized in the following corollary.

\begin{corollary}[Equilibrium Preservation through Bidirectional Apertures]
\label{cor:keq}
Catalysts create bidirectional categorical pathways with equal categorical distance in both directions:
\begin{equation}
d_{\mathcal{C}}(\text{A} \to \text{B}) = d_{\mathcal{C}}(\text{B} \to \text{A})
\label{eq:bidirectional_distance}
\end{equation}
This equality automatically preserves the equilibrium constant because both directions traverse the same topological pathway through categorical space, differing only in the direction of traversal. The thermodynamic driving force, determined by the Gibbs free energy difference $\Delta G = -RT \ln K_{\text{eq}}$, determines which direction is favored kinetically, but the categorical pathway structure is symmetric.
\end{corollary}

\begin{proof}
The categorical distance $d_{\mathcal{C}}(\text{A} \to \text{B})$ counts the number of topological transitions required to transform the phase-lock network of state A into the phase-lock network of state B. The reverse distance $d_{\mathcal{C}}(\text{B} \to \text{A})$ counts the number of topological transitions required to transform B back into A. Since these are the same transitions traversed in opposite order, the distances are equal:
\begin{equation}
d_{\mathcal{C}}(\text{A} \to \text{B}) = d_{\mathcal{C}}(\text{B} \to \text{A})
\label{eq:distance_symmetry}
\end{equation}

The rate constants are related to categorical distance through:
\begin{equation}
k_f \propto \frac{1}{d_{\mathcal{C}}(\text{A} \to \text{B})}, \quad k_r \propto \frac{1}{d_{\mathcal{C}}(\text{B} \to \text{A})}
\label{eq:rate_distance_relation}
\end{equation}

If a catalyst reduces categorical distance by providing an alternative pathway with distance $d_{\mathcal{C}}'$, both forward and reverse rates increase by the same factor:
\begin{equation}
\frac{k_f'}{k_f} = \frac{d_{\mathcal{C}}}{d_{\mathcal{C}}'} = \frac{k_r'}{k_r}
\label{eq:rate_ratio_equality}
\end{equation}

Therefore:
\begin{equation}
K_{\text{eq}}' = \frac{k_f'}{k_r'} = \frac{k_f}{k_r} = K_{\text{eq}}
\label{eq:keq_preserved_categorical}
\end{equation}

The equilibrium constant is preserved automatically through the symmetry of categorical pathways.
\end{proof}

\subsection{The Step-Exclusion Paradox}
\label{sec:step_exclusion_paradox}

The third contradiction concerns reaction intermediates. Catalysed reactions proceed through different chemical intermediates than uncatalyzed reactions, a fact established through kinetic isotope effect studies \citep{cleland2003}, intermediate trapping experiments \citep{fersht1999}, and structural characterisation of enzyme-substrate complexes \citep{hammes2002}. The temporal interpretation cannot coherently account for these different intermediates without abandoning the claim that catalysis is acceleration.

\begin{theorem}[Step-Exclusion Paradox]
\label{thm:step-exclusion}
If catalysts operated by temporal acceleration, they would need to either execute identical elementary steps faster than uncatalyzed reactions (requiring an unexplained energy source for transition state stabilisation) or skip elementary steps present in uncatalyzed reactions (implying those steps are chemically unnecessary yet universally observed). Both options are logically incoherent. Therefore, catalysts do not operate by temporal acceleration.
\end{theorem}

\begin{proof}
Consider an uncatalyzed reaction proceeding through a sequence of intermediates:
\begin{equation}
\ce{A -> B -> C -> D}
\label{eq:uncatalyzed_pathway}
\end{equation}
comprising $n = 3$ elementary steps, each characterised by a transition state with activation free energy $\Delta G_i^\ddagger$ for $i = 1, 2, 3$.

The temporal acceleration hypothesis must specify how the catalyzed reaction relates to this uncatalyzed pathway. We examine all logically possible cases.

\textbf{Case 1: Identical steps, faster execution.}

Suppose the catalyzed reaction traverses the same intermediates:
\begin{equation}
\ce{A -> B -> C -> D} \quad \text{(catalyzed, identical intermediates)}
\label{eq:catalyzed_identical}
\end{equation}
but each elementary step proceeds faster, characterised by rate constants $k_i^{\text{cat}} > k_i^{\text{uncat}}$ for $i = 1, 2, 3$.

According to transition state theory \citep{eyring1935}, the rate constant for an elementary step is given by:
\begin{equation}
k = \frac{k_B T}{h} \exp\left(-\frac{\Delta G^\ddagger}{RT}\right)
\label{eq:tst_rate}
\end{equation}
where $k_B$ is Boltzmann's constant, $h$ is Planck's constant, $T$ is temperature, $R$ is the gas constant, and $\Delta G^\ddagger$ is the activation free energy.

Increasing the rate constant requires decreasing the activation free energy:
\begin{equation}
k_i^{\text{cat}} > k_i^{\text{uncat}} \implies \Delta G_i^{\ddagger,\text{cat}} < \Delta G_i^{\ddagger,\text{uncat}}
\label{eq:barrier_reduction}
\end{equation}

The stabilization energy for each transition state is:
\begin{equation}
\Delta E_{\text{stab},i} = \Delta G_i^{\ddagger,\text{uncat}} - \Delta G_i^{\ddagger,\text{cat}} > 0
\label{eq:stabilization_energy}
\end{equation}

The total stabilization energy for the entire pathway is:
\begin{equation}
\Delta E_{\text{stab,total}} = \sum_{i=1}^{3} \Delta E_{\text{stab},i}
\label{eq:total_stabilization}
\end{equation}

This energy must originate from some physical source. However, catalysts exhibit the following properties that preclude external energy input:

\begin{enumerate}
    \item Catalysts do not consume chemical energy currencies such as ATP, GTP, or NADH. Enzymatic catalysis of reactions such as peptide bond hydrolysis or ester hydrolysis proceeds without cofactor consumption \citep{fersht1999}.

    \item Catalysts do not absorb electromagnetic radiation during the catalytic cycle. While photocatalysts exist, conventional thermal catalysts (including all enzymes discussed in this work) operate in the dark without photon absorption.

    \item Catalysts do not generate heat beyond that released by the reaction itself. The enthalpy change $\Delta H$ is identical for catalyzed and uncatalyzed reactions \citep{atkins2010}.

    \item Catalysts do not change the overall Gibbs free energy change $\Delta G$ of the reaction. The thermodynamic driving force is identical for catalyzed and uncatalyzed pathways \citep{haldane1930}.
\end{enumerate}

The only energy available is the enzyme-substrate binding energy $\Delta G_{\text{bind}}$, which is already accounted for in the catalytic cycle as the energy released upon substrate binding and consumed upon product release. This binding energy cannot simultaneously stabilize multiple transition states along the pathway without violating energy conservation. Specifically, if $\Delta G_{\text{bind}}$ is used to stabilize transition state 1, it is no longer available to stabilize transition states 2 and 3.

Furthermore, the binding energy is typically $\Delta G_{\text{bind}} \approx -5$ to $-15$ kcal/mol for enzyme-substrate complexes \citep{fersht1999}, while activation barriers for uncatalyzed reactions are often $\Delta G^\ddagger \approx 15$ to $30$ kcal/mol \citep{radzicka1996}. The binding energy is insufficient to reduce all barriers to near-zero values, yet catalysts achieve rate enhancements of $10^6$ to $10^{17}$-fold \citep{wolfenden2011}, corresponding to barrier reductions of $\Delta\Delta G^\ddagger \approx 8$ to $23$ kcal/mol.

Case 1 therefore requires an energy source that does not exist. Contradiction.

\textbf{Case 2: Fewer steps through intermediate skipping.}

Suppose the catalyzed reaction bypasses intermediates B and C, proceeding directly:
\begin{equation}
\ce{A -> D} \quad \text{(catalyzed, direct pathway)}
\label{eq:catalyzed_direct}
\end{equation}
comprising $n' = 1$ elementary step.

The uncatalyzed reaction traverses B and C because the direct pathway $\ce{A -> D}$ has a prohibitively high activation barrier:
\begin{equation}
\Delta G^\ddagger(\ce{A -> D}) \gg \max_i \Delta G_i^\ddagger(\ce{A -> B -> C -> D})
\label{eq:direct_barrier_high}
\end{equation}

If the catalyst enables the direct pathway by reducing its activation barrier:
\begin{equation}
\Delta G^{\ddagger,\text{cat}}(\ce{A -> D}) < \sum_{i=1}^{3} \Delta G_i^{\ddagger,\text{uncat}}
\label{eq:direct_barrier_reduced}
\end{equation}
then the direct pathway becomes energetically favorable.

This raises a fundamental question: why does the uncatalyzed reaction not use the direct pathway? If the direct pathway is chemically accessible (as demonstrated by the catalyzed reaction), and if the catalyst merely reduces its barrier without changing the chemical mechanism, then the direct pathway should be accessible to the uncatalyzed reaction as well, albeit at a slower rate. The uncatalyzed reaction should exhibit a minor pathway component proceeding directly from A to D, with the major pathway proceeding through B and C. However, no such minor pathway is observed experimentally \citep{fersht1999}.

Alternatively, if the direct pathway is chemically inaccessible in the absence of the catalyst, then the catalyst is not merely reducing a barrier but is fundamentally changing the reaction mechanism by enabling a pathway that does not exist for the uncatalyzed reaction. This contradicts the temporal acceleration hypothesis, which posits that catalysts accelerate existing pathways rather than creating new pathways.

The necessity of intermediates B and C in the uncatalyzed reaction would then depend on the presence or absence of the catalyst, implying that chemical necessity is contingent on external factors. This is absurd: the electronic structure of molecules and the quantum mechanical principles governing bond formation and breaking are independent of whether a catalyst is present. Chemical necessity cannot be contingent.

Case 2 therefore leads to logical incoherence. Contradiction.

\textbf{Case 3: Different intermediates.}

The experimentally observed situation is that catalyzed reactions proceed through different intermediates than uncatalyzed reactions. For example, the uncatalyzed hydrolysis of peptide bonds proceeds through a tetrahedral intermediate formed by direct water attack \citep{radzicka1996}:
\begin{equation}
\ce{R-CO-NH-R' + H2O -> R-C(OH)2-NH-R' -> R-COOH + H2N-R'}
\label{eq:uncatalyzed_hydrolysis}
\end{equation}

In contrast, serine protease-catalyzed hydrolysis proceeds through a covalent acyl-enzyme intermediate \citep{hedstrom2002}:
\begin{equation}
\ce{R-CO-NH-R' + E-Ser-OH -> R-CO-O-Ser-E + H2N-R' -> R-COOH + E-Ser-OH}
\label{eq:catalyzed_hydrolysis}
\end{equation}

The intermediates $\ce{R-C(OH)2-NH-R'}$ (tetrahedral intermediate) and $\ce{R-CO-O-Ser-E}$ (acyl-enzyme) are chemically distinct species with different bonding patterns, charge distributions, and stabilities.

If the catalyzed reaction traverses different intermediates, then it is not an accelerated version of the uncatalyzed reaction but a fundamentally different chemical process. The temporal acceleration hypothesis requires that the same chemical transformation occurs faster, but "same transformation" implies same intermediates. Different intermediates imply different transformation.

Case 3 therefore contradicts the premise of temporal acceleration.

All three cases lead to contradiction. The temporal acceleration hypothesis is untenable.
\end{proof}

\begin{remark}[Categorical Resolution: New Pathways, Not Faster Pathways]
\label{rem:categorical_intermediates}
The categorical framework resolves the step-exclusion paradox by recognizing that catalyzed reactions traverse different categorical space than uncatalyzed reactions. The enzyme-bound intermediates:
\begin{equation}
\ce{A -> E\cdotA -> E\cdotB -> E\cdotC -> E\cdotD -> D}
\label{eq:enzyme_bound_pathway}
\end{equation}
are categorically distinct from the uncatalyzed intermediates B and C because the enzyme-substrate complex $\ce{E\cdotX}$ possesses a phase-lock network topology different from that of the free substrate X. The enzyme provides additional phase-lock edges (hydrogen bonds, electrostatic interactions, van der Waals contacts) that stabilize configurations inaccessible to the free substrate.

The catalyst creates new categorical states rather than accelerating traversal of existing states. The intermediates $\ce{E\cdotB}$ and B are not the same molecule at different speeds but different topological structures in categorical space. The enzyme-bound pathway has lower categorical distance $d_{\mathcal{C}}^{\text{cat}} < d_{\mathcal{C}}^{\text{uncat}}$ because the enzyme provides topological shortcuts through phase-lock network space, not because it accelerates motion through the uncatalyzed pathway.

This resolution explains why different intermediates are observed: they are not different speeds of the same process but different routes through categorical space. It explains why energy is not required: the enzyme does not stabilize transition states of the uncatalyzed pathway but creates new states with intrinsically lower categorical distance. It explains why the uncatalyzed pathway does not use the catalyzed intermediates: those intermediates do not exist in the absence of the enzyme's phase-lock network.
\end{remark}

\subsection{Summary: Temporal Catalysis is Logically Untenable}
\label{sec:temporal_summary}

The three contradictions presented in this section—the instantaneous concentration paradox (Theorem~\ref{thm:instantaneous}), the reversible reaction paradox (Theorem~\ref{thm:reversible}), and the step-exclusion paradox (Theorem~\ref{thm:step-exclusion})—independently and collectively refute the temporal interpretation of catalysis. Each contradiction demonstrates that the hypothesis of temporal acceleration leads to logical incoherence or empirical falsehood. No resolution is possible within the temporal framework without abandoning the core claim that catalysis operates by making reactions happen faster.

The contradictions are not merely technical difficulties that might be resolved through refinement of the mathematical formalism. They are fundamental conceptual problems arising from the attempt to interpret a geometric phenomenon (configurational selection in phase-lock network space) as a temporal phenomenon (acceleration of processes in time). The resolution requires abandoning the temporal framework entirely and adopting an alternative conceptual foundation based on categorical topology, as developed in the following sections.
