\documentclass[12pt,a4paper]{article}

% Packages
\usepackage{amsmath,amssymb,amsthm}
\usepackage{mathtools}
\usepackage{physics}
\usepackage{graphicx}
\usepackage{hyperref}
\usepackage{cleveref}
\usepackage[margin=2.5cm]{geometry}
\usepackage{enumerate}
\usepackage{float}
\usepackage{booktabs}
\usepackage{natbib}

% Theorem environments
\newtheorem{theorem}{Theorem}[section]
\newtheorem{lemma}[theorem]{Lemma}
\newtheorem{corollary}[theorem]{Corollary}
\newtheorem{proposition}[theorem]{Proposition}
\theoremstyle{definition}
\newtheorem{definition}[theorem]{Definition}
\newtheorem{axiom}[theorem]{Axiom}
\theoremstyle{remark}
\newtheorem{remark}[theorem]{Remark}

% Custom commands
\newcommand{\kB}{k_{\mathrm{B}}}
\newcommand{\Sosc}{S_{\mathrm{osc}}}
\newcommand{\Scat}{S_{\mathrm{cat}}}
\newcommand{\Spart}{S_{\mathrm{part}}}
\newcommand{\Stotal}{S_{\mathrm{total}}}

\title{Unified Entropy from Oscillation, Category, and Partition: \\
\large A Thermodynamic Framework for Irreversible Categorical Operations}
\author{}
\date{}

\begin{document}

\maketitle

\begin{abstract}
We present a unified thermodynamic framework demonstrating that oscillatory dynamics, categorical structure, and partition operations yield identical entropy formulations when derived from independent first principles. Beginning with three separate derivations—entropy from bounded oscillatory systems, entropy from categorical state spaces, and entropy from partition branching structures—we prove that all three converge to the same formula $S = \kB M \ln n$, where $M$ represents the dimensional depth and $n$ the branching factor. This convergence establishes a fundamental equivalence: oscillation, category, and partition are not merely related phenomena but identical structures viewed from different perspectives. We then introduce \emph{partition lag}—the irreducible temporal gap between the act of partitioning and the partitioned result—and demonstrate that this lag generates entropy through undetermined residue. The key result is that partition operations are thermodynamically irreversible: composition cannot recover the entropy lost to partition boundaries. We apply this framework to analyse physical systems including finite geometric partitioning of aggregate properties, infinite subdivision of bounded continuous intervals, continuous-to-discrete temporal decomposition, and identity persistence under sequential component exchange. The thermodynamic analysis reveals that certain classical puzzles in philosophy dissolve when recognised as consequences of partition-induced entropy production.
\end{abstract}

\section{Introduction}
\label{sec:introduction}

The relationship between microscopic dynamics and macroscopic thermodynamics has been a central problem in physics since Boltzmann's statistical interpretation of entropy. We present a framework that unifies three apparently distinct approaches to this relationship: oscillatory mechanics, categorical enumeration, and partition theory. Our central result is that these three approaches, when developed from independent axioms, yield identical entropy formulations—demonstrating not merely an analogy but a fundamental equivalence.

The paper proceeds in three parts. In Part I, we derive entropy independently from oscillatory, categorical, and partition perspectives, then prove their mathematical equivalence. In Part II, we introduce partition lag and demonstrate that partition operations generate irreversible entropy through undetermined residue. In Part III, we apply the framework to physical systems, revealing thermodynamic resolutions to problems in mechanics and ontology.

Throughout, we employ standard thermodynamic notation with Boltzmann's constant $\kB = 1.380649 \times 10^{-23}$ J/K explicit, emphasising that our results concern physical entropy rather than abstract information measures.

%============================================================
% PART I: INDEPENDENT ENTROPY DERIVATIONS
%============================================================

\part{Entropy Unification}
\label{part:unification}

\input{sections/oscillatory-mechanics}
\input{sections/categorical-mechanics}
\input{sections/partitioning-mechanisms}
\input{sections/entropy-unification}

%============================================================
% PART II: PARTITION LAG AND IRREVERSIBILITY
%============================================================

\part{Partition Lag and Irreversibility}
\label{part:partition_lag}

\input{sections/partition-lag}

%============================================================
% PART III: PHYSICAL APPLICATIONS
%============================================================

\part{Physical Applications}
\label{part:applications}

\input{sections/heap-paradox}
\input{sections/zenos-paradox}
\input{sections/ship-of-theseus}

\section{Conclusion}
\label{sec:conclusion}

We have established three independent derivations of entropy—from oscillatory mechanics, categorical structure, and partition theory—and proved their mathematical equivalence. The unified entropy formula $S = \kB M \ln n$ emerges identically from all three perspectives, demonstrating that oscillation, category, and partition are not analogous but identical.

The partition lag mechanism reveals why composition cannot reverse partition: each partition operation generates undetermined residue that increases entropy by $\Delta S > 0$. This irreversibility is not a limitation of particular physical systems but a consequence of the fundamental structure of categorical operations.

The physical applications demonstrate that systems traditionally analysed through composition—asking how parts combine to form wholes—are more naturally understood through partition—asking how wholes decompose into parts with entropy loss. The thermodynamic framework provides quantitative predictions for the entropy cost of partition and explains why certain properties of wholes cannot be recovered from parts.

\bibliographystyle{plainnat}
\bibliography{references}

\end{document}

